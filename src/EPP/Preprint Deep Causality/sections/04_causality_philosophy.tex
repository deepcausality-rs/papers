\section{Effect Propagation: A Philosophical Foundation for Post-Quantum Causality}
\label{sec:causality_philosophy}

The concept of causality, fundamental to our understanding of the world, has evolved from ancient philosophical inquiries identifying multiple necessary factors for a cause to be true. This classical view required a spatiotemporal background, as argued by Seneca, though this notion faced philosophical debate from relationalists like Leibniz and later critiques from Bertrand Russell regarding its status in physics. 
The advent of quantum gravity, however, suggesting spacetime may be emergent from a deeper reality and causal structures can be indefinite, poses a profound challenge to traditional causality which relies on a defined background spacetime. 


In response, the author proposes the "Effect Propagation Process" as a philosophical framework for post-quantum causality. In this framework, causality is viewed not as discrete links between events in a fixed spacetime, but as the fundamental continuous transfer of influence from within a deeper quantum structure. 
The "Effect Propagation Process" framework attempts to 
provide a unified perspective for post-quantum causality that remains compatible with classical causality and aligns with leading theories in quantum gravity. 

\subsection{History of Causality}

In Timaeus (c 360 BC), one of the first known attempts to understand why things come into being, Plato explores the cause (aitia) and “contributing causes” (sunaitiai). Plato stipulates that multiple indispensable factors, the model, the maker (Demiurge), the material, and the space (receptacle), explain how the physical world with all the things in it are made\cite{SEP_Timaeus}.

Aristotle (c 350 BC) formalized the notion of causality in his Metaphysics\cite{SEP_Metaphysics}  which serves as the foundation of the classical framework of the "Four Causes”\cite{SEP_AristotleCausality}. These are:

\begin{enumerate}
    \item The material cause or that which is given in reply to the question “What is it made out of?” 
    \item The formal cause or that which is given in reply to the question “"What is it?”. What is singled out in the answer is the essence or the what-it-is-to-be something.
    \item The efficient cause or that which is given in reply to the question: “Where does change (or motion) come from?”. What is singled out in the answer is the whence of change (or motion).
    \item The final cause, the end purpose, that is which is given in reply to the question: “What is its good?”. What is singled out in the answer is that for the sake of which something is done or takes place.

\end{enumerate}

Aristotle moved beyond simplistic notions of causality to identify multiple, distinct factors necessary for explaining how things come into being and change. 

Seneca (c 56 AD) argues in letter 65\cite{Seneca_Letters} that cause and effect operate within a stage (space) and follow an order (time). Remove the stage or the order, and the conventional understanding of 'making something' or 'causing something' breaks down. His argument highlights time and place as indispensable conditions for 'making something', identifying the need for a spatiotemporal context as a prerequisite for classical causality. This focus on space and time as necessary conditions served as a precursor to later physical concepts treating spacetime as a background for causal processes.

The idea of space and time as an independent background did not go unchallenged in philosophy, even before the advent of modern physics. Gottfried Wilhelm Leibniz (1646-1716), in his famous debate with Isaac Newton's representative, rejected the concept of absolute space and absolute time as independent, fundamental containers. Instead, Leibniz proposed\cite{SEP_LeibnizPhysics} a relational view, arguing that space is merely the order of coexisting things (simultaneity), and time is merely the order of successive things. 

For Leibniz, space and time were not substances or backgrounds that existed on their own, but systems of relations between the objects and events that constitute reality. He arrived at this conclusion through rigorous philosophical reasoning, arguing from fundamental metaphysical principles, such as the Principle of Sufficient Reason, that the concept of absolute space and time was logically untenable. This relational perspective offered a significant historical philosophical alternative to the pre-eminent Newtonian worldview of his time.

\newpage

\subsection{The impact of General Relativity on Causality}
\label{subsec:general_relativity_impact}

Einstein’s theory of General Relativity\cite{Einstein_1915Papers} maintained the requirement for a spatiotemporal context as a prerequisite for causality, echoing Seneca's insight. However, GR fundamentally transformed this prerequisite from the static, absolute background of earlier physics into a dynamic spacetime manifold, warped and influenced by matter and energy. Unlike a passive stage, the spacetime of GR is an active participant in the causal processes unfolding within it.

Bertrand Russell (1872 - 1970) wrote in his 1912 essay "On the Notion of Cause"\cite{Russell_Cause1912}:

\begin{quote}
"The law of causality, I believe, like much that passes muster among philosophers, is a relic of a bygone age, surviving, like the monarchy, only because it is erroneously supposed to do no harm." 
\end{quote}


For Russell, the traditional idea of causality, a necessary, temporal asymmetrical link between distinct events, did not match the sophisticated, law-based descriptions used in successful physic. Russell's argument roots in his observation that Physics describe how the state of a system evolves. The focus is on the state of a system (e.g., position, velocity, field strength across space) and how that entire state changes continuously, rather than isolating specific, discrete events as "causes" and "effects." 

Many fundamental physical laws are symmetrical in time or involve reciprocal relationships. If state S1 at time t1 is related to state S2 at time t2 by a law, it's equally true that state S2 at time t2 is related to state S1 at time t1 by the same law. The relationship isn't a one-way street from a necessary "cause" to a dependent "effect." Knowing the state at any time allows you (ideally, in a deterministic system) to calculate the state at any other time, past or future. Therefore, which one is the "cause" and which is the "effect" becomes arbitrary and Russel demonstrated remarkable foresight with this assessment. 

\subsection{The impact of Quantum Gravity on Causality}
\label{subsec:quantum_gravity_impact}

The emergence of Quantum Gravity with its Dynamic Causal Structure\cite{hardy2005probability} directly challenges the traditional separation of cause and effect, removes the space and time frame of reference, and introduces indefinite causal structures\cite{mrini2024indefinitecausalstructurecausal} with Time-Symmetry. Therefore, the fundamental conceptualization of cause, effect, time, and space has been fundamentally challenged. 

Seneca defines the necessary conditions for causality as space and time that provides location and ordering. In general relativity, the necessary conditions have been converged into a single spacetime manifold that serves as a frame of reference to determine location and ordering. However, Quantum Gravity requires a fundamental set of rules from which spatiotemporal relationships and causal order can emerge. 
Space and time are not necessary external conditions, but potential emergent properties of the necessary internal structure of reality itself. The problem isn’t anymore whether spacetime is static or dynamic, but that that spacetime itself may emerges from the quantum level and thus positions itself as a higher order effect of a generative process.

\begin{quote}
    Instead of asking "Where and When does this cause operate?", quantum gravity asks "What is the underlying process from which the notions of 'where' and 'when', and thus causal order, emerge?” 
\end{quote}

Russell saw physics moving towards laws governing states, a view echoed in quantum gravity's search for fundamental rules or principles governing the structure from which spacetime and causal order emerge. At this stage, the understanding of causality evolved from a structure that required the existence of space and time as pre-given conditions towards emergent causality. This emergent causality does not rely on a pre-existing spacetime, but is grounded in a more fundamental level of reality—a set of underlying rules or principles (i.e. conceptualized as a 'generating function') that determine the fundamental potential for existence, relation, and the eventual manifestation of spatiotemporal properties.

If this fundamental level (or its 'output' in terms of emergent properties) didn't include states that resemble classical spacetime, then the conditions Seneca deemed necessary for "making things" (definite causal links) wouldn't appear.

From the quantum perspective, 'space and time' that Seneca identified as necessary are not fundamental inputs to causality, but are outputs or emerging properties from a deeper quantum process grounded in this fundamental level.

The conceptualization of this fundamental level as a "generating function" captures the idea of a deeper source from which the necessary condition of classical causality's spatiotemporal structure arises. It's a shift from asking "What causes X given spacetime?" to "What process generates spacetime (and thus enables X to be caused)?".

\subsection{Causality as Effect Propagation Process}
\label{subsec:effect_propagation_process}

Logically, the understanding of causality, particularly when considering the insights from modern physics that challenge classical notions of fixed spacetime and linear temporal order (as discussed in Section \ref{subsec:quantum_gravity_impact}), evolves further towards an \textbf{effect propagation process}. Framing causality in this manner moves beyond discrete, temporally-ordered cause-effect pairs towards a more fundamental view of how influence, change, or correlation is transmitted or manifested through the underlying structure of a system. This perspective profoundly informs the architecture of DeepCausality. "Effect propagation process" means:

\begin{itemize}
    \item \textbf{Focus on the Transfer of Influence:}
    The emphasis shifts from identifying isolated ``causes'' that produce distinct ``effects'' to understanding the \textit{process} by which a change, information, energy, or even a statistical correlation at one point or within one component of a system influences or leads to changes in another. This is less about a discrete ``cause A produces effect B'' event, which is often linked by an assumed external space and time, and more about a continuous or discrete transfer of state or activation \textit{through the defined structure of the modeled reality itself}.
    \textit{DeepCausality Realization:} This directly motivates the design of the \texttt{CausaloidGraph}. Reasoning within a \texttt{CausaloidGraph} is precisely the simulation of this transfer of influence. The activation of one `Causaloid` can trigger the evaluation, and potential activation, of connected `Causaloids` through the defined (hyper)edges. These edges represent the pathways of influence, and the `Causaloid` functions themselves embody the specific logic of how that influence is processed and transformed. The state of the graph at any point reflects the current propagation of effects based on input observations and context.

    \item \textbf{Detachment from Fixed Spacetime (Embracing Relational and Abstract Structures):}
    In conceptual frameworks inspired by quantum gravity, where spacetime geometry (and thus smooth paths) can be in a superposition or even non-existent at a fundamental level, ``propagation'' isn't necessarily movement along a geodesic in a fixed manifold. Instead, it signifies the propagation of influence through the connected elements defined by the system's intrinsic structure thus allowing for causal reasoning in non-Euclidean ``spaces.''
    
    \textit{DeepCausality Realization:} This principle is a cornerstone of DeepCausality’s \texttt{Context Engine}. The framework's explicit support for user-defined \texttt{Contextoid} types and its trait-based approach to spatial and temporal data (e.g., the \texttt{Spatial<V>} trait) allow the ``stage'' for causality to be defined by the problem’s intrinsic structure rather than being confined to a default Euclidean geometry. Influence can propagate through a social network, a conceptual hierarchy, or a correlation matrix, all modeled as context hypergraphs. A `Causaloid` function can query these non-Euclidean contexts, making causal inferences based on relational properties that have no direct physical analogue, thus detaching causal reasoning from a singular spacetime.

    \item \textbf{Alignment with Emergence (Classical Causality as a Limiting Case):}
    When the fundamental structure of a system \textit{does} give rise to something akin to classical spacetime (e.g., in physical systems modeled with Euclidean coordinates and a linear time progression), this underlying, more general ``effect propagation'' process would manifest as what we typically observe as propagation through that spacetime (e.g., elements traveling from one spatiotemporal point to another). Classical cause-and-effect, with its clear temporal asymmetry, becomes the macroscopic or classical limit of this deeper, symmetric and relational, propagation process.
    \textit{DeepCausality Realization:} DeepCausality accommodates this by design. Users can define `Contextoids` representing standard Euclidean `Space`, `Time`, and `SpaceTime`. `Causaloids` can then operate within this classical framework, evaluating conditions based on physical laws and temporal sequences. However, because the framework's core logic is built upon the more general principle of effect propagation through its hypergraph structures, it seamlessly handles both these classical scenarios and the non-classical, abstract scenarios without requiring a different underlying reasoning engine. The classical view is simply one specific type of structure and context that the general engine can operate upon.

    \item \textbf{Handling Indefinite or Dynamic Causal Structures:}
    In advanced physical theories or highly complex adaptive systems, the causal structure itself might be indefinite (e.g., a quantum superposition of different causal orders) or dynamically evolving. In such scenarios, ``effect propagation'' can be understood as the influence propagating through a superposition of possible pathways or through a structure whose relational links change over time. The ``effect'' isn't tied to a single, definite causal link but is a result of the propagation through all possible (weighted) connections or an evolving topology.
    \textit{DeepCausality Realization:} While full quantum indefiniteness is a long-term vision (see Section \ref{subsec:qedc}), DeepCausality takes steps towards handling dynamic structures. The \texttt{Adjustable} protocol for `Contextoids` allows the values within a context to change, thereby altering the conditions for `Causaloid` activation. More profoundly, the envisioned capability for dynamic \texttt{CausaloidGraph} adaptation (discussed in Section \ref{sec:advanced_modeling}) where the graph structure itself can be modified at runtime (e.g., adding/removing `Causaloids` or their connections based on contextual triggers or learning) directly embodies this idea of influence propagating through an evolving topology. The `Causal State Machine` (CSM) can orchestrate such changes, allowing the system's understanding of causal pathways to adapt.
\end{itemize}

Russel correctly pointed out that the asymmetric relation between a cause and its effect depends on the presence of a temporal order i.e. the time arrow. Because of the detachment from a fixed spacetime, there is conceptually no temporal order left in the notion of effect propagation process. Therefore, the entire classical concept of causality of “A happened before B” cannot be established and therefore the distinction of what is a Cause (happened before) and an effect (happened after) cannot be made anymore. As Russell foresaw regarding the time-symmetry of fundamental laws, in a realm without a fundamental temporal order, identifying one event as definitively 'cause' and another as definitively 'effect' becomes untenable."

Instead of abandoning causality altogether, merging cause and effect into one entity, as proposed by Hardy\cite{hardy2005probability}, makes causality agnostic of temporal order and, more importantly, enables the definition of causality in terms of uniform effect propagation. 

In this post-quantum context, the term "propagation" does not imply movement through a pre-existing space or over a defined time interval in the classical sense. Instead, it refers to the fundamental process by which influence, correlation, or causal efficacy is transferred or unfolds within the underlying structure of reality itself. This fundamental process is what gives rise to the appearance of propagation through spacetime in the classical, emergent limit.

Furthermore, while classical causality relies on a definite temporal order and the apparent arrow of time (cause preceding effect), within the framework of the 'effect propagation process', this directedness is understood as an emergent property, arising from the fundamental process in the classical limit, rather than being a fundamental feature of the process itself.

Replacing the notion of causality as requiring a pre-defined stage of space and time with the notion of causality being an effect propagation process then grounds post-quantum causality into a deeper quantum generative process. The notion of post-quantum causality as an effect propagation process remains agnostic of the exact quantum generative process. 

Consequently, causality is understood as an effect propagation process that emerges from the fundamental structure or set of rules (akin to a generating function) from which spatiotemporal relationships emerge. Operating within this same fundamental structure, the process dictates how fundamental degrees of freedom relate and evolve.

The notion of effect propagation process excludes the operational details of any particular physics theory and offer a coherent way of thinking about causality that aligns with the emergent nature of spacetime and the potential indefiniteness of causal order in the quantum realm. In doing so, it provides a unifying perspective for post-quantum causality by acknowledging the philosophical arguments for dynamic/relational reality which is rooted in process philosophy and relationalism.

This philosophical concept of the effect propagation process finds support in various areas of philosophy and physics. For example, effect propagation process finds support in physical theories that propose fundamental structures underlying spacetime such as Causal Set Theory and generalizes the idea of influence transfer present in standard physics i.e. Quantum Field Theory. Lastly, the notion of causality as effect propagation process offers a philosophical interpretation for mathematical tools that describe non-classical causal behavior, such as Process Matrices and Operational Frameworks like Hardy’s Causaloids.

In classic causality, the key questions are: 

\begin{quote}
    Why things exist, how do they change, and how things come into being?
\end{quote}

In post-quantum causality, the key questions are:

\begin{quote}
     why things exist, how effects propagates, and how effects emerge into being?
\end{quote}

Shifting the understanding of causality from a discrete spacetime bound event focus towards a spacetime agnostic continuous effect propagation process may help to advance complex systems that increasingly become an omnipresent reality. 

By adopting the perspective of causality as an effect propagation process, DeepCausality aims to provide a more fundamental, flexible, and robust framework. It is designed to be agnostic to the specifics of whether effects propagate through physical spacetime, a network of social relations, or an abstract conceptual space, focusing instead on the structural and functional definition of how influence is transferred and transformed within any given modeled system. This approach is crucial for building intelligent systems that can reason effectively in the diverse and often non-classical geometries of complex real-world problems.