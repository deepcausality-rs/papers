\section{Conclusion}
\label{sec:conclusion}

The Effect Propagation Process (EPP) framework rethinks causality as a continuous transfer of effects originating from a a Euclidean or non-Euclidean environment that might be subject to relativistic effects. This paper has articulated the EPP as a novel computational ontology, grounded in a relational philosophy that directly addresses the limitations of classical causal models in dynamic, complex systems.

The EPP's core metaphysical principles, Monoidic Primitives, Isomorphic Recursive Composition, and Contextual Relativity, together form a robust framework for understanding the nature of being and becoming in a coherent and computable way. 

The EPP's unique philosophical grounded design enables capabilities previously intractable for traditional methods. Its principle of Contextual Relativity allows for real-time, relativistic counterfactual analysis, where a system can explore multiple hypothetical futures by dynamically altering its frames of reference. This capacity for "what-if" reasoning, executed at scale, transforms autonomous decision-making from rigid, pre-scripted responses to adaptive, consequence-aware optimization. 

The EPP ontology offers a unified philosophical language of causality that handles new challenges, remains transparent and verifiable, and provides the essential architectural primitives for dynamic emergent causality even in relativistic contexts.