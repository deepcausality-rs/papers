\section{Future Work}
\label{sec:future_work}

The preceding chapters have established the Effect Propagation Process as a foundational framework for modeling dynamic contextual causality. The framework's philosophical underpinnings have informed its implementation in DeepCausality. However, because of the single axiomatic definition of generalized causality the EPP is build upon, the resulting vast space of new possibilities will inform a substantial amount of future work that clearly falls outside of the scope the foundational work presented in this monograph. 


\subsection{Contextual Causal Discovery:} 


A primary focus for future work is the development of a robust contextual causal discovery algorithm to automatically construct Causaloid Graphs for a given domain. The main challenge roots in the interaction between context and the causal model because and that results in triple challenge.

First, for a given domain, one context graph structure may work better than others for a given model, thus finding an "optimal" context graph w.r.t. supporting the causal model is the first major challenge. 

 Second, when holding the context graph constant, the structure of the graph has significant impact on the overall efficacy of the model and thus finding an optimal causal graph for a given context is the second major challenge. Unfortunately,  the context and causal graph cannot be separated from each other therefore it is most likely that these two challenges are deeply interrelated and require an integrated approach that searches for context representation and the causal model in tandem.  
 
Third, the causal function embedded into each causaloid also implies a not-trivial search, construct, and validate challenge. This challenge deeply interrelates with contemporary work on decomposing causality during the discovery process. However, one complication is that causaloids are isomorphic recursive and may container another causaloid, a causal collection, or another causaloid graph to represent arbitrary complex causal structure. It is not known to the author if sufficiently advanced algorithms that can discover complex nested causal structure exist. 


Addressing the challenges imposed by contextual causal learning would require a hybrid approach. For example, for finding causal relation in raw data, an adaptation of the SURD algorithm\cite{martinez2024decomposing} may result in a workable foundation. For the challenge of the entangled context and causal graph structure, algorithms from the field of search based procedural content generation\cite{togelius2011search} may offer a path forward. Dealing with the expected computational cost of resulting from the high dimensional nature of the problem of finding two interdependent hypergraphs suggest the exploration of deep neuro\-evolution adapted to high dimensional data\cite{colas2020scaling}. Furthermore, the broader exploration of novelty search\cite{lehman2011abandoning} algorithms may yield deep insights about robust contextual causal discovery. 

\subsection{Formalization of the Effect Ethos:} 

The Teloid and Effect Ethos are introduced as ontological primitives. The next logical step is the formalization of a calculus to establish the foundation for verifiable teleological reasoning that enables the design and validation of systems whose adherence to specified intent or regulations is mathematically provable. This line of study might be particular valuable for regulated industries such as robotics, avionics, and defense. 

\subsection{Exploration of Conceptual Blueprints}


The ultimate purpose of a foundational framework is its ability to provide a clear path to solving problems that were previously intractable. The following conceptual blueprints are presented as demonstrations of the EPP's expressive power. They represent a class of dynamic, and context-aware systems that were previously infeasible to solve with classical methods of computational causality.

The engineering of these systems would require a major, dedicated effort to build the necessary domain-specific tooling (e.g., relativistic physics simulators, validated models of cancer biology, or resilient decentralized communication protocols). However, these blueprints serve to illustrate how the EPP and its implementation in DeepCausality could provide the foundation upon which such tooling and dynamic systems could be built. 

\newpage