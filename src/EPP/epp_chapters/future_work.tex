\section{Future Work}
\label{sec:future_work}

The preceding chapters have established the Effect Propagation Process as a foundational framework for modeling dynamic contextual causality. The framework's philosophical underpinnings have informed its implementation in DeepCausality. However, because of the single axiomatic definition of generalized causality the EPP is built upon, the resulting vast space of new possibilities will inform a substantial amount of future work that clearly falls outside of the scope of the foundational work presented in this monograph. 


\subsection{Contextual Causal Discovery}


A primary focus for future work is the development of a robust contextual causal discovery algorithm to automatically construct Causaloid Graphs for a given domain. The main challenge roots in the interaction between context and the causal model, and that results in a triple challenge.

First, for a given domain, one context graph structure may work better than others for a given model, thus finding an "optimal" context graph with respect to supporting the causal model is the first major challenge. 

 Second, when holding the context graph constant, the structure of the graph has a significant impact on the overall efficacy of the model and thus finding an optimal causal graph for a given context is the second major challenge. Unfortunately,  the context and causal graph cannot be separated from each other; therefore, it is most likely that these two challenges are deeply interrelated and require an integrated approach that searches for context representation and the causal model in tandem.  
 
Third, the causal function embedded into each causaloid also implies a nontrivial search, construction, and validation challenge. This challenge deeply interrelates with contemporary work on decomposing causality during the discovery process. However, one complication is that causaloids are isomorphically recursive and may contain another causaloid, a causal collection, or another causaloid graph to represent arbitrarily complex causal structures. It is not known to the author if sufficiently advanced algorithms that can discover complex nested causal structures exist. 


Addressing the challenges imposed by contextual causal learning would require a hybrid approach. For example, for finding causal relations in raw data, an adaptation of the SURD algorithm\cite{martinez2024decomposing} may result in a workable foundation. For the challenge of the entangled context and causal graph structure, algorithms from the field of search-based procedural content generation\cite{togelius2011search} may offer a path forward. Dealing with the expected computational cost resulting from the high dimensional nature of the problem of finding two interdependent hypergraphs suggests the exploration of deep neuro\-evolution adapted to high dimensional data\cite{colas2020scaling}. Furthermore, the broader exploration of novelty search\cite{lehman2011abandoning} algorithms may yield deep insights about robust contextual causal discovery. 

\subsection{Formalization of the Effect Ethos}

The Teloid and Effect Ethos are introduced as ontological primitives. The next logical step is the formalization of a calculus to establish the foundation for verifiable teleological reasoning that enables the design and validation of systems whose adherence to specified intent or regulations is mathematically provable. This line of study might be particularly valuable for regulated industries such as robotics, avionics, and defense. 

%\subsection{Indefinite Causal Structures}
% 

\newpage
