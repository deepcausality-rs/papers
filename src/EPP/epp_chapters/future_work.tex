\section{Future Work}
\label{sec:future_work}

The preceding chapters have established the Effect Propagation Process as a foundational framework for modeling dynamic contextual causality. The framework's philosophical underpinnings have informed its implementation in DeepCausality. However, because of the single axiomatic definition of generalized causality the EPP is built upon, the resulting vast space of new possibilities will inform future work that falls outside of the scope of this monograph. 

\subsection{Functional Causality}
\label{sec:future_work_functional_causality}

Based on the function theoretical conjecture established in section \ref{}, future work may explore to apply concepts from category theory like monads and functors to define operations such as map, flatmap or filter for causal function. By doing so, scholars of formal methods can formulate a Causal Calculus with the same level of rigor and expressive power as the Lambda Calculus and thus pave the way for causal programming.
 
Furthermore, future work may explore the synergy between causal functions and functional programming with an extension to define a type based causal effect system relative to an explicit context to use the vast existing work on type and effect theory to make causal models increasingly statically verifiable. Instead of relying on a simple type \textit{PropagatingEffect}, a more expressive type \textit{PropagatingEffect<T, C>} is known to apply only to type T on Context C. The compiler could then statically prove that a propagating effect is applied to the correct type and, critically, to the correct context thus preventing an entire category of incorrect model transfer errors at compile time thus directly supporting verifiable safety.  
 
Functional programming methods, with their emphasis on immutability, pure functions, and referential transparency, inherently facilitate formal verification which simplifies validation and verification for causal models written in a functional causal programming language. A a functional causal programming language combined with the aforementioned type based causal effect system amount to a principled functional causality programming model rooted in function theory. The impact of functional causality programming on regulated industries can be substantial in terms of reduced validation effort and reduced certification time and cost. 

 
\subsection{Contextual Causal Discovery}


A primary focus for future work is the development of a robust contextual causal discovery algorithm to automatically construct Causaloid Graphs for a given domain. The main challenge is rooted in the interaction between context and the causal model, and that results in a triple challenge.

First, for a given domain, one context graph structure may work better than others for a given model, thus finding an "optimal" context graph with respect to supporting the causal model is the first major challenge. 

 Second, when holding the context graph constant, the structure of the graph has a significant impact on the overall efficacy of the model and thus finding an optimal causal graph for a given context is the second major challenge. Unfortunately,  the context and causal graph cannot be separated from each other; therefore, it is most likely that these two challenges are deeply interrelated and require an integrated approach that searches for context representation and the causal model in tandem.  
 
Third, the causal function embedded into each causaloid also implies a nontrivial search, construction, and validation challenge. This challenge deeply interrelates with contemporary work on decomposing causality during the discovery process. However, one complication is that causaloids are isomorphically recursive and may contain another causaloid, a causal collection, or another causaloid graph to represent arbitrarily complex causal structures. It is not known to the author if sufficiently advanced algorithms that can discover complex nested causal structures exist. 


Addressing the challenges imposed by contextual causal learning would require a hybrid approach. For example, for finding causal relations in raw data, an adaptation of the SURD algorithm\cite{martinez2024decomposing} may result in a workable foundation. For the challenge of the entangled context and causal graph structure, algorithms from the field of search-based procedural content generation\cite{togelius2011search} may offer a path forward. Dealing with the expected computational cost resulting from the high-dimensional nature of the problem of finding two interdependent hypergraphs suggests the exploration of deep neuro\-evolution adapted to high dimensional data\cite{colas2020scaling}. Furthermore, the broader exploration of novelty search\cite{lehman2011abandoning} algorithms may yield deep insights about robust contextual causal discovery. 


\subsection{Teleological Inquiry}

An intelligent agent must be proactive, capable of navigating a complex future to achieve its goals. One critical prerequisite to autonomous decision making is teleological inquiry. This would require a  formalization of a calculus to establish the foundation for verifiable teleological reasoning that enables the design and validation of systems whose adherence to specified intent or regulations is mathematically provable. From there, a teleological inquiry becomes feasible. 

For a teleological inquiry, the outcome aims for alignment with the Effect Ethos, is held constant as the goal, whereas the decision space is a set of potential pathways through alternative contexts. The challenge is to decide which alternative course of action leads to an outcome that is most closely aligned with the encoded Effect Ethos. This requires a form of path-seeking and alignment inquiry. However, the classical formulation of inquiry as a search problem, as seen, for example in Zetetic Logic\cite{Millson2020Zetetic}, is fundamentally incompatible with the EPP's real-time operational capabilities. Instead, the EPP proposes a multi-step combination of alternative contexts with qualitative diversity as foundation for an efficient teleological inquiry.\newline 

\textbf{1) Candidate Action Generation:}

At a decision point, the system generates a finite set of immediate, mutually exclusive Causal Actions it could take. For a drone, this could be an alternative trajectory. The existing context informs a heuristic that limits the number of potential causal actions to allow only actions that are feasible i.e., within the remaining power budget.\newline

\textbf{2) The Deontic Filter:}

Before any computationally expensive simulation is performed, every candidate action is passed through a Deontic Filter. This filter applies only the subset of Teloids that represent absolute, non-negotiable rules encoded as impermissible norms. Any action that violates a hard deontic rule is immediately discarded and is not simulated. This "fail-fast" step prunes the action space, ensuring that computational resources are only spent on options that are, at a minimum, not forbidden.

\textbf{3) Parallel Counterfactual Simulation:}

The EPP's instantiates a temporary, hypothetical Context for each candidate action and runs a fast, forward simulation to predict the future state that would result if the proposed action were taken. This creates N predicted alternative futures.\newline

\textbf{4) Unified Scoring:}

For each of the simulated "alternative futures," the EPP now calculates a single, unified Teleological Alignment Score (TAS). This score is a holistic measure of "quality" that seamlessly blends two criteria:

\textit{Goal Achievement:} Positive scores for making progress towards the primary teleological objective i.e., reaching a destination coordinate. 

\textit{Ethic/Safety score:} Each outcome is also scored relative to adherence to ethics and safety rules encoded in the effect ethos. Positive scores represent relative adherence to the ethical boundaries. Negative scores representing the "cost" of violating Teloids (e.g., penalties for high fuel consumption, excessive engine strain, or entering a restricted airspace).

The TAS represents the quantitative expression of the Effect Ethos for each specific potential future.

\textbf{5) Update the Tactical Option Map:}

Instead of selecting any one particular set of proposed actions, the EPP maintains a persistent, low-dimensional grid called the Tactical Option Map. Each cell in this grid represents a region of the diverse behavioral space, for example, low energy options that sorts all actions by the remaining energy score, or a time to target options sort all actions by the lowest time to target regardless of the remaining energy. 


At this stage, the system possesses an entire map of the best-known way to achieve different types of behaviors that supports vastly more sophisticated decision making processes:

\begin{itemize} 
	\item \textbf{Default selection:} Assuming no anomalies have been detected, the EPP can select the action with the highest overall TAS from the entire on the Tactical Option Map.
	\item \textbf{Contingency Planning:} If the highest scoring TAS option somehow becomes untenable i.e. due to changed visibility i.e. a drone is flying through a tunnel, the EPP can instantly search the Tactical Option Map for viable alternatives without complex re-planning. 
	\item \textbf{Strategic Flexibility:} Because of the  Tactical Option Map, priorities can change at at time and be quickly implemented by switching to different alternative in the Tactical Option Map. 
 
\end{itemize} 
 
The proposed teleological inquiry builds entirely on techniques that are either rooted in the EPP i.e., alternative contexts and implemented in DeepCausality or in proven algorithms such as qualitative diversity that have been used in the industry for close to two decades. When implemented, the potential for (semi) autonomous agents, however, can be substantial. As an example, a search and rescue drone may operate under a default directive of low energy to thermal scan the largest possible area for survivors in a disaster area. It is possible that another drone has detected multiple potential survivors, but requires support to scan the surrounding area in parallel with multiple other drones. In this case, the command center can override the priority directive with "fastest time" to dispatch one or more drones to the new target area and provide rapid expansion of thermal scans in the designated area of interest. Here, the teleological inquiry re-plans ad-hoc for the new priority, adjust the trajectory, and dispatches to the new target coordinates using the new trajectory calculated based on the new priority. One step further, the same teleological inquiry can also be used to re-configure the core mission of the drone. For example, suppose a thermal signature indicates multiple survivors have been identified with a visual confirms multiple injuries. Then, the drone can determine to pause thermal scan, descend, drop off a pre-packed first aid kit with a positioning beacon that guides the rescue crew thus saving valuable time. Following the drop, the drone may proceed with the thermal scan.  
  
  
% \subsection{Quantum Probability}


% Another interesting conjecture is that the uncertain type enables a particularity Lucian Hardy pointed out in his work on probability theories with dynamic causal structure. Specifically, Hardy observed that, in principle, a causaloid can unify classical probability and quantum probability. The uncertain type enables the seamless integration of quantum probability in the EPP while preserving the uncertain distribution. While this has not been implemented yet, it is a promising area of future work.

\subsection{Indefinite Causal Structures}

The frontier of modern physics, particularly in the pursuit of a unified theory of quantum gravity, has introduced the profound notion of Indefinite Causal Structures (ICS), wherein the temporal ordering of events may exist in a quantum superposition. While Indefinite Causal Structures originate in fundamental physics, its implications for computation and the modeling of complex systems are significant.  However, a full formal treatment of quantum indefinite causal structures is beyond the scope of this foundational monograph, but the EPP provides a uniquely suitable foundation for extending the domain of computational causality to explore indefinite causal structures.

The EPP’s foundational compatibility with ICS results from its core axiomatic principles. The EPP's principled detachment from a presupposed spacetime removes the rigid constraint of temporal precedence that has historically served as the bedrock for causal reasoning. As ICS directly refutes this fixed "happen-before" relation, the EPP has already established the necessary foundation to consider a reality where A causes B and B causes A can coexist simultaneously within a single process. The EPP’s single axiom of causality as functional dependency, E2 = f(E1), is deliberately unconstrained, allowing the causal function f to be instantiated as a process matrix, which is the precise mathematical formalism required to describe applied indefinite causal processes such as the quantum switch.

A prospective implementation of indefinite causal structures in the EPP would involve a series of targeted modifications. First, the PropagatingEffect requires a new variant that represents a probability distribution over a set of potential outcomes. Consequently, the CausaloidGraph itself would require an extension to its formal definition, enabling it to represent a superposition of causal orderings. This means, the superposition of a causal graph is defined as a set of different graphs with different causal connectivity patterns to allow for the simultaneous co-existence of diametrically different causal orderings. Then, the EPP itself needs to be generalized to a process capable of evaluating all pathways within the superposition set of all causal graphs, aggregating their outcomes to compute a final, coherent probability distribution for the final state of indefinite causal structure.

When implemented, the potential payoff extends into critical domains of engineering and computational science. It would provide the first comprehensive tool for the unified, multi-paradigm simulation of quantum computers and communication networks, where classical control systems modeled with standard EPP modalities could interact with an indefinite causal structure and devise interactions via the causal state machine. In fields such as drug discovery and molecular dynamics, where the precise sequence of binding events is governed by quantum-level causality, an ICS-enabled EPP could offer a more faithful modeling of reality, potentially accelerating the design of novel molecules. Similarly, in quantum communication, the classical control system could be modeled using standard EPP, while the quantum core and its interconnections are modeled using the new ICS extensions, creating a unified, multi-paradigm simulation environment.

Extending the EPP to incorporate indefinite causal structures promises to yield a powerful new class of computational tools for modeling the complex, dynamic, and fundamentally quantum realities for a new category of dynamic quantum causal systems.  
  
  \subsection{Discussion}

The EPP offers a deep, wide, and rich selection of multiple areas of future work that ranges from the integration of advanced quantum concepts like indefinite causal structure, covers new methods for causal learning, and expands  computational ethics with the proposed teleological inquiry. The exploration of Indefinite Causal Structures seeks to formalize the generation of outcomes from a superposition of causal realities. The pursuit of Contextual Causal Discovery aims to generate the causal model itself from the intricate interplay of causal logic and its context. The teleological inquiry seeks to guide the generation of future actions toward desirable and verifiable ends. 

The presented areas of future work remain necessarily incomplete, as the full spectrum of possibilities opened by the EPP still has to be discovered. Each presented avenue of future work presents formidable technical and scientific challenges that imposes a demand for careful consideration of the epistemological and ethical consequences of engineering systems capable of dynamic self-modification. While the EPP provides the tools for this exploration, the responsibility, however, for its principled application remains an endeavor for the scientific and engineering communities.


\newpage
