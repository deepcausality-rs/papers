\section{The Metaphysics of the Effect Propagation Process}
\label{sec:metaphysics}

The metaphysics of the Effect Propagation Process establishes a set of principles underlying the ontology. We begin by defining the three core metaphysical concepts: Monoidic Primitives, Isomorphic Recursive Composition, and Contextual Relativity and then proceed to derive from these first principles the metaphysic of dynamics and becoming of the EPP. Conbined, these form the foundation of ontological design used throughout the EPP and its implementation DeepCausality. 

Ontological design is a constructive, engineering-oriented form of philosophy to specify the necessary and sufficient conditions of a new system to exist and operate coherently. The EPP's metaphysics is, therefore, the blueprint for a computable reality. It establishes the axiomatic foundation upon which the ontology is built, the epistemology is derived, and the implementation is realized. 

\subsection{Monoidic Primitives} 
\label{sec:metaphysics_monoidic_primitives}

A monoid is defined as an abstract algebraic structure that comprises of:

\begin{enumerate}
	\item A set of elements with a certain type.
	\item A binary operation that combines any two elements of the set results in a third element of the same set.
	\item An identity element, which, when combined with any other element, leaves it unchanged.
\end{enumerate}

  Monoidic elements that can be combined with each other is fundamental to the composability of the EPP.
  
\subsection{Isomorphic Recursive Composition} 
\label{sec:metaphysics_isomorphic_recursive_composition}


Isomorphism means "having the same shape” in the sense that isotropic elements all share the same form. Recursion refers to a structure containing itself. Isomorphism enables the combination of different types of monoidic primitives whereas recursion allows self-referential nesting. Combining these two concepts results in isomorphic recursive composition. This composition enables a monoidic primitive to derive its significance from its structural relation to other monoidic primitives. Critically, to ensure the  non-reducible  of form of monoidic primitives within the isomorphic recursive composition, it  must have a singleton representation of itself. 

\subsection{The Metaphysics of Being} 
\label{sec:metaphysics_being}

The metaphysics of being is structures in a classical Aristotelian notion (Book Zeta\cite{furth1985metaphysics}): 

\begin{itemize}
	\item Monoidic Primitives (Matter): The EPP is composed of fundamental, identifiable elements.
	\item Isomorphic Recursive Composition (Form): Gives the primitive elements its form. 
\end{itemize}

Combined, the monoidic primitives and the isomorphic recursive composition amount form an Aristotelian hylomorphic compound. The `Monoidic Primitives` constitute the "matter" (*hyle*) of and the `Isomorphic Recursive Composition` provides the "form" (*morphe*) that arranges its elements into a structured, meaningful whole.

This hylomorphic compound is the  \textit{archê kai aitia} of the system's existence. The specific form imposed upon the primitives is the foundational principle and explanation for its properties. Therefore, the hylomorphic compound is by definition static and describes what the system is at a snapshot in time, but it contains no inherent mechanism of change. 

\newpage

\subsection{The Metaphysics of Dynamics} 
\label{sec:metaphysics_dynamics}

The classical hylomorphism describes a static substance as the combination of its matter and form. Therefore, to account for the dynamism inherent in the EPP, the \textit{archê kai aitia} needs to capture the dynamics within an existing substance. However, because of the spacetime agnostic design of the EPP, dynamics can only be defines relative to its engulfing context. Therefore, the principle of contextual relativity operationalizes the expression of Substantial-Structural co-determination relative to its context. This is an inherent  principle of how a substance with a fixed identity can exhibit variable states. By its definition, dynamics is predictable and bound to an existing substance and context. 

\subsection{The Metaphysics of Becoming} 
\label{sec:metaphysics_becoming}


The principle of Higher-Order Emergence captures the profound process of creation of new substance from within an existing substance. Emergence brings into being a new mater, a new form, both of it, or new dynamics. The principle of Higher-Order Emergence operationalizes two different modalities.  

First-Order Emergence explains the creation of new substance. While contextual relativity applies to an existing substance, it cannot bring into being a new substance. The principle of `First-Order Emergence \` posits a generative capacity within the EPP, a capability of imposing a novel \textit{archê kai aitia} upon a set of primitives. This is the mechanism by which a new substance from within an existing context becomes into being.

In this modality, the distinction between the system and its context begins to dissolve. The result is a non-deterministic co-evolution where the spectrum of subsequent causal structures and contextual facts cannot be predicted any longer. The "reason for being" \textit{aitia} of any given state is no longer a fixed principle but is itself an emergent property of the ongoing, self-modifying process. However, the process of becoming is invariant because of the first order designation. 

Higher-Order Emergence refers to moves beyond the creation of a new substance from within an existing one. Instead, it describes a state where the generative capacity that enables `Emergence` acts upon itself recursively in what amounts to  dynamic co-mergence of form, matter, and dynamics through recursive higher order emergence. 

In this modality, the process of becoming itself becomes dynamic. This higher order emergence represents the EPP's most advanced state of becoming, one that necessitates a new epistemology of emergence to explore its inherent emergent properties.

\subsection{Discussion}
\label{sec:metaphysics_discussion}

The metaphysics of the Effect Propagation Process defines the substance of the EPP, its dynamics, 
and its emergence.  Metaphysically, the principal of Higher-Order Emergence demands further elaboration. 
Fundamentally, it implies that first-order emergence, the capacity to create new substance, is simply a less recursive specialization of the governing higher order principle.The principal of Higher-Order Emergence leads to outcomes that are not necessarily guaranteed to be decidable let alone deterministic any longer
and therefore it foreshadows three crisis:

\begin{enumerate}
	\item The Crisis of Justification
	\item The Crisis of Truth
	\item The Crisis of Explainability
\end{enumerate}


The crisis of justification immediately results from the fact that it must be decided how to chose one state of emergence over another one and how? The justification must be there otherwise the decision cannot be made, but this would render higher-order emergence fundamentally undecidable. 

The crisis of truth results from the fact that, if emergence generates a new context, then how do we know that
the facts in the newly generated context are true? If emergence generates new causal rules that uses new facts from a generated context, how do we know the outcome is true? If facts are fluid, verification is impossible. Therefore, truth must be re-established otherwise it undermines trust in operational safety of the EPP. 

The crisis of explainability means that in co-emergence, it might not be possible any longer to explain the outcome because of the previous crisis of truth and the crisis of justification. Furthermore, if the process of emergence itself cannot be explained, how could possible derived artifacts be explained? How can a system be held accountable when its not explainable. It is not possible, and therefore the crisis of explainability roots in the very core of higher-order emergence.

The introduction of higher order emergence also raises the question of the genesis process, 
the origin of the emergence itself. Fundamentally, the genesis process imposes a decision: Do we allow any kind of machine intelligence to modify if its genesis process or not? The author argues for a unequivocal no. 
Considering the alternative, when a system that can evolve its own genesis process, it fundamentally becomes uncontrollable and unexplainable. 

Therefore, the genesis process of emergence  has to remain at the sole discretion of a human designer to ensure its explainability and a fundamental alignment with human values. The metaphysics itself cannot establish the core ethos or telos, only a human designer can do that. Under no circumstances should the genesis process in parts or in its entirety ever be created or modified by a machine intelligence because it cannot possible have the innate ethos of a human being and thus cannot possible align itself with humanity. 

The existence of the genesis process also raises the thorny issue of whose ethics and values to codify and why? Which human designer? Who guards the guardians of the genesis? How to balance conflict demands from different stakeholders? These are  immense normative and political challenge and, a metaphysics alone, cannot possibly answer a fundamentally societal set of questions. 

As a consequence, the genesis process itself is a decision with far fetching higher order effects. The mitigation of unintended higher order consequences, will lead to the necessity of an immutable genesis telos, an underlying intent that serves as a criterion to discern whether emerging states are intended. 


\subsection{Summary} 
\label{sec:metaphysics_summary}

The complete metaphysics of the EPP is the synthesis of three core principles:

\begin{enumerate}
\item Substance (Being): The hylomorphic compound of Monoidic Primitives and Isomorphic Recursive Composition.
\item Dynamics (Changing): Governed by the principle of Contextual Relativity.
\item Emergence (Becoming): Governed by the principle of Higher-Order Emergence 
\end{enumerate}


The EPP's core metaphysical principles, Monoidic Primitives, Isomorphic Recursive Composition combined with the principles of and Contextual Relativity and  Higher-Order Emergence form the foundation of the EPP. The The EPP provides a framework that distinguishes between two tiers of emergence. First-Order Emergence describes the system's capacity to generate new  substance from a stable set of generative rules. Higher-Order Emergence, describes the system's ultimate capacity to evolve itself via a recursive and open-ended process of generative emergence. The introduction of higher-order emergence implies that the ultimate outcome of the emergent process is not guaranteed to be decidable let alone deterministic any longer. This is a deliberate design decision to provide multiple modalities for different requirements. For dynamic systems, contextual relativistic dynamics should suffice. For handling regime change where the new structure can be decided a-priori, first-order emergence should suffice. However, when handling dynamic relativistic regime change in response to an evolving context, higher-order emergence becomes necessary to capture the dynamic co-emergence. The EPP, through the introduction of Higher-Order Emergence, establishes a foundation for exploring the interrelation between emergent systems and human value. However, it also presents a new set of unique challenges that demand a new epistemology and ontology of the EPP. 

\newpage