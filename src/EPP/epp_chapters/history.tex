\section{Philosophy of Causality}
\label{sec:philosophy}

\subsection{The Classical Philosophy of Causality}
\label{sec:philosophy_foundation}


\subsubsection{Plato}
\label{sec:history_plato}

Plato is believed to be the first to have explored the cause in a systematic way, most notably in his dialogue Timaeus\cite{archer1888timaeus} (c. 360 BC). In this work, Plato explains the creation of the cosmos through the actions of a divine craftsman, the Demiurge. The Demiurge looks to the eternal Forms as a model to bring order to the pre-existing, chaotic matter. Plato’s concept of causality is thus tied to his broader metaphysical theory of Forms, where true causes are the eternal patterns or blueprints of which the physical world is merely a copy. He identifies several contributing factors necessary for the creation of the world, including the Demiurge (the efficient cause), the Forms (the formal cause), and the Receptacle (the material space)\cite{archer1888timaeus}. 

\subsubsection{Aristotle}
\label{sec:history_aristotle}

A student of Plato, Aristotle (c. 350 BC) formalized the notion of causality in his Metaphysics\cite{heidegger1995aristotleMetaphysics} with the ``Four Causes''\cite{falcon2006aristotlecausality}:

\begin{enumerate}
    \item The material cause or that which is given in reply to the question, ``What is it made out of?''
    \item The formal cause or that which is given in reply to the question, ``What is it?''. What is singled out in the answer is the essence of the what-it-is-to-be something.
    \item The efficient cause or that which is given in reply to the question, ``Where does change (or motion) come from?''. What is singled out in the answer is the whence of change (or motion).
    \item The final cause, the end purpose, is given in reply to the question, ``What is its good?''. What is singled out in the answer is that for the sake of which something is done or takes place.
\end{enumerate}

Aristotle’s framework provided a comprehensive vocabulary for analyzing causality that became foundational to Western scientific and philosophical thought for nearly two millennia.

\subsubsection{Seneca}
\label{sec:history_seneca}

Seneca (c. 56 AD) argues in Letter 65\cite{SenecaLetters} that cause and effect operate within a stage (space) and follow an order (time). Remove the stage or the order, and the conventional understanding of 'making something' or 'causing something' breaks down. His argument highlights time and space as indispensable prerequisites for classical causality. His focus on space and time as necessary conditions served as a precursor for physical concepts that treat spacetime as a background for causal processes.

\subsection{The Realist View: Discovering Causal Laws}
\label{ssec:philosophy_realist}

The realist tradition posits that causal relationships are objective, structural features of the world to be discovered.

\subsubsection{Gottfried Wilhelm Leibniz}
\label{sec:history_leibniz}

The idea of space and time as a background for causality, however, did not remain unchallenged. Gottfried Wilhelm Leibniz (1646--1716) rejected the concept of absolute space and absolute time as independent, fundamental constructs. Instead, Leibniz proposed\cite{LeibnizPhysicsSEP} a relational view in which space is the simultaneous relation of coexisting things and time is the relational order of successive things. Through rigorous first principles analysis, Leibniz argued that the concept of absolute space and time was logically untenable. His relational perspective offered a significant alternative to the preeminent Newtonian worldview of his time.

\subsubsection{John Stuart Mill}
\label{sec:history_stuart_mill}

John Stuart Mill (1806-1873) shifted the focus of causality from metaphysical inquiry to a more practical, methodological problem. In his influential work, \textit{A System of Logic, Ratiocinative and Inductive}\cite{mill2023system}, Mill developed a set of principles, now known as "Mill's Methods," to serve as a guide for discovering causal connections through empirical observation. These methods are designed to systematically eliminate non-causal factors and isolate true causes. Mill's work provided a formal basis for the inductive reasoning that underpins modern experimental science.


\subsubsection{Albert Einstein}
\label{sec:history_einstein}

Albert Einstein (1879--1955) departed from Newtonian physics with his theory of general relativity\cite{EinsteinPapers1915} (GR), in which he established that space and time are one manifold, spacetime, that is bent by the gravitational influence of large masses. General relativity preserves the prerequisite of a spatiotemporal context for causality, echoing Seneca's key insight. However, the notion of a dynamic spacetime requires a dynamic view of causality to fit into the dynamic spacetime manifold.

\subsection{The Empiricist View: Constructing Causal Models}
\label{ssec:philosophy_empiricist}

The empiricist tradition, born from a deep skepticism, argues that causality is not a directly observable force. Instead, it is a mental model that we, as observers, construct to make sense of the statistical regularities we perceive in the world.


\subsubsection{David Hume}
\label{sec:history_hume}

The Scottish philosopher David Hume (1711–1776) launched a more fundamental attack on the notion of causality itself. In his A Treatise of Human Nature\cite{hume2000treatise}, Hume argued that we can never have direct empirical evidence of a necessary connection between a cause and an effect. All we can observe is their constant conjunction in space and time. The idea of "causality" as an unseen force is, for Hume, a product of the human mind develop after observing repeated patterns. Hume's skepticism decouples the idea of causality from any metaphysical necessity and re-grounds it in observational patterns\cite{hume2000treatise}.

\subsubsection{Immanuel Kant}

The German philosopher Immanuel Kant (1724 - 1804) agreed with Hume's premise that causality is not something we can empirically observe in the world. However, he disagreed with Hume's conclusion that it is merely a habit of the mind. In his Critique of Pure Reason \cite{kant2024critique}, Kant argued that causality is an a priori category of understanding. It is a fundamental, built-in "rule" that the mind itself imposes on the raw data of sensory experience to make that experience coherent and understandable. We do not learn causality from the world; rather, we cannot experience the world in an intelligible way without already presupposing the principle of cause and effect. Kant reframes causality from an objective feature of the world (the Realist view) or a mere psychological habit (Hume's view) into a fundamental component of the cognitive architecture of any rational agent. 

\subsubsection{Hans Reichenbach}

Building on Hume's empiricism, Hans Reichenbach (1891 - 19543) was among the first to propose a formal, probabilistic solution to the problem of induction. He argued that causal relationships are not deterministic laws but statistical structures. His 'Principle of the Common Cause' provided the foundational logic for inferring causal structures from statistical correlations\cite{reichenbach1991direction} that later became known as the Reichenbach's Principle used by Pearl to define the concept of a confounder in modern graphical causal models\cite{pearl2000causality}.

\subsubsection{Karl Popper}

The challenge of induction, as framed by Hume, was famously addressed by the philosopher Karl Popper  (1902 – 1994). Popper argued that causal laws can never be verified by observation, only falsified. From this perspective, a causal theory is not a description of a hidden force, but a bold conjecture—a falsifiable model—that is held to be provisionally true only as long as it survives rigorous empirical testing\cite{popper2005logic}.


\subsection{The Practical View: Causality in Physics}
\label{sec:philosophy_physics}

\subsubsection{Bertrand Russell}
\label{sec:history_Russell}

Bertrand Russell (1872--1970) observed that successful physics has its roots in sophisticated, law-based descriptions of how a system evolves dynamically. In modern physics, the focus is on the state of a system (e.g., position, velocity, field strength) and how that entire state evolves continuously and dynamically. Therefore, for Russell, the idea of classical causality, a strict happen-before relation, no longer matches the reality of modern physics. Consequently, Russell wrote in his 1912 essay ``On the Notion of Cause''\cite{RussellOnCause}:  \textit{The law of causality, [...], is a relic of a bygone age.''} 
Many modern physics laws are time-symmetric, which means that if state S1 at time t1 is related to state S2 at time t2 by a law, it is equally true that state S2 at time t2 is related to state S1 at time t1. This relationship is not a simple, linear, one-way street from a necessary ``cause'' to a dependent ``effect.'' Knowing the state at any time allows to calculate the state at any other time, past or future. Therefore, which state is the ``cause'' and which one is the ``effect'' becomes arbitrary. Russell was not opposed to causality itself; instead, his primary argument was that the traditional philosophical interpretation of causality as a fundamental, temporally asymmetric, and directed link is not what Russell observed in physics. Russell saw physics as a discipline to find, formulate, and test time-symmetric laws of dynamic change. 

\subsubsection{Causality in Quantum Physics}

Russels view of physics describing time-symmetric laws of dynamic change was taken one step further in quantum physics because there the understanding of causality evolved from a structure that required the pre-existence of space toward a dynamic generative process from which causality emerges\cite{mrini2024indefinitecausalstructurecausal}. This emergent causality does not rely on a pre-existing spacetime but is grounded in a set of underlying rules (i.e., conceptualized as a 'generating function') that determines the fundamental materialization of spatiotemporal properties. The conceptualization of this fundamental level as a ``generating function'' captures the idea of a quantum process from which the necessary condition of classical causality's spatiotemporal structure arises. It is a shift from asking, ``What causes X, given spacetime?'' to ``What process generates spacetime (and thus enables X to be caused)?''.


\subsection{The Functional View of Causality}
\label{sec:philosophy_functional}

The preceding philosophical traditions, from the Realists to the Empiricists, share a common language rooted in the concepts of objects, causes, and effects. Their view differ in whether these concepts correspond to objective reality or are mental constructs, but the concepts of causality themselves remain the same. The practical view from physics, as articulated by Bertrand Russell, suggests that there might be a third view of causality. Russel correctly observed that the lack of time symmetry disqualifies causality from its application to modern physics that models dynamic change. And indeed, causality does struggle with complex dynamic because of the inherent assumption of linear and asymmetric time.  The author agrees with the premise and therefore generalizes causality as a spacetime agnostic functional dependency.   

This functional view of causality is formalized in the EPP's single axiom from which the entire framework is derived:
\begin{quotation}
\textbf{$E_{2} = f(E_{1})$}
\end{quotation}

The functional view redefines causality in the most general terms, the relationship between an input effect (E1) and an output effect (E2) via a transformation denoted as a function f(). The full reasoning behind the functional view and that, in fact, it subsumes the classical definition of causality is elaborates in section \ref{sec:epp}. From a philosophical perspective, the functional view of causality lead to a cascade of consequences:

\begin{enumerate}
	\item Inherently Agnostic: The nature of the function f and the effects E are unconstrained and thus agnostic. 
	\item Inherently Computable: Functions compute and thus can be programmed.
	\item Inherently Composable: Category theory is well established for functional caomposition. 
	\item Inherently Dynamic: Functions and higher order functions enable dynamics. 
	\item Inherently Formalized: Function theory has been studied for over a century and is well understood.
\end{enumerate}

The realistic view of causality as an objective reality remains valid and yet compatible with the functional view because, from
a realistic perspective, a function is nothing more than an objective, structural feature of the world. 
The empiricist view of causality as a model inferred from observed data also remains valid because, from the empirical perspective, the function is nothing more than the formal representation of the causal model derived from data.
The practical view, from physics, however, stands to gain because here the function can represent an arbitrarily complex and dynamic physics process without any assumption of time. Within the EPP, one causal function might be deterministic, another one probabilistic, and yet another one describing a complex physical model using differential equations and yet all can be expressed using the functional foundation of the EPP. 

The rest of the presented monograph explores the higher order effects that emerge from the functional view of causality. It begins with a review of the existing work in computational causality in section \ref{sec:related_work} and proceeds in section \ref{sec:motivation} with establishing a practical motivation for dynamic causality. The main contribution starts with rebuilding causality from first principles on a new functional foundation in section \ref{sec:epp}. Because of the introduced dynamics, a metaphysics discusses the necessary modalities to structure dynamic causality in section \ref{sec:metaphysics}. Dynamic causality, however introduces a set of profound challenges that are discussed in the subsequent epistemology (section \ref{sec:epp_epistemology}), Teleology (section \ref{sec:teleology}) and Ontology (section \ref{sec:epp_ontology}). Section \ref{sec:formalization} then formalizes the effect propagation process that is derived from its single axiomatic foundation. 

\newpage
