\section{History of Causality}
\label{sec:history}

\subsection{Plato}
\label{sec:history_plato}



Plato is believed to be the first to have explored the cause in a systematic way, most notably in his dialogue Timaeus\cite{archer1888timaeus} (c. 360 BC). In this work, Plato explains the creation of the cosmos through the actions of a divine craftsman, the Demiurge. The Demiurge looks to the eternal Forms as a model to bring order to the pre-existing, chaotic matter. Plato’s concept of causality is thus tied to his broader metaphysical theory of Forms, where true causes are the eternal patterns or blueprints of which the physical world is merely a copy. He identifies several contributing factors necessary for the creation of the world, including the Demiurge (the efficient cause), the Forms (the formal cause), and the Receptacle (the material space)\cite{archer1888timaeus}. 

\subsection{Aristotle}
\label{sec:history_aristotle}

A student of Plato, Aristotle (c. 350 BC) formalized the notion of causality in his Metaphysics\cite{heidegger1995aristotleMetaphysics} with the ``Four Causes''\cite{falcon2006aristotlecausality}:

\begin{enumerate}
    \item The material cause or that which is given in reply to the question, ``What is it made out of?''
    \item The formal cause or that which is given in reply to the question, ``What is it?''. What is singled out in the answer is the essence of the what-it-is-to-be something.
    \item The efficient cause or that which is given in reply to the question, ``Where does change (or motion) come from?''. What is singled out in the answer is the whence of change (or motion).
    \item The final cause, the end purpose, is given in reply to the question, ``What is its good?''. What is singled out in the answer is that for the sake of which something is done or takes place.
\end{enumerate}

Aristotle’s framework provided a comprehensive vocabulary for analyzing causality that became foundational to Western scientific and philosophical thought for nearly two millennia.

\subsection{Seneca}
\label{sec:history_seneca}

Seneca (c. 56 AD) argues in Letter 65\cite{SenecaLetters} that cause and effect operate within a stage (space) and follow an order (time). Remove the stage or the order, and the conventional understanding of 'making something' or 'causing something' breaks down. His argument highlights time and space as indispensable prerequisites for classical causality. His focus on space and time as necessary conditions served as a precursor for physical concepts that treat spacetime as a background for causal processes.

\subsection{Gottfried Wilhelm Leibniz}
\label{sec:history_leibniz}

The idea of space and time as a background for causality, however, did not remain unchallenged. Gottfried Wilhelm Leibniz (1646--1716) rejected the concept of absolute space and absolute time as independent, fundamental constructs. Instead, Leibniz proposed\cite{LeibnizPhysicsSEP} a relational view in which space is the simultaneous relation of coexisting things and time is the relational order of successive things.
Through rigorous first principles analysis, Leibniz argued that the concept of absolute space and time was logically untenable.
His relational perspective offered a significant alternative to the preeminent Newtonian worldview of his time.

\subsection{John Stuart Mill}
\label{sec:history_stuart_mill}

John Stuart Mill (1806-1873) shifted the focus of causality from metaphysical inquiry to a more practical, methodological problem. In his influential work, \textit{A System of Logic, Ratiocinative and Inductive}\cite{mill2023system}, Mill developed a set of principles, now known as "Mill's Methods," to serve as a guide for discovering causal connections through empirical observation. These methods are designed to systematically eliminate non-causal factors and isolate true causes. Mill's work provided a formal basis for the inductive reasoning that underpins modern experimental science.

\newpage


\subsection{Albert Einstein}
\label{sec:history_einstein}

Albert Einstein (1879--1955) departed from Newtonian physics with his theory of general relativity\cite{EinsteinPapers1915} (GR), in which he established that space and time are one manifold, spacetime, that is bent by the gravitational influence of large masses. General relativity preserves the prerequisite of a spatiotemporal context for causality, echoing Seneca's key insight. However, the notion of a dynamic spacetime requires a dynamic view of causality to fit into the dynamic spacetime manifold.

\subsection{Bertrand Russell}
\label{sec:history_Russell}

Bertrand Russell (1872--1970) observed that successful physics has its roots in sophisticated, law-based descriptions of how a system evolves dynamically. In modern physics, the focus is on the state of a system (e.g., position, velocity, field strength across space) and how that entire state evolves continuously and dynamically. Therefore, for Russell, the idea of classical causality, a strict happen-before relation, no longer matches the reality of modern physics. Consequently, Russell wrote in his 1912 essay ``On the Notion of Cause''\cite{RussellOnCause}:


\begin{quote}
    \begin{center}
            \textit{The law of causality, [...], is a relic of a bygone age.''} - Bertrand Russell
    \end{center}
\end{quote}


Many modern physics laws are time-symmetric, which means that if state S1 at time t1 is related to state S2 at time t2 by a law, it is equally true that state S2 at time t2 is related to state S1 at time t1. This relationship is not a simple, linear, one-way street from a necessary ``cause'' to a dependent ``effect.'' Knowing the state at any time allows you to calculate the state at any other time, past or future. Therefore, which state is the ``cause'' and which one is the ``effect'' becomes arbitrary. Russell was not opposed to causality itself; instead, his primary argument was that the traditional philosophical interpretation of causality as a fundamental, temporally asymmetric, and directed link is not what he observed in physics. His critique resonates with challenges encountered in contemporary quantum research.

\subsection{Quantum Physics}
\label{sec:history_quantum}

Quantum field theory\cite{peskin2018introduction} (QFT) stands out as one of the most rigorously tested theories, with unparalleled predictive accuracy in the history of science. QFT predictions in quantum electrodynamics have been experimentally verified up to an astonishing accuracy of one part in a billion or better. The standard model of physics, built on top of QFT, despite being one of the most successful theories of all time, accurately describes three of the four known fundamental forces: electromagnetism, the strong nuclear force, and the weak nuclear force. Notably, gravity remains absent due to complex discrepancies between Einstein’s (pre-quantum) theory of relativity and quantum field theory.

The unification of general relativity with quantum field theory would complete the standard model of physics, but doing so faces a non-trivial impasse, as Lucian Hardy formulated: ``Quantum theory is a probabilistic theory with a fixed causal structure. General relativity is a deterministic theory, but where the causal structure is dynamic''\cite{HardyDynamicCausalStructure}.. The fundamental dynamical laws of quantum physics (excluding the weak nuclear force) are accepted as T-symmetric. However, the transition of a quantum state from a probabilistic, T-symmetric state into a definite and time-asymmetric state, while observed, is not well understood and is the subject of ongoing research in the field of quantum measurement.

The quantum superposition of states, on the other hand, is well-observed, well-documented, and, as a result, is accepted as a fundamental building block of quantum physics. Quantum superposition inspires the exploration of concepts like indefinite causal structures in theories aiming to unify quantum mechanics and gravity. Therefore, the classical conceptualization of cause and effect embedded into a background spacetime might not exist at the fundamental level of quantum gravity. In quantum gravity, space and time are not external conditions but potential emergent properties of the internal quantum structure itself. The problem is no longer whether spacetime is static or dynamic, but that spacetime itself emerges from the quantum level and thus positions itself as a higher-order effect of a generative process.

Russell saw physics moving towards laws governing states, a view echoed in quantum gravity's search for fundamental rules governing the structure from which spacetime and causal order emerge. At this stage, the understanding of causality evolved from a structure that required the existence of space toward a dynamic generative process from which causality emerges. This emergent causality does not rely on a pre-existing spacetime but is grounded in a more fundamental level of reality—a set of underlying rules (i.e., conceptualized as a 'generating function') that determines the fundamental manifestation of spatiotemporal properties. The conceptualization of this fundamental level as a ``generating function'' captures the idea of a quantum process from which the necessary condition of classical causality's spatiotemporal structure arises. It is a shift from asking, ``What causes X, given spacetime?'' to ``What process generates spacetime (and thus enables X to be caused)?''.

\newpage
