\section{Epilogue}
\label{sec:epilogue}

 The Effect Propagation Process, as detailed in the preceding sections, provides a consistent, first-principles foundation for dynamic causality and a new philosophy-grounded language to study dynamic, emergent, and contextual causality. It is important to acknowledge that the higher-order effects resulting from the single axiomatic foundation most likely will remain incomplete for the foreseeable future despite the EPP\'s attempt to describe and formalize an initial set. The vast space of new possibilities opened up by  an irreducible single axiomatic definition will require some meaningful time to explore and understand. The presented three modalities of dynamic causality already demonstrated very clearly that the higher-order consequences need to be carefully considered to ensure safe and reliable dynamic systems built upon the EPP. 

The implementation of the EPP in the DeepCausality\footnote{\url{https://deepcausality.com}} demonstrates the practicality of hypergeometric computational causality for fast context-aware causal reasoning across Euclidean and non-Euclidean spaces. Furthermore, DeepCausality, as a reference implementation, provides an excellent foundation to study dynamic causality further. And indeed, the new foundation of dynamic causality already suggests  several domains of inquiry: 

\begin{itemize}
	\item \textbf{The Theory of Hybrid Causal Models:} The EPP already enables uniform causal reasoning over deterministic and probabilistic modes and the Causaloid is a polymorphic container that can host SCMs, Bayesian networks, deterministic rules, and neural networks. One area the EPP opens up is the exploration of a theory of hybrid causal models to explore hybrid causal composition and discovery.    
	\item \textbf{Formal verification of the CSM:} The Causal State Machine (CSM) bridges insights from causal reasoning to actions and that opens up the area of formal verification of the CSM especially for dynamic causal models with a mixture of static and dynamic contexts.  	
	\item \textbf{The Formalization of Causal Emergence:} While the EPP introduced causal emergence, its foundation is far from settled; therefore, the study of causal emergence promises further discoveries.
	\item \textbf{The Dynamics of Context:} The EPP established external context and its modalities, but especially the intersection between a dynamically changing environment and a dynamic context offers a promising area for study. This is particularly valuable for control systems in autonomous unmanned vehicles that have to adapt to new terrain. 
	\item \textbf{The Teleology of Emergence:} Directly related to the calculus of purpose follows the study of teleological boundaries of causal emergence which is particularly relevant to the development of safe emergent algorithms. 
\end{itemize}


The presented EPP serves as an open invitation. It is an invitation to the architects and engineers of complex systems who require formal proof of safety and explainability. It is an invitation to the underwriters of risk, who must quantify, model, and price complex risks that are contextual and dynamic. And it is an invitation to the regulators and policymakers tasked with ensuring the safe and compliant integration of complex systems into the fabric of society. Trust is the single most valuable asset any society has and the EPP exists to facilitate trust building and certification of complex dynamic causal systems. 

\newpage
