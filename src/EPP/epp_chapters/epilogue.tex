\section{Epilogue}
\label{sec:epilogue}

 The Effect Propagation Process, as detailed in the preceding sections, provides a consistent, first-principles foundation for dynamic causality and a new philosophy grounded language to study dynamic, emergent, and contextual causality.
 
The metaphysics of Effect Propagation Process establishes the essence of the EPP and establishes its core dynamics that lead to an orthogonal design that consistently applies to all levels of the EPP. 

The ontology of the Effect Propagation Process foreshadows the complex structures the EPP is designed to model by scaling the modality of the EPP. For a static EPP, a positivist epistemology remains sufficient. For a dynamic EPP, the epistemology evolves towards an interpretivism perspective, and for an emergent EPP, a pragmatism perspective on the epistemology becomes necessary.

The epistemology of the EPP explores the meaning truth and how it scales with the modality of the EPP. For a static EPP, the meaning of truth aligns with the classical correspondence theory. However, in a dynamic EPP, the meaning of truth shifts towards a coherent adaptability approach. In an emergent EPP, the meaning of truth evolves towards pragmatic efficacy where the validity of relativistic, emergent causal relationships is established by their functional utility.


The implementation of the EPP in the DeepCausality\footnote{\url{https://deepcausality.com}} demonstrates the practicality of hypergeometric computational causality for fast context-aware causal reasoning across Euclidean and non-Euclidean spaces. Furthermore, DeepCausality, as a reference implementation, provides an excellent foundation to study dynamic causality further. And indeed, the new foundation of dynamic causality already suggests  several domains of inquiry: 

\begin{itemize}
	\item \textbf{The Formalization of Causal Emergence:} While the EPP introduced causal emergence, its foundation is far from settled therefore the study of causal emergence promises further discoveries.
	\item \textbf{The Dynamics of Context:} The EPP established external context and its modalities, but especially the intersection between a dynamically changing environment and a dynamic context offers promising area for study. This is particular valuable for control systems in autonomous unmanned vehicles that have to adapt to new terrain. 
	\item \textbf{A Calculus of Purpose:} The Teloid and Effect Ethos are introduced as ontological primitives. The next logical step is the formalization of a calculus to establish the foundation for verifiable teleological reasoning that enables the design and validation of systems whose adherence to specified intent or regulations is mathematically provable. This line of study might be particular valuable for regulated industries such as robotics, avionics, and defense. 
	\item \textbf{The Teleology of Emergence:} Directly related to the calculus of purpose follows the study to teleological boundaries of causal emergence which is particular relevant to the development of safe emergent algorithms. 
	\item \textbf{Certification of Regulatory Requirements:} The Teloid provides a direct, transparent, traceable, and auditable link between a system's behavior and a codified safety-critical rule. Therefore, further study is warranted to explore the applicability of the EPP and Teloid foundation to the world of rigorous certification (e.g., DO-178C) to solidify operational trust.  
\end{itemize}


The presented EPP serves as an open invitation. It is an invitation to the architects and engineers of complex systems who require formal proof of safety and explainability. It is an invitation to the underwriters of risk, who must quantify, model, and price complex risks that is contextual and dynamic. And it is an invitation to the regulators and policymakers tasked with ensuring the safe and compliant integration of complex systems into the fabric of society. Trust is the single most valuable asset any society has and the EPP exists to facilitate trust building and certification of complex dynamic causal systems. 

\newpage