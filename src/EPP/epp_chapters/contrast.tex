\section{Contrast}
\label{sec:contrast}

The classical concept of causality existed for millennia and served mankind well until the dawn of the quantum era. The Effect Propagation Process does not seek to replace the classical notion of causality, but instead seeks to advance and generalize the concept of causality to match the now more advanced understanding of fundamental science. To this end, this section contrasts the Effect Propagation Process (EPP) with the classical definition of causality to pinpoint the exact differences:

\subsection{Detachment from Fixed Spacetime}

\textbf{Classical causality:} 
Classical causality assumes an implicit Euclidean context with a fixed spacetime. This is a direct consequence of Seneca’s definition. The advent of general relativity changed the notion of a fixed spacetime background towards a dynamic spacetime background with the sole implication that causality became dynamic relative to its engulfing spacetime.

\textbf{EPP:} 
The EPP removes the constraint of spacetime altogether and does not assume any particular fabric in which effects propagate. While classic causality cannot operate in arbitrary, non-spatiotemporal, non-Euclidean structures, EPP  can.


\subsection{Agnostic to Temporal Order}

\textbf{Classical causality:} 
Classical causality requires a strict linear temporal order, i.e., a cause must precede its effect.

\textbf{EPP:} 
 As EPP removes spacetime altogether, it does not impose any particular temporal order. By extension, there is also no imposed spatial order with the understanding that the EPP would require the underlying fabric to define temporal or spatial order if it were to use these orders during effect propagation.
Instead of seeing temporal order as a fixed background, temporal order is seen as an emergent property. This removes the constraint of fitting all causal interactions into a rigid before-after sequence, opening the door to modeling complex systems with feedback loops.

\subsection{Causaloid as Uniform Building Block}

\textbf{Classical causality:} 
Classical causality discern a cause from an effect merely by the imbued temporal order which stipulates that a cause must happen before its effect.

\textbf{EPP:} 
In the absence of a temporal order in EPP, there is no meaningful way to discern a cause from an effect, and consequently, the EPP does not make this distinction any longer. Instead, the EPP relies on Hardy’s causaloid which folds cause and effect into a single entity of testable effect transfer.
The absence of a temporal order does not imply the absence of time; it means the absence of a strict order. That means, EPP allows the modeling of complex systems with bi-directional causal influence, a feat that is hard to accomplish with the classical definition of causality.

\newpage

\subsection{Focus on the Generative Function}

\textbf{Classical causality:} 
Classical causality tries to simplify reality into tractable causal chains operating on observable states.


\textbf{EPP:} 
EPP does not presume the existence of a linear causal chain. Instead, it assumes the existence of a generating function from which the observable fabric (i.e., spacetime) and causal relationships emerge. The effects then propagate through the observable fabric.

\subsection{Embracing Indefinite Causal Order}

\textbf{Classical causality:} 
Classical causality assumes the existing of a definitive causal structure that, once discovered, can be modeled. This leads directly to the notoriously hard causal discovery problem as it is not yet clear how to find a definitive causal structure assuming it exists.

\textbf{EPP:} 
EPP does not assume the pre-existence of a definitive causal structure. Instead, it embraces indefinite causal order in which it is not yet clear if A happened before B or B happened before A. Conceptually, this mechanism not only allows for the existence of superposed causal pathways, but enables the handling of emergent causal structures. While the practicality of superposed causal pathways remains an open topic, there is a clear use case for emergent causal structures in dynamic systems. Moving away from a presumed definitive causal structure towards emergent causal structures in dynamic systems elevates causal modeling to deal with the intricate uncertainty of dynamic systems.
