\section{Validity}
\label{sec:validity}

The validity of the Effect Propagation Process must be assessed according to the nature of its contribution. The EPP is
a philosophy-informed, formal computational framework designed to provide a new, more expressive language for modeling
dynamic, contextual causal systems. Therefore, its validity rests on the criteria appropriate for such a framework,
analogous to how one would assess a mathematical logic or a new programming language:

\begin{itemize}
    \item \textbf{Internal Validity} (Soundness \& Consistency): The framework's internal validity is determined by its logical and
  mathematical soundness and coherence. As detailed in the accompanying specification, the EPP is built on a foundation
  of first-principles reasoning to ensure its components are consistent and its operations are sound.
  \item \textbf{External Validity} (Expressive Power \& Scoped Generalization): The framework's external validity is demonstrated by its
  robustness against alternative interpretations of its premises and how well it defines the boundaries of its
  generalization.
\end{itemize}



\subsection{Internal validity}
\label{sec:validity_internal}

The \textbf{soundness} of EPP derives from the first principled logical progression of the presented argument. The stated problem of inadequacy of classical causality in light of the challenges imposed by Quantum Gravity (emergent spacetime, indefinite causal order) is well-recognized and documented\cite{MriniHardyIndefinite}.. From this recognized problem, the argument for EPP progresses as stated below:

\begin{enumerate}
    \item Establishes classical causality and its historical critiques.
    \item Introduces the challenges from modern physics (GR, QG).
    \item Shows why these challenges render classical causality insufficient.
    \item Proposes EPP as a coherent response to these challenges.
    \item Contrasts EPP with classical causality.
    \item Discusses its ontological, epistemological, and teleological implications.
    \item Acknowledges and addresses threats to its validity.
\end{enumerate}

The soundness is further strengthened by its inspiration from established scientific theories (QFT, GR) and foundational work in quantum gravity. While quantum gravity as a scientific theory remains a work in progress, the conceptual challenges that arise from it are valid regardless of how any particular theory may explain the underlying quantum mechanisms.

\newpage

The \textbf{consistency} of the EPP framework arises from a handful of carefully stated conclusions.

\subsubsection{Spacetime Agnosticism \& Causaloids}

\textbf{Premise:} On a quantum level, spacetime may not exist.

\textbf{Conclusion:} Therefore, remove spacetime from EPP.

\textbf{Premise:} EPP does not have a defined spacetime.

\textbf{Conclusion:} Cause and effect cannot be separated anymore because there is no a priori temporal order.

\textbf{Premise:} Cause and effect cannot be discerned because of missing temporal order.

\textbf{Conclusion:} Fold cause and effect into one entity, the causaloid, that is independent of temporal (and spatial) order.

Note, the last conclusion holds because of the temporal order required for classical causality. The only logical alternative conclusion from the premise of missing temporal order would be to abandon causality altogether. However, this conflicts with the reality in which causal relationships indeed exist; therefore, the alternative has been deemed unsound.

\subsubsection{Effect Propagation as the Essence of Causality}

\textbf{Premise:} The causaloid, as a building block, is independent of temporal (and spatial) order.

\textbf{Conclusion:} Define causality by what it does (propagate effects) instead of what it was thought to be (a temporal order dependent atomic relationship).

The careful reader may raise a concern over the choice of words (i.e., propagate effects), since “transferring information” or “transmitting influence” might be equally valid choices. True, that is indeed a fair point. However, the term information has specific meaning in information theory and computer science, and likewise, the term influence has specific meaning in social science; therefore the author settled cautiously on “effect” mainly to prevent conflating different meanings.

\subsubsection{Compatibility with Classical Causality}

The full logical argument of how classical causality is derived from EPP is in the section “Causality as Effect Propagation Process”.
While classical causality can be derived from EPP, the reverse is not true because EPP cannot be derived from classical causality because classical causality requires a background spacetime. That deduction proves that EPP is more abstract in the sense of more general than classical causality. Therefore, it follows that EPP naturally applies to areas where classical causality cannot be used any longer. The internal validity of EPP roots in its internal soundness and consistency that stems from its first-principles reasoning. Therefore, ambiguity, contradictory claims, and unjustified leaps in logic are avoided to the extent it is possible. Minor mistakes might be possible and the author is open to suggestions  to improve EPP further.

\newpage

\subsection{External validity}
\label{sec:validity_external}

Establishing the boundaries of generalization as a proxy for external validity requires a delicate balance of realistically acknowledging what EPP can address versus avoiding overstating  any particular capability. Related to external validity is always the possibility of an alternative interpretation of the underlying premises.

\subsubsection{Alternative interpretations}

There are several potential alternative interpretations of the premises underlying the Effect Propagation Process framework.

\textbf{Russell was more right than acknowledged}

One can take the position that, if Russell deemed causality as a relict of a bygone era, then why not openly ask to abandon causality altogether and focus solely on descriptive and correlation-based data science?

DARPA disagrees\cite{DARPA_ANSR}:

\begin{quote}
    “In the real world, observations are often correlated and a product of an underlying causal mechanism, which can be modeled and understood.”
\end{quote}

The problem is not a simple choice between correlation and causation. The deeper issue, as contemporary philosophers
like Luciano Floridi have argued, is that the predominant mental model underlying complex system design remains rooted
in an outdated Aristotelian and Newtonian "Ur-philosophy" of fixed space and time. This leads to tools that are
inadequate for the intricate complexity of reality. To address this core critique, the EPP lifts causality into a
contextual generalization required to model dynamic systems that no longer adhere to a fixed background spacetime. This
includes systems with non-Euclidean geometries, multiple interacting contexts, and emergent causal structures, as found
in domains like avionics. The question of falsifiability, therefore, applies not to the EPP framework itself, but to the
specific, testable models that are constructed within it. A model of an avionics system built using the EPP is indeed
falsifiable against simulation data; the framework itself simply provides the language for expressing that model.

The author argues, a similar shift towards a richer contextualization of advanced dynamic models needs to happen in the
correlation-based methodologies of deep learning as well to build tools more suitable for today's reality. A
transfer of core EPP concepts, i.e., the contextual hypergraph, uniform Euclidean and non-Euclidean geometries, into the
foundations of deep learning is welcomed by the author.

\textbf{Pluralism of causal concepts}

Instead of a unified framework like EPP, one can argue in favor of the existing pluralistic reality where multiple disjoint concepts of (computational) causality exist for different levels of causal analysis.

As stated before, EPP does not seek to replace classical causality and all tools that are built atop the classical
definition of causality. Instead, EPP seeks to advance the core concept of causality to meet increasingly challenging
demands. Due to the novelty of this foundational work, a single coherent framework is preferred until it becomes clear
which parts may branch out and become more specialized domains over time.

\newpage

\subsubsection{Boundaries of Generalization}

The main boundary to EPP comes down to preventing the application of any quantum principle to macroscopic systems.

EPP does not endorse nor attempt to apply any quantum principle to non-quantum systems. Instead, it takes inspiration from quantum gravity to advance causality. Furthermore, it is crucial to understand that EPP does not propose that macroscopic complex systems are quantum mechanical in the way they physically operate. Instead, EPP attempts to transfer concepts from quantum gravity to advance causality further to model advanced dynamic systems with complex feedback loops that are notoriously hard to model with conventional causality tools.

Another boundary relates to the lower limit of complexity that can be solved with EPP. For example, abandoning the cause-effect distinction via the causaloid is too radical for many applications. EPP remains compatible with classical causality and therefore can potentially be used where classical causality would apply. However, when a simpler classical causality methodology can solve the problem at hand, it should be preferred over any more complex methodology.

To prevent conceptual overreach, the author does not claim to have invented post-quantum causality even though its inherent independence of spacetime would suggest this view. Furthermore, the author does not claim that the Effect Propagation Process framework would solve open questions in quantum physics or quantum gravity. However, because of the early stage of quantum gravity research, it is far from settled how causality fits into the suggested conceptualization of emergent spacetime. Instead, the Effect Propagation Process, as a philosophical framework inspired by concepts of quantum gravity, is defined as a spacetime-agnostic continuous effect propagation process.

\subsubsection{Method Selection Criteria: Classical Causality vs. EPP}

In cases where methods of classical causality and conventional machine learning do not solve the problem at hand, methodologies rooted in EPP might be preferred. The following decision matrix supports the assessment of when to use which methods. This matrix provides guidance on selecting an appropriate causal modeling approach based on the temporal and spatial complexity of the system under investigation.

% Please add the following required packages to your document preamble:
\begin{table}[hb]
\begin{tabular}{llll}
Feature Assessed &
  System Characteristic &
  Classical Methods &
  EPP Methods \\ \hline
\multicolumn{1}{|l|}{Temporal Complexity} &
  \multicolumn{1}{l|}{} &
  \multicolumn{1}{l|}{} &
  \multicolumn{1}{l|}{} \\ \hline
\multicolumn{1}{|l|}{} &
  \multicolumn{1}{l|}{Single time scale + linear progression} &
  \multicolumn{1}{l|}{Sufficient} &
  \multicolumn{1}{l|}{} \\ \hline
\multicolumn{1}{|l|}{} &
  \multicolumn{1}{l|}{Multiple time scales OR non-linear temporal relationships} &
  \multicolumn{1}{l|}{May struggle} &
  \multicolumn{1}{l|}{Consider EPP} \\ \hline
\multicolumn{1}{|l|}{} &
  \multicolumn{1}{l|}{Multiple time scales AND non-linear temporal relationships} &
  \multicolumn{1}{l|}{Insufficient} &
  \multicolumn{1}{l|}{EPP Required} \\ \hline
\multicolumn{1}{|l|}{Spatial Structure} &
  \multicolumn{1}{l|}{} &
  \multicolumn{1}{l|}{} &
  \multicolumn{1}{l|}{} \\ \hline
\multicolumn{1}{|l|}{} &
  \multicolumn{1}{l|}{Euclidean space with fixed coordinates} &
  \multicolumn{1}{l|}{Sufficient} &
  \multicolumn{1}{l|}{} \\ \hline
\multicolumn{1}{|l|}{} &
  \multicolumn{1}{l|}{Non-Euclidean OR dynamic spatial relationships} &
  \multicolumn{1}{l|}{Limited/Difficult} &
  \multicolumn{1}{l|}{EPP Advantageous} \\ \hline
\multicolumn{1}{|l|}{} &
  \multicolumn{1}{l|}{Non-Euclidean AND dynamic spatial relationships} &
  \multicolumn{1}{l|}{Very Limited} &
  \multicolumn{1}{l|}{EPP Required} \\ \hline
\multicolumn{1}{|l|}{Combined Complexity} &
  \multicolumn{1}{l|}{} &
  \multicolumn{1}{l|}{} &
  \multicolumn{1}{l|}{} \\ \hline
\multicolumn{1}{|l|}{Scenario 1} &
  \multicolumn{1}{l|}{Linear Time \& Euclidean Space} &
  \multicolumn{1}{l|}{Preferred} &
  \multicolumn{1}{l|}{} \\ \hline
\multicolumn{1}{|l|}{Scenario 2} &
  \multicolumn{1}{l|}{Non-Linear Time OR Complex Spatial} &
  \multicolumn{1}{l|}{Limited/Difficult} &
  \multicolumn{1}{l|}{Consider EPP} \\ \hline
\multicolumn{1}{|l|}{Scenario 3} &
  \multicolumn{1}{l|}{Non-Linear Time AND Complex Spatial} &
  \multicolumn{1}{l|}{Insufficient} &
  \multicolumn{1}{l|}{EPP Required} \\ \hline
\end{tabular}
\caption{Causal Method Selection Matrix}
\label{tab:method_matrix}
\end{table}

\textbf{Emergent Causality:} If the problem involves emergent causal structures (where the causal graph itself is not fixed and changes dynamically based on context, i.e., dynamic regime shifts), EPP-based methodologies become the only viable option.


\newpage