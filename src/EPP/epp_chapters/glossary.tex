\section*{Glossary}
\addcontentsline{toc}{section}{Glossary} % Add it to the Table of Contents

This glossary provides definitions for the key formal terms and symbols used throughout this paper to describe the Effect Propagation Process (EPP) and its components.

\begin{description}[style=nextline] % Using [style=nextline] if you have enumitem package and want this style

    \item[Effect Propagation Process (EPP)] 
    The overarching philosophical framework and the formalized dynamic process (\(\Pi_{EPP}\)) of effect transfer within a CausaloidGraph, conditioned by the Contextual Fabric.

    \item[Contextual Fabric (\(\mathcal{C}\))] 
    The structured environment, composed of Context Hypergraphs, within which effects propagate and causal relationships are conditioned. It is formalized as a Context Collection \(\mathcal{C}_{sys}\).

    \item[Contextoid (\(v\))] 
    The atomic unit of contextual information, defined as a tuple \(v = (id_v, \text{payload}_v, \text{adj}_v)\). It encapsulates data, time, space, or spacetime values. (See Definition 3.1)

    \item[Contextoid Payload (\(\text{payload}_v\))] 
    The tagged union representing the actual data, temporal, spatial, or spatiotemporal value held by a Contextoid.

    \item[Context Hypergraph (\(C\))] 
    A structured collection of Contextoids (\(V_C\)) and the N-ary relationships (Hyperedges \(E_C\)) between them, defined as \(C = (V_C, E_C, ID_C, \text{Name}_C)\). (See Definition 3.2)

    \item[Context Collection (\(\mathcal{C}_{sys}\))] 
    A finite set of distinct Context Hypergraphs, \(\mathcal{C}_{sys} = \{C_1, C_2, \dots, C_k\}\), representing the total available contextual information. (See Definition 3.4)

    \item[Context Accessor (\(\text{ContextAccessor}(\mathcal{C}_{refs})\))] 
    A functional interface providing read-access for Causaloids to a specified subset of Context Hypergraphs (\(\mathcal{C}_{refs} \subseteq \mathcal{C}_{sys}\)). (See Definition 3.5)

    \item[Causaloid (\(\chi\))] 
    The fundamental, operational unit of causal interaction, defined as \( \chi = (id_\chi, \text{type}_\chi, f_\chi, \mathcal{C}_{refs}, \text{desc}_\chi, \mathcal{A}_{linked}, I_{linked}) \). It encapsulates a testable mechanism for effect transfer. (See Definition 4.1)

    \item[Causaloid Type (\(\text{type}_\chi\))] 
    Specifies the structural nature of a Causaloid (Singleton, Collection, or Graph), determining how its causal function \(f_\chi\) is realized.

    \item[Causal Function (\(f_\chi\))] 
    The operational logic associated with a Causaloid \(\chi\), which maps input observations (\(\mathcal{O}_{\text{type}}\)) and accessed context to an activation status (\(\{\text{true, false}\}\)). (See Section 4.2.2)

    \item[CausaloidGraph (\(G\))] 
    A hypergraph, defined as \( G = (V_G, E_G, ID_G, \text{Name}_G) \), whose nodes (\(V_G\)) typically encapsulate Causaloids and whose hyperedges (\(E_G\)) define pathways and logic for effect propagation. Supports recursive isomorphism. (See Definition 4.2)

    \item[Causal Node (\(v_g\))] 
    A node within a CausaloidGraph \(G\), defined as \(v_g = (id_g, \text{payload}_g)\), where \(\text{payload}_g\) can be a Causaloid, a collection of Causaloids, or another CausaloidGraph.

    \item[Causal Hyperedge (\(e_g\))] 
    A hyperedge within a CausaloidGraph \(G\), defined as \(e_g = (V_{\text{source}}, V_{\text{target}}, \text{logic}_e)\), representing a directed functional relationship.

    \item[State of CausaloidGraph (\(S_G\))] 
    A function \(S_G: V_G \to \{\text{active}, \text{inactive}\}\) mapping each causal node in a CausaloidGraph to its current activation status. (See Definition 4.3)

    \item[Effect (\(\varepsilon\))] 
    Primarily the activation state (active/inactive) of a Causaloid within EPP, and the transfer of this status or derived information. (See Section 5.1.1)

    \item[Input Observations/Triggers (\(O_{input}, O_{trig}\))] 
    External data or events (\(O_{input}\) being the set of all possible, \(O_{trig}\) being the current set) that can initiate or influence the Effect Propagation Process. (See Definition 5.1)

    \item[EPP Transition Function (\(\Pi_{EPP}\))] 
    The core dynamic function \( \Pi_{EPP} : (G, S_G, \mathcal{C}_{sys}, O_{trig}) \to S_G' \), describing how the state of a CausaloidGraph evolves. (See Definition 5.2)

    \item[Operational Generative Function (\(\Phi_{gen}\))] 
    A formal function or set of functions representing meta-level rules for the dynamic evolution of the Contextual Fabric (\(\Phi_{gen\_C}\)), CausaloidGraphs (\(\Phi_{gen\_G}\)), or both (\(\Phi_{gen\_Total}\)). (See Section 5.2)

    \item[Causal State Machine (CSM, \(M\))] 
    An operational component, defined by its set of state-action pairs \(\mathcal{SA}_M\), that links recognized Causal States (\(q\)) to deterministic Causal Actions (\(a\)). (See Definition 6.3)

    \item[Causal State (\(q\))] 
    A specific condition within a CSM, \( q = (id_q, \text{data}_q, \chi_q, \text{version}_q) \), whose activation is determined by its associated Causaloid \(\chi_q\). (See Definition 6.1)

    \item[Causal Action (\(a\))] 
    A deterministic operation within a CSM, \( a = (\text{exec}_a, \text{descr}_a, \text{version}_a) \), triggered by an active Causal State. (See Definition 6.2)

\end{description}