\section{The Epistemology of the Effect Propagation Process}
\label{sec:epp_epistemology}

Epistemological approaches to acquiring knowledge in research fall into three categories: positivism, interpretivism, and pragmatism. Positivism concerns itself with observable facts based on the scientific method and thus seeks to achieve generalizability and objectivity. Interpretivism maintains that our knowledge depends greatly on our interpretation of observations of human actions, experiences, and environments thus making interpretive research more subjective. Pragmatism focuses on practical effects or solutions to address problems that are suitable for existing situations or conditions. The epistemology of pragmatism is that knowledge is a self-correcting process based on experience thus, it must be evaluated and revised in view of subsequent experience.

The  presented EPP epistemology changes depending on whether the context is static or dynamic, and, equally profound, whether the EPP is static, dynamic, or emergent.

\textbf{Ontology of Knowledge sources}

In the EPP, the context is designed as the source of factual knowledge. For context, facts may remain invariant (e.g. the value of Pi) or receive continuous updates. The designation whether a context is static or dynamic refers to its structure, not to the factual data in it. Furthermore, a context might be shared between two or more defined EPP and an EPP may use one or more context(s) thus simplifying modeling complex domains.
The EPP encodes each causal relationship in a designated Causaloid. The Causaloid encodes the causal rule, whereas the context encodes supporting data required to apply the rules. The Causaloid may use external data or data from the context to apply its rule.
For example, a context may encode a continuous signal feed from a LIDAR sensor and the Causaloid encodes a rule to infer whether an obstacle has been detected. In this case, the context provides all data. In another scenario, a context may encode several known defect patterns, a Causaloid tests incoming image data for the defect data from the context, but uses incoming real-time image feeds from a manufacturing monitoring system to determine if any of the produced items contain known defects. In this case, the Causaloid relies on context and external data. Therefore, the Effect Propagation Process emits a flexible knowledge ontology by relying on one or more contexts and potentially multiple external data sources.

\textbf{Knowledge Derivation}

The EPP derivates knowledge by applying data to the Causaloid that models that causal relationship to determine whether the causal relation holds true within the applicable context. Consequently, multi-stage reasoning maps directly to the topology of the EPP itself because each effect from a Causaloid propagates further through the EPP topology, which is the structure of the EPP manifested as all connected Causaloids.
From this perspective, a “line of reasoning” literally becomes a pathway through the EPP topology.
Through the topological approach of knowledge derivation, the Effect Propagation Process provides a flexible way to model complex, contextual, multi-causal domains.

\textbf{Justification of Derived Knowledge}

Discerning the truthfulness of knowledge is one key element of epistemology. The EPP with its explicit context, explicit causal function (Causaloid), and explicit support for external data provides all pre-requirements to support the full explainability of each inference. Furthermore, in the case of multi-stage reasoning, the sequence of applied Causaloids establishes the order or explainability.
Fundamentally, the EPP leads to explainable causal inference because of complete data, context, and inference function when assuming a static EPP. For a dynamic or emergent EPP, explainability might not be guaranteed for all potential state transitions. An implementation of the EPP has to specify the exact details to support explainable inference and where to establish sensible constraints on explainability.

The gravitas of the EPP epistemology is rooted in its flexible, contextualized ontology, a powerful knowledge derivation mechanism, and its intrinsic support of explainable causal inference. The epistemology varies depending on whether the EPP process is static, dynamic, or emergent.

\subsection{Static EPP Epistemology}

For a static Effect Propagation Process, the knowledge is explicitly modeled during the design stage and confined in the context. The quantitative nature of explicitly modeled context and EPP leads to the positivism of the resulting epistemology.

\textbf{Static context}

A static context emits an invariant structure after it's defined, therefore a static EPP combined with a static context allows for the strongest deterministic guarantees albeit at the expense of flexibility. Static contexts remain an invaluable tool to model contextual data that remain structurally invariant, which is a common situation when integrating external data sources. The content, structure, richness, and accuracy of that static context profoundly determine the epistemology of what can be known through the EPP.

\textbf{Dynamic context}

In a dynamic context, the context structure itself evolves e.g. new elements (i.e. quarter of a year) are added as the data feed progresses. By definition, a dynamic context relies on a generating function to gauge the dynamic changes of the context. The impact on the epistemology of a static EPP remains minimal though.
Fundamentally, dynamic contexts are used when structural elements occur at either regular intervals or otherwise determinable occurrences, and therefore, the EPP can model these elements regardless of whether they have been added to the context yet.
For example, a Causaloid that determines whether the sum of the previous three monthly financial reports matches the quarterly financial reports for the current quarter might be a precondition if the “current” quarter in the context has been updated. Therefore, dynamic contexts simplify domain modeling while leaving the epistemology modeled in a static EPP intact.

\subsection{Dynamic EPP Epistemology}

For a dynamic Effect Propagation Process, the dynamics are captured in generative functions that evolve either the EPP, the context, or both. Conceptually, these generative functions could range from deterministic, rule-based algorithms that construct or modify Causaloids and Context structures based on predefined logic or specific triggers, to more adaptive mechanisms. For instance, a generative function for a dynamic Context might be a higher-order function that, given the current state and new inputs, returns an updated Context graph, a practice well established in functional programming to build dynamic systems.

The ontology of knowledge may evolve as a result of the evolving EPP and the impact of the epistemology remains deepening not only the EPP evolution itself but the interaction with its context as it can happen that both, the EPP and its context evolve dynamically.

\textbf{Static context}

For a dynamic EPP, a static context may serve as the foundational layer that captures core data that remain structurally invariant. As with a static EPP, the static context determines fundamentally what determines the epistemology of what can be known through the dynamic EPP.

\textbf{Dynamic context}

For a dynamic context, though, the impact on the epistemology captured in a dynamic EPP can be profound. For example, with the advent of model context protocol (MCP), which lets LLMs call into tools to retrieve or modify data, a causaloid in a dynamic EPP may trigger an MCP invocation, which then updates the context by expanding its structure, and then triggers a generative function that creates a new Causaloid based on the retrieved contextual data, which then analyzes either a newly created part of the context, or a new external data feed created by the MCP. As a consequence, the epistemology in this case depends on a dynamic EPP-Context co-evolution.

\textbf{Dynamic Co-Evolution}

When both, the EPP and the engulfing context evolve dynamically in what can be seen as a co-evolution, then no fixed epistemology can be established anymore because the inference based on generated Causaloid over newly added sub-structures of the context may or may not occur depending on the occurrence of the underlying trigger event(s). One could estimate a potential epistemology by using a Rubin causal model\cite{rubin2005causal} (RCM) by comparing potential reasoning outcomes under different scenarios in which a Causaloid was generated versus when not.

More profoundly, it might not be possible any longer to use automated explainability to discern the appropriateness and relevance of the generated Causaloids and contextual shifts in response to external changes. This introduces an element of interpretivism to the resulting epistemology: the derived knowledge requires the observer to apply a conceptual framework for understanding the system's complex and dynamic evolving behavior.

\subsection{Emergent EPP Epistemology}

Unlike a dynamic EPP, an emergent EPP does not evolve anymore based on pre-determined triggers that initiate pre-defined generative functions. Instead, an EPP is considered emergent when the underlying generation process leads to novel causal configurations, reasoning pathways, or new generative principles for Causaloids context interactions that were not explicitly encoded beforehand.
The generation process may incorporate principles from evolutionary computation, novelty search, or machine learning embedded in the EPP itself. While the full exploration of AI-driven generative functions for EPP remains future work, the foundational idea of using programmable functions to dynamically define and evolve both the EPP and its context is a natural extension of EPP's core design philosophy.

Regardless of the mechanism, emergent EPP does not interact with its Context using pre-defined procedures but instead relies on procedures generated by the EPP itself. The key indicator of emergence is that its novel behavior was not foreseeable by its initial designers.

\textbf{Static context}

Like a static or dynamic EPP, when the static context has been defined upfront, it determines fundamentally what determines the epistemology of what can be known through the dynamic EPP within the contextual boundaries.

Unlike a static or dynamic EPP, an emergent EPP may or may not generate a new static context and that indeed alters the Epistemology emerging from an emerging EPP.

\textbf{Dynamic context}

Likewise, when an emerging EPP creates or modifies a dynamic context, the emerging Epistemology cannot be determined any longer because of the resulting co-emergence of the EPP and its context.

\textbf{Dynamic Co-Emergence}

An EPP co-emerges with its context when the underlying generation process leads to novel causal configurations that were not explicitly encoded beforehand. This can happen when the EPP contains methods of machine learning that evolve the EPP itself in response to a dynamically changing context. As a result of the dynamic, non-deterministic self-modification of the EPP itself, the spectrum of subsequent factual representation in the context and the emerging causal structures cannot be predicted any longer.

Therefore, determining the truthfulness of the emerging causality imposes a non-trivial challenge that adds an elevated level of pragmatism to the epistemology. The pragmatism becomes necessary because it is not guaranteed that the underlying dynamic context always leads to a truthful representation of the world it seeks to model, but the generated causal relationships may not always be correct either. Both can happen due to generative errors during the EPP. Generative errors may result from complex interactions that contain steps that, in isolation, are correct, but when combined in a certain order may lead to an incorrect outcome. This is a typical characteristic of increasingly complex dynamic systems that need to be considered by taking an operational stance on truth.

\subsection{Operationalized meaning of truth in EPP}

The meaning of truth evolves depending on the modality of the EPP because the underlying reference for a true statement varies depending on the chosen modality. Per the definition of knowledge, a true belief must entail a high degree of justification and come from a reliable source to count as knowledge. It is the underlying justification process that depends on the modality of the EPP that causes the shift in the meaning of truth to vary.

For a static Effect Propagation Process, the meaning of truth aligns with the classical correspondence theory. That means, that if the context encodes accurate facts and the causal relationships are true, all derived forms of knowledge must be true.
Justification rests upon the verifiable mapping between the EPP's explicit model encoded in its context and the part of reality it purports to represent. The static EPP implicitly operates under the assumption that its model is a faithful mirror of objective facts. Here, the truth of an inference is determined by the adherence to the contextual facts and encoded causal relationships. As a result, a static EPP leads to deterministic verifiability within the confined boundaries of its context.
As the EPP transitions into a dynamic modality, the meaning of truth begins to shift towards a coherent adaptability to dynamic interaction with a changing context. A dynamically modified causal relationship is deemed true if it maintains consistency with the facts in its evolving context. In a dynamic modality,  the justification of knowledge becomes contextually and temporally aware. Therefore, truth is assessed by the EPP's capacity to maintain a relevant and internally consistent causal understanding amidst navigating a temporal dynamic context.  This leans towards a coherence theory of truth, where coherence itself must be evaluated relative to the EPP’s intricate temporal structures.


For a contextual co-emergent EPP, the meaning of truth shifts further toward pragmatic efficacy. This shift becomes necessary because of the emergence of relativistic causal relationships from the EPP that co-evolve with its context. Here, establishing an objective a priori truth becomes elusive since the fabric that would traditionally serve as a stable reference for truth is itself emerging dynamically alongside the causal inference made from it. Instead, the truth of an emergent causal inference is established by its utility in enabling the EPP to navigate its environment within its temporally complex context.
This pragmatic efficacy means that truth, defined by its functional value, becomes inherently system-relative and context-dependent.

Indeed, a functional value could serve as a fitness function guiding the emergent process itself thus raising fundamental questions about alignment. Consequently, pragmatic efficacy can lead to multiple, functionally 'true' yet distinct causal understandings, each valid within its own emergent trajectory and its relativistic interrelation with its context.

This dynamic interplay, where the EPP generates both its context and the Causaloids that encode the causal relationships that operate within that context presents a research opportunity. It allows for the exploration of relativistic emergent causality and how coherent and pragmatically effective causal understandings can arise in systems that lack a fixed predefined spacetime. This might be of interest in theories of fundamental physics where spacetime itself may be an emergent phenomenon arising from more fundamental processes.

\subsection{Causal Emergence}
\label{sec:causal_emergence}

The problem of modeling no a-priori causal structures motivates a different view of causality that sets the stage for tackling  causal emergence. The Effect Propagation Process framework's detachment from fixed spacetime and its focus on a generative function establish the foundation" for causal emergence. Static causal discovery, for example Pearl’s DAGs framework, assumes a fixed causal structure and thus cannot handle causal emergence. Granger causality assumes that time-dependent variables change, but the causal structure remains fixed and therefore cannot handle causal emergence either. This argument holds true for any dynamic system with fixed causality because of the inability of the underlying methodology to handle spacetime-agnostic causal structure. The Effect Propagation Process instead proposes that the causal relationships themselves emerge, change, and may even disappear. The implications of this approach lead to a fundamental reassessment of how to operationalize causality:

\textbf{Causal discovery}

Instead of trying to find a fixed causal structure, EPP models how causal relationships emerge from underlying (dynamic) processes. The existing work on causal discovery remains valid; the EPP, however, takes the idea one step further by incorporating a dynamic generative process. Further research will verify the utility of this perspective, but at least it expands the notion of discovering a static structure to describing a dynamic process.

\textbf{Causal transferability}

Instead of trying to capture the exact conditions under which a causal relationship holds true, the EPP specifies all presumptions as a generative function, which makes it fundamentally testable and thus transferability can be decided.

\textbf{Causal dynamics}

The inspiration from causality in quantum gravity was carefully chosen because of its unique ability to reconcile dynamic and static structures and its handling of deterministic and probabilistic modality. The underlying idea in quantum gravity is that the spacetime fabric of reality itself emerges from an underlying process. While we do not have the scientific methods and technology to verify this idea on the quantum level, we can carefully, within boundaries, transfer the idea to the EPP notion that causality itself emerges from an underlying generative process and, therefore, model the dynamics of causal emergence. The properties of EPP become apparent when looking from the lens of modeling the dynamics of causal emergence.

\begin{itemize}
    \item \textbf{Temporal order becomes irrelevant:} Because causal relations can emerge from an underlying process
    \item \textbf{Spacetime-agnostic becomes necessary:} Because the generative process is concerned with establishing relations of effect propagation, the exact fabric through which those effects propagate is conceptualized as an external context; therefore, EPP itself has to be spacetime-agnostic.
    \item \textbf{Hardy's Causaloids are necessary:} Because the EPP itself is spacetime-agnostic, a different representation of causal relationship that is also spacetime-agnostic becomes necessary and the causaloid proposed by Lucian Hardy has been deemed the best fit.
    \item \textbf{Centrality of the generative function:} Because the EPP can represent causal relations as either static, dynamic, or emergent, the generative function takes on a central role to express those causal relations. Furthermore, a generative function may generate the engulfing context as a specific fabric for the effect propagation process.
\end{itemize}


The Effect Propagation Process framework constitutes a  foundational shift to viewing causality itself as emergent and thus redefines what causality means in dynamic systems. Because of its flexibility, EPP can express static causal relationships similar to Pearl’s Causal DAG, it can handle dynamic causal systems similar to Dynamic Bayesian Networks, but then goes further and enables dynamic causal emergence. Dynamic causal emergence has real-world applications:

\begin{itemize}
    \item \textbf{Financial Markets:}  Causal relationships between assets change based on market conditions. EPP can be used to model how these relationships emerge and dissolve.
    \item \textbf{Biological Systems:} Gene regulatory networks where modulating relationships emerge based on cellular state.
    \item \textbf{Social Systems:} How influence relationships emerge and change in social networks.
\end{itemize}