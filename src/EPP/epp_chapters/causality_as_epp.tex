\section{Causality as Effect Propagation Process}
\label{sec:epp}

\subsection{Background}
\label{sec:epp_background}

The philosophy of the Effect Propagation Process (EPP) builds upon several important contributions in philosophy and science. 

The EPP fundamentally rejects the classical Newtonian conception of a static absolute background spacetime. Historically, the idea finds precedent in Alfred North Whitehead who argued that reality is not composed of enduring substances but of dynamic, interconnected "actual occasions." This view inspired the shift towards a dynamic effect propagation process. 

Luciano Floridi's view that the design principles for  dynamic systems require a relational paradigm was profoundly inspirational to the formalization of the Effect Propagation Process. The EPP leverages the hypergraph as its foundational structure to model rich and complex relationships across all its elements. To do so, the ontology provides the foundational principles for handling disjoint categories of knowledge, not by exhaustively modeling all content, but by defining the foundational categories (e.g., space, time, data) and the dynamics of change itself.

Next, the EPP is inspired by Einstein's theory of General Relativity, which demonstrated that spacetime is a dynamic fabric, its geometry determined by the matter within it, which in turn dictates the motion of matter. The EPP's concept of a Contextual Relativity that is both influenced by and influences the entities within is a direct metaphysical analogue of this profound physical insight. 

The EPP's Emergence and Dynamic Co-Emergence principles  are inspired by the profound work of Lucien Hardy on formulating a framework for Quantum Gravity. Hardy noted the core conflict: general relativity is a theory with a dynamic causal structure, while quantum theory operates on a fixed causal structure. More profoundly, Hardy points out that Quantum Theory leans probabilist causal structures whereas General Relativity uses deterministic causal structures. This insight inspired the multi modal reasoning across probabilistic and deterministic representations of causality in the EPP. 

Hardy introduced the "causaloid," a concept that encapsulates a region of spacetime and the causal connections within it. The EPP draws direct inspiration from Hardy’s pioneering work by using the term Causaloid honoring Hardy's concept of a unified, self-contained unit of causality, though it has been adapted for a more general, computational context. Furthermore, the EPP's concept of Dynamic Co-Emergence is a direct architectural answer to the challenge of indefinite causal structures Hardy identified. The EPP's Emergence principle provides a computational mechanism to dynamically generate and select a new causal structure (a new archê kai aitia) that did not exist before in response to a trigger event. This co-evolution of the causal structure and its context is a computational mechanism for realizing a system with dynamic indefinite causal structures that, on demand, become into being. 

The contribution of the EPP ontology is to synthesize all these best in class concepts and formalize them into a deeply integrated ontology that defines the specific primitives (Causaloid, Context), compositions (Isomorphic Recursion) before introducing Contextual Relativity as the governing principle of predictable dynamics and the Generative Process as the governing principle of dynamic emergence. 


\subsection{Definitions}
\label{sec:epp_definition}

The notion of a generative process that underlies the fabric of spacetime leads to the implication that causality has to evolve beyond the strict “before-after” relation towards a spacetime-agnostic view. The classical definition of causality, taken from Judea Pearl's foundational work\cite{pearl2000causality}: 

\begin{quote}
    IF (cause) A then (effect) B.
    
    AND 
    
    IF NOT (cause) A, then NOT (effect) B.
\end{quote}

When removing time from causality, it is indeed no longer possible to discern cause from effect because, in the absence of time, there is no “happen-before” relation any longer, and therefore, the designation of cause or effect indeed becomes arbitrary, just as Russell hinted at earlier on. When removing space from causality, the location of a cause or effect in space is not possible anymore because space itself is no longer available. The absence of spacetime raises the question: 
\begin{quote}
\begin{center}
    \textit{What is the essence of causality?}
\end{center}
\end{quote}

Logically, the answer comes in three parts:

\begin{enumerate}
    \item Causality is a process.
    \item Causality deals with effects.
    \item Causality describes how effects propagate.
\end{enumerate}

The first one is self-explanatory because causality occurs in dynamic systems that change and therefore, causality must be a process.

The second one is less obvious, because one might think that causality is all about the “cause” that brings the effects into existence. However, let’s think the other way around: We know that X is the cause of effect E, because E happens when X happens and because E does not happen when X does not happen either. Therefore, we can describe a cause in terms of its effects. Therefore, it is true that causality deals with effects.

The third one, effect propagation, stretches the imagination and is less obvious. When we rewrite the previous definition of classical causality in terms of effect propagation, we see that there is no loss of information:

\begin{quote}
    If X happens, then its effect propagates to Effect E.

    AND
    
    If X does not happen, then its effect does not propagate to Effect E.
\end{quote}

In this definition, X does not have a designated label and instead is described in terms of its emitting effect. Therefore, X can be seen as a preceding effect, which then propagates its effect further. Therefore, causality becomes an effect propagation process. 

In this definition, X does not have a designated label and instead is described in terms of its emitting effect. Therefore, X can be seen as a preceding effect, which then propagates its effect further. Therefore, causality becomes an effect propagation process.

The effect propagation process definition is more general and treats the classical happen-before definition of causality as a specialized derived form. When you designate the preceding effect to be a “cause”, then you can rewrite the general definition back into the classical definition thus the general and the specialized definition of causality remain congruent. This detail is important because the generalized effect propagation process definition would not be sound if it were unable to express a specialized variation. Framing causality as an effect propagation process leads to several implications:


\subsection{Effect Propagation}
\label{sec:epp}

In EPP,'effect propagation' denotes the operation of a specific, definable mechanism (via the causaloid) that links states through the underlying fabric regardless of how that fabric might be defined.

The first part, the causaloid, operationalizes effect transfer within a system without relying on any metaphysical causal power. The effect transfer encapsulated in the causaloid is  intrinsic to its mechanistic definition. The identification of relevant causaloids requires explicit modeling and hypothesis testing or (future) discovery processes. EPP provides the formatl structure for representation of Causeloids once discovered.

The second part, the non-defined fabric, through which effects propagate, serves the purpose of externalizing the fabric as a specific, definable context. Instead of assuming an implicit background spacetime, the EPP externalizes the fabric through which effects propagate as a specific context that is defined by a different generative function.
Therefore, EPP fundamentally moves away from invoking a metaphysical causal power towards definable, testable, and verifiable functions (Causaloids) that define the functional relationships describing how a system changes if these functions   are active.

\subsection{Spacetime Agnosticism}
\label{sec:epp_spacetime}

In quantum gravity, where spacetime geometry might be in a superposition or non-existent at the fundamental level, "propagation" isn't a movement along a geodesic in a manifold. It's the propagation of effects through a network of states or elements defined by the fundamental structure. This is a prerequisite for handling indefinite causal order.


\subsection{Indefinite Causal Order}
\label{sec:epp_indefinite_causal_order}

In situations where the causal structure itself is indefinite (a superposition on the quantum level, a not-yet-emerged state in GR), "effect propagation" can be understood as the influence propagating through a superposition of possible pathways through the fundamental structure. The "effect" isn't tied to a single, definite causal link but is a result of the propagation through all possible pathways.

\subsection{Causaloid}
\label{sec:epp_causaloid}

In the Effect Propagation Process framework, due to the detachment from a fixed spacetime, this fundamental temporal order is absent. Consequently, the entire classical concept of causality, where a cause must happen before its effect, can no longer be fundamentally established. The distinction between a definitive 'Cause' and a definitive 'Effect' becomes untenable as Russell foresaw. When the separation between cause and effect becomes untenable, then the obvious question arises: why even preserve an untenable separation?

Therefore, the Effect Propagation Process framework adopts the causaloid, a uniform entity proposed by Hardy\cite{HardyDynamicCausalStructure}, that merges the ‘cause' and 'effect' into one entity. Instead of dealing with two nearly identical concepts discernible from each other by temporal order, the causaloid is one concept that defines causality in terms of its effect transfer without presupposing a fixed spacetime background. It is important to note that the EPP framework adopts the conceptual role of the Causaloid as a spacetime-agnostic unit of causal interaction, inspired by Hardy’s work on Quantum Gravity, but it does not uses Hardy's formal definition. Instead, the EPP formulates the Causaloid as an abstract structure, thereby decoupling it from any particular physical theory while preserving its core philosophical utility and making it practical implementable in software.

In the post-quantum context, the term "propagation" does not imply movement through a pre-existing space or time in the classical sense. Instead, it refers to the fundamental process by which an effect is transferred within the underlying structure of reality itself. This fundamental process is what gives rise to the appearance of propagation through spacetime in the classical view. Furthermore, while classical causality relies on a definite temporal order, the Effect Propagation Process treats temporal order as an emergent property, arising from the fundamental process itself.

While the Effect Propagation Process involves the transfer of effects within the fundamental structure, it is crucial to distinguish this from mere accidental correlation. The process reflects the fundamental way the underlying structure of reality establishes dependencies between its components and how it is gives rise to the non-accidental relationships we recognize as observed causal relations.This fundamental determination, rather than simple co-occurrence, is what the "Effect Propagation Process" captures at the deepest level.

Consequently, causality is understood as an effect propagation process that emerges from the fundamental structure or set of rules (akin to a generating function) from which spatiotemporal relationships emerge. This philosophical concept of the Effect Propagation Process finds support in physical theories that propose fundamental structures underlying spacetime, such as Causal Set Theory, and generalizes the idea of effect transfer present in Quantum Field Theory.

Lastly, the Effect Propagation Process offers a philosophical interpretation for mathematical tools that describe non-classical causal behavior, such as Process Matrices. However, the Effect Propagation Process remains agnostic to the operational details of any particular physics theory and, instead, it offers a coherent way of thinking about causality that aligns with the absence of finite spacetime in the quantum realm.

\newpage
