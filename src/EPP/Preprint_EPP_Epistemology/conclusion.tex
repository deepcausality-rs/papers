\section{Conclusion}
\label{sec:Conclusion}


The epistemology of the Effect Propagation Process reflects the complex systems it is designed to model by scaling with the modality of the EPP. For a static EPP, a positivist epistemology remains sufficient. For a dynamic EPP, the epistemology evolves towards an interpretivism perspective, and for an emergent EPP, a pragmatism perspective on the epistemology becomes necessary.

Likewise, for the justification of knowledge in an EPP, the underlying notion of truth scales with the modality of the EPP. For a static EPP, the meaning of truth aligns with the classical correspondence theory. However, in a dynamic EPP, the meaning of truth shifts towards a coherent adaptability approach. In an emergent EPP, the meaning of truth evolves towards pragmatic efficacy where the validity of relativistic, emergent causal relationships is established by their functional utility.

The shifting epistemology implies that the role of human involvement equally scales with the EPP modality. For a static EPP, the human remains the designer, whereas for a dynamic EPP, the role extends towards an observer to understand system dynamics by observation and analysis. For an emergent EPP, however, the role of an observer might be delegated to an artificial intelligence while the human role evolves further towards assessing the quality and appropriateness of the emerging EPP.

The rich epistemology of the EPP enables novel explorations of understanding evolving causal knowledge in emergent systems. Further research is warranted to ensure the safety and alignment of emergent systems.