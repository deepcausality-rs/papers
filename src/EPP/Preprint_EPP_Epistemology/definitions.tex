\section{Definitions}
\label{sec:Definitions}

\subsection{Epistemology}
\label{subsec:Def_Epistemology}

Epistemology refers to the study of understanding knowledge and sets out to explain the sources of knowledge, how we derive knowledge, and the distinction between justified belief and opinion\cite{sol2022understanding}.

\subsection{Knowledge}
\label{subsec:Def_Knowledge}

A consensus is that knowledge is defined as justified true belief. Thus, the three key components of knowledge include truth, belief, and justification. By truth, it is meant that something is in correspondence to facts or reality. Belief refers to the state of mind that a person has about reality and truth\cite{sol2022understanding}.
When considering a proposition, belief results in three different states of mind: We may believe and accept it as true, disbelief and rejection as false, or withhold belief for further judgment. The last key component of knowledge is justification.
A true belief must entail a high degree of justification to count as knowledge. Sources of knowledge include, but are not limited, to perception, memory, reason, and testimony. Knowledge must come from a reliable source to be considered true. Epistemological reasoning broadly falls into two categories: a priori (non-empirical i.e. deduction) and a posteriori (empirical) justification of knowledge\cite{sol2022understanding}.


\subsection{Positivism, Interpretivism, and Pragmatism.}
\label{subsec:Def_Positivism}

Epistemological approaches to acquiring knowledge in research fall into three categories: positivism, interpretivism, and pragmatism. Positivism concerns itself with observable facts based on the scientific method and thus seeks to achieve generalizability and objectivity. Interpretivism maintains that our knowledge depends greatly on our interpretation of observations of human actions, experiences, and environments thus making interpretive research more subjective. Pragmatism focuses on practical effects or solutions to address problems that are suitable for existing situations or conditions. The epistemology of pragmatism is that knowledge is a self-correcting process based on experience thus, it must be evaluated and revised in view of subsequent experience\cite{sol2022understanding}.

\subsection{Classical Causality}
\label{subsec:Classical_Causality}

The classical definition of causality, taken from Judea Pearl's foundational work\cite{pearl2009causality}:


\begin{quote}
\begin{center}
        IF (cause) A then (effect) B

    AND 
    
    IF NOT (cause) A, then NOT (effect) B
\end{center}
\end{quote}


The classical definition does not entail a transmitter of the causal effect and thus assumes an engulfing spacetime. It is a subtle, but important assumption that, among others, requires time linearity to preserve the introduced time asymmetry. Time asymmetry follows from the directional definition (via the negation, if-not-then-not) that the causal relationship must flow from the cause to the effect.


\subsection{Effect Propagation Process}
\label{subsec:Effect_Propagation_Process}

The effect propagation process\cite{Hansen2025EPP} generalizes causality by removing the spacetime constraint by replacing the implicitly assumed spacetime with the stated propagation which serves as an effect transmitter. The generalized definition reads:

\begin{quote}
\begin{center}
        If X happens, then its effect propagates to Effect E.
    
    AND
    
    If X does not happen, then its effect does not propagate to Effect E.
\end{center}
\end{quote}


In this definition, X does not have a designated label and instead is described in terms of its emitting effect. Therefore, X can be seen as a preceding effect, which then propagates its effect further, and thus causality becomes an effect propagation process. The propagation does not entail any notion of time linearity because the emitting effect flows from one effect to another, but crucially, in the absence of the previous cause/effect distinction, the implied assumption of linear temporal order has been removed.
Furthermore, the definition does not imply any spatial constraints and thus makes the definition fundamentally spacetime agnostic. The logical constraints (if-then and if-not-then-not) remain present to preserve the essence of causality and maintain a clear distinction from correlation.

Because X and E are seen as effects, there is no meaningful distinction between cause and effect any longer. This directly leads to the EPP’s adoption of the Causaloid, a uniform entity proposed by Hardy\cite{HardyDynamicCausalStructure}, that merges the ‘cause' and 'effect' into one entity. Instead of dealing with two nearly identical concepts discernible from each other by temporal order, the causaloid is one concept that defines causality in terms of its effect transfer without presupposing a fixed spacetime background\cite{HardyDynamicCausalStructure}.

In EPP, effect propagation denotes the operation of a specific, definable mechanism (via the causaloid) that links states through the underlying fabric regardless of how that fabric might be defined. The first part, the causaloid, operationalizes effect transfer within a system without relying on any metaphysical causal power. The effect transfer encapsulated in the causaloid is intrinsic to its mechanistic definition. The identification of relevant causaloids requires explicit modeling and hypothesis testing or (future) discovery processes. 

EPP provides the formal structure for the representation of Causeloids once discovered. The second part, the non-defined fabric, through which effects propagate, serves the purpose of externalizing the fabric as an explicitly defined context. Instead of assuming an implicit background spacetime, the EPP externalizes the fabric through which effects propagate as a specific context that is defined by a different generative function. Therefore, EPP fundamentally embraces definable, testable, and verifiable functions (Causaloids) that define the functional relationships describing how a system changes if these functions are active.

Furthermore, the Effect Propagation Process might be static in the sense of being defined ahead of time, dynamic in the sense of being defined at runtime from a generative function, or emerging in the sense that the EPP results from previous generative stages of itself.

\newpage

\subsection{Externalized Context}
\label{subsec:Externalized_Context}

The Effect Propagation Process generalizes the preceding internal implicit spacetime assumption by replacing it with an external and explicitly defined context.
A context can be static or dynamic, depending on the situation. The context structure is defined beforehand for a static context, whereas for a dynamic context, the structure is generated dynamically at runtime from a generative function.

Unlike the EPP, a context is constrained from being recursive. Self-referential recursion in the absence of a stop criterion may lead to infinite time or even spacetime loops, which conflict with explainability and touches upon exotic scientific ideas. Therefore, for EPP's current formulation, context is constrained from being recursive to prevent problematic corner cases and to ensure a straightforward implementation of the presented concepts.

Regardless of the specific structure, the context is explicitly defined, either by a (static) structure or by a generating function that generates the context structure dynamically.

Data in any context changes over time, hence requiring constant updating. A change of context structure may happen and may affect the effect propagation process. Lastly, the need to handle multiple contexts emerges from the simple need to address multiple sources of contextual data.

With the separation of the Effect Propagation Process and its context, several interrelations emerge that directly impact the understanding of derived knowledge. Furthermore, since the context serves as the factual source of the EPP, and the EPP defines generalized causal inference, the resulting epistemology depends on the structure of the context and the EPP.

