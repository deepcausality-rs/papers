\section{Discussion}
\label{bsec:Discussion}

The Effect Propagation Process with its static, dynamic, and emergent modalities leads to a remarkably rich and multifaceted epistemological landscape that ranges from positivism to pragmatic emergence.

When a static EPP operates in a static context, the epistemology aligns with positivism. Knowledge is derived from explicitly modeled, observable facts and follows explicit rules, leading to verifiable and deterministic causal chains. The objective is a generalizable inference based on a predefined world model and is best suited for confined use cases such as control systems that operate autonomously within defined boundaries.

However, with the introduction of dynamics via a dynamic context or a dynamic EPP, the system actively constructs and revises its causal structure in response to, and in concert with, a changing environment. As a result, system boundaries evolve and shift dynamically. While the explainability of individual steps might be maintained, the relevance of newly generated Causaloids and contextual interpretations necessitate an observer's engagement.
The observer's engagement adds an element of interpretivism because the meaning and validity of the system's knowledge are co-constructed through the observer's understanding of its complex, evolving behavior.

The 'observer' in these interpretive scenarios is not uniform by definition. Instead, the observer could be the EPP's initial designer who focuses on mechanistic integrity, an end-user who evaluates the domain-specific utility, a regulator who is scrutinizing the EPP for compliance and safety, or even another AI system that integrates EPP outputs into its operational logic. Each of these distinct observers brings a unique interpretive lens and, an advanced EPP may need to support multiple, potentially differing, observational perspectives and their corresponding reporting requirements. This leads to a critical observation: the 'justified knowledge' derived from interpretive engagement with an EPP is not absolute. Instead, the 'truth' or 'validity' of a complex causal inference can become dependent on the observer's framework, with different interpretations holding legitimate, albeit perspectival, justification, thus demanding a clear understanding of whose interpretation is being applied and for what purpose.

In an EPP that dynamically co-emerges with its context, the system may generate novel configurations and reasoning unforeseen by its designers. The justification of knowledge in such a system leans on sophisticated interpretive methodologies to discern meaning and validity from dynamic complex emergent structures. Even if the system’s genesis is not explainable by its initial design any longer, the emergent causal 'understanding' is fundamentally validated by its success in dealing with its co-evolving environment. The system doesn't just adapt; it becomes its own evolving theory of its world.

It is important to recognize that these epistemological stances, positivism, interpretivism, and pragmatism, are not mutually exclusive stages in an EPP. In practice, sophisticated EPP implementations will likely lead to hybrid systems with a static foundation, several different dynamic parts, and potentially emergent parts. At each level of the system, the appropriate epistemological view helps to analyze the mode at hand. For a static subsystem, the positivist view suffices, for a dynamic subsystem requires interpretivist engagement from the designer(s), and for emergent parts, pragmatic validation becomes paramount.

These three epistemological stages have profound implications for the role of the human in the process. With static EPPs, the human is the architect, directly encoding causal knowledge. With dynamic EPPs, the role shifts towards a meta-designer, setting up generative principles and then observing and interpreting the evolving system as it unfolds. Dynamic Bayesian Networks\cite{dagum1992dynamic} may help to develop tools to model transition probabilities between emergent states. The interpretivism perspective becomes the new center of deriving meaning from the dynamic EPP. Lastly, when dealing with emergent EPPs, the role shifts again towards a pragmatic approach to the emerging self-correcting knowledge. to understand, validate, and safely harness that emergent knowledge.

The epistemology of co-emergent EPP and its context raises multiple questions:

\begin{itemize}
    \item When the EPP evolves with its context, how it ensure the resulting causal inference remains valid?
    \item How can the emergent process be safeguarded?
    \item How can the emergent process remain aligned with established principles of AI safety?
\end{itemize}

At this early stage, none of the questions can be answered and more research in the area of ensuring safety in dynamic causal emergence over dynamic effect propagation is warranted.

Because of the inherent indefinite nature of EPP context co-emergence, one option to ensure safety is to codify the co-emergence as much as possible in an external control system that establishes principled boundaries to safeguard the process. Principled boundaries can be thought of as similar in spirit to Asimov's Laws of Robotics which the system must adhere to under all conditions. However, this presupposes a mechanism capable of verifying that it is possible to verify that each emergent step complies with a set of principled boundaries, and that might require additional research.

\newpage
