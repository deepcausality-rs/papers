\section{Introduction}
\label{sec:introduction}

The study of cause and effect served humanity for millennia and provided a foundation for scientific inquiry and intervention. Contemporary frameworks for computational causality, from Pearl's Structural Causal Models to Granger's time-series analysis, provide a formal basis for causal inference within systems governed by fixed causal structures. These classical models, however, are predicated on a set of core assumptions, including a fixed background spacetime, linear temporal progression, and static causal structures. However, at the frontiers of science and engineering a new category of challenges arise where the rules of causality itself become dynamic. For these complex dynamic systems, the classical assumption of a static causal structure embedded in a fixed background spacetime is no longer applicable and imposes a fundamental limitation. From the dynamic regime shifts in financial markets to the context-dependent safety of autonomous vehicles, there is a need for a new foundation to model systems where causal structures themselves can evolve dynamically.

This monograph introduces the Effect Propagation Process (EPP), an axiomatic foundation for composition and evaluation of dynamic causal models. Its purpose is to serve as a foundation upon which a new generation of domain specific dynamic causal theories can be built for systems that operate beyond the classical scope. Categorically, the EPP is closest to a meta-calculus because it defines the abstract mechanisms of dynamic causality while leaving the specifics to a derived dynamic causal model. However, the EPP also borrows from other categories:

\begin{itemize}
\item A Meta-Theory: The EPP provides the philosophical and logical foundation via its metaphysics and ontology upon which one can build a theory of dynamic causality.
\item A Meta-Algebra: The EPP provides the formal, abstract language via its formalization of the Causaloids, Contextoids, and their structural relationships.
\item A Meta-Calculus: The EPP provides the formal, computational elements of dynamic causality via the Effect Propagation Process and the Deontic Inference Cycle.
\end{itemize}

The EPP's single axiomatic foundation that defines causality as a spacetime-agnostic functional dependency necessitates a trifecta of computable, first-class primitives. The detachment from a fixed background spacetime requires an explicit and dynamic the Context. The EPP design of the context enables Euclidean and non-Euclidean representation and linear and non-linear temporal structures and thus supports modeling rich, dynamic, and complex operational environments.

The functional nature of the axiom requires a polymorphic container for any specific causal calculus, the Causaloid. The causaloid, a concept borrowed from physicist Lucian Hardy, unifies cause and effect into one single abstract entity that solve a fundamental problem of structural composition by enabling isomorphic recursive causal structures. The EPP introduces three modalities of dynamic causality: dynamic, adaptive, and emergent. While dynamic and adaptive causality remain deterministic, the introduction of emergent causality where causal structures itself co-emerge with its context, also introduces non-determinism with the implication that verifiability is no longer possible.

The Causal State Machine is a formal mechanism that translates causal reasoning into actions that follow directly from a causal inference. The Teloid and Effect Ethos provide an operational guardrail to causal state machines based on a defeasible deontic calculus. The Effect Ethos enable operating a dynamic causal system in a non-deterministic, emergent environment while ensuring compliance as code and enforcable mission parameters. 

The presented Effect Propagation Process has been fully implemented in the open source DeepCausality project hosted at the Linux Foundation for Data \& Ai since 2023. At the time of writing, the implementaiton has reached a total of fifty thousand lines of Rust code, with an average unit test coverage of 95\%. While the EPP provides the formal methods to construct, and test domain-specific dynamic causal theories, the DeepCausality project enables the implementation, simulation, and application of derived dynamic causal systems.   

The presented monograph can be read from multiple angles. For the reader with a background in Philosophy, a natural path is to begin with the overview of the EPP (Chapter \ref{sec:epp}), and then delve into its philosophical foundation: the Metaphysics (Chapter \ref{sec:metaphysics}), the Epistemology (Chapter \ref{sec:epp_epistemology}), and the Teleology (Chapter \ref{sec:teleology}).
For the reader with a background in Formal Methods, the EPP's core concepts are in Chapter \ref{sec:epp}), then the formal Ontology (Chapter \ref{sec:epp_ontology}) and the Formalization of the meta-calculus (Chapter \ref{sec:formalization}) and its soundness in Chapter \ref{sec:validity}.
A reader with a background in Engineering may start with the Motivation (Chapter \ref{sec:motivation}), understand the high-level EPP framework (Chapter  \ref{sec:epp}), and then see how these concepts are applied in the DeepCausality Implementation (Chapter \ref{sec:implementation}). The Metaphysics in (Chapter \ref{sec:metaphysics}) provides details about the orthogonal design used throughout the implementation.
The reader with a background in Computational Causality may start with the critique of classical models in the Motivation (Chapter \ref{sec:motivation}) and the Related Work (Chapter \ref{sec:related_work}) is a good start. Following the EPP's core concepts (Chapter \ref{sec:epp}), the next step is the Formalization (Chapter \ref{sec:formalization}) and validity (Chapter \ref{sec:validity}). From there, the Future Work (Chapter \ref{sec:future_work}) will be of particular interest.

\newpage
 