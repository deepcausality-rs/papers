\section{Introduction}
\label{sec:introduction}

The study of cause and effect has served mankind for millennia and provides a foundation for scientific inquiry and intervention.
Contemporary frameworks for computational causality, particularly those based on Directed Acyclic Graphs, offer powerful tools for reasoning within this domain.
These classical models, however, are predicated on a set of core assumptions, including a fixed background spacetime, linear temporal progression,
and static causal structures.
These assumptions become a limitation when addressing a class of complex, dynamic systems where the causal relationships themselves are subject to change.

This monograph presents the Effect Propagation Process (EPP), a theory of dynamic causality developed to address these limitations.
The EPP is foremost a theory that enables a cascade of novel capabilities: it allows the EPP to model causal relationships independent of any specific structure of space and time,
to externalize context as a first class entity, to manage structural complexity through isomorphic recursive composition,
to introduce dynamics via contextual relativity, and to add emergence for structural self modification.

These capabilities are enabled by a foundational premise: the detachment of causality from a presupposed spacetime. This step necessitates a re-evaluation of a causal relation, shifting its conceptualization to a more general process of effect propagation. The EPP's architecture is built upon two key innovations.

First, addressing the recognized need for a rich and expressive context that informs causal relations, the EPP externalizes context as a first class structure that is agnostic to space, time, and data type, enabling the use of Euclidean, non Euclidean, and symbolic representations.
Second, The EPP's use of isomorphic recursive composition provides a generalized mechanism for managing arbitrary complex causal structures by
organizing them in hierarchical order. The mechanism is enabled by adopting the Causaloid, a concept first introduced by the physicist Lucien Hardy,
as a spacetime agnostic unit of causal interaction and provides the unit of causality in the EPP.

The framework uses this architecture to distinguish between two fundamental modes of change.
The first is Dynamics, where EPP treats causality as a contextual process, handling predictable internal and external changes through the principle of contextual relativity.
The second mode is Emergence.

The EPP treats dynamic causality as a fundamentally contextual process and thus uses contextual relativity to handle internal and external dynamics.
Internal dynamics refer to causal reasoning relative to one or more attached context using the effect propagation process.
External change to the context are handled via the Adjustable mechanism that enables either updating contextual data to new values or adjusting contextual data,
for example when correcting for a detected error.

This capability for emergence presents a necessary trade-off: One path is to retain classical determinism, a choice which would confine models to non-evolving systems.
The other path is to enable the modeling of emergent causal structures, which provides greater expressive power at the expense of determinism.
The Effect Propagation Process is explicitly designed around the second path, accepting this trade-off as a prerequisite for modeling systems that can genuinely adapt and evolve.
The decision to enable dynamic emergence has significant implications for establishing verification and trust, because a system’s causal rules can co-evolve with its context.
The EPP is designed to address these implications directly through a first-principle philosophy foundation that includes a dedicated metaphysics, ontology, and epistemology.

The EPP integrates insights and methods from three distinct domains. It begins with a philosophical foundation formulated
in a dedicated metaphysics, ontology, and epistemology that establishes the first principles and dynamics of the EPP.
Next, these principles are then translated into a rigorous formalization to ensure logical consistency and precision of the EPP.
The EPP theory is grounded in reality through its principled implementation in the open-source
DeepCausality project hosted at the Linux Foundation. The DeepCausality project leverages an orthogonal
design derived from the EPP metaphysics and ontology to achieve noticeable performance levels for causal reasoning.

Finally, the ethical consequences of modeling emergent causality motivate the need for new building blocks for reliable systems.
The monograph concludes by proposing a path toward a verifiable and computable framework for system safety composed
of a prospective "Teloid" (a unit of purpose) and a retrospective "Effect Ethos."

\newpage