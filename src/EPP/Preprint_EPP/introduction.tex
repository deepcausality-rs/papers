\section{Introduction}
\label{sec:introduction}

The study about cause and effect has served mankind for millennia and provided powerful tools to inform interventions.
However, the foundational frameworks of computational causality, such as those based on Directed Acyclic Graphs, 
presuppose a fixed, linear, and Euclidean world. This monograph addresses the well-recognized challenge 
of extending causal reasoning to dynamic system engineering, where systems are often characterized 
by non-Euclidean data representations, non-linear temporal structures, and dynamic causality. 
Existing methods of computational causality, while valid and effective in their respective domains, 
were not designed for this class of complex, dynamic problem.

In response, this monograph presents the Effect Propagation Process (EPP): a theory of dynamic causality. 
The EPP adopts and formalizes a computable Causaloid, a concept first introduced by the physicist Lucien Hardy. 
Hardy proposed the Causaloid as a logical construct that merges the classical dualism of cause and effect 
into one unified entity that is agnostic of a fixed spacetime.

The EPP is foremost a theory of dynamic causality that became possible because of the adoption of the causaloid 
and its adoption lead to a cascade of novel capabilities: it allows the EPP to model causal relationships 
that are independent of any specific structure of space and time, to externalize context as a first-class entity, 
to introduce dynamics via contextual relativity, and to manage structural complexity through isomorphic recursive composition.

The EPP is therefore a synthesis that integrates the philosophical need for a relational foundation
as suggested by Floridi, the statistical need for hierarchical models as advanced by Blei, 
and the physical need for spacetime-agnostic causality as pioneered by Hardy. 

The EPP integrates insights and methods from three distinct domains. It begins with a philosophical foundation formulated 
in a dedicated metaphysics, ontology, and epistemology that establishes the first principles and dynamics of the EPP. 
Next, these principles are then translated into a rigorous formalization to ensure logical consistency and precision of the EPP. 
Finally, the theory is grounded in reality through its principled implementation in the open-source 
DeepCausality project hosted at the Linux Foundation.

This monograph is structured to guide the reader through the presented material. It begins by detailing the philosophical foundations 
before presenting the complete formalization. The subsequent chapters detail the implementation, including performance benchmarks 
and working examples in relativistic navigation. The monograph concludes by exploring the profound implications of this theory of dynamic causality 
and by arguing that the EPP's architecture provides a plausible path towards a computable "Effect Ethos" designed for verifiable alignment.

\newpage