\documentclass{article}

\usepackage{../styles/PRIMEarxiv} % imports style
\usepackage[utf8]{inputenc} % allow utf-8 input
\usepackage[T1]{fontenc}    % use 8-bit T1 fonts
\usepackage{hyperref}       % hyperlinks
\usepackage{url}            % simple URL typesetting
\usepackage{booktabs}       % professional-quality tables
\usepackage{amsfonts}       % blackboard math symbols
\usepackage{nicefrac}       % compact symbols for 1/2, etc.
\usepackage{microtype}      % microtypography
\usepackage{siunitx}
\usepackage{lipsum}
\usepackage{fancyhdr}       % header
\usepackage{graphicx}       % graphics
\usepackage{float}          % For more control over figure 
\graphicspath{{media/}}     % organize your images and other figures under media/ folder
\usepackage{amsmath}
\usepackage{amssymb}
\usepackage{amsfonts}
\usepackage{listings}
\usepackage{titlesec}
\usepackage{needspace}
\usepackage{glossaries}
\usepackage{enumitem}
 % imports packages 

% imports code formatting used across the document 



\usepackage{color}
\usepackage{listings}
\definecolor{GrayCodeBlock}{RGB}{241,241,241}
\definecolor{BlackText}{RGB}{110,107,94}
\definecolor{RedTypename}{RGB}{182,86,17}
\definecolor{GreenString}{RGB}{96,172,57}
\definecolor{PurpleKeyword}{RGB}{184,84,212}
\definecolor{GrayComment}{RGB}{170,170,170}
\definecolor{GoldDocumentation}{RGB}{180,165,45}
\lstdefinelanguage{rust}
{
    columns=fullflexible,
    keepspaces=true,
    frame=single,
    framesep=0pt,
    framerule=0pt,
    framexleftmargin=4pt,
    framexrightmargin=4pt,
    framextopmargin=5pt,
    framexbottommargin=3pt,
    xleftmargin=4pt,
    xrightmargin=4pt,
    backgroundcolor=\color{GrayCodeBlock},
    basicstyle=\ttfamily\color{BlackText},
    keywords={
        true,false,
        unsafe,async,await,move,
        use,pub,crate,super,self,mod,
        struct,enum,fn,const,static,let,mut,ref,type,impl,dyn,trait,where,as,
        break,continue,if,else,while,for,loop,match,return,yield,in
    },
    keywordstyle=\color{PurpleKeyword},
    ndkeywords={  bool,u8,u16,u32,u64,u128,i8,i16,i32,i64,i128,char,str,Self,Option,Some,None,Result,Ok,Err,String,Box,Vec,Rc,Arc,Cell,RefCell,HashMap,BTreeMap,
        macro_rules
    },
    ndkeywordstyle=\color{RedTypename},
    comment=[l][\color{GrayComment}\slshape]{//},
    morecomment=[s][\color{GrayComment}\slshape]{/*}{*/},
    morecomment=[l][\color{GoldDocumentation}\slshape]{///},
    morecomment=[s][\color{GoldDocumentation}\slshape]{/*!}{*/},
    morecomment=[l][\color{GoldDocumentation}\slshape]{//!},
    morecomment=[s][\color{RedTypename}]{\#![}{]},
    morecomment=[s][\color{RedTypename}]{\#[}{]},
    stringstyle=\color{GreenString},showstringspaces=false,
    string=[b]"
}

%Formating code listings
% https://nasa.github.io/nasa-latex-docs/html/examples/listing.html
\usepackage{listings} 
\lstset
{ %Formatting for code in appendix
    basicstyle=\footnotesize,
    numbers=left,
    stepnumber=1,
    showstringspaces=false,
    tabsize=1,
    breaklines=true,
    breakatwhitespace=false,
}

%Header
\pagestyle{fancy}
\thispagestyle{empty}
\rhead{ \textit{ }} 

% Update your Headers here
\fancyhead[LO]{The Effect Propagation Process (EPP): An Axiomatic Foundation for Dynamic Causality}
  
%% Title
\title{The Effect Propagation Process (EPP):\newline An Axiomatic Foundation for Dynamic Causality}

%% Author
\author{
  Marvin Hansen \\
  \texttt{marvin.hansen@gmail.com} \\
   Date: \today
}


\begin{document}
\maketitle

\begin{abstract}
Contemporary frameworks for computational causality provide a formal basis for inference within systems governed by static causal structures. These models, however, are predicated on assumptions of a fixed background spacetime and linear temporal progression, which become fundamental limitations when addressing complex dynamic systems where the causal laws themselves can evolve.

This monograph introduces the Effect Propagation Process (EPP), a formal meta-calculus for dynamic causality. It provides a single axiomatic foundation, a formal language, and a set of computable primitives for constructing, composing, and verifying domain-specific, causal theories. The EPP's single axiomatic foundation that causality is a spacetime-agnostic functional dependency necessitates a trifecta of core architectural components. The Context is an explicit and dynamic hypergraph that models the operational environment, supporting non-Euclidean and non-linear temporal structures. The Causaloid is a polymorphic, composable unit of causal mechanism that can encapsulate any specific object-level calculus. The Effect Ethos is a  deontic guardrail that uses a defeasible calculus to ensure a system's actions remain verifiably aligned with its safety and mission objectives.

The EPP is fully implemented in the open-source DeepCausality peoject, demonstrating the work's practical application. Together, the EPP and its implementation in DeepCausality provide a foundation for the design, simulation, and implementation of a new generation of robust, context-aware, and verifiably aligned dynamic causal systems.

\end{abstract}

\newpage

%% Table of Contents
\tableofcontents

\newpage

%% Introduction
\section{Introduction}
\label{sec:introduction}

The study of cause and effect has served humanity for millennia and provided a foundation for scientific inquiry and understanding. Contemporary frameworks for computational causality, from Pearl's Structural Causal Models to Granger's time-series analysis, provide a formal basis for causal inference within systems governed by fixed causal structures. These classical models, however, are predicated on a set of core assumptions, including a fixed background spacetime, linear temporal progression, and static causal structures. However, at the frontiers of science and engineering, a new category of challenges arises where the rules of causality themselves become dynamic. For these complex dynamic systems, the classical assumption of a static causal structure embedded in a fixed background spacetime is no longer applicable and imposes a fundamental limitation. From the dynamic regime shifts in financial markets with complex, multi-scale temporal feedback loops to context-dependent safety of autonomous vehicles, there is a need for a new foundation to model systems where causal structures themselves can evolve dynamically.

The presented monograph introduces the Effect Propagation Process (EPP), a single axiomatic foundation for dynamic causal models. Its purpose is to serve as a foundation upon which a new generation of domain-specific dynamic causal models can be built. Categorically, the EPP is closest to a meta-calculus because it defines the abstract mechanisms of dynamic causality while leaving the specifics to a derived dynamic causal model. However, the EPP also borrows from other categories:

\begin{itemize}
\item A Meta-Theory: The EPP provides the philosophical and logical foundation via its metaphysics and ontology upon which one can build a theory of dynamic causality.
\item A Meta-Algebra: The EPP provides the formal, abstract language via its formalization of Causaloids, Contextoids, and their structural relationships.
\item A Meta-Calculus: The EPP provides the formal, computational elements of dynamic causality via the Effect Propagation Process and the Deontic Inference Cycle.
\end{itemize}
  
The EPP adds dynamics as a first class principle to causality based on a single axiomatic foundation that generalizes causality as a spacetime-agnostic functional dependency. The operationalization of the EPP's single axiomatic foundation necessitates a trifecta of computable, first-class primitives: The Context, the Causaloid, and the Causal State Machine.

 The detachment from a fixed background spacetime requires an explicit and dynamic Context. The EPP design of the context enables Euclidean and non-Euclidean representation and linear and non-linear temporal structures, thus supporting the modeling of rich, dynamic, and complex operational environments.

The functional nature of the axiom requires a polymorphic container for any specific causal calculus, the Causaloid. The causaloid, a concept borrowed from physicist Lucian Hardy, unifies cause and effect into a single abstract entity that solves a fundamental problem of structural composition by enabling isomorphic recursive causal structures. The EPP introduces three modalities of dynamic causality: dynamic, adaptive, and emergent. While dynamic and adaptive causality remain deterministic, the introduction of emergent causality, where causal structures co-emerge with their context, also introduces non-determinism with the implication that static verifiability is no longer possible. The Effect Ethos provides an operational guardrail to emergent causality based on a defeasible deontic calculus.

The Causal State Machine is a formal mechanism that translates causal reasoning into actions. A causal state machine separates its state from an action to allow an optional Effect Ethos to verify a proposed action against a set of norms to decide the permissibility of an action relative to the encoded ethos. The causal state machine combined with an Effect Ethos enable dynamic causal system in a dynamic environment while ensuring compliance as code to enforce alignment of derived actions.

\subsection{Precedent}

The presented Effect Propagation Process builds upon rich and diverse preceding scholarly work. 

\textbf{Whitehead's Process Philosophy}

The EPP's primary departure point is a fundamental rejection of the classical Newtonian conception of a static, absolute background spacetime. This move is deeply rooted in the tradition of process philosophy, which argues that reality is not composed of enduring, static substances but is a dynamic flow of interconnected events. This idea finds its clearest expression in the work of Alfred North Whitehead, who posited a universe of "actual occasions"\cite{whitehead2010process}, and Henri Bergson, who described reality as a continuous "creative evolution"\cite{bergson2022creative}. Their shared insight of reality as a process inspires the EPP's foundational redefinition of causality itself, shifting from a static, happen-before relation to a dynamic process of effect propagation.

\newpage

\textbf{Einstein's Theory of General Relativity}

Einstein's theory of General Relativity\cite{EinsteinPapers1915}, demonstrate that spacetime is a dynamic fabric, 
its geometry shaped through the gravity of the matter within it, which in turn influences the motion of matter. 
The EPP's concept of a Contextual Relativity that is both influenced by and influences the entities within 
is a direct analogue of this profound physical insight.

\textbf{Pearl's SCM}

Judea Pearl, with his SCM, established the foundation upon which the entire field of computational work was subsequently built. His foundational work in "Causality: Models, Reasoning, and Inference" \cite{pearl2000causality} was as influential as his foundational critique in "Theoretical Impediments to Machine Learning with Seven Sparks from the Causal Revolution" \cite{pearl2018theoretical}. In particular, his contribution to the algorithmization of counterfactuals has proven instrumental for the development of contextual counterfactuals.

\textbf{Bareinboim's Transportability of Causal Effects}

Bareinboim's calculus of transportability \cite{bareinboim2012transportability} and his subsequent work on data fusion formalize the very problem of contextual variance that the EPP's explicit context is designed to manage at a computational level. Where Bareinboim provides the definitive logical framework for reasoning about moving causality between discrete contexts, the EPP provides the computational primitive, the dynamic, queryable, and multi-modal Context, to operationalize this reasoning as part of the EPP.

\textbf{Forbus's Defeasible Deontic Calculus}

Kenneth Forbus's work on formalizing deontic calculus \cite{olson2024DDIC} has proven invaluable to solve the complex topic of conflicting norms in the effect ethos. In practice, it is rarely possible to write conflict-free norms, and during the development of the effect ethos, a recurring theme was the acceptance of this reality. The subsequent search for a solution led to the adoption of Forbus's Defeasible Deontic Calculus as the primary means to resolve normative conflicts.

\textbf{Russel's Critique on Causality}

Bertrand Russell critique on causality  formulated in his 1912 essay ``On the Notion of Cause''\cite{RussellOnCause} directly lead to the realization that it's not necessarily causality that is central to Russels objection, but the underlying assumption of time asymmetry that is at odds with his observation that most successful theories in physics are based on time symmetry. Russel hinted at a profound truth because while physics routinely models dynamic change in complex systems, computational causality consistently struggles with capturing dynamic causality.  While there is truth to causal invariance, there is also the reality that dynamic systems emit different causal structures depending on dynamic change and that is where computational causality is fundamentally at odds with physics and to an extent with reality. From there, it became clear that for causality to handle dynamics requires a new foundation of causality itself.

\textbf{Hardy's Causaloid}

Lucian Hardy introduced the "causaloid,"\cite{HardyDynamicCausalStructure} a concept that encapsulates a spatial region and the causal connections within 
as a foundation for his work on finding a theory of Quantum Gravity. Critically, unlike all prior forms of causality, 
Hardy's causaloid is spacetime-agnostic because it folds cause and effect into one entity and thus removes the need for temporal order. His seminal work "Probability Theories with Dynamic Causal Structure" \cite{hardy2005probability} had a three-fold impact.
First, during the foundational work of lifting causality into geometric structures, his causaloid formalism has
proven instrumental in the formation of isomorphic recursive causal data structures. 
Second, his insight that the causaloid formalism puts deterministic and probabilistic structures on equal footing directly led to the multi-modal reasoning of the EPP. 
Third, his demonstration that fundamental differences of theoretical foundations are contained in a causaloid directly informed the representation of causal relations 
as a causal function, which then resulted in the single axiomatic formulation of the EPP.

\textbf{Bornholdt's Uncertain Type}

Bornholdt's et al. contribution in "Uncertain< T >: A First-Order Type for Uncertain Data" \cite{bornholt2014uncertain} directly informed the unification of deterministic and probabilistic reasoning in the EPP. Instead of representing a value with a single number, the Uncertain type in the EPP represents a probability with a full probability distribution (e.g., Normal, Bernoulli) or even a complex computation graph that produces a distribution. More importantly, the EPP reasoning logic can seamlessly "lift" simpler Deterministic and Probabilistic effects into the Uncertain distribution, aggregate all distributions, and then infer a logical combination of all the input distributions without loss of information. The final output is an Uncertain<bool> that collapses rich uncertainties into a single value at the very last moment while preserving second-order properties such as the standard deviation or confidence level. As a result, decisions, and crucially, deontic reasoning become more robust under uncertainty.


\subsection{Contribution}

The Effect Propagation Process contribution spans three distinctive areas. First and foremost a single axiomatic foundation  formally derives the entire EPP from first principles. Second, the EPP establishes dynamic and emergent causal structures and an accompanying computable ethics framework for verifiable safety of dynamic causal systems. Third, a reference implementation in Rust demonstrates the feasibility of dynamic causality.   


\textbf{A Single Axiomatic Foundation of Dynamic Causality}

The EPP's single axiom, causality as a spacetime-agnostic functional dependency generalizes causality so that the classical definition becomes a special case of it. Section \ref{sec:philosophy} establishes the underlying philosophical foundation of the EPP. The details of the single axiomatic foundation are established in section \ref{sec:epp_definition}, the implications of the generalization are discussed throughout section \ref{sec:metaphysics} and \ref{sec:teleology}, and the formalization substantiates the EPP in section \ref{sec:formalization} further. The generalization is further substantiated in section \ref{sec:epp_subsumption_classical} where five established methods of classical causality are expressed through the EPP  and in Appendix A with a formal proof that the EPP subsumes the SCM. The SCM was chosen for the proof because of its historical and foundational significance to the field of computational causality. The reframing of causality as a single axiomatic functional dependency establishes explicit context and the causaloid as first-class structures. On important conjecture that follows from the axiom of functional dependency is that the field of function theory becomes applicable to causality as derived in section \ref{sub:epp_general_def_causality} and further discussed in section \ref{sec:future_work_functional_causality}. The impact of the new foundation results in a single, coherent language to model advanced dynamic causality in physics, software, finance, and system biology with equal rigor.

\textbf{A Formalism for Dynamic and Emergent Causal Structures}


The EPP is designed form the ground up to model meta-causality, or the causality of causal change. The EPP defines an operational Generative Function is a function that can modify the causal model, its context, or both dynamically to describe causal emergence. Section \ref{sec:metaphysics} establishes the foundation of causal emergence, the higher-order implications are discussed in section \ref{sec:teleology}, and section \ref{sec:formalization_epp} establishes the formalization. The impact of causal emergence enables systems that can generate novel causal strategies in response to unforeseen circumstances. When safeguarded by the Effect Ethos, this provides a path to creating robust and resilient systems that can safely navigate a dynamically changing world.

\textbf{A Computable Ethics Framework for Verifiable Safety}

Traditional causal models are overwhelmingly focused on passive inference and prediction. They provide a powerful way to answer "what if" questions but lack a formal, verifiable, and safe bridge from that knowledge to taking action. Safety, ethics, and alignment are treated as post-hoc problems to be solved with testing and ad-hoc rules.

The EPP is the first computational causality framework to treat verifiable safety as first-class formal primitives directly integrated into the foundation itself. The motivation for the Effect Ethos is rooted in safeguarding causal emergence, and the EPP achieves this with two mechanisms:

\begin{itemize}
\item The Causal State Machine (CSM):  Links a causal inference to a deterministic action.
\item The Effect Ethos: The Effect Ethos allows a system to formally verify that a proposed action is compliant with its mission and safety objectives before it is executed.
\end{itemize}

The impact of the Effect Ethos results in a feasible roadmap for building trustworthy, high-stakes autonomous systems that adhere to contextual rules of engagement encoded in an immutable ethos. The effect ethos inverts the entire verification and validation model by shifting the safety objectives of a system into the pre-design stage to convert existing rules of engagement into a testable effect ethos, which enables design-time safety and provable alignment.

\textbf{A Reference Implementation in Rust}

The presented Effect Propagation Process has been fully implemented in the open\-source DeepCausality\footnote{\url{www.deepcausality.com}} project hosted at the Linux Foundation for Data \& AI. The reference implementation contributes three sub-projects. UltraGraph, a high performance, two-phase hyper-graph data structure used in the DeepCausality project. Uncertain, an implementation of Bornholdt's Uncertain Type in Rust. DeepCausality, the reference implementation of the EPP. The five established methods of classical causality expressed through the EPP in section \ref{sec:epp_subsumption_classical} are demonstrated as code examples\footnote{\url{https://github.com/deepcausality-rs/deep_causality/tree/main/examples}} in the DeepCausality project. At the time of writing, the DeepCausality mono-repository has reached a total of fifty thousand lines of Rust code with a sustained three months average test coverage\footnote{\url{https://app.codecov.io/gh/deepcausality-rs/deep_causality}} of 95\%. 


\subsection{Structure}

The presented monograph can be read from multiple angles. For the reader with a background in Philosophy, a natural path is to begin with the philosophy of causality (Chapter \ref{}) and the overview of the EPP (Chapter \ref{sec:epp}), and then delve into its philosophical foundation: the Metaphysics (Chapter \ref{sec:metaphysics}), the Epistemology (Chapter \ref{sec:epp_epistemology}), and the Teleology (Chapter \ref{sec:teleology}).

For the reader with a background in Formal Methods, the EPP's core concepts are in Chapter \ref{sec:epp}), then the formal Ontology (Chapter \ref{sec:epp_ontology}), the Formalization of the meta-calculus (Chapter \ref{sec:formalization}), and its soundness in Chapter \ref{sec:validity}.

A reader with a background in Engineering may start with the Motivation (Chapter \ref{sec:motivation}), understand the high-level EPP framework (Chapter \ref{sec:epp}), and then see how these concepts are applied in the DeepCausality Implementation (Chapter \ref{sec:implementation}). The Metaphysics in (Chapter \ref{sec:metaphysics}) provides details about the orthogonal design used throughout the implementation.

The reader with a background in Computational Causality may start with the critique of classical models in the Motivation (Chapter \ref{sec:motivation}) and the Related Work (Chapter \ref{sec:related_work}) is a good start. Following the EPP's core concepts (Chapter \ref{sec:epp}), the next step is the Formalization (Chapter \ref{sec:formalization}) and validity (Chapter \ref{sec:validity}). From there, the Future Work (Chapter \ref{sec:future_work}) will be of particular interest.

\newpage


%% History
\section{Philosophy of Causality}
\label{sec:philosophy}

\subsection{The Classical Philosophy of Causality}
\label{sec:philosophy_foundation}


\subsubsection{Plato}
\label{sec:history_plato}

Plato is believed to be the first to have explored the cause in a systematic way, most notably in his dialogue Timaeus\cite{archer1888timaeus} (c. 360 BC). In this work, Plato explains the creation of the cosmos through the actions of a divine craftsman, the Demiurge. The Demiurge looks to the eternal Forms as a model to bring order to the pre-existing, chaotic matter. Plato’s concept of causality is thus tied to his broader metaphysical theory of Forms, where true causes are the eternal patterns or blueprints of which the physical world is merely a copy. He identifies several contributing factors necessary for the creation of the world, including the Demiurge (the efficient cause), the Forms (the formal cause), and the Receptacle (the material space)\cite{archer1888timaeus}. 

\subsubsection{Aristotle}
\label{sec:history_aristotle}

A student of Plato, Aristotle (c. 350 BC) formalized the notion of causality in his Metaphysics\cite{heidegger1995aristotleMetaphysics} with the ``Four Causes''\cite{falcon2006aristotlecausality}:

\begin{enumerate}
    \item The material cause or that which is given in reply to the question, ``What is it made out of?''
    \item The formal cause or that which is given in reply to the question, ``What is it?''. What is singled out in the answer is the essence of the what-it-is-to-be something.
    \item The efficient cause or that which is given in reply to the question, ``Where does change (or motion) come from?''. What is singled out in the answer is the whence of change (or motion).
    \item The final cause, the end purpose, is given in reply to the question, ``What is its good?''. What is singled out in the answer is that for the sake of which something is done or takes place.
\end{enumerate}

Aristotle’s framework provided a comprehensive vocabulary for analyzing causality that became foundational to Western scientific and philosophical thought for nearly two millennia.

\subsubsection{Seneca}
\label{sec:history_seneca}

Seneca (c. 56 AD) argues in Letter 65\cite{SenecaLetters} that cause and effect operate within a stage (space) and follow an order (time). Remove the stage or the order, and the conventional understanding of 'making something' or 'causing something' breaks down. His argument highlights time and space as indispensable prerequisites for classical causality. His focus on space and time as necessary conditions served as a precursor for physical concepts that treat spacetime as a background for causal processes.

\subsection{The Realist View: Discovering Causal Laws}
\label{ssec:philosophy_realist}

The realist tradition posits that causal relationships are objective, structural features of the world to be discovered.

\subsubsection{Gottfried Wilhelm Leibniz}
\label{sec:history_leibniz}

The idea of space and time as a background for causality, however, did not remain unchallenged. Gottfried Wilhelm Leibniz (1646--1716) rejected the concept of absolute space and absolute time as independent, fundamental constructs. Instead, Leibniz proposed\cite{LeibnizPhysicsSEP} a relational view in which space is the simultaneous relation of coexisting things and time is the relational order of successive things. Through rigorous first principles analysis, Leibniz argued that the concept of absolute space and time was logically untenable. His relational perspective offered a significant alternative to the preeminent Newtonian worldview of his time.

\subsubsection{John Stuart Mill}
\label{sec:history_stuart_mill}

John Stuart Mill (1806-1873) shifted the focus of causality from metaphysical inquiry to a more practical, methodological problem. In his influential work, \textit{A System of Logic, Ratiocinative and Inductive}\cite{mill2023system}, Mill developed a set of principles, now known as "Mill's Methods," to serve as a guide for discovering causal connections through empirical observation. These methods are designed to systematically eliminate non-causal factors and isolate true causes. Mill's work provided a formal basis for the inductive reasoning that underpins modern experimental science.


\subsubsection{Albert Einstein}
\label{sec:history_einstein}

Albert Einstein (1879--1955) departed from Newtonian physics with his theory of general relativity\cite{EinsteinPapers1915} (GR), in which he established that space and time are one manifold, spacetime, that is bent by the gravitational influence of large masses. General relativity preserves the prerequisite of a spatiotemporal context for causality, echoing Seneca's key insight. However, the notion of a dynamic spacetime requires a dynamic view of causality to fit into the dynamic spacetime manifold.

\subsection{The Empiricist View: Constructing Causal Models}
\label{ssec:philosophy_empiricist}

The empiricist tradition, born from a deep skepticism, argues that causality is not a directly observable force. Instead, it is a mental model that we, as observers, construct to make sense of the statistical regularities we perceive in the world.


\subsubsection{David Hume}
\label{sec:history_hume}

The Scottish philosopher David Hume (1711–1776) launched a more fundamental attack on the notion of causality itself. In his A Treatise of Human Nature\cite{hume2000treatise}, Hume argued that we can never have direct empirical evidence of a necessary connection between a cause and an effect. All we can observe is their constant conjunction in space and time. The idea of "causality" as an unseen force is, for Hume, a product of the human mind develop after observing repeated patterns. Hume's skepticism decouples the idea of causality from any metaphysical necessity and re-grounds it in observational patterns\cite{hume2000treatise}.

\subsubsection{Immanuel Kant}

The German philosopher Immanuel Kant (1724 - 1804) agreed with Hume's premise that causality is not something we can empirically observe in the world. However, he disagreed with Hume's conclusion that it is merely a habit of the mind. In his Critique of Pure Reason \cite{kant2024critique}, Kant argued that causality is an a priori category of understanding. It is a fundamental, built-in "rule" that the mind itself imposes on the raw data of sensory experience to make that experience coherent and understandable. We do not learn causality from the world; rather, we cannot experience the world in an intelligible way without already presupposing the principle of cause and effect. Kant reframes causality from an objective feature of the world (the Realist view) or a mere psychological habit (Hume's view) into a fundamental component of the cognitive architecture of any rational agent. 

\subsubsection{Hans Reichenbach}

Building on Hume's empiricism, Hans Reichenbach (1891 - 19543) was among the first to propose a formal, probabilistic solution to the problem of induction. He argued that causal relationships are not deterministic laws but statistical structures. His 'Principle of the Common Cause' provided the foundational logic for inferring causal structures from statistical correlations\cite{reichenbach1991direction} that later became known as the Reichenbach's Principle used by Pearl to define the concept of a confounder in modern graphical causal models\cite{pearl2000causality}.

\subsubsection{Karl Popper}

The challenge of induction, as framed by Hume, was famously addressed by the philosopher Karl Popper  (1902 – 1994). Popper argued that causal laws can never be verified by observation, only falsified. From this perspective, a causal theory is not a description of a hidden force, but a bold conjecture—a falsifiable model—that is held to be provisionally true only as long as it survives rigorous empirical testing\cite{popper2005logic}.


\subsection{The Practical View: Causality in Physics}
\label{sec:philosophy_physics}

\subsubsection{Bertrand Russell}
\label{sec:history_Russell}

Bertrand Russell (1872--1970) observed that successful physics has its roots in sophisticated, law-based descriptions of how a system evolves dynamically. In modern physics, the focus is on the state of a system (e.g., position, velocity, field strength) and how that entire state evolves continuously and dynamically. Therefore, for Russell, the idea of classical causality, a strict happen-before relation, no longer matches the reality of modern physics. Consequently, Russell wrote in his 1912 essay ``On the Notion of Cause''\cite{RussellOnCause}:  \textit{The law of causality, [...], is a relic of a bygone age.''} 
Many modern physics laws are time-symmetric, which means that if state S1 at time t1 is related to state S2 at time t2 by a law, it is equally true that state S2 at time t2 is related to state S1 at time t1. This relationship is not a simple, linear, one-way street from a necessary ``cause'' to a dependent ``effect.'' Knowing the state at any time allows to calculate the state at any other time, past or future. Therefore, which state is the ``cause'' and which one is the ``effect'' becomes arbitrary. Russell was not opposed to causality itself; instead, his primary argument was that the traditional philosophical interpretation of causality as a fundamental, temporally asymmetric, and directed link is not what Russell observed in physics. Russell saw physics as a discipline to find, formulate, and test time-symmetric laws of dynamic change. 

\subsubsection{Causality in Quantum Physics}

Russels view of physics describing time-symmetric laws of dynamic change was taken one step further in quantum physics because there the understanding of causality evolved from a structure that required the pre-existence of space toward a dynamic generative process from which causality emerges\cite{mrini2024indefinitecausalstructurecausal}. This emergent causality does not rely on a pre-existing spacetime but is grounded in a set of underlying rules (i.e., conceptualized as a 'generating function') that determines the fundamental materialization of spatiotemporal properties. The conceptualization of this fundamental level as a ``generating function'' captures the idea of a quantum process from which the necessary condition of classical causality's spatiotemporal structure arises. It is a shift from asking, ``What causes X, given spacetime?'' to ``What process generates spacetime (and thus enables X to be caused)?''.


\subsection{The Functional View of Causality}
\label{sec:philosophy_functional}

The preceding philosophical traditions, from the Realists to the Empiricists, share a common language rooted in the concepts of objects, causes, and effects. Their view differ in whether these concepts correspond to objective reality or are mental constructs, but the concepts of causality themselves remain the same. The practical view from physics, as articulated by Bertrand Russell, suggests that there might be a third view of causality. Russel correctly observed that the lack of time symmetry disqualifies causality from its application to modern physics that models dynamic change. And indeed, causality does struggle with complex dynamic because of the inherent assumption of linear and asymmetric time.  The author agrees with the premise and therefore generalizes causality as a spacetime agnostic functional dependency.   

This functional view of causality is formalized in the EPP's single axiom from which the entire framework is derived:
\begin{quotation}
\textbf{$E_{2} = f(E_{1})$}
\end{quotation}

The functional view redefines causality in the most general terms, the relationship between an input effect (E1) and an output effect (E2) via a transformation denoted as a function f(). The full reasoning behind the functional view and that, in fact, it subsumes the classical definition of causality is elaborates in section \ref{sec:epp}. From a philosophical perspective, the functional view of causality lead to a cascade of consequences:

\begin{enumerate}
	\item Inherently Agnostic: The nature of the function f and the effects E are unconstrained and thus agnostic. 
	\item Inherently Computable: Functions compute and thus can be programmed.
	\item Inherently Composable: Category theory is well established for functional caomposition. 
	\item Inherently Dynamic: Functions and higher order functions enable dynamics. 
	\item Inherently Formalized: Function theory has been studied for over a century and is well understood.
\end{enumerate}

The realistic view of causality as an objective reality remains valid and yet compatible with the functional view because, from
a realistic perspective, a function is nothing more than an objective, structural feature of the world. 
The empiricist view of causality as a model inferred from observed data also remains valid because, from the empirical perspective, the function is nothing more than the formal representation of the causal model derived from data.
The practical view, from physics, however, stands to gain because here the function can represent an arbitrarily complex and dynamic physics process without any assumption of time. Within the EPP, one causal function might be deterministic, another one probabilistic, and yet another one describing a complex physical model using differential equations and yet all can be expressed using the functional foundation of the EPP. 

The rest of the presented monograph explores the higher order effects that emerge from the functional view of causality. It begins with a review of the existing work in computational causality in section \ref{sec:related_work} and proceeds in section \ref{sec:motivation} with establishing a practical motivation for dynamic causality. The main contribution starts with rebuilding causality from first principles on a new functional foundation in section \ref{sec:epp}. Because of the introduced dynamics, a metaphysics discusses the necessary modalities to structure dynamic causality in section \ref{sec:metaphysics}. Dynamic causality, however introduces a set of profound challenges that are discussed in the subsequent epistemology (section \ref{sec:epp_epistemology}), Teleology (section \ref{sec:teleology}) and Ontology (section \ref{sec:epp_ontology}). Section \ref{sec:formalization} then formalizes the effect propagation process that is derived from its single axiomatic foundation. 

\newpage


%% Related Work
\section{Related Work}
\label{sec:related_work}

The study of causality and causal inference aims to distinguish genuine cause-and-effect relationships from mere associations. Traditionally, establishing causality often relied on carefully controlled randomized controlled trials. However, significant theoretical advancements have shown that causal knowledge can be inferred from observational data by examining patterns of conditional independence among variables, given explicit assumptions \cite{pearl2018theoretical}.


\subsection{Foundational Theories of Causal Inference}
\label{subsec:foundational_theories}

The endeavor to formalize and compute causal relationships draws upon several influential theoretical frameworks. Understanding these foundations is crucial for situating contemporary advancements and appreciating the nuances of different approaches to causal reasoning.

\subsubsection{Directed Acyclic Graphs (DAGs)}

A foundational framework for representing causal structures is based on graphical causal models, most notably Directed Acyclic Graphs (DAGs) \cite{verma1986causal, pearl1988probabilistic, Glymour2019Review, koller2009probabilistic}. In these models, variables are typically represented by nodes, and directed edges indicate direct causal influences \cite{pearl1988probabilistic}. The impact of interventions, conceptualized by operators like the \texttt{do}-operator which sets a variable's value independently of its usual causes, can be analyzed within this framework to predict outcomes under hypothetical scenarios \cite{pearl1988probabilistic, Pearl2009Causality}. The theoretical underpinnings of Structural Causal Models (SCMs), which are closely related to graphical models, have been extensively studied \cite{pearl2000causality, Peters2017Elements, bareinboim2020causal, janzing2016algorithmic, Peters2022Causal}. Methods exist for handling complex scenarios, including incorporating latent variables \cite{Mohan2021Graphical, richardson2003causal} and understanding the relationship between different causal models \cite{Verma1990Equivalence, pearl2018theoretical}. Policy interventions in specific graphical structures, such as Lauritzen-Wermuth-Freydenburg (LWF) latent-variable chain graphs, have also been investigated \cite{sherman2020general}. This includes work providing a novel identification result for effects of policy interventions in these graphs \cite{sherman2020general}.

\subsubsection{Structural Causal Model (SCM)}

The dominant paradigm in modern computational causality is arguably the Structural Causal Model (SCM) framework, extensively developed by Judea Pearl and his colleagues \cite{Pearl2009Causality}. An SCM consists of a set of variables, some of which are designated as exogenous (external, uncaused within the model) and others as endogenous (their values are determined by other variables within the model). The relationships between these variables are represented by a set of structural equations, typically of the form $X_i = f_i(\mathbf{PA}_i, U_i)$, where \(\mathbf{PA}_i\) are the direct causal parents of $X_i$ in the associated causal graph, and $U_i$ are exogenous error terms representing unmodeled influences or inherent stochasticity. These structural equations are considered to represent autonomous, invariant causal mechanisms. Graphically, SCMs are typically depicted using {Directed Acyclic Graphs (DAGs), where nodes represent variables and directed edges $X_j \to X_i$ indicate that $X_j$ is a direct cause of $X_i$ (i.e., $X_j \in \mathbf{PA}_i$). This graphical representation provides an intuitive way to encode causal assumptions and to determine statistical independencies via the criterion of \textit{d-separation}.

A cornerstone of Pearl's framework is the \textit{do-calculus}, a set of three axiomatic rules that allows for the inference of the effects of interventions from a combination of observational data and the causal graph structure, even when direct experimentation is not possible \cite{Pearl2009Causality}. An intervention, denoted $do(X_j=x_j')$, represents an external manipulation that sets the variable $X_j$ to a specific value $x_j'$, thereby severing the links from its original parents $\mathbf{PA}_j$ and altering the system's natural dynamics. The ability to calculate post-intervention distributions, $P(Y | do(X=x))$, is central to predicting the consequences of actions and policies. Bayesian Networks, which are DAGs coupled with conditional probability distributions $P(X_i | \mathbf{PA}_i)$, are closely related to SCMs and are often used to represent the observational probability distribution $P(\mathbf{X})$ entailed by an SCM under specific assumptions about the error terms $U_i$. They provide a powerful tool for probabilistic inference under passive observation, but require the \textit{do-calculus} or similar interventional logic to reason about causal effects.

\newpage

\subsubsection{Counterfactuals: What if?}
\label{subsec:counterfactuals}

While discovering causal structures and predicting the effects of interventions(``What if we do $X=x$?'') are fundamental tasks in causal inference, the ability to compute counterfactual queries represents a deeper and often more insightful level of causal reasoning \cite{Pearl2009Causality}. Counterfactuals address questions about alternative realities or ``what might have been''(``What if $X$ had been $x'$, given that we observed $X=x$ and $Y=y$?''). This form of reasoning is crucial for tasks such as understanding individual responsibility, learning from past mistakes, diagnosing failures, and fine-tuning policies. It requires moving beyond population-level effects of interventions to consider specific individuals or units in specific factual circumstances \cite{Pearl2009Causality, Balke1994Probabilistic}.

Within Pearl's Structural Causal Model (SCM) framework, computing a counterfactual, denoted as $Y_x(u)$ (the value $Y$ would have taken in unit $u$ had $X$ been $x$), involves a three-step algorithmic process \cite{Pearl2009Causality}:
\begin{enumerate}
    \item \textbf{Abduction:} Use the available factual evidence (e.g., observed values of some variables) to update the probability distribution over the exogenous variables $U$. This step accounts for the specific unit or situation under consideration by inferring the background conditions consistent with the observed facts.
    \item \textbf{Action:} Modify the original SCM by replacing the structural equation for the counterfactual antecedent $X$ with $X=x'$(the hypothetical condition), effectively performing a ``mini-surgery'' on the model as in the \textit{do}-calculus. The equations for other variables remain unchanged, reflecting the principle that interventions only alter the targeted mechanism directly.
    \item \textbf{Prediction:} Compute the probability of the counterfactual consequent $Y$ using the modified model and the updated distribution of $U$ (from the abduction step). This yields the probability $P(Y_{x'} = y' | \text{evidence})$.
\end{enumerate}

This process allows for a principled way to reason about hypothetical scenarios that differ from what was actually observed, effectively comparing parallel possible worlds \cite{Pearl2009Causality, Morgan2015Counterfactuals}. Foundational work also explored the bounding and identification of specific types of counterfactual queries related to probabilities of causation \cite{tian2000probabilities}. Formal systems like Pearl's \textit{do}-calculus provide tools for determining if causal effects under intervention are identifiable from observational data \cite{tian2002identification}, and algorithms exist to automate this process \cite{shpitser2012efficient}, which are often prerequisite steps before full counterfactual queries can be comprehensively addressed.

The extension of these concepts to practical applications and more complex settings remains an active area of research. For instance, model-agnostic approaches aim to enable counterfactual reasoning without full specification of the SCM, particularly in dynamic environments where systems evolve over time. Furthermore, the domain of causal bandits, which focuses on online decision-making and learning under uncertainty, increasingly incorporates causal background knowledge and aspects of counterfactual reasoning to optimize sequences of actions and learn policies more efficiently than purely correlational reinforcement learning approaches \cite{Lattimore2016Causal, Lee2018Structural, Zhang2022Causal, Bilodeau2022Adaptively}. The capacity for counterfactual reasoning thus forms a critical component of advanced intelligent systems that can not only predict and act, but also reflect, learn, and adapt based on a deep understanding of cause and effect in alternative scenarios.

More recent work explores model-agnostic approaches to counterfactual reasoning, particularly in dynamic environments \cite{Berrevoets2021ModelAgnostic}, and investigates optimizing treatment effects in such settings \cite{Berrevoets2022Treatment}. Causal bandits also incorporate causal background knowledge into online decision-making problems \cite{Lattimore2016Causal, Lee2018Structural, Zhang2022Causal, Bilodeau2022Adaptively}.


\subsubsection{Potential Outcomes}
\label{subsec:potential_outcomes}

Alongside SCMs, Potential Outcomes Framework, also known as the Rubin Causal Model (RCM) \cite{holland1986statistics}, offers another rigorous foundation for causal inference, with early conceptualizations by Neyman \cite{splawa1990application} and formally developed for observational studies by Rubin \cite{rubin1974estimating}. It has been particularly influential in statistics, econometrics, and the social sciences. This framework defines the causal effect of a treatment (or exposure) on an individual unit by considering the potential outcomes that unit would exhibit under different treatment assignments. For a binary treatment $T \in \{0,1\}$, each unit $i$ is conceptualized as having two potential outcomes: $Y_i(1)$, the outcome if unit $i$ receives the treatment, and $Y_i(0)$, the outcome if unit $i$ receives the control. The individual treatment effect (ITE) is then $Y_i(1) - Y_i(0)$. A core challenge, often termed the ``fundamental problem of causal inference,'' is that only one of these potential outcomes can be observed for any given unit \cite{holland1986statistics}.

Inference in this framework hinges on crucial assumptions, such as the Stable Unit Treatment Value Assumption (SUTVA), which posits no interference between units and well-defined treatment versions. When all confounders are believed to be observed, the key assumption is \textbf{ignorability} (or unconfoundedness), which states that treatment assignment is independent of potential outcomes, conditional on the observed covariates \cite{rosenbaum1983central}. Under this assumption, causal effects can be estimated using methods like matching, stratification, or inverse probability weighting based on propensity scores \cite{rosenbaum1983central}.

In many settings, the belief that all confounders have been measured is not plausible. To address this, applied researchers have developed a powerful toolkit of \textbf{quasi-experimental methods} that can enable causal identification even in the presence of unobserved confounding. These methods are cornerstones of modern computational causality and include:
\begin{itemize}
    \item \textbf{Instrumental Variables (IV):} This technique is used to handle unobserved confounding by leveraging an ``instrument'': a variable that is correlated with the treatment but is not causally related to the outcome except through its effect on the treatment. The instrument provides a source of exogenous variation in the treatment, allowing for the estimation of a causal effect that is not biased by the unobserved confounders \cite{angrist1996identification}.
    \item \textbf{Difference-in-Differences (DiD):} Leveraging panel data (observations of the same units over time), DiD estimates the effect of a treatment by comparing the change in the outcome for a treated group before and after an intervention to the change in the outcome for an untreated group over the same time period. This method controls for unobserved confounders that are constant over time by differencing them out\cite{callaway2021difference}.
    \item \textbf{Regression Discontinuity Design (RDD):} RDD is applicable when the treatment is assigned based on a sharp cutoff in a continuous variable (the ``running variable''). By comparing the outcomes of units just below the cutoff to those just above it, RDD can provide an unbiased estimate of the local causal effect at the threshold, mimicking a randomized experiment in a narrow window around the cutoff \cite{imbens2008regression}.
\end{itemize}

While SCMs provide an explicit language for encoding causal mechanisms, the Potential Outcomes framework, supported by these robust quasi-experimental methods, provides a powerful applied toolkit for estimating causal effects from complex observational data.

\subsection{Invariant Prediction and Out-of-Distribution Generalization}

A central challenge in modern machine learning remains robust out-of-distribution (OOD) generalization, as models trained under the standard I.I.D. assumption often fail when deployed in new or shifting environments. A powerful approach to this problem is rooted in the causal principle of invariance: the idea that while statistical correlations can be spurious and brittle, true causal mechanisms remain stable across different contexts \cite{pearl2000causality}.

This principle has been operationalized into a formal framework for both causal discovery and robust prediction. The seminal work in this area is Invariant Causal Prediction (ICP), developed by Peters, Bühlmann, and Meinshausen \cite{peters2016invariant}. The ICP framework leverages data from multiple distinct ``environments'' or ``settings.'' It posits that a set of variables constitutes the direct causes of a target if and only if the conditional distribution of the target given those variables remains invariant across all environments. By searching for a set of predictors that yields such a stable predictive model, ICP can identify causal relationships and produce a model that is robust to the types of distributional shifts observed during training.

This idea has been influential in the deep learning community, inspiring methods aimed at learning invariant representations. A prominent example is Invariant Risk Minimization (IRM), which seeks to learn a data representation such that the optimal classifier on top of that representation is the same for every training environment \cite{arjovsky2019invariant}. The goal is to isolate invariant causal features from spurious, environment-specific correlations. Other related approaches, such as Risk Extrapolation (REx) \cite{krueger2021out}, also aim to improve OOD performance by enforcing penalties on models whose performance is unstable across environments. Collectively, this body of work formalizes the intuition that a model based on causal structure should generalize better than one that merely interpolates the training data, representing a major school of thought in building more reliable and robust machine learning systems.

\newpage

\subsection{Causal Discovery}

A central task in the field is causal discovery, which focuses on learning the causal structure, represented by the graph, from observed data alone. A comprehensive survey categorizes existing methods for causal discovery on both independent and identically distributed (I.I.D.) data and time series data, including approaches for both types of data. According to this survey, categories include Constraint-based, Score-based, FCM-based, Hybrid-based, Continuous-Optimization-based, or Prior-Knowledge-based. Constraint-based methods infer relationships by testing for conditional independencies in the data \cite{Glymour2019Review, Spirtes2000Causation, Eberhardt2017Introduction}. Score-based methods search over potential graph structures and evaluate them based on how well they fit the data, often including a penalty for complexity \cite{Chickering2002Optimal}. The KGS method \cite{hasan2022kcrl}, for example, leverages prior causal information such as the presence or absence of a causal edge to guide a greedy score-based causal discovery process towards a more restricted and accurate search space. It demonstrates how incorporating different types of edge constraints can enhance both accuracy and runtime for graph discovery and candidate scoring, concluding that any type of edge information is useful. This method relates to the KCRL framework \cite{hasan2022kcrl}. Continuous optimization techniques formulate causal discovery as an optimization problem, potentially involving differentiable approaches that can handle constraints like acyclicity. The NOTEARS framework is one such example \cite{Zheng2018Dags}, and studies have analyzed its performance and proposed post-processing algorithms to enhance its precision and efficiency. A study provides an in-depth analysis of the NOTEARS framework for causal structure learning, proposing a local search post-processing algorithm that significantly increased the precision of NOTEARS and other algorithms \cite{hasan2024surveycausaldicovery}.


\subsection{Causal Inference and Discovery for (Hyper)graphs}
\label{subsec:causal_graphs_hypergraphs}

The representation of causal relationships via graphical models, predominantly Directed Acyclic Graphs (DAGs) as foundational to Structural Causal Models (SCMs) \cite{Pearl2009Causality}, is a cornerstone of computational causality. Much research has focused on discovering these graph structures from observational or interventional data (causal discovery) and subsequently estimating causal effects based on the identified graph (causal inference). While traditional methods often assume simpler pairwise relationships, the inherent complexity of many real-world systems necessitates considering more intricate relational structures.

Recent work has begun to explicitly tackle causal inference in settings involving multi-way interactions best represented by hypergraphs. Ma et al. \cite{ma2022learning} directly address the problem of estimating Individual Treatment Effects (ITE) on hypergraphs, specifically accounting for high-order interference where group interactions (modeled by hyperedges) influence individual outcomes. Their proposed HyperSCI framework leverages hypergraph neural networks to model these spillover effects and uses representation learning to control for confounders, demonstrating the utility of explicitly considering hypergraph topology for ITE estimation from observational data. This represents a significant step beyond assuming only pairwise interference, which is common in ordinary graph-based causal inference. While the work by Ma et al. focuses on statistical ITE estimation on a \textit{given} hypergraph, it highlights the increasing recognition of hypergraph structures as vital for certain causal problems.

The broader field of graph mining and network science also provides a rich backdrop, with techniques for link prediction and understanding influence spread, though these often operate at a correlational level rather than a strictly causal one. The challenge remains to bridge network science concepts with formal causal reasoning in these complex relational systems.


\subsection{Causal Inference for Time Series}
\label{subsec:causal_timeseries}

Causal inference for time series data introduces a unique set of challenges and opportunities compared to static, cross-sectional settings. The inherent temporal ordering of observations provides strong, intuitive information about potential causal directionality---causes generally precede their effects---but also necessitates methods that can handle auto-correlation, non-stationarity, feedback loops, and varying time lags in causal influences.

A foundational concept in this domain is Granger Causality, originally developed by Clive Granger for economic time series \cite{Granger1969Investigating, granger1980testing}. A time series $X_t$ is said to Granger-cause another time series $Y_t$ if past values of $X_t$ contain information that helps predict future values of $Y_t$ beyond the information already contained in past values of $Y_t$ itself. This is typically tested using vector autoregression (VAR) models and statistical tests on the coefficients of lagged variables \cite{geweke1982measurement, padav2021granger}. While widely applied, standard Granger causality is primarily about predictive improvement and may not always align with true mechanistic causation, especially in the presence of unobserved confounders, instantaneous effects, or non-linear relationships. Extensions and refinements have been developed to address some of these limitations, including non-linear Granger causality tests and methods incorporating multivariate information criteria.

To explicitly model evolving causal relationships and dependencies over time, dynamic graphical models have been developed. The Dynamic Uncertain Causality Graph (DUCG) \cite{Zhang2012Dynamic} is one such framework, specifically designed to represent and reason about causal relationships that themselves change as a system evolves. DUCGs find applications in complex dynamic systems, such as fault diagnosis in nuclear power plants where understanding the temporal progression of component failures is critical \cite{Deng2018Cubic, Hu2017Accident}. These models often aim to unify diagnostic reasoning (what caused an observed state?) with treatment or control strategies (what intervention will lead to a desired future state?) \cite{Deng2020Towards}.

More recently, deep learning techniques have been increasingly applied to causal discovery and inference in time series. For example, the Time-Series Causal Discovery Framework (TCDF) utilizes attention-based convolutional neural networks to learn causal relationships, explicitly trying to identify relevant time lags and dependencies \cite{Nauta2019Causal}. Research in this direction often focuses on challenges such as optimizing hyperparameters for these complex models, ensuring robustness to varying noise levels and non-stationarities in the data, improving the interpretability of attention mechanisms to understand which past events are deemed causally salient, and developing robust causal validation methods beyond simple predictive accuracy. Other important research avenues in time series causality include:
\begin{itemize}
    \item \textbf{Handling Unobserved Confounders:} Just as in static settings, unobserved common causes can induce spurious relationships between time series. Methods that attempt to detect or adjust for such confounding, perhaps using instrumental variable approaches adapted for time series or by searching for specific types of conditional independencies, are crucial.
    \item \textbf{State-Space Models and Causal Inference:} Integrating causal concepts with state-space models (e.g., Kalman filters and their non-linear extensions) allows for reasoning about causality between latent (unobserved) states as well as observed variables.
    \item \textbf{Interventional Time Series Analysis:} Developing methods to estimate the effect of specific interventions applied at certain points in time on the future trajectory of one or more time series. This is vital for policy evaluation and system control.
    \item \textbf{Causal Discovery from Irregularly Sampled or High-Dimensional Time Series:} Many real-world time series (e.g., medical patient data, sensor networks) are not regularly sampled or involve a very large number of variables, posing challenges for traditional methods.
    \item \textbf{Information-Theoretic Approaches:} Methods like Transfer Entropy [Schreiber, 2000, \textit{Measuring information transfer}] provide a non-parametric way to quantify directed information flow between time series, offering an alternative perspective to Granger causality, especially for detecting non-linear interactions.
\end{itemize}

The temporal dimension thus adds significant complexity but also provides a powerful constraint (time ordering) that can be leveraged for causal reasoning, making this a vibrant and critical area of ongoing research.


\subsection{The Role of Context in Causal Inference}
\label{subsec:context_temporality_causality}

While foundational causal frameworks like SCMs implicitly allow for conditioning variables, the explicit, structured, and dynamic modeling of \textit{context} as a multi-faceted entity is a growing area of focus, crucial for applying causal inference to complex, real-world systems. The Jiao et al. survey \cite{jiao2024causal} highlights numerous deep learning applications where contextual understanding is paramount, from visual commonsense reasoning to multimodal interactions, and notes the challenges posed by contextual shifts and confounders.

Berrevoets et al. \cite{berrevoets2022navigatingcausaldeeplearning, berrevoets2024causaldeeplearning} propose a conceptual ``map of causal deep learning'' (CDL) that explicitly incorporates dimensions for structural knowledge, parametric assumptions, and significantly, a temporal dimension. They argue that time is not merely another variable but introduces unique considerations in causal settings, such as the fundamental principle that causes precede effects, and the potential for feedback loops or evolving relationships in dynamic systems. Their framework aims to help researchers and practitioners categorize CDL methods based on how they handle these dimensions, including whether they operate on static data or explicitly model temporal dynamics. For instance, they differentiate models based on whether they assume ``no structure,'' ``plausible causal structures'' (often derived from statistical independencies), or a ``full causal structure'' as input, and similarly categorize parametric assumptions from non-parametric to fully known factors. The temporal axis distinguishes between static models and those designed for time-series data where variables are observed repeatedly. While this work by Berrevoets et al. primarily offers a \textit{taxonomy and conceptual guide} for the emerging field of CDL rather than a specific implemented reasoning engine, it underscores the increasing recognition of structured context, and especially temporality, as a first-class concern in bridging deep learning with robust causal inference.
This emphasis on temporal context aligns with established work in time series causality, such as Granger causality \cite{Granger1969Investigating} and dynamic graphical models like DUCGs \cite{Zhang2012Dynamic}, which inherently focus on how relationships evolve over time. However, modern approaches, including those at the intersection of deep learning and causality, seek richer representations of temporal context beyond simple lagged variables. For example, methods like TCDF \cite{Nauta2019Causal} attempt to learn relevant temporal dependencies and attention patterns. The challenge remains to develop frameworks that can uniformly reason over diverse types of contextual information (static attributes, explicit temporal sequences, spatial relationships (both Euclidean and non-Euclidean), and even abstract conceptual states) and integrate this rich contextual understanding directly into the causal reasoning process. The ability to model multiple, potentially interacting contexts, and to allow these contexts to be dynamically updated, is key to building causal AI systems that can adapt to real-world complexities.


\subsection{Hierarchical Causality}
\label{subsec:hierarchical_causality}

A significant frontier in causal inference is its extension to handle hierarchical or nested data structures, a common feature in fields from social science to biology. While hierarchical Bayesian modeling is a standard statistical tool for such data, causal modeling has traditionally forced a difficult choice: either aggregate the data to the unit level, losing valuable information, or ignore the group structure, risking incorrect inferences.
Recent work by Weinstein and Blei\cite{weinstein2024hierarchical} formalizes this problem by introducing Hierarchical Causal Models (HCMs). They extend Structural Causal Models (SCMs) by incorporating the concept of plates from graphical modeling to explicitly represent the nested structure. The central and powerful insight of their work is that disaggregating data and modeling the hierarchy can enable causal identification even in situations where it would be impossible with ``flat'' or aggregated data. For instance, with an unobserved unit-level confounder (e.g., a school's budget), the within-unit variation (e.g., student-level randomization) provides a ``natural experiment'' that can be leveraged to control for the confounder.
To operationalize Hierarchical Causal Models, they develop a systematic, nonparametric identification procedure that extends the do-calculus. The HCM framework provides a formal causal justification for many existing methods, such as fixed-effects and difference-in-difference models, which can be seen as specific parametric instances of an HCM. Their work provides a broad and rigorous toolkit for analyzing cause and effect in multi-level systems, formally connecting the principles of hierarchical modeling with the inferential power of graphical causal models.

\subsection{(Geometric) Deep Learning for Causal Inference and Representation}
\label{subsec:geometric_dl_causality}

Deep learning has achieved remarkable success in various tasks, such as neural architecture search \cite{Baker2017aDesigning, Bender2018Understanding} and techniques for handling complex relationships in data, such as those explored using hypergraphs \cite{Ouvrard2020Hypergraphs, Berge1973Graphs}. Hypergraphs, introduced by Berge in 1973 \cite{Berge1973Graphs}, can model multi-way relationships and have found applications in areas like visualization \cite{Alsallakh2016State, Jacomy2014ForceAtlas2}, partitioning \cite{Catalyurek1999Hypergraph, Devine2006Parallel, Yang2017Hypergraph}, and recommender systems \cite{Zheng2018Novel, Zhou2007Learning, Wu2018Nonnegative, Jin2015Low, Zhu2015ContentBased, Zhu2016Heterogeneous}. Link prediction, particularly in multiplex networks, is another active area where deep learning is applied \cite{Potluru2020Deeplex, Zhang2018Link}.

The intersection of deep learning with causal inference is a rapidly expanding research area, aiming to leverage the expressive power of neural networks to address challenges in causal representation learning, discovery, and effect estimation \cite{deng2022deep, jiao2024causal}. Many approaches focus on adapting deep learning architectures to better estimate treatment effects from observational data, often by learning balanced representations of covariates to mitigate confounding bias or by modeling complex response surfaces. Ramachandra \cite{ramachandra2018deep} proposes the use of deep autoencoders for generalized neighbor matching to estimate ITE, focusing on dimensionality reduction while preserving local neighborhood structure, and also suggests using Deep Neural Networks (DNNs) for improved propensity score estimation (PropensityNet). These methods exemplify the application of standard deep learning architectures to enhance specific statistical tasks within the potential outcomes framework, typically under assumptions such as the Stable Unit Treatment Value Assumption (SUTVA), which precludes interference.


A notable direction involves incorporating prior causal knowledge into deep generative models, enabling the generation of data that respects a given causal graph. While combining causal discovery with generative modeling is a goal, these methods are often constrained by the fundamental limitations of causal discovery \cite{Glymour2019Review}. Specific efforts include incorporating causal graphical prior knowledge into predictive modeling \cite{Teshima2021Incorporating} and matching learned causal effects with domain priors in neural networks \cite{Kancheti2022Matching}. Applications in finance have also utilized informed machine learning frameworks based on a priori causal graphs for prediction tasks. The area of causal reinforcement learning and causal bandits also represents significant related work in combining causality with learning agents that interact with environments \cite{Lattimore2016Causal, Lee2018Structural, Zhang2022Causal, Bilodeau2022Adaptively, Lu2020Regret, Tennenholtz2021Bandits}.

More fundamentally, researchers are exploring how geometric deep learning principles can inform the design of causal models capable of handling complex data structures and respecting informational constraints. Acciaio et al. \cite{acciaio2024designing} introduce a ``universal causal geometric DL framework,'' featuring the Geometric Hypertransformer (GHT). Their work is concerned with the universal approximation of causal maps between discrete-time path spaces, which may be non-Euclidean metric spaces such as Wasserstein spaces, while strictly respecting the forward flow of information inherent in causal processes. The GHT employs hypernetworks to adapt its parameters over time and aims to provide theoretical guarantees for approximating Hölder continuous functions between these complex spaces. Although highly theoretical and with a focus on applications in stochastic analysis and mathematical finance, this line of work signifies a deep engagement with geometric structures, non-Euclidean spaces, and transformer-like attention mechanisms within a causal learning context. Their ``geometric attention mechanism'' operating on Quantizable and Approximately Simplicial (QAS) spaces represents a sophisticated approach to handling non-Euclidean output geometries. A survey by Jiao et al. \cite{jiao2024causal} details various methods where deep learning is applied to causal discovery (e.g., leveraging neural networks for GraN-DAG or extensions of NOTEARS) or to augment specific causal inference tasks within existing deep learning modalities (e.g., developing causal attention mechanisms in computer vision, or applying causal methods to Graph Neural Networks). This body of work collectively seeks to imbue deep learning models with a degree of causal awareness or to use their representational power to overcome limitations in traditional causal inference techniques. The ongoing challenge is to move beyond enhancing specific sub-tasks towards building more integrated and principled causal deep learning architectures.

\subsection{Causal Representation Learning}

Causal Representation Learning (CRL) has emerged as a field dedicated to this problem, aiming to learn latent representations that are not just statistically useful but are also causally meaningful \cite{Scholkopf2021Toward}. The central goal is to learn a mapping from high-dimensional observations to a latent space where the dimensions correspond to independent, underlying causal factors.
Early efforts in deep learning focused on learning ``disentangled'' representations, with the hope that unsupervised methods could automatically separate data into its factors of variation. However, foundational work by Locatello et al. demonstrated that learning disentangled representations without inductive biases is theoretically impossible \cite{Locatello2019Challenging}. This finding has spurred the CRL community to propose that causal structure is the necessary inductive bias to achieve meaningful and identifiable disentanglement. The core assumption is that real-world changes often arise from interventions on a sparse set of high-level causal mechanisms. A model that captures these mechanisms in its latent space should therefore be more robust and generalize better out-of-distribution. A key technical challenge in this area is the identifiability of the latent causal variables—ensuring that the learned representation is unique and corresponds to the true underlying causal structure.

\subsection{The Causaloid Framework}

A significant effort towards establishing a framework for probabilistic theories with dynamic causal structure was presented by Hardy \cite{hardy2005probability}. Hardy proposes a framework aimed at unifying quantum theory (QT) and general relativity (GR) as a step towards quantum gravity (QG). The core of this unification lies in a generalized theory of causality, capable of describing both the probabilistic nature and fixed causal structure of QT, as well as the deterministic nature and dynamic causal structure of GR, within a single formalism. The core of this new framework of unified causality is the ``causaloid,'' a mathematical object designed to encapsulate all information about the causal relationships within a physical system. The framework begins from an operational standpoint, focusing on ``recorded data'' which consists of ``actions'' and ``observations'' associated with ``elementary regions'' of spacetime.

The causaloid itself is a theory-specific mathematical entity, primarily represented by a collection of ``lambda matrices'' (\(\Lambda\)). These matrices quantify how the complexity of describing a composite region (specifically, the number of fiducial measurements needed to determine its state) is reduced due to causal connections between its component elementary regions. Associated with any region $R$ and an experimental procedure $F_R$ resulting in outcome $X_R$ are ``r-vectors,'' denoted $r(X_R, F_R)(R)$, which are analogous to operators in QT \cite{hardy2005probability}. A key innovation is the ``causaloid product,'' which is governed by the causaloid (via the lambda matrices). This product combines r-vectors of sub-regions to form the r-vector for a composite region, e.g., $r(R_1 \cup R_2) = r(R_1) \hat{\;} r(R_2)$. This product aims to unify the different ways systems are composed in QT, such as tensor products for space-like separated systems and sequential (matrix) products for timelike evolutions. Probabilities for joint outcomes, conditioned on experimental settings, are then derived from these r-vectors. A crucial feature of the causaloid formalism is that it does not impose a fixed causal structure or a background time a priori. Instead, the causal relations are implicitly defined by the causaloid itself.

Hardy demonstrated how both classical probability theory and quantum theory can be cast within this framework, with the differences between theories being encoded entirely in the specification of their respective causaloids. The ultimate aim is to provide a structure wherein the dynamic causal aspects of GR can be consistently combined with the probabilistic nature of QT. Hardy also introduces ``causaloid diagrams'' as a visual tool to represent and compute the causaloid based on local lambda matrices for nodes (elementary regions) and links (pairwise connections), particularly under simplifying assumptions met by QT and classical probability \cite{hardy2005probability}.

\newpage

\subsection{Computational Causality Libraries}

A vibrant ecosystem of Python libraries has emerged over time, providing tools for various aspects of causal inference, discovery, and analysis. These libraries typically build upon foundational causal theories and aim to make causal methods accessible to data scientists, researchers, and engineers. This report summarizes several key libraries shaping the Python landscape for computational causality.

\subsubsection{DoWhy (Microsoft)}

Developed by Microsoft Research, DoWhy\footnote{https://github.com/py-why/dowhy} \cite{sharma2020dowhy} is perhaps one of the most well-known libraries aiming to provide an end-to-end workflow for causal inference. Its philosophy centers on explicitly separating the causal modeling assumptions from the statistical estimation steps, adhering to a four-stage process: 1) Modeling the causal assumptions (often using graphical models), 2) Identifying the target causal estimand based on the model, 3) Estimating the causal effect using appropriate statistical methods (like propensity scores, regression, instrumental variables), and 4) Refuting the obtained estimate through robustness checks. DoWhy aims to unify concepts from both Pearl's Structural Causal Models (SCMs) and the Potential Outcomes framework. It integrates with other libraries like EconML and CausalML for specific estimation tasks and is designed to be a general-purpose tool for applied causal analysis.

\subsubsection{EconML (Microsoft)}

EconML\footnote{https://github.com/py-why/EconML} \cite{oprescu2019econml} focuses specifically on estimating heterogeneous treatment effects (HTE) – understanding how the effect of an intervention or treatment varies across different individuals or subgroups. It heavily leverages machine learning techniques to model complex conditional outcome expectations and propensity scores while incorporating causal identification strategies to ensure the validity of the effect estimates. Key methodologies implemented include Double Machine Learning (DML), Orthogonal Random Forests, Deep Instrumental Variables (DeepIV), and various ``meta-learners'' (S-learner, T-learner, X-learner) that adapt standard ML models for causal effect estimation. Meanwhile, DoWhy and EconML have been brought together in the PyWhy\footnote{https://www.pywhy.org} project to foster industry wide collaboration.

\subsubsection{CausalML (Uber)}


Developed initially at Uber, CausalML\footnote{https://github.com/uber/causalml} \cite{zhao2023causal} is another library primarily focused on treatment effect estimation and, notably, uplift modeling. Uplift modeling specifically aims to estimate the incremental impact of an intervention on an individual's behavior – identifying who would be positively influenced by an action (e.g., receiving a promotion) compared to doing nothing. CausalML provides implementations of various uplift algorithms, including tree-based methods (causal trees/forests) and meta-learners similar to those in EconML. It's geared towards practical industry applications, especially in customer relationship management (CRM) and marketing, where optimizing interventions based on predicted individual uplift is a key objective.

\subsubsection{CausalNex (McKinsey)}

CausalNex\footnote{https://github.com/mckinsey/causalnex} takes a different approach, focusing more strongly on causal discovery and the use of Bayesian Networks for causal reasoning. It provides tools to learn causal graph structures from data, potentially incorporating domain knowledge to constrain the search space. It implements structure learning algorithms (like NOTEARS) and allows users to fit Bayesian Networks to the data based on the learned (or provided) graph structure. Once the network is built, users can perform queries (e.g., conditional probability queries, interventions via the do-calculus if the graph assumptions hold) to understand relationships and simulate scenarios within the modeled system. CausalNex is particularly useful for exploring and visualizing complex systems where understanding the network of causal influences is a primary goal.


\subsubsection{Meridian (Google)}

Meridian\footnote{https://github.com/google/meridian} is a marketing mix model (MMM) that uses Bayesian causal inference methods to offer better insights into online and offline marketing channels. Meridian provides methodologies to support calibration of MMM with experiments and other prior information, and to optimize target ad frequency by utilizing reach and frequency data.

\subsection{Causal Inference with Large Language Models (LLMs)}

The intersection of causal inference and Large Language Models (LLMs) has emerged as a vibrant and rapidly developing research frontier. Foundational work has mapped out the potential for using LLMs as interactive causal knowledge engines, capable of answering queries about causal relationships, interventions, and counterfactuals by drawing on the vast information embedded in their training data \cite{kiciman2023causal}. This opens up the possibility of automating parts of the causal modeling process that have traditionally been highly manual.
However, a fundamental question shadows this potential: whether LLMs, trained on vast quantities of observational and correlational text, can distinguish true causation from spurious association. Research specifically investigating this issue has shown that LLMs often struggle to infer causation correctly when faced with scenarios where correlation and causation are deliberately misaligned, highlighting a significant risk of the models simply repeating the statistical patterns in their training data \cite{jin2023can}.
To address this challenge, the research community has focused on creating structured and comprehensive benchmarks to systematically assess the causal reasoning capabilities of LLMs. For instance, the CLADDER framework was developed to generate complex, language-based causal problems from underlying structural causal models, allowing for a controlled and fine-grained analysis of model performance and failure modes \cite{jin2023cladder}. In a similar vein, CausalBench provides another comprehensive benchmark suite designed to evaluate a wide array of causal reasoning skills, from basic causal discovery to complex counterfactual inference \cite{wang2024causalbench}. Collectively, this body of work indicates that while LLMs show promise, they are not yet reliable causal reasoners.

\subsection{In-context Causal Reasoning with Large Language Models}

Recent advancements have seen the application of large-scale language models (LLMs) to tasks in causal inference, leveraging their ability for in-context learning. This emergent capability allows models to perform causal reasoning on the fly, based on the provided context, without requiring parameter updates. A notable direction of this research is in-context counterfactual reasoning. Miller, Schölkopf, and Guo\cite{miller2025counterfactualreasoninganalysisincontext} demonstrate that language models can perform counterfactual reasoning in a controlled synthetic environment. Their work suggests that for a wide variety of functions, counterfactual reasoning can be reduced to a transformation of in-context observations. The authors find that self-attention, model depth, and data diversity are key drivers of this capability in transformer architectures. This line of inquiry extends to sequential data, providing preliminary evidence for the potential of counterfactual story generation. Building on the concept of in-context learning, Schölkopf et al.\cite{robertson2025dopfnincontextlearningcausal} have proposed Do-PFN, a pre-trained foundation model designed to predict interventional outcomes from purely observational data. Their model is pre-trained on a wide variety of synthetic causal structures, which enables it to meta-learn how to perform causal inference. Do-PFN has shown strong performance in estimating causal effects, even without knowledge of the underlying causal graph, a significant departure from traditional methods that require such information.


\subsection{Causal Inference at Industry Scale}

An open challenge for the practical application of causal inference remains scalability. In enterprise environments such as Netflix, Google, and Meta, causal questions must be answered using datasets with millions or billions of observations. Causal inference at scale has exposed a significant gap between theoretical algorithms and practical feasibility due to two known limitations:
\newline
The first is causal discovery, where the goal is to learn the graph structure from data. Score-based search methods are generally NP-hard in the worst case \cite{Chickering2002Optimal}, and even faster constraint-based methods like the PC algorithm face a combinatorial explosion of required conditional independence tests in dense or high-dimensional graphs \cite{Kalisch2007Estimating}.
\newline
 The second is in causal effect estimation, especially in the presence of high-dimensional confounding. To address this, a significant body of work has emerged around Double/Debiased Machine Learning (DML) \cite{Chernozhukov2018Double}. The DML framework provides a recipe for using powerful, arbitrary machine learning models to flexibly control for a large number of confounders without introducing bias into the final treatment effect estimate. This line of research is explicitly motivated by the need to apply causal inference in settings where the number of variables makes traditional statistical methods intractable.
 In response to these challenges, a major engineering focus has been the development of causal inference platforms. These internal systems are designed to automate and scale causal analyses, enabling data scientists to run thousands of experiments and quasi-experiments reliably and efficiently.

\newpage

The study of causality and causal inference aims to distinguish genuine cause-and-effect relationships from mere associations. Traditionally, establishing causality often relied on carefully controlled randomized controlled trials. However, significant theoretical advancements have shown that causal knowledge can be inferred from observational data by examining patterns of conditional independence among variables, given explicit assumptions \cite{pearl2018theoretical}.


%% Effect Propagation Process
\section{Causality as Effect Propagation Process}
\label{sec:epp}

The foundational premise of the EPP is the detachment of causality from a presupposed spacetime. This premise necessitates a re-evaluation of the causal relation itself, shifting its conceptualization to a more general process of effect propagation. From this re-evaluation, the core architectural 
components of the EPP, the Contextual Fabric, the Causaloid, and the Causaloid Graph, are derived.


\subsection{Background}
\label{sec:epp_background}

The philosophy of the Effect Propagation Process (EPP) builds upon several important contributions in philosophy and science. 

The EPP fundamentally rejects the classical Newtonian conception of a static absolute background spacetime. Historically, the idea finds precedent in Alfred North Whitehead who argued that reality is not composed of enduring substances but of dynamic, interconnected "actual occasions." This view inspired the shift towards a dynamic effect propagation process. 

Luciano Floridi's view that the design principles for  dynamic systems require a relational paradigm was profoundly inspirational to the formalization of the Effect Propagation Process. The EPP leverages the hypergraph as its foundational structure to model rich and complex relationships across all its elements. To do so, the ontology provides the foundational principles for handling disjoint categories of knowledge, not by exhaustively modeling all content, but by defining the foundational categories (e.g., space, time, data) and the dynamics of change itself.

Next, the EPP is inspired by Einstein's theory of General Relativity, which demonstrated that spacetime is a dynamic fabric, its geometry determined by the matter within it, which in turn dictates the motion of matter. The EPP's concept of a Contextual Relativity that is both influenced by and influences the entities within is a direct metaphysical analogue of this profound physical insight. 

Physicist Lucian Hardy introduced the "causaloid," a concept that encapsulates a region of spacetime and the causal connections within it for his work on Quantum Gravity. The EPP draws direct inspiration from Hardy’s pioneering work by using the term Causaloid honoring Hardy's concept of a unified, self-contained unit of causality, though it has been adapted for a more general, computational context. F


The EPP synthesize these concepts and formalize them into  into a deeply integrated foundation for
dynamic causality. The EPP contributes a set of mechanism to operationalize spacetime agnostic contextual dynamic causality:

\begin{itemize}
	\item Generalized definition of causality as effect propagation 
	\item Externalized context
	\ 
\end{itemize}  


\subsection{Definitions}
\label{sec:epp_definition}

The notion of a generative process that underlies the fabric of spacetime leads to the implication that causality has to evolve beyond the strict “before-after” relation towards a spacetime-agnostic view. The classical definition of causality, taken from Judea Pearl's foundational work\cite{pearl2000causality}: 

\begin{quote}
    IF (cause) A then (effect) B.
    
    AND 
    
    IF NOT (cause) A, then NOT (effect) B.
\end{quote}

When removing time from causality, it is indeed no longer possible to discern cause from effect because, in the absence of time, there is no “happen-before” relation any longer, and therefore, the designation of cause or effect indeed becomes arbitrary, just as Russell hinted at earlier on. When removing space from causality, the location of a cause or effect in space is not possible anymore because space itself is no longer available. 

\newpage

The absence of spacetime raises the question: \textit{What is the essence of causality?}

Logically, the answer comes in three parts:

\begin{enumerate}
    \item Causality is a process.
    \item Causality deals with effects.
    \item Causality describes how effects propagate.
\end{enumerate}

The first one is self-explanatory because causality occurs in dynamic systems that change and therefore, causality must be a process.

The second one is less obvious, because one might think that causality is all about the “cause” that brings the effects into existence. However, let’s think the other way around: We know that X is the cause of effect E, because E happens when X happens and because E does not happen when X does not happen either. Therefore, we can describe a cause in terms of its effects. Therefore, it is true that causality deals with effects.

The third one, effect propagation, stretches the imagination and is less obvious. When we rewrite the previous definition of classical causality in terms of effect propagation, we see that there is no loss of information:

\begin{quote}
    If X happens, then its effect propagates to Effect E.

    AND
    
    If X does not happen, then its effect does not propagate to Effect E.
\end{quote}

In this definition, X does not have a designated label and instead is described in terms of its emitting effect. Therefore, X can be seen as a preceding effect, which then propagates its effect further. Therefore, causality becomes an effect propagation process. The effect propagation process definition is more general and treats the classical happen-before definition of causality as a specialized derived form. When you designate the preceding effect to be a “cause”, then you can rewrite the general definition back into the classical definition thus the general and the specialized definition of causality remain congruent. To operationalize the effect propagation process, the EPP formalizes the "effect" itself as a dedicated "PropagatingEffect".


\subsection{PropagatingEffect}
\label{sec:propagating_effect}

The output of a Causalal evaluation is the PropagatingEffect, a monoidic primitive of influence. The PropagatingEffect is a unit of influence that travels through cause to the next to transit its effect.  
It serves as a unified inference outcome across different reasoning modalities and can represent:

\begin{itemize}
	\item Deterministic effect: A definitive boolean outcome ("true/false").
	\item Probabilistic effect: A quantitative outcome, such as a probability score or an estimate.
	\item Contextual Link: A reference to a specific fact in the context. 
\end{itemize}

The Contextual Link accommodates for advanced causal reasoning via non-numerical representations by
writing a complex reasoning outcome directly into the context and then propagating the reference to
the next reasoning stage which reads the complex reasoning outcome and processes it further. 

\subsection{Context}
\label{sec:epp_context}

A key contribution of the EPP is the externalization of context as a first-class entity.
The context of a causal model is a hypergraph, that encapsulates supporting data. 
Each node in this hypergraph is a Contextoid, a unit of information that can represent:

\begin{itemize}
	\item Data
	\item Time
	\item Space
	\item Spacetime
	\item Symbol 
\end{itemize}


The causal logic is kept distinct from the contextual data it operates on. It also directly enables the  agnosticism to the structure of space and time, accommodating Euclidean, non-Euclidean, and symbolic representations within the same architecture. Furthermore,  causal logic  may operate on one or more contexts and, equally important, a particular context might be shared between different causal logic thus enable efficient and salable context representation in complex dynamic systems. Furthermore, the Symbolic context type combined with the Contextual Link establishes a foundation for a uniform integration of multi-modal causal reasoning with advanced neuro-symbolic reasoning.

\subsection{Causaloid}
\label{sec:epp_causaloid}

In the Effect Propagation Process framework, due to the detachment from a fixed spacetime, this fundamental temporal order is absent. Consequently, the entire classical concept of causality, where a cause must happen before its effect, can no longer be fundamentally established. The distinction between a definitive 'Cause' and a definitive 'Effect' becomes untenable as Russell foresaw. When the separation between cause and effect becomes untenable, then the obvious question arises: why even preserve an untenable separation?

Therefore, the Effect Propagation Process framework adopts the causaloid, a uniform entity proposed by Hardy\cite{HardyDynamicCausalStructure}, that merges the ‘cause' and 'effect' into one entity. Instead of dealing with two nearly identical concepts discernible from each other by temporal order, the causaloid is one concept that defines causality in terms of a causal function that determines its effect which then propagates to the next causaloid. without presupposing a fixed spacetime background. The nature of the causal function is not prescribed, allowing the Causaloid to encapsulate diverse logical forms, including but not imited to:

\begin{itemize}
	\item A deterministic rule (IF temp > 100).
 	\item A formal Structural Causal Model (SCM).
	\item A probabilistic estimate or Bayesian network.
	\item A specialized neural network.
\end{itemize}

For the deterministic case, the causal function takes some evidence as input, applies boolean operators (AND, OR) or comparators, and returns a boolean value as its PropagatingEffect.

For a more complex causal scenario, the causal function encapsulates a set of structural causal equations,
applies the corresponding calculus and returns a probability distribution value as its PropagatingEffect.

In case of a probabilistic estimate or a Bayesian network, the causal function implements a Conditional Probability Table (CPT) or a similar probabilistic model, applies a probabilistic calculus i.e. the chain rule of probability, and returns another probability as its PropagatingEffect. If a Causaloid receives multiple PropagatingEffects, each carrying a probability, the receiving Causaloid implements the aggregation of all probabilities. This is a deliberate architectural principle rooted in the EPP's primary role as a flexible, hybrid framework. A specialized neural network embedded into a causal function will most likely return a classification score as its PropagatingEffect. In case the output of the neural network results in a complex type, for example 
generative data, then it is sensible to write its output into the appropriate context as a contextoid and return the context and contextoid ID as the PropagatingEffect. 


It is important to note that the EPP framework adopts the conceptual role of the Causaloid as a spacetime-agnostic unit of causal interaction, inspired by Hardy’s work on Quantum Gravity, but it does not uses Hardy's formal definition that requires a complex process matrix.  Instead, the EPP formulates the Causaloid as an abstract data structure that embeds a causal function, thereby decoupling it from any particular physical theory while preserving its core philosophical utility and making it practical implementable in software.

The term "propagation" refers to the fundamental process by which an effect is transferred within the structure from one Causaloid to another. This fundamental process is what gives rise to the appearance of propagation through spacetime in the classical view. Furthermore, while classical causality relies on a definite temporal order, the Effect Propagation Process treats temporal order as an emergent property, arising from the fundamental process itself.

While the Effect Propagation Process involves the transfer of effects within the fundamental structure, it is crucial to distinguish this from mere accidental correlation. The process reflects the fundamental way the underlying structure of reality establishes dependencies between its components and how it is gives rise to the non-accidental relationships we recognize as observed causal relations.This fundamental determination, rather than simple co-occurrence, is what the "Effect Propagation Process" captures at the deepest level. The Effect Propagation Process redefines what causality means in dynamic systems. Because of its flexibility, the EPP can express static causal relationships similar to Pearl’s Causal DAG, it can handle probabilistic causal systems similar in spirit to Dynamic Bayesian Networks, but then goes further and unifies both paradigms into one that is static and dynamic, deterministic and probabilistic while remaining spacetime agnostic. 

\newpage

\subsection{Causaloid Graph}
\label{sec:epp_causaloid_graph}

Modeling real-world dynamic causal systems requires a mechanism capable of managing complexity.
Classical computational causality relies on algebra, which is rich in formalization,
but has its limits when complexity grows and thus limits scalability. The EPP adopts 
a geometric approach by expressing causal models as a hypergraph. The EPP
exchanges the arithmetic complexity of solving large equation systems for the challenge of managing structural complexity that comes from the geomtrization of causality.
The EPP addresses the structural challenge through isomorphic recursive composition 
that enable concise expression of complex causal structures. 

A causal hypergraph may contain any number of nodes with any number of relations to other nodes, with each node representing a causaloid. A causaloid uniformly represents three distinct levels of abstraction:

\begin{itemize}
 	\item Singleton Causaloid: The base case, representing a single, indivisible causal mechanism.
 	\item A Collection of Causaloids: A set of Causaloids that can be evaluated with an aggregate logic. 
 	\item A  Causaloid Graph: A node can encapsulate an entire graph
\end{itemize}

Recursive isomorphism allows to built causal models in a modular and hierarchical fashion. 
A complex sub-system can be modeled as a self-contained Causaloid Graph, then encapsulated into a single node to be used as a component in a larger, higher-level model. The causal graph enables the concise expression of deeply layered systems without sacrificing logical integrity.

The Effect Propagation Process is the operational dynamic on this graph. When triggered, Causaloids are evaluated. Their outcomes, the \textit{PropagatingEffects}, propagate along the hyperedges to other Causaloids, which in turn evaluate their own functions, thus continuing the process until the graph traversal completes and a final, reasoned inference is reached.

\subsection{The Causal State Machine}
\label{sec:epp_csm}

The causaloid and causal graph provides the mechanism for causal inference, 
but they lack the ability of intervention. The EPP addresses this through the Causal State Machine (CSM), which serves as the formal bridge between causal reasoning and deterministic intervention.

The CSM originates in Finite State Machine (FSM) in that it aims to formalize state transition.
However, a defining property of the Finite State Machine is its explicit 'Finiteness': the entire set of possible system states must be known at design time. The FSM paradigm is highly effective for closed-world problems where all conditions are predictable and known. However, the finiteness of states becomes untenable when applied to dynamic causality. The Causal State Machine generalizes the FSM and adapts it 
to operationlize interventions for dynamic causality through two mechanism:
 
\begin{itemize}
	\item A "Causal State" is an Inferred Predicate. 
	\item The "Causal Action" is a Deterministic Intervention based on the  Causal State.
\end{itemize}

In a classical FSM, a state is an identifier from a pre-defined list (e.g., "State A").
In the CSM, a Causal State is a inferential predicate defined as a specific Causaloid whose truthfulness is evaluated. The CSM does not need to know all possible states in advance. It only requires the causal logic (the Causaloids) necessary to infer whether the encoded predicate in the "Causal State" is true. 

Each Causal State is formally linked to a Causal Action. This is a deterministic, programmatic function that is executed if and only if its corresponding Causal State is inferred to be true. This action represents a real-world intervention.

The CSM is a inference-to-action state machine that is both deterministic in its execution and dynamic in its definition. The CSM is deterministic within its encoded caudal states and actions, but also dynamic in its definition as it can be extended at run-time by adding new Causal States and Actions, enabling the control logic of a system to evolve in tandem with the causal understanding of its environment. 

\newpage

\subsection{Mapping Pearl's Ladder of Causation to the EPP}
\label{sec:epp_ladder_causation}

Judea Pearl's Ladder of Causation\cite{pearl2000causality} defines three distinct levels of ability required for causal reasoning: Association, Intervention, and Counterfactuals. The EPP achieves these three rungs of the ladder by different means than the established methods of the SCM and Causal DAG.  

\textbf{Rung 1: Association}

The first rung, Association, concerns reasoning from observational data. 
It answers the question, "What is the likelihood of Y, given that we have observed X?" In classical models, this is handled by conditional probabilities i.e $P(Y|X)$.

In the EPP, association is a structured, operational process:

\begin{enumerate}
	\item The observation $(X)$ is formalized as Evidence and presented as an input to a Causaloid.
	\item The background condition $(|)$ is represented by the context, which provides the necessary supporting data for the reasoning process.
	\item The causal inference of $(Y)$ is performed by the causal function embedded within each Causaloid. This function takes the Evidence and any required information from the Context as its inputs and computes a result.
	\item The inference result is emitted as a PropagatingEffect, which then travels along the graph's hyperedges, serving as Evidence for subsequent Causaloids.
\end{enumerate}

Thus, "seeing" in the EPP is the operational dynamic where initial Evidence triggers a cascade of computations via the causal functions throughout the Causaloid Graph, leading to a final, reasoned inference.

\textbf{Rung 2: Intervention}

The second rung, Intervention, involves predicting the effects of deliberate actions. It answers the question, "What would Y be if we do X?" This is formalized in Pearl's framework by the do-operator, which simulates an intervention on the causal model itself.

The EPP provides a mechanism of intervention through the Causal State Machine (CSM). 
The CSM  links causal inferences to deterministic actions:

\begin{itemize}
	\item A Causal State is defined as an inferential predicate. It is a specific Causaloid evaluated against a specific Context.
	\item This Causal State is mapped to a Causal Action, a verifiable function that executes when its corresponding state is inferred to be true.

\end{itemize}

This Causal Action is the EPP's intervention. It is a programmatic function that changes state. 
This change is then reflected as an update to the context, creating a complete feedback loop of inference, action, and new observation.

\textbf{Rung 3: Counterfactuals}

The third rung, Counterfactuals, involves reasoning about alternative possibilities given a known outcome. It answers the retrospective question, "What would Y have been if X had been different, given that we actually observed Z?"

The EPP's architecture provides a Contextual Counterfactual mechanism, which leverages the EPP's externalization of context:

\begin{enumerate}
	\item Abduction is Context Pinning
	\item Action is Contextual Alternation
	\item Prediction is Re-evaluation over an altered context. 
\end{enumerate}

The factual observation $(Z)$ is already explicitly represented within the primary context. 
The abduction step is therefore equivalent to identifying and "pinning" this factual context.

The hypothetical premise ("if X had been different") is then established by creating a new, hypothetical context $(C_counterfactual)$ by cloning the primary context $C_factual$ and modifying the value of the relevant Contextoid.The system then executes the exact same, unmodified Causaloid Graph, but uses $C_counterfactual$ as its frame of reference. The resulting inference is the answer to the counterfactual query. 

\newpage

The EPP's mechanisms of causation differ from Structural Causal Models, 
but they fulfill the same fundamental goals of 'seeing,' 'doing,' and 'imagining' Judea Pearl established
via the ladder of causation. Table \ref{tab:ladder_comparison} summarizes the comparison of the EPP to the existing methods of computational causality.
 

\begin{table}[h!]
\centering
\caption{Comparison of Causal Ladder Implementations}
\label{tab:ladder_comparison}
\begin{tabular}{|l|p{5.5cm}|p{5.5cm}|}
\hline
\textbf{Ladder Rung} & \textbf{Pearl's Framework (SCM/DAG)} & \textbf{Effect Propagation Process (EPP)} \\
\hline
\textbf{1. Association} &
Statistical analysis; calculating conditional probability $P(Y|X)$. &
Execution of a \texttt{causal function} on \texttt{Evidence} within a factual \texttt{Context} ($C_{\text{factual}}$). \\
\hline
\textbf{2. Intervention} &
The \textit{do}-operator; surgical modification of the model's structural equations. &
Execution of a \texttt{Causal Action} by the \texttt{Causal State Machine (CSM)} in response to an inferred state. \\
\hline
\textbf{3. Counterfactuals} &
Three-step algorithm: Abduction (solving for latent variables), Action (model surgery), and Prediction. &
Three-step process: Context Pinning, Contextual Alternation, and Re-evaluation of the \textit{unmodified model} over an \textit{altered context} ($C_{\text{counterfactual}}$). \\
\hline
\end{tabular}
\end{table}

The decision to separate causal logic (the Causaloid Graph) from its data (the Context) that underpins
contextual alternation leads to some welcome properties. For example, through contextual alternation,  counter-factual reasoning becomes an "embarrassingly parallel" problem because, if 100 alternate contexts are derived, all of them are independent from each other and thus can be evaluated in parallel. 

\subsection{Mapping Dynamic Bayesian Networks to the EPP}
\label{sec:epp_Dynamic_Bayesian_Networks}

Dynamic Bayesian Network is the established framework for modeling dynamic causal systems. A DBN models a temporal process by "unrolling" a causal graph over discrete time slices, creating separate nodes for a variable at each point in time. The EPP represents this same process   by evaluating a single, static causal model over a dynamic, temporal context hypergraph. The mapping of DBN to EPP components constructs as following:
 
 \textbf{Time Modeling:}
 
 In a DBN, time is an implicit index of the temporal variables. In the EPP, time is made explicit within a
 context by representing each time slice (e.g., $X_{t-1}$, $X_t$, $X_{t+1}$) as a Tempoid, a temporal type of Contextoid, with their sequential relationship defined by hyper-edges within the context hypergraph. In the EPP data might be attached to a Tempoid as a dedicted node of a different type i.e. Datoid. 

 \textbf{State Variables:}
 
state variable in a DBN (e.g., the concept of Weather across time) corresponds to a single Causaloid in the Causaloid Graph. The Causaloid represents the variable's underlying causal mechanism and the state of "Weather at time t" is the result of evaluating the "Weather" Causaloid contextual using the Tempoid that maps to the time t. A contextual temporal graph allows a Causaloid to access uniformly multiple slices of time at different scales i.e. weekly average rainfall and today's rainfall to inform its causal logic. 

\textbf{Dependencies:}

The directed edges in a DBN can reference  within the same time slice or across different time slice.
For edges within the same time slice, (e.g., $\text{Weather}_t \to \text{Umbrella}_t$), the causal function simply references its causal logic to the one tempoid at time t in the graph. If "Weather" is a complex state object, then the causaloid may loads it from another context before evaluating the causal rule that would lead to Umbrella become true.

For edges between different time slices (e.g., $\text{Weather}_{t-1} \to \text{Weather}_t$),  are represented by the hyperedges in the Causaloid Graph by referencing two different Tempoid nodes. Practically, one would implement a dynamic temporal graph index with accessors for frequently used "current" or "previous" values. 


\textbf{Conditional Probability Tables (CPT):}

Conditional Probability Tables (CPTs): The CPT defining a variable's probability given its parents
(e.g., $P(X_t \mid Z_t, X_{t-1})$) is implemented as the causal function within the Causaloid.

\newpage

\textbf{Execution Flow:}

To compute the state of the system at time t, the causal function of a Causaloid queries the Context to determine the current time slice. It receives the PropagatingEffects from its parent Causaloids (representing their states at the appropriate time slices) and uses its internal CPT logic to compute a new probability distribution. This distribution is then emitted as a new probabilistic PropagatingEffect. This process maps the EPP to the "filtering" or "unrolling" inference of a DBN.

\begin{table}[h!]
\centering
\caption{Mapping of Dynamic Bayesian Network (DBN) Concepts to EPP}
\label{tab:dbn_mapping}
\begin{tabular}{|l|p{5.5cm}|p{5.5cm}|}
\hline
\textbf{DBN Concept} & \textbf{Description} & \textbf{EPP Architectural Counterpart} \\
\hline
\textbf{Time Slices} &
The discrete steps ($t, t+1, \dots$) over which the model is unrolled. &
A sequence of \texttt{Tempoid} nodes within the \texttt{Contextual Fabric}. \\
\hline
\textbf{State Variable} &
A variable that has a state at each time slice (e.g., $X_t$). &
A single, time-invariant \texttt{Causaloid} representing the variable's causal mechanism. \\
\hline
\textbf{Dependencies (Edges)} &
Directed edges representing causal influence within and between time slices. &
Hyperedges in the \texttt{Causaloid Graph} connecting the relevant Causaloids. \\
\hline
\textbf{Conditional Probability Table (CPT)} &
The function defining a variable's probability given its parents, $P(X_t | \text{Parents}(X_t))$. &
The specific implementation of the \texttt{causal function} within the corresponding \texttt{Causaloid}. \\
\hline
\end{tabular}
\end{table}

% \subsection{MappingGranger Causality to the EPP}
% \label{sec:epp_Granger_Causality}


\newpage


%% Metaphysics
\section{The Metaphysics of the Effect Propagation Process}
\label{sec:metaphysics}

The metaphysics of the Effect Propagation Process establishes a set of underlying principles widely used across the EPP. The three core metaphysical concepts are defined as: Monoidic Primitives, Isomorphic Recursive Composition, and Contextual Relativity. Then we derive from these first principles the metaphysics of being, dynamics, becoming, and doing. Combined, these form the foundation of orthogonal design used throughout the EPP and its implementation DeepCausality. Ontological design is a constructive, engineering-oriented form of philosophy to specify the necessary and sufficient conditions of a new system to exist and operate coherently. The EPP's metaphysics is, therefore, the blueprint for a computable reality. It establishes the axiomatic foundation upon which the ontology is built, the epistemology is derived, and the implementation is realized.

\subsection{Monoidic Primitives} 
\label{sec:metaphysics_monoidic_primitives}

A monoid is defined as an abstract algebraic structure that comprises:

\begin{enumerate}
	\item A set of elements with a certain type.
	\item A binary operation that combines any two elements of the set results in a third element of the same set.
	\item An identity element, which, when combined with any other element, leaves it unchanged.
\end{enumerate}

  Monoidic elements that can be combined with each other are fundamental to the composability of the EPP.
  
\subsection{Isomorphic Recursive Composition} 
\label{sec:metaphysics_isomorphic_recursive_composition}


Isomorphism means "having the same shape” in the sense that isomorphic elements all share the same form. Recursion refers to a structure containing itself. Isomorphism enables the combination of different types of monoidic primitives whereas recursion allows self-referential nesting. Combining these two concepts results in isomorphic recursive composition. This composition enables a monoidic primitive to derive its significance from its structural relation to other monoidic primitives. Critically, to ensure the non-reducibility of form of monoidic primitives within the isomorphic recursive composition, it must have a singleton representation of itself. 

\subsection{The Metaphysics of Context} 
\label{sec:metaphysics_context}

The metaphysics of context is structured as a classical "matter", the hyle, through the
monoidic primitive of the contextoid. The contextoid entails one unit of context, that can be
space, time, spacetime, symbol, or data. The context is free of recursion to prevent paradoxical states such as time-loops. Furthermore, the metaphysics does not prescribe any particular shape or structure of the context to ensure a context can be represented in a structure best suited for the domains at hand.


\subsection{The Metaphysics of Being} 
\label{sec:metaphysics_being}

The metaphysics of being is structured in a classical Aristotelian notion (Book Zeta\cite{furth1985metaphysics}): 

\begin{itemize}
	\item Monoidic Primitives (Matter): The EPP is composed of fundamental, identifiable elements.
	\item Isomorphic Recursive Composition (Form): Gives the primitive elements its form. 
\end{itemize}

Combined, the monoidic primitives and the isomorphic recursive composition form an Aristotelian hylomorphic compound. The `Monoidic Primitives` constitute the "matter" (hyle) and the `Isomorphic Recursive Composition` provides the "form" (morphe) that arranges its elements into a structured, meaningful whole.

This hylomorphic compound is the  \textit{archê kai aitia} of the system's existence. The specific form imposed upon the primitives is the foundational principle and explanation for its properties. Therefore, the hylomorphic compound is by definition static and describes what the system is at a snapshot in time, but it contains no inherent mechanism of change. 

\newpage

\subsection{The Metaphysics of Dynamics} 
\label{sec:metaphysics_dynamics}

The classical hylomorphism describes a static substance as the combination of its matter and form. Therefore, to account for the dynamism inherent in the EPP, the \textit{archê kai aitia} needs to capture the dynamics within an existing substance. However, because of the spacetime agnostic design of the EPP, dynamics can only be defined relative to its engulfing context. Therefore, the principle of contextual relativity operationalizes the expression of Substantial-Structural co-determination relative to its context. This is an inherent  principle of how a substance with a fixed identity can exhibit variable states. By its definition, dynamics is predictable and bound to an existing substance and context. 

The dynamics in the EPP operates relative to its engulfing context and within its structure. The structure might be static or dynamic, but for dynamic the separation between hyle (context), logos (logic) and its structure remains intact. That means, a causal logic can only operate along pathways. Dynamics cannot leave a pathway, it must follow it regardless of whether these paths were established upfront or created dynamically

\subsection{The Metaphysics of Adaptation} 
\label{sec:metaphysics_adaptation}

 For Adaptive dynamics as in adaptive reasoning, there is an element of interaction between the logos and its structure. The logos, a causaloid, can alter adaptively the pathway through the causal structure depending on its internal reasoning. Specifically, when a causal model is structured in such a way that it contains sub-models for specific scenarios, then a causaloid in adaptive reasoning mode can determine dynamically which sub-model is more applicable and then dispatch the reasoning path to the sub-model and with that alter the causal reasoning pathway. Notice, as long as the targets of the dispatch exist and the dispatch logic is decidable, the modality of adaptive dynamics remains deterministic. Although adaptive dynamics can change the reasoning pathway, however, it cannot fundamentally change or evolve the structure itself.  


\subsection{The Metaphysics of Structural Change} 
\label{sec:metaphysics_becoming}

The principle of contextual relativity (Dynamics) accounts for how a system can change its state predictably within a fixed structure and enables reasoning along a fixed path. Adaptive reasoning extends the  modality towards a dynamic pathway through a static or dynamic structure. However, none of it can account for how the structure itself might evolve, as is required in advanced use cases like autonomous navigation. For structural change, the principle of emergence allows for two modalities.

\begin{itemize}
	\item Fixed structural change
	\item Dynamic structural change
\end{itemize}

Fixed structural changes limit systems to pre-defined all possible structural changes that are triggered by known conditions.
Consequently, the system remains fully deterministic and verifiable, but at the expense of adaptability. 
Conversely, dynamic structural changes enable adaptability by  generating novel causal structures that were not explicitly encoded before,
but at the loss of determinism and verifiability. This fundamental trade-off determinism at the expense of adaptability versus
adaptability at the expense of  determinism demands a fundamental decision. Also, fixed structural change can be expressed 
as a specialized form of dynamic structural change with a momentum of change held constant. The reverse, however, is not possible. 

The Effect Propagation Process as a theory of dynamic causality deliberately embraces dynamic structural changes.
The limitations of fixed structural changes are too restrictive for the next generation of intelligent systems.
The consequences of defining and implementing dynamic structural changes, while substantial, 
are accepted as the price of enabling truly dynamic causality in a dynamic environment. 
Therefore, the metaphysics of becoming defines the required principle of emergence, 
the subsequent ontology structures the mechanism of emergence, and the epistemology captures 
the implications. To capture dynamic structural change, the principle of Higher-Order Emergence  becomes necessary.

\subsection{The Metaphysics of Becoming} 
\label{sec:metaphysics_becoming}

The principle of Higher-Order Emergence captures the profound process of creation of new substance from within an existing substance. 
Emergence brings into being a new matter, a new form, both of them, or new dynamics. The principle of Higher-Order Emergence operationalizes two different modalities.  

First-Order Emergence explains the creation of new substance. While contextual relativity applies to an existing substance, it cannot bring into being a new substance. The principle of `First-Order Emergence` posits a generative capacity within the EPP, a capability of imposing a novel \textit{archê kai aitia} upon a set of primitives. This is the mechanism by which a new substance from within an existing context comes into being.

In this modality, the distinction between the system and its context begins to dissolve. The result is a non-deterministic co-evolution where the spectrum of subsequent causal structures and contextual facts cannot be predicted any longer. The "reason for being" \textit{aitia} of any given state is no longer a fixed principle but is itself an emergent property of the ongoing, self-modifying process. However, the process of becoming is invariant because of the first-order designation. 

Higher-Order Emergence moves beyond the creation of a new substance from within an existing one. Instead, it describes a state where the generative capacity that enables `Emergence` acts upon itself recursively in what amounts to a dynamic co-mergence of form, matter, and dynamics through recursive higher order emergence. 

In this modality, the process of becoming itself becomes dynamic. This higher order emergence represents the EPP's most advanced state of becoming, one that necessitates a new epistemology of emergence to explore its inherent emergent properties.


Metaphysically, the principle of Higher-Order Emergence demands further elaboration. Fundamentally, it implies that first-order emergence, the capacity to create new substance, is simply a less recursive specialization of the governing higher order principle.The principle of Higher-Order Emergence leads to outcomes that are not necessarily guaranteed to be decidable let alone deterministic any longer and therefore it foreshadows three crises:

\begin{enumerate}
	\item The Crisis of Justification
	\item The Crisis of Truth
	\item The Crisis of Explainability
\end{enumerate}


The crisis of justification immediately results from the fact that it must be decided how to choose one state of emergence over another and how. The justification must be there otherwise the decision cannot be made, but this would render higher-order emergence fundamentally undecidable. 

The crisis of truth results from the fact that, if emergence generates a new context, then how do we know that
the facts in the newly generated context are true? If emergence generates new causal rules that uses new facts from a generated context, how do we know the outcome is true? If facts are fluid, verification is impossible. Therefore, truth must be re-established otherwise it undermines trust in operational safety of the EPP. 

The crisis of explainability means that in co-emergence, it might not be possible any longer to explain the outcome because of the previous crises of truth and the crisis of justification. Furthermore, if the process of emergence itself cannot be explained, how could possible derived artifacts be explained? How can a system be held accountable when it's not explainable. It is not possible, and therefore the crisis of explainability roots in the very core of higher-order emergence.

The introduction of higher order emergence also raises the question of the genesis process, 
the origin of the emergence itself. Fundamentally, the genesis process imposes a decision: Do we allow any kind of machine intelligence to modify its genesis process or not? The author argues for an unequivocal no. 
Considering the alternative, when a system that can evolve its own genesis process, it would fundamentally become uncontrollable and unexplainable. 

Therefore, the genesis process of emergence  has to remain at the sole discretion of a human designer to ensure its explainability and a fundamental alignment with human values. The metaphysics itself cannot establish the core ethos or telos, only a human designer can do that. Under no circumstances should the genesis process in parts or in its entirety ever be created or modified by a machine intelligence because it cannot possibly have the innate ethos of a human being and thus cannot possibly align itself with humanity. 

The existence of the genesis process also raises the thorny issue of whose ethics and values to codify and why? Which human designer? Who guards the guardians of the genesis? How to balance conflicts between different stakeholders? These are  immense normative and political challenges and, a metaphysics alone, cannot possibly answer these fundamentally societal questions. 

As a consequence, the genesis process itself is a decision with far reaching higher order effects. The mitigation of inevitable unintended higher order consequences, will lead to the necessity of an immutable genesis telos, an underlying intent that serves as a criterion to discern whether emerging states are intended. 


\subsection{The Metaphysics of Axiology} 
\label{sec:metaphysics_axiology}

The reasoning (logos) alone does not have any impact without action, but action without reason (logos) may not lead to a desired outcome. The link between reasoning and action is captured in axiology, the study of value, ethics, intent, and aesthetics. For the EPP, two aspects of axiology are used. 

The Telos, intent, which is the specific, singular intent or overarching goal applied proactively and the ethos, a framework of rules retrospectively applied. The goal of the telos is to prevent incorrect causal reasoning within a causaloid, for example, resulting from invalid context reading i.e. because of a faulty sensor. 

The ethos with its finer set of rules is applied after all reasoning has been completed, but before an action can be taken. A well known property of complex systems is that, even when all components operate within a valid range, the final outcome can potentially still be invalid because of unforeseen state combinations. Therefore, the goal of the ethos is to prevent unforeseen consequences from happening by vetoing actions that would violate the rules codified in it. 

In the EPP, the causaloid folds cause and effect into one single monoidic entity and the PropagatingEffect folds input and output into one single monoidic entity. In the same way, the teloid folds intent and ethos into one single monoidic entity with a twofold function:

\textbf{Teloid:} A single teloid codifies a single intent or a single rule. A causaloid can query one or more teloids to inform whether its reasoning stays within codified intent to ensure alignment. 

\textbf{Effect Ethos:} Multiple teloids, using isomorphic recursion, form a scalable rule set that codify the effect ethos governing the final PropagatingEffect. Before a reasoning outcome stages into an action, it must pass the governing effect ethos to ensure that the proposed action is within the overall applicable rule set. Only after the checks enforced by the effect ethos have passed, a proposed action can be executed. 

Together, the teloid and the effect ethos safeguard operations of the EPP regardless of the modality of dynamics. Furthermore, the three crises caused by emergence are mitigated:

\begin{enumerate}
	\item The Effect Ethos justifies why an action was taken. 
	\item The Teloid codifies the intent and thus attests to the ground truth.
	\item Both, the Teloid and Effect Ethos explain a system even when it is emergent.
\end{enumerate}

Because dynamics and emergence are first class principles in the EPP, so are the
teloid and effect ethos to safely govern the EPP and its actions. 

\subsection{The Metaphysics of Doing} 
\label{sec:metaphysics_doing}

In the EPP, the transition from insight to action is governed by a causal state machine that is based on two principles.

 First, the causal state that takes as input any terminal PropagatingEffect and decides deterministically whether to act. The separation between reasoning outcome and deciding whether to act upon the reasoning outcome allows for arbitrary complex decision logic especially if the terminal PropagatingEffect is, for example, symbolic or probabilistic. In this causal state, the effect ethos intercepts the line of reasoning and decides whether the final determination to act falls within its rules. 
 
 Second, the causal action encodes what action to take conditional on the causal state. The causal action is an arbitrary and unconstrained function to allow for the broadest possible set of action an EPP system can take. 
 
 Combined, the causal state and action form the causal state machine governed by the effect ethos to safely transition from insight to action while respecting all rules encoded in the accompanying ethos. 

\subsection{Discussion} 
\label{sec:metaphysics_summary}

The complete metaphysics of the EPP is the synthesis of three core principles:

\begin{enumerate}
	\item Context: The hyle, the eternalized context. 
	\item Substance (Being): The hylomorphic compound of Primitives and Composition.
	\item Dynamics (Changing): Governed by the principle of Contextual Relativity.
	\item Adaptive (Adapting): Governed by the principle of adaptive reasoning. 
	\item Emergence (Becoming): Governed by the principle of Higher-Order Emergence. 
	\item Effect Ethos (Judging): Governed by its axiology. 
	\item Acting (Doing): Governed by the effect ethos. 
\end{enumerate}


The EPP's core metaphysical principles, Monoidic Primitives, Isomorphic Recursive Composition combined with the principles of and Contextual Relativity and  Higher-Order Emergence form the foundation of the EPP. The EPP provides a framework that distinguishes between two tiers of emergence. First-Order Emergence describes the system's capacity to generate new  substance from a stable set of generative rules. Higher-Order Emergence, describes the system's ultimate capacity to evolve itself via a recursive and open-ended process of generative emergence. The introduction of higher-order emergence implies that the ultimate outcome of the emergent process is not guaranteed to be decidable let alone deterministic any longer. This is a deliberate design decision to provide multiple modalities for different requirements. For dynamic systems, contextual relativistic dynamics should suffice. For handling regime change where the new structure can be decided a-priori, first-order emergence should suffice. However, when handling dynamic relativistic regime change in response to an evolving context, higher-order emergence becomes necessary to capture the dynamic co-emergence, but it removes the deterministic foundation of causality, which introduces three foundational crises. The introduction of the teloid and effect ethos address these crises by establishing a priori and a posteriori policy check to ensure safe, reliable, and explainable operations. The EPP, through the introduction of Higher-Order Emergence and the effect ethos, establishes a foundation for exploring the interrelation between emergent systems and human value. 

\newpage


%% Epistemology
\section{The Epistemology of the Effect Propagation Process}
\label{sec:epp_epistemology}

Epistemological approaches to acquiring knowledge in research fall into three categories: positivism, interpretivism, and pragmatism. Positivism concerns itself with observable facts based on the scientific method and thus seeks to achieve generalizability and objectivity. Interpretivism maintains that our knowledge depends greatly on our interpretation of observations of human actions, experiences, and environments thus making interpretive research more subjective. Pragmatism focuses on practical effects or solutions to address problems that are suitable for existing situations or conditions. The epistemology of pragmatism is that knowledge is a self-correcting process based on experience thus, it must be evaluated and revised in view of subsequent experience.

The  presented EPP epistemology changes depending on whether the context is static or dynamic, and, equally profound, whether the EPP is static, dynamic, or emergent.

\textbf{Ontology of Knowledge sources}

In the EPP, the context is designed as the source of factual knowledge. For context, facts may remain invariant (e.g. the value of Pi) or receive continuous updates. The designation whether a context is static or dynamic refers to its structure, not to the factual data in it. Furthermore, a context might be shared between two or more defined EPP and an EPP may use one or more context(s) thus simplifying modeling complex domains.
The EPP encodes each causal relationship in a designated Causaloid. The Causaloid encodes the causal rule, whereas the context encodes supporting data required to apply the rules. The Causaloid may use external data or data from the context to apply its rule.
For example, a context may encode a continuous signal feed from a LIDAR sensor and the Causaloid encodes a rule to infer whether an obstacle has been detected. In this case, the context provides all data. In another scenario, a context may encode several known defect patterns, a Causaloid tests incoming image data for the defect data from the context, but uses incoming real-time image feeds from a manufacturing monitoring system to determine if any of the produced items contain known defects. In this case, the Causaloid relies on context and external data. Therefore, the Effect Propagation Process emits a flexible knowledge ontology by relying on one or more contexts and potentially multiple external data sources.

\textbf{Knowledge Derivation}

The EPP derivates knowledge by applying data to the Causaloid that models that causal relationship to determine whether the causal relation holds true within the applicable context. Consequently, multi-stage reasoning maps directly to the topology of the EPP itself because each effect from a Causaloid propagates further through the EPP topology, which is the structure of the EPP manifested as all connected Causaloids.
From this perspective, a “line of reasoning” literally becomes a pathway through the EPP topology.
Through the topological approach of knowledge derivation, the Effect Propagation Process provides a flexible way to model complex, contextual, multi-causal domains.

\textbf{Justification of Derived Knowledge}

Discerning the truthfulness of knowledge is one key element of epistemology. The EPP with its explicit context, explicit causal function (Causaloid), and explicit support for external data provides all pre-requirements to support the full explainability of each inference. Furthermore, in the case of multi-stage reasoning, the sequence of applied Causaloids establishes the order or explainability.
Fundamentally, the EPP leads to explainable causal inference because of complete data, context, and inference function when assuming a static EPP. For a dynamic or emergent EPP, explainability might not be guaranteed for all potential state transitions. An implementation of the EPP has to specify the exact details to support explainable inference and where to establish sensible constraints on explainability.

The gravitas of the EPP epistemology is rooted in its flexible, contextualized ontology, a powerful knowledge derivation mechanism, and its intrinsic support of explainable causal inference. The epistemology varies depending on whether the EPP process is static, dynamic, or emergent.

\subsection{Static EPP Epistemology}

For a static Effect Propagation Process, the knowledge is explicitly modeled during the design stage and confined in the context. The quantitative nature of explicitly modeled context and EPP leads to the positivism of the resulting epistemology.

\textbf{Static context}

A static context emits an invariant structure after it's defined, therefore a static EPP combined with a static context allows for the strongest deterministic guarantees albeit at the expense of flexibility. Static contexts remain an invaluable tool to model contextual data that remain structurally invariant, which is a common situation when integrating external data sources. The content, structure, richness, and accuracy of that static context profoundly determine the epistemology of what can be known through the EPP.

\textbf{Dynamic context}

In a dynamic context, the context structure itself evolves e.g. new elements (i.e. quarter of a year) are added as the data feed progresses. By definition, a dynamic context relies on a generating function to gauge the dynamic changes of the context. The impact on the epistemology of a static EPP remains minimal though.
Fundamentally, dynamic contexts are used when structural elements occur at either regular intervals or otherwise determinable occurrences, and therefore, the EPP can model these elements regardless of whether they have been added to the context yet.
For example, a Causaloid that determines whether the sum of the previous three monthly financial reports matches the quarterly financial reports for the current quarter might be a precondition if the “current” quarter in the context has been updated. Therefore, dynamic contexts simplify domain modeling while leaving the epistemology modeled in a static EPP intact.

\subsection{Dynamic EPP Epistemology}

For a dynamic Effect Propagation Process, the dynamics are captured in generative functions that evolve either the EPP, the context, or both. Conceptually, these generative functions could range from deterministic, rule-based algorithms that construct or modify Causaloids and Context structures based on predefined logic or specific triggers, to more adaptive mechanisms. For instance, a generative function for a dynamic Context might be a higher-order function that, given the current state and new inputs, returns an updated Context graph, a practice well established in functional programming to build dynamic systems.

The ontology of knowledge may evolve as a result of the evolving EPP and the impact of the epistemology remains deepening not only the EPP evolution itself but the interaction with its context as it can happen that both, the EPP and its context evolve dynamically.

\textbf{Static context}

For a dynamic EPP, a static context may serve as the foundational layer that captures core data that remain structurally invariant. As with a static EPP, the static context determines fundamentally what determines the epistemology of what can be known through the dynamic EPP.

\textbf{Dynamic context}

For a dynamic context, though, the impact on the epistemology captured in a dynamic EPP can be profound. For example, with the advent of model context protocol (MCP), which lets LLMs call into tools to retrieve or modify data, a causaloid in a dynamic EPP may trigger an MCP invocation, which then updates the context by expanding its structure, and then triggers a generative function that creates a new Causaloid based on the retrieved contextual data, which then analyzes either a newly created part of the context, or a new external data feed created by the MCP. As a consequence, the epistemology in this case depends on a dynamic EPP-Context co-evolution.

\textbf{Dynamic Co-Evolution}

When both, the EPP and the engulfing context evolve dynamically in what can be seen as a co-evolution, then no fixed epistemology can be established anymore because the inference based on generated Causaloid over newly added sub-structures of the context may or may not occur depending on the occurrence of the underlying trigger event(s). One could estimate a potential epistemology by using a Rubin causal model\cite{rubin2005causal} (RCM) by comparing potential reasoning outcomes under different scenarios in which a Causaloid was generated versus when not.

More profoundly, it might not be possible any longer to use automated explainability to discern the appropriateness and relevance of the generated Causaloids and contextual shifts in response to external changes. This introduces an element of interpretivism to the resulting epistemology: the derived knowledge requires the observer to apply a conceptual framework for understanding the system's complex and dynamic evolving behavior.

\subsection{Emergent EPP Epistemology}

Unlike a dynamic EPP, an emergent EPP does not evolve anymore based on pre-determined triggers that initiate pre-defined generative functions. Instead, an EPP is considered emergent when the underlying generation process leads to novel causal configurations, reasoning pathways, or new generative principles for Causaloids context interactions that were not explicitly encoded beforehand.
The generation process may incorporate principles from evolutionary computation, novelty search, or machine learning embedded in the EPP itself. While the full exploration of AI-driven generative functions for EPP remains future work, the foundational idea of using programmable functions to dynamically define and evolve both the EPP and its context is a natural extension of EPP's core design philosophy.

Regardless of the mechanism, emergent EPP does not interact with its Context using pre-defined procedures but instead relies on procedures generated by the EPP itself. The key indicator of emergence is that its novel behavior was not foreseeable by its initial designers.

\textbf{Static context}

Like a static or dynamic EPP, when the static context has been defined upfront, it determines fundamentally what determines the epistemology of what can be known through the dynamic EPP within the contextual boundaries.

Unlike a static or dynamic EPP, an emergent EPP may or may not generate a new static context and that indeed alters the Epistemology emerging from an emerging EPP.

\textbf{Dynamic context}

Likewise, when an emerging EPP creates or modifies a dynamic context, the emerging Epistemology cannot be determined any longer because of the resulting co-emergence of the EPP and its context.

\textbf{Dynamic Co-Emergence}

An EPP co-emerges with its context when the underlying generation process leads to novel causal configurations that were not explicitly encoded beforehand. This can happen when the EPP contains methods of machine learning that evolve the EPP itself in response to a dynamically changing context. As a result of the dynamic, non-deterministic self-modification of the EPP itself, the spectrum of subsequent factual representation in the context and the emerging causal structures cannot be predicted any longer.

Therefore, determining the truthfulness of the emerging causality imposes a non-trivial challenge that adds an elevated level of pragmatism to the epistemology. The pragmatism becomes necessary because it is not guaranteed that the underlying dynamic context always leads to a truthful representation of the world it seeks to model, but the generated causal relationships may not always be correct either. Both can happen due to generative errors during the EPP. Generative errors may result from complex interactions that contain steps that, in isolation, are correct, but when combined in a certain order may lead to an incorrect outcome. This is a typical characteristic of increasingly complex dynamic systems that need to be considered by taking an operational stance on truth.

\subsection{Operationalized meaning of truth in EPP}

The meaning of truth evolves depending on the modality of the EPP because the underlying reference for a true statement varies depending on the chosen modality. Per the definition of knowledge, a true belief must entail a high degree of justification and come from a reliable source to count as knowledge. It is the underlying justification process that depends on the modality of the EPP that causes the shift in the meaning of truth to vary.

For a static Effect Propagation Process, the meaning of truth aligns with the classical correspondence theory. That means, that if the context encodes accurate facts and the causal relationships are true, all derived forms of knowledge must be true.
Justification rests upon the verifiable mapping between the EPP's explicit model encoded in its context and the part of reality it purports to represent. The static EPP implicitly operates under the assumption that its model is a faithful mirror of objective facts. Here, the truth of an inference is determined by the adherence to the contextual facts and encoded causal relationships. As a result, a static EPP leads to deterministic verifiability within the confined boundaries of its context.
As the EPP transitions into a dynamic modality, the meaning of truth begins to shift towards a coherent adaptability to dynamic interaction with a changing context. A dynamically modified causal relationship is deemed true if it maintains consistency with the facts in its evolving context. In a dynamic modality,  the justification of knowledge becomes contextually and temporally aware. Therefore, truth is assessed by the EPP's capacity to maintain a relevant and internally consistent causal understanding amidst navigating a temporal dynamic context.  This leans towards a coherence theory of truth, where coherence itself must be evaluated relative to the EPP’s intricate temporal structures.


For a contextual co-emergent EPP, the meaning of truth shifts further toward pragmatic efficacy. This shift becomes necessary because of the emergence of relativistic causal relationships from the EPP that co-evolve with its context. Here, establishing an objective a priori truth becomes elusive since the fabric that would traditionally serve as a stable reference for truth is itself emerging dynamically alongside the causal inference made from it. Instead, the truth of an emergent causal inference is established by its utility in enabling the EPP to navigate its environment within its temporally complex context.
This pragmatic efficacy means that truth, defined by its functional value, becomes inherently system-relative and context-dependent.

Indeed, a functional value could serve as a fitness function guiding the emergent process itself thus raising fundamental questions about alignment. Consequently, pragmatic efficacy can lead to multiple, functionally 'true' yet distinct causal understandings, each valid within its own emergent trajectory and its relativistic interrelation with its context.

This dynamic interplay, where the EPP generates both its context and the Causaloids that encode the causal relationships that operate within that context presents a research opportunity. It allows for the exploration of relativistic emergent causality and how coherent and pragmatically effective causal understandings can arise in systems that lack a fixed predefined spacetime. This might be of interest in theories of fundamental physics where spacetime itself may be an emergent phenomenon arising from more fundamental processes.

\subsection{Causal Emergence}
\label{sec:causal_emergence}

The problem of modeling no a-priori causal structures motivates a different view of causality that sets the stage for tackling  causal emergence. The Effect Propagation Process framework's detachment from fixed spacetime and its focus on a generative function establish the foundation" for causal emergence. Static causal discovery, for example Pearl’s DAGs framework, assumes a fixed causal structure and thus cannot handle causal emergence. Granger causality assumes that time-dependent variables change, but the causal structure remains fixed and therefore cannot handle causal emergence either. This argument holds true for any dynamic system with fixed causality because of the inability of the underlying methodology to handle spacetime-agnostic causal structure. The Effect Propagation Process instead proposes that the causal relationships themselves emerge, change, and may even disappear. The implications of this approach lead to a fundamental reassessment of how to operationalize causality:

\textbf{Causal discovery}

Instead of trying to find a fixed causal structure, EPP models how causal relationships emerge from underlying (dynamic) processes. The existing work on causal discovery remains valid; the EPP, however, takes the idea takes the idea one step further by incorporating a dynamic generative process. Further research will verify the utility of this perspective, but at least it expands the notion of discovering a static structure to describing a dynamic process.

\textbf{Causal transferability}

Instead of trying to capture the exact conditions under which a causal relationship holds true, the EPP specifies all presumptions as a generative function, which makes it fundamentally testable and thus transferability can be decided.

\textbf{Causal dynamics}

The inspiration from causality in quantum gravity was carefully chosen because of its unique ability to reconcile dynamic and static structures and its handling of deterministic and probabilistic modality. The underlying idea in quantum gravity is that the spacetime fabric of reality itself emerges from an underlying process. While we do not have the scientific methods and technology to verify this idea on the quantum level, we can carefully, within boundaries, transfer the idea to the EPP notion that causality itself emerges from an underlying generative process and, therefore, model the dynamics of causal emergence. The properties of EPP become apparent when looking from the lens of modeling the dynamics of causal emergence.

\begin{itemize}
    \item \textbf{Temporal order becomes irrelevant:} Because causal relations can emerge from an underlying process
    \item \textbf{Spacetime-agnostic becomes necessary:} Because the generative process is concerned with establishing relations of effect propagation, the exact fabric through which those effects propagate is conceptualized as an external context; therefore, EPP itself has to be spacetime-agnostic.
    \item \textbf{Hardy's Causaloids are necessary:} Because the EPP itself is spacetime-agnostic, a different representation of causal relationship that is also spacetime-agnostic becomes necessary and the causaloid proposed by Lucian Hardy has been deemed the best fit.
    \item \textbf{Centrality of the generative function:} Because the EPP can represent causal relations as either static, dynamic, or emergent, the generative function takes on a central role to express those causal relations. Furthermore, a generative function may generate the engulfing context as a specific fabric for the effect propagation process.
\end{itemize}


The Effect Propagation Process framework constitutes a  foundational shift to viewing causality itself as emergent and thus redefines what causality means in dynamic systems. Because of its flexibility, EPP can express static causal relationships similar to Pearl’s Causal DAG, it can handle dynamic causal systems similar to Dynamic Bayesian Networks, but then goes further and enables dynamic causal emergence. Dynamic causal emergence has real-world applications:

\begin{itemize}
    \item \textbf{Financial Markets:}  Causal relationships between assets change based on market conditions. EPP can be used to model how these relationships emerge and dissolve.
    \item \textbf{Biological Systems:} Gene regulatory networks where modulating relationships emerge based on cellular state.
    \item \textbf{Social Systems:} How influence relationships emerge and change in social networks.
\end{itemize}

%% Teleology
\section{The Teleology of the Effect Propagation Process}
\label{sec:teleology}

\subsection{Overview}

The preceding chapters have established the Effect Propagation Process as a conceptual and formal framework for modeling dynamic causality. Higher-Order Emergence provides a formal language for describing systems capable of recursively evolving their own causal and contextual structures. However, the concept of emergence gains enormous expressiveness to handle dynamically evolving situations, but at the expense of determinism. The loss of determinism leads to a profound challenge of alignment because, when context and causal structure dynamically co-evolve, how do we ensure the resulting system states remain aligned with mission objectives? The metaphysics of the EPP already established that neither verification nor alignment is possible without an additional verification mechanism. In response, the EPP establishes an effect ethos as a mechanism to verify alignment with mission objectives or codified values to ensure that, regardless of dynamic modalities, the system always operates within defined parameters. The EPP and its implementation DeepCausality provide a synergistic foundation for the effect ethos, because:

\begin{itemize}
    \item Real-world safety problems are not confined to simple geometries. Safety systems for avionics, maritime, and robotics require native support
  for non-Euclidean geometries.
    \item Ethics deals with moral dilemmas that evolve dynamically. As a complex situation changes, so do priorities. Therefore, the Effect Ethos needs access to a dynamically evolving context to resolve conflicting objectives by  adjusting priorities relative to the context. 
    \item Ethics aims to prevent unwanted consequences. Thus causal state machines are required to have an intermediate step that applies ethics during the translation of reasoning insights into actions to prevent actions that would violate the encoded ethos. 
\end{itemize}

The EPP already provides a solid foundation for the effect ethos by providing context, access to complex geometries, and causal state machines. However, the metaphysics in section \ref{sec:metaphysics} of the EPP also established three fundamental crises, of truth, justification, and explainability, as fundamental properties of higher-order emergence.  The crisis of truth has proven to be particularly problematic as elaborated in the epistemology in section \ref{sec:epp_epistemology}. The metaphysics of axiology establishes two new primitives for a normative framework that shifts the  anchor away from an ambiguous epistemology (what is true) to clear and decidable teleology (what is its intent) to mitigate the three identified crises:

\begin{itemize}
    \item \textbf{The Teloid:} A computable unit of purpose. Functioning as a prospective guard of intent, a Teloid is a verifiable function that intercepts a proposed action from a Causal State Machine and evaluates it against a defined
  goal or policy before execution. This introduces a real, deliberative step of teleological verification against stated
  intent deeply integrated into the system's core reasoning engine. The Teloid ethical decisions are relative to the system's current context to ensure correct contextual priorities. 
  
    \item \textbf{The Effect Ethos:}  A framework for validating outcomes. Functioning as a retrospective validator, the Effect Ethos assesses the holistic, emergent state of the system after a reasoning cycle to ensure fundamental principles such
  as safety, fairness, or regulatory compliance have been upheld. The Effect Ethos leverages the EPP's isomorphic
  design to construct a verifiable 'machine ethos' from simpler Teloid primitives, creating a composable and mechanistic
  ethical framework from first principles as an integral part of the EPP.
\end{itemize}


Combined, the Teloid and Effect Ethos form an architecture within which ethics becomes a computable. 
The distinction between a prospective "Teloid" (guarding actions) and a retrospective "Effect Ethos"
(validating outcomes) exists for a specific reason. A proposed action is vetted upfront against a set of codified rules to prevent catastrophic failures before they can happen. However, a reasoning outcome, especially when the reasoning is conducted throughout a complex causal hypergraph connected to multiple static and dynamic contexts, can only be evaluated after completion. Many real-world ethical dilemmas involve balancing a locally "correct" action (which a Teloid might permit) against a holistically undesirable emergent outcome that the Effect Ethos may prevent. For instance, a series of individually-approved financial trades could, in aggregate, run against global risk management or potentially violate complex regulatory requirements. In that case, the Effect Ethos prevents the violation before it can happen. 

The central challenge faced by the effect ethos is the translation of abstract, relative, and often ambiguous principles such as "safe," "fair," or "efficient" into a verifiable and computable format. Complicating matters further, operational rules often have a hierarchical order where some rules override others, but only in some defined cases. Therefore, contextual conflict resolution is assumed to be the norm and needs to be addressed accordingly. 

\subsection{Defeasible Deontic Foundation}


Addressing the identified challenge, the foundation of the effect ethos is directly inspired by the Defeasible Deontic Inheritance Calculus\cite{olson2024DDIC} (DDIC) pioneered by Olson, Salas-Damian, and Forbus. The DDIC is an axiomatic system designed for the purpose of resolving conflicts between evolving norms in a dynamic context and therefore it fulfills the core EPP requirements. The DDIC defines a norm as a formal tuple (Agent, Behavior, Context, Deontic Modal) and its rules are explicit, symbolic inference rules

The DDIC formalizes two conflict resolution heuristics: Lex Posterior (the later rule wins) and Lex Specialis (the more specific rule wins).  Critically, it achieves conflict resolution "defeasible inheritance", which means a later or more specific norm can defeat the inheritance of a more general or earlier rule. The DDIC norm tuple explicitly includes a parameter for Context and thus ensures contextual relevance for conflict resolution. Because DDIC is a formal calculus, its reasoning process is traceable. A decision to permit or forbid an action can be explained by providing the chain of axioms and defeaters that were triggered\cite{olson2024DDIC}. 

\subsection{Encoding Ethics in a Teloid}

The DDIC defines a norm as a formal structure representing an ethical rule. The EPP adapts the formal structure of the DDIC and maps its components directly to the EPP architectural primitives. A Teloid is, therefore, the concrete instantiation of a single DDIC norm. The mapping of the DDIC components is shown in table \ref{tab:mapping-ddic}. 


\begin{table}[h!]
\caption{Mapping DDIC components to the EPP}
\label{tab:mapping-ddic}
\begin{tabular}{|l|l|l|}
\hline
\textbf{DDIC} & \textbf{EPP}  & \textbf{Description}                     \\ \hline
Agent         & Model         & The EPP model itself is the agent whose behavior is being governed. \\ \hline
Norm          & Teloid        & The teloid represents a single DDIC norm \\ \hline
Deontic Modal & Teloid Type   & The normative status: Obligatory, Impermissible, or Optional        \\ \hline
Behavior      & Causal Action & The proposed action from the CSM being evaluated                    \\ \hline
Context       & Context Query & A query against the Context              \\ \hline
\end{tabular}
\end{table}

The calculus of defeasible inheritance enables the EPP to reason efficiently about the implications of its norms. For example, a Teloid forbidding a general behavior will cause its normative status to "inherit" downwards, making more specific behaviors also impermissible. Crucially, this inheritance is "defeasible," meaning it can be blocked or defeated by a more specific or more recent Teloid. For instance, a general permission to change lanes can be defeated by a newer, more specific prohibition against doing so in a construction zone. Furthermore, to handle nuanced decisions between permissible actions, a Teloid with an Optional modal can also contain an associated cost function. This cost function, evaluated against the Context, provides a quantitative measure used by the Effect Ethos to adjudicate between multiple non-forbidden options, ensuring that even optional behaviors are selected in a resource-aware and efficient manner. This provides a formal, predictable, and traceable mechanism for handling exceptions and resolving conflicts. 

\subsection{Deontic Inference}

The Effect Ethos conducts deontic inference via the  the DDIC calculus using a multi-step process that is deeply integrated into the foundation of the EPP:

\begin{itemize}
	\item  Interception and Contextual Filtering
	\item  Belief Inference via Inheritance
	\item  Conflict Resolution via Defeasibility
	\item  Verdict Finding via Ethical Consensus
\end{itemize}

\textbf{Interception and Contextual Filtering}

When a Causal State reaches a conclusion that would activate a Causal Action, the Effect Ethos intercepts the Causal State to determine its teleological scope. This scope defines the nature and extent of the ethical review required. For non-critical actions, the scope may be empty, allowing the action to proceed without review. For critical actions, the Causal State defines its teleological scope through one or more tags. The tagging mechanism is a cornerstone of the EPP's modularity. Teloids are independently authored with their own corresponding tags, allowing them to function as general-purpose, reusable norms. A single Teloid for battery saving, tagged $low_power$, can be applied to any Causal State that triggers a power-related action, preventing the creation of overly specialized rules.


The Effect Ethos uses the tags from the intercepted Causal State to select a relevant subset of all Teloids for the ethical assessment. It then proceeds to query the Context only for this smaller, relevant set, ensuring it has the latest contextual data for its evaluation and can identify any Teloids that have become stale in case of missing context information. This two-level filtering, first by tag, then by context, ensures that the subsequent deontic inference is highly relevant, computationally efficient, and fast enough for real-time applications.

The causal state designates the applicable teloids to decide the authorization of its conclusion. The Causal State Machine, see section \ref{sec:epp_csm}, separates the decision to activate an action from the action itself to allow for complex decision logic. The Effect Ethos uses the mechanism as intended because the decision whether an action is permissible can only be made upfront. Because the applicable Teloids are designated in the causal state, the Effect Ethos only verifies a smaller but relevant subset of all Teloids thus accelerating the decision process.


\textbf{Belief Inference via Inheritance}

The Effect Ethos then takes this active set of Teloids and applies the DDIC's inheritance to infer a set of normative beliefs. It forward-chain reasons from the active norms to derive all their implications. It is expected that conflicting normative beliefs will occur frequently at this stage because of the dynamic nature of the EPP reasoning and the dynamic context.  

\textbf{Conflict Resolution via Defeasibility}

As the Effect Ethos infers new beliefs, it simultaneously checks the "defeater" conditions defined in the DDIC's axioms. The DDIC fomalizes two mechanisms for norm conflicts resolution.  However, the DDIC  acknowledges that Lex Superior, the rule with the highest priority wins, as a key strategy for norm conflict resolution, but does not formalizes it in its calculus most likely because a norm's priority is an external property that, unlike the other two modalities, cannot be derived from the logical content of the norms themselves. The EPP, does include Lex Superior for its practicality to simplify a normative order and give more design flexibility. In total, the EPP supports three modalities for normative conflict resolution:  

\textbf{Lex Posterior (The More Recent Rule Wins):}  This principle is based on the time a norm was stated. If two norms conflict, the one with the later timestamp is given precedence and defeats the earlier one\cite{olson2024DDIC}.

\textbf{Lex Specialis (The More Specific Rule Wins):} This principle states that a more specific norm should override a more general one. The  DDIC establishes that in many cases, what appears to be a conflict resolved by Lex Specialis is actually just a natural consequence of deontic inheritance, where a more specific rule adds an exception or refinement to a general one without creating a true logical conflict\cite{olson2024DDIC}.  

\textbf{Lex Superior (The Highest Priority Rule Wins): } This principle is based on the priority property of the norm with the and establishes that the rule with the highest priority is given precedence and defeats the rule with the lower priority.  

The conflict resolution step uses these three principles to systematically eliminates contradictions and to producing a final, coherent, and conflict-free set of normative beliefs.

\textbf{Verdict Finding via Consensus}

The Effect Ethos examines this final set of norms and seeks a verdict via a  consensus. While the DDIC proves that, through defeasible inheritance, it can resolve the logical contradictions to yield a conflict-free set of norms, it does not have a mechanism of reaching consensus among all remaining norms. The EPP addresses this by establishing an order based  mechanism to reach a consensus based verdict. The uses a clear precedence $(Impermissible > Obligatory > Optional)$ to handle the output of a the previous deontic conflict resolution. 

If the proposed Causal Action is present with a single norm that yields impermissible, the Effect Ethos returns a Forbidden verdict, providing the specific Teloid(s) that led to the prohibition as a justification. In practice, it is expected that the set contains a mixture of modalities i.e., some are obligatory and must be adhered to, some might be impermissible, and others might be optional. Thus the consensus rules are as following:

The first consensus rule is that impermissible overrides all other modalities. 
The second  consensus rule is that  obligatory overrides optional. 
And the third consensus rule is that optional supports either of the other ones when its associated cost warrants it. In case the final consensus reaches an optional conclusion, the total  associated cost must fall below a defined threshold to convert the verdict into permissible, otherwise it will be designated as impermissible due to unjustifiable associated cost. In case the final verdict reaches an obligatory consensus, it then provides the specific Teloid(s) as a justification for the permission.

The Effect Ethos makes ethics a first-class component of the EPP that is rooted in the formal DDIC calculus. As a result, the reasons for any decision are fully transparent, traceable and auditable. 
\newpage

\subsection{Discussion}

The Teloid and Effect Ethos are directly recognizable as "computable policy" and "auditable safety
layers" that broadly translate into two new categories:

\begin{itemize}
  \item \textbf{Compliance-as-Code:} The idea of modular Teloids for regulations (e.g., a "Reg-T Teloid") that could be audited
  directly would lower regulatory risk (fines) and operational cost (standardization).
  \item \textbf{Verifiable Safety for Autonomous Systems:} This provides a concrete architecture for satisfying safety standards (
  like ISO 26262 for automotive), which is currently a major challenge for any autonomous systems.
\end{itemize}


One practical application of Compliance-as-Code would be the formal verification of adherence to regulatory requirements
directly embedded into the model itself. It is not unthinkable that regulators might want to see audits of the codifying
teloids as a means to ascertain and monitor regulatory compliance. Another practical application is the development of
modular reference Teloids that codify specific regulations for certain domains with mandatory industry rules,
for example in finance, to lower the cost of compliance. For autonomous systems, embedding specific safety rules
becomes not only streamlined, but easier to audit, verify, and simulate. Lastly, while neither the Teloid nor the Effect
Ethos can directly answer the question of whether a specific inference or proposed action is the right thing
with respect to its context, at least these are feasible primitives to build a solution to answer those questions.

Challenges will arise mostly from formalization and verification of the proposed Teloid and Effect Ethos. Specifically,
at least the following questions need to be addressed in future development:

\begin{itemize}
    \item How do we formally verify the Teloid itself
    \item How do we prove that a composite Effect Ethos is complete and covers all necessary edge cases?
    \item How do we prove, even if a composite Effect Ethos is correct, that it will be deterministically applied?
\end{itemize}

The Teloid and Effect Ethos are presented as future work because addressing these immense challenges clearly
falls outside the scope of the presented EPP, but still warrant further consideration. 
While the formalization is
a subject for extensive future work, the implementation can re-use existing concepts and primitives already built in
DeepCausality and thus substantiate the feasibility of the proposal.


The capability for higher-order emergence carries the risk of uncontrolled or undesirable system evolution. The "Crisis of Truth" is not a theoretical abstraction but a practical safety concern. The proposed architecture of the Teloid and Effect Ethos is the primary mechanism for managing the risks that result from dynamic emergence. The Teloid can be engineered to constrain the generative process by rejecting proposed structural modifications that violate predefined safety, ethical, or operational policies. However, no set of prospective rules can be proven complete. The retrospective Effect Ethos provides a second layer of defense, assessing holistic outcomes where individually correct actions might lead to an undesirable emergent state. 

It is crucial, however, to recognize the pragmatic reality of applying EPP: real-world systems will be hybrid models. The majority of their components will be static or governed by predictable dynamics. Only a small but critical subset of the system will be designed to be truly emergent.
Traditional brute-force testing is computationally infeasible due to combinatorial explosion. 
Likewise, formal verification, while powerful for deterministic systems, may not be applicable to a system whose state space can evolve dynamically relative to a dynamic context.
The most viable and rigorous path forward is adversarial stress-testing of the teloids and effect ethos. 
It is possible to systematically search for emergent loopholes and stress-test the Effect Ethos
by using Deep Reinforcement Learning to intelligently and adversarially explore the state space of the learned world model.

Adversarial stress-testing  does not offer absolute safety guarantees. The potential for unforeseen behavior in a sufficiently complex system remains, as risk is intrinsic to the nature of dynamic emergence. It represents, however, a principled and practical engineering discipline for managing that unavoidable risk. 
The alternative is to either forgo the benefits of adaptive dynamic systems or to deploy them without a comparably rigorous validation strategy. 
The proposed Teloid and Effect Ethos, validated through adversarial stress-testing, serve as the tools for navigating causal emergence responsibly.

Managing the intrinsic risk of emergent causality is not a challenge for a single methodology; the problem represents an ongoing challenge for the fields of AI safety, formal verification, and causality. The EPP, with its transparent and auditable architecture, is therefore offered as a high-fidelity testbed for exploring these foundational issues. 
The author acknowledges that the exploration of causal emergence requires deep inquiry, probing questions, and different perspectives from a multitude of diverse stakeholders. 
The transparent and open-governance of the DeepCausality project, hosted at the LF AI \& Data Foundation, provides a vendor-neutral venue for facilitating such an essential discussion.

\newpage


%% Ontology
\section{The Ontology of the Effect Propagation Process}
\label{sec:epp_ontology}

\subsection{The Ontology of Being}
\label{sec:ontology_being}

The static ontology defines the concrete primitives that are the fundamental building blocks, the "matter" (hyle) and "form" (morphe) of the  EPP, as defined by the Metaphysics of Being. Each primitive is a direct instantiation of the Monoidic Primitives and Isomorphic Recursive Composition, designed to  represent distinct categories of existence within the EPP's universe. This section details these foundational entities, which, when combined, form the foundation upon which all dynamic processes and emergent phenomena are built.

\subsubsection{The Causaloid: A Unit of Causality}
\label{sec:ontology_causaloid}

Causality is fractal. A high-level cause (e.g., "economic recession") can be decomposed into a network of smaller, interacting causes (e.g., "inflation," "supply  chain disruption," "interest rate hikes"), each of which can be further decomposed into smaller causes. Classical causality has always struggled with this reality because structural composition is fundamentally at odds with both the SCM representation and the  causal DAG. 

The EPP proposes to represent causality closer to its fractal nature: The Causaloid. In line with 
the EPP metaphysics, the Causaloid is a  monoid.   

The identity element of the monoid is the Singleton `Causaloid`. This is the fundamental unit of causality. It has an id as identity and a causal function that captures the causal relationship. The form of the Causaloid is isomorphically recursive to enable uniform composition. The causaloid is isomorphic in the sense that a causaloid can have different types and yet share the shape of a causaloid. For example, a causaloid can be:

\begin{enumerate}
	\item A singleton causaloid that is a single cause.
	\item A collection of causaloids.
	\item A graph in which each node is another causaloid. 
\end{enumerate}

Applying the concept of isomorphically recursive composition means that any number of causaloids of different types uniformly compose. A complex causal graph can be encapsulated into a single Causaloid, which then becomes a node in another causal graph that itself is part of a causal collection. This creates a mechanism where:

\begin{itemize}
	\item Complexity is manageable
	\item Scalability is inherent
	\item Abstraction is first-class
\end{itemize}

Complexity is manageable because each causaloid can be built and tested in isolation, or as part of a specific subgraph. The mechanism scales from a singular unit to larger collections up to complex graphs. While the system complexity scales, the conceptual overhead remains constant because a single concept, the causaloid, scales from simple to complex, from small to large. Abstraction is first-class because causal reasoning happens uniformly regardless of the underlying complexity. The reasoning mechanism remains the same regardless of whether a causaloid is a singleton, a collection, or a graph.

The monodic binary operation for composition results from the fact that causaloids are isomorphic, which allows to combine a causal collection into a new singleton causaloid. Likewise, an arbitrarily complex causal graph composes uniformly any number of causaloids as its nodes and is in itself a causaloid that composes with other causaloids. 

Isomorphic Recursive Composition for causality defines an ontology where the distinction between "part" and "whole" is fluid. Every "whole" (a composite Causaloid) can become a "part" in a larger whole without ever changing its fundamental nature as a Unit of  Causality.

\subsubsection{The Contextoid: A Unit of Context}
\label{sec:ontology_contextoid}

The Contextoid represents the monoidic primitive of state, an atomic, non-recursive, and identifiable unit of factual information. The EPP ontology makes a strict distinction between these two primitives: Causaloids are recursive and represent reasoning; Contextoids are isomorphic and represent the ground truth upon which reasoning operates. The ontology of the Contextoid does not enforce a single, fixed representation for concepts like "space" or "time." Instead, it defines abstract categories:

\begin{itemize}
	\item Datoid -  A monoidic unit of data-like context. 
	\item Spaceoid - A monoidic unit of space-like context.
	\item SpaceTempoid - A monoidic unit of spacetime-like context.
	\item Symboid -  A monoidic unit of symbol-like context.
	\item Tempoid -  A monoidic unit of time-like context.
\end{itemize}

The categorical isomorphism leads to the most critical design principle of this primitive: Contextoids are structurally non-recursive. A Tempoid cannot contain another Tempoid; a Spaceoid cannot contain another Spaceoid. The structural prohibition of recursion guarantees the logical consistency of the Context. It makes it impossible to construct a paradoxical state, such as a time loop where a moment in time is defined as preceding itself. By ensuring the factual bedrock is non-paradoxical and acyclic, the EPP provides the stable, non-contradictory ground truth required for sound, higher-level reasoning to operate. 

Conversely, the same categorical isomorphism enables heterogeneous composition within each category. For example, the Spaceoid category can represent a point in a flat, Euclidean space alongside another representing a point in a non-Euclidean GeoSpace. From the perspective of the ontology, both are valid "spatial facts," i.e., instances of a Spaceoid and thus uniformly treated as space-like. This categorical isomorphism enables the EPP to model heterogeneous systems where different parts of reality demand different mathematical representations of space, time, spacetime, symbol, and data. However, the absence of 
isomorphic recursion deprives the contextoid of its structure. 


\subsubsection{The Context: A structure for Contextoids}
\label{sec:ontology_context}


The Context is the structure that gives meaning through relations to the Contextoids. In the EPP, the Context is a first-class ontological entity: a hypergraph structure that holds all Contextoid primitives and the explicit, typed relationships between them. It is partially monodic in the sense that it has an identity, but unlike other primitives, it has no monoidic composition hence its name “Context” reflects that it is not monoidic and thus does not compose. The deliberate choice roots in the fact that, while its elements, the contextoids, are structurally isomorphic and compose, the engulfing structure, the context, does not compose to prevent the complexity that arises from merging graphs, but more importantly, to prevent incorrect states in this critical structure. Despite this decision, the context, however, is neither singular nor absolute. 

\textbf{Static vs Dynamic Context}

A static context is established upfront and is structurally assumed to remain invariant during its lifetime. The utility comes from encoding static knowledge, for example, the ICD-10 medical ontology that is standardized for a given version. A static representation for each version of the ICD ensures there is no mixing of different standards. 

A dynamic context is one whose structure is dynamically built and modified during its lifetime. The utility applies to situations where context is fluent but structurally known. For example, a monitoring system that receives data feeds from maintenance drones has to add contextoids for drones coming online and streaming data and, likewise, remove contextoids for drones that get out of range and stop the data stream. The dynamic context can only add, modify, or remove contextoids of known types to ensure strict operational safety. 

\newpage

\textbf{Single vs Multiple Contexts}


The EPP ontology explicitly supports  multiple frames of reference via multiple contexts. A causal model can be linked to a primary context  representing the current, observable reality, while also having access to any number of additional contexts. These auxiliary contexts can represent simulated worlds, historical states, counterfactual scenarios, or backup data sources for real-time data feeds.

\textbf{Contextual Counterfactuals}

This capacity for multiple contexts establishes the foundation for relativistic contextual counterfactual analysis. It allows the system to ask "what if" questions by creating hypothetical realities (a new extra context), modifying specific Contextoids within them (e.g., "what if temperature were 5% higher?”), and then evaluating the same Causaloid logic against these altered frames of reference. This can be achieved through architectural patterns such as  "hot/cold" context partitioning, where the subset of Contextoids relevant for the counterfactual analysis are centralized into one dedicated context, from which the alternate versions are derived. When combined with high-performance data structures, this  enables the execution of thousands of counterfactual simulations in real-time, even on resource-constrained embedded devices.

\subsubsection{The Evidence: A Unit of Facts}
\label{sec:ontology_evidence}


Evidence represents a specific monoidic fact presented to a Causaloid for evaluation that may originate from outside the system, i.e., from a sensor reading, is extracted from a contextoid, or is derived from a previous chain of reasoning from a causaloid. A causaloid uses Evidence as input and the context for supporting data used during the analysis. The output of a Causaloid can then be directly verified against the specific Evidence it received, which is fundamental for explainability in complex systems.

Evidence is a generalized isomorphically recursive container designed to support unified reasoning across multiple modalities:

\begin{itemize}
	\item Deterministic: Boolean values ("true/false").
	\item Numerical values: Numbers (e.g., sensor readings).
	\item Probabilistic values (e.g., confidence scores, likelihoods).
	\item Contextual Links (a Contextoid within a Context), enabling the Causaloid to access complex, structured, and non-numerical facts.
\end{itemize}

Evidence is recursive and may contain:
\begin{itemize}
	\item Maps of other Evidence primitives
	\item Graphs of Evidence primitives enable a Causaloid to access complex, relational data.
\end{itemize}

The multi-modal Evidence enables the EPP to reason over the full  spectrum of information found in complex systems.

\subsubsection{The Propagating Effect: A Unit of Influence}
\label{sec:ontology_propagating_effect}

The output of a causal evaluation is the PropagatingEffect, a monoidic primitive of influence. The PropagatingEffect is the operational heart of the "Effect Propagation Process" itself, a unit of influence that travels through the causal graph, from one Causaloid to the next, driving the continuous effect propagation process. The PropagatingEffect is a unified inference outcome across different modalities. It is isomorphic in that it can represent:


\begin{itemize}
	\item Deterministic effect: A definitive boolean outcome ("true/false").
	\item Probabilistic effect: A quantitative outcome, such as a probability score or an estimate.
	\item  Contextual Link: A reference to a specific Contextoid within a Context.
\end{itemize}

For the Contextual Link,  a causaloid writes its reasoning outcome into a contextoid and then propagates its "effect" as a Contextual link to direct the next Causaloid to use the structured information in the contextoid for further analysis. This Contextoid then becomes the Evidence for the subsequent step in the reasoning chain, enabling dynamic, data-driven causal pathways for non-numerical representation as required, for example, for causal symbolic reasoning enabled by the symboid contextoid type. 

\subsection{The Ontology of Becoming: Dynamics}
\label{sec:ontology_dynamics}

The ontology of dynamics in the EPP is governed by the metaphysical principle of contextual relativity. Contextual means that the significance of a monoidic primitive is derived from its relation to its context. Contextual relativity means that state is an emergent property arising relative to its context with the implication that the same object may have different states when used in different contexts, but, more profoundly, the same object may alter its state when context itself is relativistically altered and propagates its contextual adjustment to all connected monoidic primitives. 
Contextual Relativity is expressed in the EPP in two ways:

First, a Causaloid's reasoning is relative to the `Context` it is evaluated against. The EPP ontology explicitly supports multiple frames of reference for each causaloid. A Causaloid asking "Is the pressure critical?" can return true when evaluated against a Context representing a high-altitude environment and false when evaluated against one representing sea level. The causal logic is the same, but the truth it produces is relative to the frame of reference. This is very much in line with the observer principle in the general theory of relativity. 

Second, the state of a Contextoid itself is subject to relativistic forces imbued upon its engulfing fabric. For example, a spacetime contextoid may need to adjust its temporal value for  gravity-induced time dilation. A quaternion contextoid may need to adjust its rotation value relative to incoming sensor data. Therefore, the EPP ontology defines two critical operations to handle contextual relativity via two operations:

\begin{enumerate}
	\item Adjust
	\item Update
\end{enumerate}

Adjust means an existing value is adjusted for relativistic effects or corrected for sensor drift. The core property of an adjustment operation is to take a correction value, say an offset, and apply it to the existing value. 

Update means an existing value is replaced with a new value, for example, a new sensor reading. The core property of an update operation is to take a data value and replace the current one with the new one. 

These two operations, adjust and update, determine the difference between a strictly static contextoid and a dynamic contextoid. A static one is assumed to be invariant throughout its lifetime, for example by holding a set of immutable facts, i.e., a thermal threshold. A dynamic contextoid is one that gets either adjusted or updated, for example when new sensor reading becomes available. A simple causaloid then reads the thermal threshold from one contextoid, and reads the current thermal sensor data from a dynamic contextoid, and determines if it is getting closer to the thermal limit. Critically, when transferring the model to a different operational environment that requires a different thermal threshold, the replacement of one contextoid is sufficient to ensure the correct functionality of the modeled thermal system.  

\subsection{The Ontology of Emergence}
\label{sec:ontology_emergence}

The ontology of Emergence is captured in the mechanism of the Generative Process, a four-stage command-execution cycle that transforms external stimuli or internal states into structural modifications of the EPP itself. 

\begin{enumerate}
	\item The Generative Trigger
	\item The Generative Command
	\item The Generative Process
	\item The Generative Outcome 
\end{enumerate}

This four-stage process applies to both first-order emergence and higher-order emergence because of its recursive design. Ontologically, each step results in a specific and verifiable object, which lays the foundation for a principled implementation at a later stage. 


\subsubsection{The Generative Trigger}
\label{sec:ontology_emgerence_gen_trigger}

Emergence begins with a Generative Trigger. This primitive represents the initial impetus for change, signaling to the EPP that a modification of its internal structure becomes necessary. Triggers can originate from external stimuli, the passage of time, or an explicit external command. A generative trigger can arise from internal states, such as the detection of an anomaly, a deviation from a desired goal, or from a change in external state such as the detection of a fundamental change in the environment, i.e., transiting from day to night. The Generative Trigger acts as the catalyst to initiate the generative process.


\subsubsection{The Generative Command}
\label{sec:ontology_emgerence_gen_command}

Upon receiving a Generative Trigger, a Generative Command is constructed. This primitive is a declarative, verifiable blueprint of a desired action or structural modification. It is the formal expression of the EPP's intent to alter its own substance. Generative Commands are explicit instructions for change, such as the creation, update, or deletion of a Causaloid or Contextoid, the modification of contextual relationships by adding or removing edges in the hypergraph. 

The Generative Command is both isomorphic and recursive. It is isomorphic in that it unifies a diverse set of possible actions under a single primitive type. These actions include, but are not limited to:

\begin{itemize}
	\item No operation. \textit{NoOp}:
	\item  Commands for managing causal logic: \textit{CreateCausaloid}, \textit{UpdateCausaloid}, \textit{DeleteCausaloid}
	\item Commands for managing contexts: \textit{CreateBaseContext}, \textit{CreateExtraContext}, \textit{UpdateContext}, \textit{DeleteContext}
	\item Commands for managing facts within a context: \textit{AddContextoidToContext}, \textit{UpdateContextoidInContext}, \textit{DeleteContextoidFromContext}
	\item A user-defined command for higher-order emergence: \textit{Evolve}.
\end{itemize}

The Generative Command is recursive through its Composite variant. This variant allows a single Generative Command to contain an ordered sequence (a "stack") of other  Generative Commands. This recursive capability enables the EPP to express complex, multi-step transformations as a single, atomic unit of intent. For example, a single Composite command could specify: "add root node, then add Causaloid A, then add an edge between root and A." This ensures that even intricate sequences of operations are treated as a coherent, atomic transaction, maintaining the integrity and verifiability of the system's emergence.  

The Evolve command is one notable outlier in that it is not an atomic command. Instead, Evolve is a meta-command; it is an instruction to replace the mechanism that generates commands itself. By definition, Evolve is no longer guaranteed to be  deterministic.  If the choice of what to evolve into is not itself deterministically derived, then the system's future behavior becomes non-deterministic and potentially unpredictable. Even if the new command is deterministic, the decision to evolve and how to evolve represents a point where the system's fundamental archê kai aitia of change is altered and it is conceptually unclear what this may entail in practice. 

\subsubsection{The Generative Process}
\label{sec:ontology_emgerence_gen_process}

The Generative Process is the operation responsible for actuating the Generative Command. It interprets the command and translates it into concrete, structural modifications within the EPP's ontology. 

This stage represents the materialization of the intended change, where new Causaloids are instantiated, Context graphs are reconfigured, or Contextoids are added, modified, or removed. The Generative Process transforms the abstract intent into tangible alterations of the EPP's substance.

The Generative Process is designed to be robust and auditable. It processes each Generative Command deterministically, ensuring that the system's state transitions are predictable given a specific command. Its primary function is to ensure the integrity of the EPP's internal structure during the process of self-modification. 


\subsubsection{The Generative Outcome}
\label{sec:ontology_emgerence_gen_outcome}

The generative outcome is a primitive that holds both the modified or generated artifact and relevant metadata about the process itself. It attests to the “what happened” during the process in the form of a log detailing every state transition and asserts whether the outcome is in accordance with the stated intention of the generative command. This explicit record of the system's self-modification means that the EPP maintains a complete and verifiable history of its own evolution; a crucial step for auditability, debugging, and the continuous learning process, providing the ground truth for subsequent development of emergence in complex dynamic systems.



\subsection{The Ontology of Action}
\label{sec:ontology_action}

\newpage

\subsection{Discussion}
\label{sec:ontology_discussion}

The ontology of the Effect Propagation Process addresses the limitations of classical causal models in dynamic, complex systems. The EPP's ontology deeply integrates a set of primitives: the Causaloid as a recursive unit of logic, the Contextoid as a non-recursive unit of fact, and the Context as a dynamic, multi-frame relational environment. These primitives, combined with the multi-modal Evidence and PropagatingEffect, form a robust foundation for expressing causality not as a static, linear chain, but as a continuous process of effect propagation. 

The EPP's explicit ontological primitives and its four-stage Generative Process provide a  unique computational platform for exploring the nature of emergence itself. While the  principles of Dynamic State Mutation and Dynamic Structural Evolution offer clear pathways for  predictable adaptation, the concept of Dynamic Co-Emergence, through the Evolve command, opens a frontier for investigating self-referential self-modification of dynamic systems.

The Evolve command, as defined within the EPP's ontology, represents the system's capacity to  fundamentally alter its own generative principles. This capability, while theoretically profound, introduces complex questions regarding predictability, control, and safety. The EPP's transparent, auditable  architecture where every Generative Command and Execution Result is explicit offers an unprecedented opportunity for empirical investigation into these phenomena.

By designing and observing systems that engage in self-referential self-modification, practitioners can gain firsthand insights into the dynamics of emergent behavior, the challenges of maintaining verifiability in adaptive systems, and the potential pathways towards engineering truly autonomous and self-adapting dynamic systems. This area represents a rich vein for future research, where the EPP's foundational principles can be tested and expanded through direct computational experimentation.

\newpage


%% Formalization  

%% ======================================================================
%% Formalization 
%% ======================================================================


\section{The Formalization of the Effect Propagation Process}
\label{sec:formalization}


%% ======================================================================
%  Axioms
%% ======================================================================
\subsection{Axioms}
\label{sec:formalization_axioms}

The theoretical foundation of the Effect Propagation Process is derived from a single axiom, which is itself a necessary consequence of detaching causality from a presupposed spacetime.

\begin{quotation}
\noindent\textbf{Axiom of Causality:} Every causal phenomenon is an instance of a spacetime-agnostic functional dependency. Let $\mathcal{C}$ be the universal set of all causal phenomena. The axiom states:
\begin{equation}
\forall c \in \mathcal{C}, \exists f \text{ such that } c \equiv (E_2 = f(E_1))
\end{equation}
where $E_1$ and $E_2$ are effects and $f$ is a function. The nature of $E$ and $f$ are deliberately unconstrained.
\end{quotation}

%% ======================================================================
%  Formal Derivation of the EPP Primitives
%% ======================================================================
\subsection{Formal Derivation of the EPP Primitives}
\label{sec:formalization_derivation}

The core architectural primitives of the Effect Propagation Process are not a collection of independent design choices, but are necessary logical consequences that follow from the single causal axiom. This section provides a formal summary of that derivation.

\begin{enumerate}
    \item \textbf{The Contextual Fabric ($\mathcal{H}$): The Background Condition.}
    A function $f(E_1)$ may be conditioned on a set of background variables, $K$. The general form of the axiom is thus $E_2 = f(E_1, K)$. 
    A mechanism to hold and provide this background information $K$ is therefore necessary.
    \begin{equation*}
        \implies \text{Let the \textbf{Contextual Fabric} ($\mathcal{H}$) be this mechanism.}
    \end{equation*}
    
    \item \textbf{The Causaloid ($\chi$): The Functional Primitive.}
    Since every causal phenomenon $c$ is defined by a function $f$, a computational primitive to encapsulate $f$ is a logical necessity.
    \begin{equation*}
        \implies \text{Let the \textbf{Causaloid} ($\chi$) be this primitive, whose primary component is the \textbf{causal function} ($f_\chi$).}
    \end{equation*}

    \item \textbf{The PropagatingEffect ($\Pi_E$): The Isomorphic Message.}
    A function requires inputs and produces an output. Given the compositional nature of functions (e.g., $g(f(E))$), where the output of one serves as the input to the next, a single, uniform message type is logically necessary.
    \begin{equation*}
        \implies \text{Let the \textbf{PropagatingEffect} ($\Pi_E$) be this isomorphic primitive serving as input to and output from a Causaloid.}
    \end{equation*}

    \item \textbf{The Causaloid Graph ($\mathcal{L}$): The Structural Composition.}
    A system may contain multiple causal phenomena ($c_1, c_2, \dots, c_n$) that are themselves causally related. A structure is required to represent the set of Causaloids $\{\chi_1, \chi_2, \dots, \chi_n\}$ and the dependency pathways of their `PropagatingEffect`s.
    \begin{equation*}
        \implies \text{Let the \textbf{Causaloid Graph} ($\mathcal{L}$) be this structure, defined as a hypergraph where nodes are Causaloids and hyperedges are the propagation pathways.}
    \end{equation*}

\end{enumerate}

The subsequent sections of this chapter provide the detailed set-theoretic definitions of these logically necessary primitives, which together form the operational core of the Effect Propagation Process.

%% ======================================================================
%  Contextual Fabric 
%% ======================================================================
\subsection[Contextual Fabric (C)]{Contextual Fabric (\(\mathcal{C}\))}
\label{sec:formalization_context}

The Effect Propagation Process (EPP), being spacetime-agnostic, requires a formally defined structure through which effects propagate and within which causal relationships are conditioned. This structure is termed the \textbf{Contextual Fabric}, denoted abstractly as \(\mathcal{C}\). The Contextual Fabric is not a monolithic entity but can be composed of multiple, distinct contextual realms, each potentially representing different aspects of a system's environment or internal state (e.g., physical space, temporal scales, relational networks, data streams). This section provides a set-theoretic formalization of this fabric, starting from its most granular component, the Contextoid, and building up to collections of Context Hypergraphs.

\newpage

\subsubsection[The Contextoid (v)]{The Contextoid (\(v\))}
\label{ssec:contextoid_formal}

The atomic unit of contextual information within the EPP framework is the \textbf{Contextoid}. A Contextoid encapsulates a single, identifiable piece of data, which can be temporal, spatial, spatiotemporal, or a general data value.

\textbf{Contextoid Definition}

A Contextoid \( v \) is an element of a set of all possible contextoids \( V_{\mathcal{C}} \) within a given Context Hypergraph (defined in Section \ref{ssec:context_hypergraph_formal}). It is formally defined as a tuple:
\[ v = (id_v, \text{payload}_v, \text{adj}_v) \]


\textbf{Contextoid Identifier}

Let \(\mathbb{I}\) be a universal set of unique identifiers.
Then \( id_v \in \mathbb{I} \) is a unique identifier for the contextoid \(v\). This ensures each piece of contextual information can be uniquely referenced within its encompassing Context Hypergraph \(C\).


\textbf{Contextoid Payload} 

The \( \text{payload}_v \) represents the actual contextual information encapsulated by the Contextoid. It is defined as a tagged union, allowing for diverse types of context:
\[ \text{payload}_v \in \{ \text{Data}(d) \mid d \in \mathcal{D}_T \} \cup \{ \text{Time}(t) \mid t \in \mathcal{T} \} \cup \{ \text{Space}(s) \mid s \in \mathcal{S} \} \cup \{ \text{SpaceTime}(st) \mid st \in \mathcal{ST} \} \]
where $\mathcal{D}_T, \mathcal{T}, \mathcal{S}, \text{ and } \mathcal{ST}$ are predefined sets representing the domains of all possible data, temporal, spatial, and spatiotemporal values, respectively.


            \paragraph[Data Payload]{Data Payload (\(Data(d), d \in \mathcal{D}_T\))}\label{par:data_payload}
            
            The set \(\mathcal{D}_T\) represents the domain of all possible data values that a Data Contextoid can hold. The specific nature of \(d\) is generic (of type \(T\)) and can encapsulate arbitrary information relevant to the causal model, such as sensor readings, calculated metrics, textual features, or symbolic states.


            \paragraph[Time Payload]{Time Payload (\(Time(t), t \in \mathcal{T}\))}\label{par:time_payload}
            
            The set \(\mathcal{T}\) represents the domain of all possible temporal values. A temporal value \(t \in \mathcal{T}\) is typically structured as an ordered pair:
            \[ t = (\text{scale}, \text{unit}) \]
            
                \subparagraph[Temporal Scale and Unit]{Formalizing Temporal Scale (\(T_{scale}\)) and Unit (\(T_{unit}\))}
                \label{subpar:temporal_scale_unit}
                
                Let \(\mathcal{T}_{\text{scale}}\) be an enumerated set of possible temporal granularities, e.g., 
                \[ \mathcal{T}_{\text{scale}} = \{\text{Year, Month, Day, Hour, Minute, Second, Nanosecond, EventStep}\} \]
                Let \(\mathcal{T}_{\text{unit}}\) be the domain of values for the temporal unit, typically non-negative integers (\(\mathbb{N}_0\)) or a suitable ordered set, representing the count at the given \(\text{scale}\).
                Thus, \(\text{scale} \in \mathcal{T}_{\text{scale}}\) and \(\text{unit} \in \mathcal{T}_{\text{unit}}\).


            \paragraph[Space Payload]{Space Payload (\(Space(s), s \in \mathcal{S}\))}\label{par:space_payload}
            
            The set \(\mathcal{S}\) represents the domain of all possible spatial values. A spatial value \(s \in \mathcal{S}\) can be represented by coordinates, for example, as a tuple for up to three dimensions:
            \[ s = (x, y, z) \]
            where \(x, y, z\) are optional components.
            
                \subparagraph[Coordinates]{Formalizing Coordinates (\(T_{coord}\))}
                \label{subpar:coordinates}
                
                Let \(T_{coord}\) be the domain for coordinate values (e.g., \(\mathbb{R}\), \(\mathbb{Z}\), or a generic type \(V\)). Each coordinate component \(x, y, z \in T_{coord} \cup \{\text{null}\}\), allowing for variable dimensionality.
                
                \subparagraph[Spatial Interpretation]{Formalizing Spatial Interpretation (Euclidean/Non-Euclidean via \texttt{Spatial<V>})}
                \label{subpar:spatial_interpretation}
                
                The geometric interpretation of spatial coordinates (e.g., distance metrics, neighborhood relations) is not fixed but is governed by functions associated with the specific instantiation of the spatial context. In a computational framework like DeepCausality, this is typically managed through a trait or interface, analogous to the \texttt{Spatial<V>} trait discussed in the EPP philosophy \cite{Hansen2025EPP}. This allows \(s\) to represent points in Euclidean spaces, nodes in a graph (non-Euclidean), or other abstract relational structures.


            \paragraph[SpaceTime Payload]{SpaceTime Payload (\(SpaceTime(st), st \in \mathcal{ST}\))}\label{par:spacetime_payload}
            
            The set \(\mathcal{ST}\) represents the domain of all possible spatiotemporal values. A value \(st \in \mathcal{ST}\) combines spatial and temporal information, typically as a composite structure:
            \[ st = (\text{space\_value}, \text{time\_value}) \]
            where \(\text{space\_value} \in \mathcal{S}\) and \(\text{time\_value} \in \mathcal{T}\). This represents a specific spatial configuration at a particular temporal point or interval.


\textbf{Contextoid Adjustability Protocol }
        
        The component \( \text{adj}_v \) represents optional functions implementing the EPP's Adjustable protocol, allowing for the dynamic modification of a Contextoid's payload \( \text{payload}_v \).
        Let \(\mathcal{V}_{\text{payload}}\) be the set of all possible payload values and \(\mathcal{V}_{\text{adj\_factor}}\) be the set of all possible adjustment factors. The functions are:
        \begin{itemize}
            \item \( \text{update}_v: \mathcal{V}_{\text{payload}} \to \text{void} \): This function replaces the current \( \text{payload}_v \) of the Contextoid with a new payload value.
            \item \( \text{adjust}_v: \mathcal{V}_{\text{adj\_factor}} \to \text{void} \): This function modifies the current \( \text{payload}_v \) based on an adjustment factor.
        \end{itemize}
        The 'void' return type indicates that these functions perform a stateful update on the Contextoid \(v\). Their absence implies the Contextoid's payload is immutable post-instantiation.

\subsubsection[The Context Hypergraph (C)]{The Context Hypergraph (\(C\))}
    \label{ssec:context_hypergraph_formal}

    Individual Contextoids are organized into a structured collection called a \textbf{Context Hypergraph}. This hypergraph structure allows for the representation of complex, N-ary relationships between different pieces of contextual information, which is crucial for modeling the intricate dependencies found in real-world systems.

    \paragraph*{Context Hypergraph Definition.} 
    An individual Context Hypergraph \( C \) is defined as a tuple:
    \[ C = (V_C, E_C, ID_C, \text{Name}_C) \]
    where \(V_C\) is its finite set of \textbf{Contextoid} nodes (each conforming to Definition 3.1, thus \( V_C = \{v_1, v_2, \dots, v_n\} \)), \(E_C\) is its finite set of \textbf{Hyperedges}, \(ID_C \in \mathbb{N}\) is a unique identifier for \(C\), and \(\text{Name}_C\) is a descriptive name.

    \paragraph*{Set of Context Hyperedges (\(E_C\)).}
    A Hyperedge \( e \in E_C \) represents a relationship among a subset of Contextoids in \(V_C\). It is defined as a tuple:
    \[ e = (V_e, \text{kind}_e, \text{label}_e) \]
    where:
    \begin{itemize} \setlength\itemsep{0em} % Reduces space between items
        \item \( V_e \subseteq V_C \) is a non-empty subset of Contextoids in \(C\) connected by this hyperedge, with \(|V_e| \ge 1\).
        \item \( \text{kind}_e \in \mathcal{K}_{\text{relation}} \) specifies the type or nature of the relationship, drawn from a predefined set of possible relation kinds \(\mathcal{K}_{\text{relation}}\) (e.g., \(\mathcal{K}_{\text{relation}} = \{\text{containment, proximity, temporal\_sequence, logical\_and, synonymy}\}\)).
        \item \( \text{label}_e \) is an optional descriptive label for the hyperedge (e.g., a string).
    \end{itemize}
    The use of hyperedges allows a single relational link (\(e\)) to connect an arbitrary number of Contextoids (\(V_e\)), enabling the representation of multi-way contextual dependencies.

\subsubsection[Context Collection (C\_sys)]{Context Collection (\(C_{sys}\))} 
    \label{ssec:context_collection_formal}

    In many complex scenarios, causal reasoning may need to draw upon multiple, potentially distinct but interacting, contextual realms. EPP accommodates this through the concept of a \textbf{Context Collection}.

    \noindent\textbf{Definition 3.4 (Context Collection):} A System Context, or Context Collection, \(C_{sys}\), is defined as a finite set of distinct Context Hypergraphs:  \[ C_{sys} = \{C_1, C_2, \dots, C_k\} \] where each \( C_i \) is an individual Context Hypergraph as defined in Definition 3.2. Each \(C_i \in C_{sys}\) possesses a unique identifier \(ID_{C_i} \in \mathbb{N}\) (distinct from other Context Hypergraphs in the collection) and a descriptive \(\text{Name}_{C_i}\).


    \subsubsection[Context Accessor (ContextAccessor C\_refs)]{Context Accessor (\(\text{ContextAccessor}(\mathcal{C}_{refs})\))}
    \label{ssec:context_accessor_formal}

    To enable Causaloids (defined in Section \ref{sec:formalization_causal_units}) to interact with the Contextual Fabric, a mechanism for querying and retrieving contextual information is required. This is conceptualized as a \textbf{Context Accessor}.

    \noindent\textbf{Definition 3.5 (Context Accessor):} A Context Accessor, denoted \(\text{ContextAccessor}(\mathcal{C}_{refs})\), is a functional interface that provides read-access to a specified subset of Context Hypergraphs \(\mathcal{C}_{refs} \subseteq \mathcal{C}_{sys}\).
    Its operations, denoted abstractly, would typically include functions \(f_{\text{access}}\) such that: % Changed f_{access} to f_{\text{access}}
    \begin{itemize} 
        \item \(\text{getContextoid}(id_v, ID_C) \to v \cup \{\text{null}\}\): Retrieves a Contextoid \(v\) by its identifier \(id_v\) from a specific Context Hypergraph \(C\) (identified by \(ID_C\)) within \(\mathcal{C}_{refs}\).
        \item \(\text{getHyperedges}(v_i, \text{kind}_e) \to \{e_1, \dots, e_m\}\): Retrieves all hyperedges of a specific kind \(\text{kind}_e\) that involve a given Contextoid \(v_i\).
    \end{itemize}
    The precise set of functions within the Context Accessor depends on the requirements of the Causaloid functions that utilize it. This formalism defines the accessor as a means to obtain relevant contextual data for causal evaluation based on identifiers and relational queries.


%% ======================================================================
%% Causal Units and Structures
%% ======================================================================

\subsection[Causal Units and Structures (G)]{Causal Units and Structures (\(\mathcal{G}\))} % Using \mathcal{G} for 
\label{sec:formalization_causal_units}

Having formalized the Contextual Fabric (\(\mathcal{C}\)) within which effects propagate, we now turn to the formalization of the causal entities themselves. The Effect Propagation Process (EPP) posits that causal influence is mediated by discrete, operational units which can be composed into complex structures. This section defines these Causal Units and Structures, denoted abstractly as \(\mathcal{G}\). The central entity is the Causaloid (\(\chi\)), representing an individual causal mechanism, and these are organized into CausaloidGraphs (\(G\)) to model intricate webs of causal relationships.

In the EPP philosophy, Causaloids are grounded in a process of observation, assumption validation, and inference. While a full formalization of this empirical grounding process is extensive, we briefly define its key components as they inform the structure of a Causaloid, particularly its linkage to underlying evidence or hypotheses. Let \(\mathbb{I}\) be a universal set of identifiers.

    
\textbf{Observation Instance (o\_data)}
        
        An \textbf{Observation Instance} \(o_{data}\) represents an empirical data point or a collection of related measurements relevant to a potential causal link. It typically includes an identifier, the observed value(s), and any associated outcome or effect. Formally, one might define \(o_{data} = (id_{obs} \in \mathbb{I}, \text{val}_{obs}, \text{eff}_{obs})\). A dataset \(\mathcal{D}_{obs}\) would be a set of such observations.

\textbf{Assumption Instance (A\_smp)}

        An \textbf{Assumption Instance} \(A_{smp}\) articulates a condition believed to hold, under which causal interpretations are made. It includes an identifier, a description, and critically, an evaluation function \(f_{asmp}\) that tests the assumption against observational data \(\mathcal{D}_{obs}\) and/or contextual information \(\mathcal{C}_{sys}\). Formally, \(A_{smp} = (id_{asmp} \in \mathbb{I}, \text{desc}_{asmp}, f_{asmp}, \text{status}_{asmp})\), where \(f_{asmp}: \mathcal{P}(\mathcal{D}_{obs}) \times \text{ContextAccessor}(\mathcal{C}_{relevant}) \to \{\text{true, false}\}\). (\(\mathcal{P}(\mathcal{D}_{obs})\) denotes the power set or a relevant subset of observations).


\textbf{Inference Instance (I\_inf)}

An \textbf{Inference Instance} \(I_{inf}\) represents a tested hypothesis about a potential causal link, derived from observations under validated assumptions. It typically includes an identifier, the inferential question, the strength of the observed relationship, and a threshold for asserting the inference. Formally, \(I_{inf} = (id_{inf} \in \mathbb{I}, \text{question}_{inf}, \text{strength}_{obs}, \text{threshold}_{inf}, \text{status}_{inf})\).


\subsubsection[The Causaloid (chi)]{The Causaloid (\(\chi\))}
\label{ssec:causaloid_formal}

The fundamental unit of causal interaction within EPP is the \textbf{Causaloid}, inspired by Hardy's work \cite{HardyDynamicCausalStructure} but adapted here as an abstract, operational entity for generalized causal modeling. It encapsulates a single, testable causal mechanism or hypothesis, whose activation depends on input observations and contextual information. Its definition reflects the versatile structural forms found in the DeepCausality implementation.

\textbf{Causaloid Definition}

    
    \textbf{Definition 4.1 (Causaloid):} A Causaloid \( \chi \) is defined as a tuple:
    \[ \chi = (id_\chi, \text{type}_\chi, f_\chi, \mathcal{C}_{refs}, \text{desc}_\chi, \mathcal{A}_{linked}, I_{linked}) \]
    where:
    \begin{itemize}
        \item \( id_\chi \in \mathbb{I} \) is a unique identifier for the Causaloid.
        \item \( \text{type}_\chi \in \{\text{Singleton, Collection, Graph}\} \) specifies the structural nature of the Causaloid, determining how its causal function \(f_\chi\) is realized. This corresponds to the \texttt{CausalType} enum in the Rust implementation.
        \item \( f_\chi \) is the \textbf{causal function or logic} associated with the Causaloid. Its precise signature and operation depend on \(\text{type}_\chi\) (detailed in Section \ref{sssec:causal_function_logic_formal_revised}). It fundamentally maps inputs and context to an activation status \(\{\text{true (active)}, \text{false (inactive)}\}\).
        \item \( \mathcal{C}_{refs} \subseteq \mathcal{C}_{sys} \) is a set of references to Context Hypergraphs that provide contextual information for the evaluation of \(f_\chi\).
        \item \( \text{desc}_\chi \) is a human-readable description (e.g., a string) of the causal mechanism or hypothesis. This corresponds to the output of an \texttt{explain()} method in an implementation.
        \item \( \mathcal{A}_{linked} \) is an optional set of identifiers \(\{id_{asmp_1}, \dots\}\) referring to Assumption Instances (Section \ref{sssec:assumption_instance}) upon which this Causaloid's validity depends.
        \item \( I_{linked} \) is an optional identifier \(id_{inf}\) referring to an Inference Instance (Section \ref{sssec:inference_instance}) that may have led to this Causaloid's formulation.
    \end{itemize}


\textbf{Causal Function Logic (f\_chi)}
    
    
    The causal function \(f_\chi\) embodies the specific operational logic of the Causaloid \(\chi\). Its behavior is contingent on \(\text{type}_\chi\):

    \begin{itemize}
        \item If \( \text{type}_\chi = \text{Singleton} \):
            \(f_\chi\) is a direct evaluation function, \(f_{\chi,S}: \mathcal{O}_{\text{type}} \times \text{ContextAccessor}(\mathcal{C}_{refs}) \to \{\text{true}, \text{false}\}\). This function directly tests a specific causal hypothesis against an input observation (of type \(\mathcal{O}_{\text{type}}\), often a \texttt{NumericalValue} in implementations) and the accessed context. 

        \item If \( \text{type}_\chi = \text{Collection} \):
            The Causaloid \(\chi\) encapsulates an ordered or unordered set of other Causaloids, \(\mathcal{X}_{coll} = \{\chi'_1, \chi'_2, \dots, \chi'_p\}\). 
            In this case, \(f_\chi\) represents an aggregate reasoning logic over the activation states of the Causaloids in \(\mathcal{X}_{coll}\). For example, \(f_{\chi,C}\) might evaluate to true if all \(\chi'_j \in \mathcal{X}_{coll}\) are active, or if any one of them is active. The evaluation of each \(\chi'_j\) would itself involve its own causal function \(f_{\chi'_j}\). The input \(\mathcal{O}_{\text{type}}\) might be a collection of observations, one for each member of \(\mathcal{X}_{coll}\), or a shared observation.

        \item If \( \text{type}_\chi = \text{Graph} \):
            The Causaloid \(\chi\) encapsulates an entire CausaloidGraph \(G'=(V_{G'}, E_{G'}, \dots)\) (as defined in Section \ref{ssec:causaloidgraph_formal}). 
            Here, \(f_\chi\) represents the outcome of the full Effect Propagation Process \(\Pi_{EPP}\) (defined in Section \ref{ssec:epp_process_formal}) operating on the encapsulated graph \(G'\). For instance, \(f_{\chi,G}\) might evaluate to true if a specific target node in \(G'\) becomes active after propagation, or if the overall graph reaches a certain state. The input \(\mathcal{O}_{\text{type}}\) would serve as the initial trigger(s) for propagation within \(G'\).
    \end{itemize}
    Regardless of \(\text{type}_\chi\), the causal function \(f_\chi\) ultimately determines the binary activation state of \(\chi\), providing the "testable effect transfer" mechanism central to EPP.

    
    \subsubsection[The CausaloidGraph (\(G\))]{The CausaloidGraph (\(G\))}
    \label{ssec:causaloidgraph_formal} % Added _formal

    Individual Causaloids are composed into \textbf{CausaloidGraphs} to represent complex webs of interconnected causal relationships. A CausaloidGraph is itself a hypergraph.

    
\textbf{CausaloidGraph Definition}
        
        \textbf{Definition 4.2 (CausaloidGraph):} A CausaloidGraph \( G \) is defined as a tuple:
        \[ G = (V_G, E_G, ID_G, \text{Name}_G) \]
        where:
        \begin{itemize}
            \item \( V_G \) is a finite set of causal nodes.
            \item \( E_G \) is a finite set of causal hyperedges, representing functional relationships or propagation pathways between nodes in \(V_G\).
            \item \( ID_G \in \mathbb{I} \) (or \(\mathbb{N}\)) is a unique identifier for the CausaloidGraph.
            \item \( \text{Name}_G \) is a descriptive name for the CausaloidGraph (e.g., a string).
        \end{itemize}

        
\textbf{Set of Causal Nodes (V\_G)}
        
        Each causal node \( v_g \in V_G \) is defined as a tuple \(v_g = (id_g, \text{payload}_g)\), where \(id_g \in \mathbb{I}\) is its unique identifier within \(G\).
        The \( \text{payload}_g \) embodies the principle of \textbf{recursive isomorphism} central to EPP, allowing for hierarchical model construction. It can be one of:
        \begin{itemize}
            \item A single Causaloid \(\chi\) (as per Definition 4.1).
            \item A collection of Causaloids \(\{\chi_1, \chi_2, \dots, \chi_m\}\), where the collection itself might have aggregate evaluation logic (e.g., "active if any \(\chi_i\) is active").
            \item Another entire CausaloidGraph \(G'\), enabling the nesting of causal sub-models.
        \end{itemize}


\textbf{Set of Causal Hyperedges}
        
        
        Each causal hyperedge \( e_g \in E_G \) represents a directed functional relationship or pathway for effect propagation. It is defined as a tuple:
        \[ e_g = (V_{\text{source}}, V_{\text{target}}, \text{logic}_e) \]
        where:
        \begin{itemize}
            \item \( V_{\text{source}} \subseteq V_G \) is a non-empty set of source causal nodes.
            \item \( V_{\text{target}} \subseteq V_G \) is a non-empty set of target causal nodes.
            \item \( \text{logic}_e \) defines the functional relationship or condition under which effects propagate from \(V_{\text{source}}\) to \(V_{\text{target}}\). This logic might range from simple conjunction/disjunction of source node states to more complex functions that determine how the activation of source nodes influences target nodes.
        \end{itemize}
        This hypergraph structure allows for many-to-many causal relationships.

        
\textbf{State of the CausaloidGraph (S\_G)}
        
        The state of a CausaloidGraph \( G \) at any given point in its evaluation is defined by the activation states of its constituent causal nodes. 
        
        \textbf{Definition 4.3 (State of CausaloidGraph):} The state \( S_G \) is a function mapping each causal node in \(V_G\) to an activation status:
        \[ S_G: V_G \to \{\text{active}, \text{inactive}\} \]
        This state evolves as effects propagate through the graph according to the Causaloid functions and the logic of the hyperedges, as will be detailed in Section \ref{sec:epp_core}.

%% ======================================================================
%%  Effect Propagation Process
%% ======================================================================


\subsection[Effect Propagation, Dynamics, and Emergence]{Dynamics and Emergence}
\label{sec:formalization_epp}

The preceding sections formalized the static structural components of the Contextual Fabric (\(\mathcal{C}\)) and Causal Units/Structures (\(\mathcal{G}\)). This section now delves into the heart of the Effect Propagation Process (EPP): its core dynamics, the mechanisms for generating and evolving these structures, and how these formalisms embody the foundational philosophical principles of EPP. This operationalization is crucial for understanding how EPP moves beyond classical causality to address complex, dynamic, and emergent systems.

\textbf{The Effect Propagation Process (Pi\_EPP)}


The core dynamic of EPP is the propagation of effects through a CausaloidGraph \(G\), influenced by the Contextual Fabric \(\mathcal{C}_{sys}\) and triggered by observations or internal states. This process, denoted \(\Pi_{EPP}\), describes how the activation state \(S_G\) of the CausaloidGraph evolves from an initial state \(S_G\) to an updated state \(S_G'\).


\textbf{Defining an Effect (epsilon) within EPP}
    
    Within the EPP formalism, an **"Effect"** (\(\varepsilon\)) is primarily represented by:
    \begin{enumerate}
        \item The activation state (\(\text{active}\) or \(\text{inactive}\)) of a Causaloid (\(\chi\)) or a causal node (\(v_g\)) within a CausaloidGraph, as determined by its causal function \(f_\chi\).
        \item The transfer of this activation status, or information derived from it, to other connected Causaloids according to the graph structure and hyperedge logic.
    \end{enumerate}
    An active state \(S_G(v_g) = \text{active}\) signifies that the causal mechanism encapsulated by \(v_g\) has met its conditions for effect transfer. The *nature* of the effect is qualitatively described by \(\text{desc}_\chi\), and its *consequence* is how its activation (or inactivation) influences other parts of the CausaloidGraph.

    
\textbf{Input Observations and Triggers (O\_input)}
    
    The propagation process \(\Pi_{EPP}\) is typically initiated or influenced by external inputs or observations.
    
    \textbf{Definition 5.1 (Input Trigger Set):} Let \(\mathcal{O}_{input}\) be the set of all possible input observations relevant to the CausaloidGraph \(G\). An input \(o_{in} \in \mathcal{O}_{input}\) is an element compatible with the \(\mathcal{O}_{\text{type}}\) expected by one or more Causaloid functions \(f_\chi\) within \(G\). 
    At any given evaluation cycle, a set of specific input observations \(O_{trig} \subseteq \mathcal{O}_{input}\) serves as triggers. These triggers might be:
    \begin{itemize}
        \item Directed at specific Causaloids in \(G\) (e.g., root nodes, or nodes representing sensors).
        \item Representing updates to the Context Collection \(\mathcal{C}_{sys}\) which, in turn, affect the evaluation of context-aware Causaloids.
    \end{itemize}

    
\textbf{The Propagation Transition Function}
    
    The propagation of effects within a CausaloidGraph \(G\) is formalized as a state transition function \(\Pi_{EPP}\). This function describes how the graph's state evolves.

    \textbf{Definition 5.2 (EPP Transition Function):} The EPP Transition Function \(\Pi_{EPP}\) is a mapping:
    \[ \Pi_{EPP} : (G, S_G, \mathcal{C}_{sys}, O_{trig}) \to S_G' \]
    where \(G=(V_G, E_G, \dots)\) is a CausaloidGraph, \(S_G: V_G \to \{\text{active, inactive}\}\) is its current state, \(\mathcal{C}_{sys}\) is the Context Collection, \(O_{trig}\) is the set of current input triggers, and \(S_G': V_G \to \{\text{active, inactive}\}\) is the updated state of the CausaloidGraph.

    The computation of \(S_G'\) from \(S_G\) is achieved through an iterative evaluation process driven by the structure of \(G\) and the logic of its components):
    \begin{enumerate}
        \item \textbf{Initialization:} Some nodes in \(V_G\) may have their states in \(S_G\) initially set or updated directly by \(O_{trig}\).
        \item \textbf{Iterative Evaluation \& Propagation:} The process iteratively (or recursively, depending on the traversal strategy) considers nodes \(v\_g \in V\_G\):
            \begin{enumerate}
                \item The activation of a node \(v_g\) (i.e., \(S_G'(v_g)\)) is determined by evaluating its causal function \(f_{\chi_{v_g}}\). This evaluation takes into account:
                    \begin{itemize}
                        \item Relevant observations from \(O_{trig}\) or the activation states of its source nodes (elements of \(V_{\text{source}}\) from incoming hyperedges \(e_g\)).
                        \item Contextual information obtained via \(\text{ContextAccessor}(\mathcal{C}_{refs_{\chi_{v_g}}})\).
                    \end{itemize}
                \item The influence of source nodes \(V_{\text{source}}\) on target nodes \(V_{\text{target}}\) is mediated by the \(\text{logic}_e\) of the connecting hyperedge \(e_g = (V_{\text{source}}, V_{\text{target}}, \text{logic}_e)\). This \(\text{logic}_e\) determines if and how the states of source nodes contribute to triggering the evaluation of target nodes.
            \end{enumerate}
        \item \textbf{Convergence:} The process continues until no further changes in node states occur for the given \(O_{trig}\) and \(\mathcal{C}_{sys}\), resulting in the stable updated state \(S_G'\). For CausaloidGraphs with cycles, specific convergence criteria or iteration limits may be necessary.
    \end{enumerate}
    This operational definition emphasizes that effect propagation is a structured traversal and evaluation, where the "rules" of propagation are embedded in both the individual Causaloid functions \(f_\chi\) and the connective logic \(\text{logic}_e\) of the CausaloidGraph's hyperedges.

    


\subsubsection[The Operational Generative Function (Phi\_gen)]{The Operational Generative Function (\(\Phi_{gen}\))}
    \label{ssec:generative_function_formal_merged} % Renamed label

    The EPP philosophy posits that the Contextual Fabric and even the Causal Structures themselves can be dynamic and emergent, shaped by an underlying "generative function." This formalism operationalizes this concept as \(\Phi_{gen}\), a function or set of functions that govern the evolution of \(\mathcal{C}_{sys}\) and/or \(G\).


\textbf{Conceptual Role of Phi\_gen in EPP}

        
        \(\Phi_{gen}\) represents the meta-level rules or processes that can alter the EPP's structural components. It embodies the system's capacity for adaptation, learning, or emergence beyond simple state changes within a fixed structure. While the EPP framework itself does not mandate a specific physical interpretation for an ultimate generative function, it provides the means to model systems where such generative dynamics are at play operationally.

        
\textbf{Generation and Dynamics of Contexts (Phi\_gen\_C)}
  
        
        Let \(\Phi_{gen\_C}\) be the component of \(\Phi_{gen}\) responsible for the evolution of the Context Collection.
        
        \textbf{Definition 5.3 (Context Generative Function):}
        \[ \Phi_{gen\_C} : (\mathcal{C}_{sys}, O_{in}, E_{ext}) \to \mathcal{C}_{sys}' \]
        where \(O_{in}\) represents relevant system inputs/observations and \(E_{ext}\) represents external events or triggers. \(\mathcal{C}_{sys}'\) is the updated Context Collection.
        This function can:
        \begin{itemize}
            \item Modify Contextoids \(v \in V_C\) within a \(C_i \in \mathcal{C}_{sys}\) via their \(adj_v\) protocol.
            \item Modify the structure of a Context Hypergraph \(C_i\), e.g., by adding/removing Contextoids from \(V_{C_i}\) or Hyperedges from \(E_{C_i}\).
            \item Add new Context Hypergraphs to \(\mathcal{C}_{sys}\) or remove existing ones.
        \end{itemize}
        The specific rules defining \(\Phi_{gen\_C}\) are domain-dependent but allow for contexts that structurally adapt to new information or changing environmental conditions (e.g., the dynamic temporal hypergraph described in the EPP philosophy \cite{Hansen2025EPP}).


\textbf{Generation and Dynamics of Causal Structures (Phi\_gen\_G)]}

        
        Let \(\Phi_{gen\_G}\) be the component of \(\Phi_{gen}\) responsible for the evolution of CausaloidGraphs, enabling emergent causality.

        \textbf{Definition 5.4 (Causal Structure Generative Function):}
        \[ \Phi_{gen\_G} : (G, S_G, \mathcal{C}_{sys}, O_{in}, E_{ext}) \to (G', S_G') \]
        This function can modify the CausaloidGraph \(G=(V_G, E_G, \dots)\) into a new graph \(G'=(V_G', E_G', \dots)\) with a corresponding initial state \(S_G'\). Modifications can include:
        \begin{itemize}
            \item Adding or removing causal nodes \(v_g\) from \(V_G\).
            \item Modifying the payload of existing nodes (e.g., changing a Causaloid's function \(f_\chi\) or its linked assumptions \(\mathcal{A}_{linked}\)).
            \item Adding or removing causal hyperedges \(e_g\) from \(E_G\), or altering their \(\text{logic}_e\).
        \end{itemize}
        This function formalizes the EPP's capacity for causal emergence, where the causal "rules of the game" themselves evolve in response to the system's experience and context.

        
\textbf{Co-evolution of Context and Causal Structure (Phi\_gen\_Total)]}
        
        In the most general case, context and causal structure can co-evolve.
        
        \textbf{Definition 5.5 (Total Generative Function):}
        \[ \Phi_{gen\_Total} : (G, S_G, \mathcal{C}_{sys}, O_{in}, E_{ext}) \to (G', S_G', \mathcal{C}_{sys}') \]
        This represents the combined action of \(\Phi_{gen\_C}\) and \(\Phi_{gen\_G}\), allowing for feedback between structural changes in context and structural changes in causal models.

\subsubsection[Core EPP Principles]{Core EPP Principles}
\label{ssec:epp_principles_formal_merged} % Renamed label

    The formalism presented supports and makes explicit the core philosophical principles of EPP.

\textbf{Spacetime Agnosticism}

        
        The EPP formalism is inherently spacetime-agnostic. Neither the Contextoid (Def 3.1), Context Hypergraph (Def 3.2), Causaloid (Def 4.1), nor CausaloidGraph (Def 4.2) definitions mandate a spatiotemporal nature for the underlying fabric or relations. Spatial or temporal properties are introduced only if specific \(\text{payload}_v\) types (Time, Space, SpaceTime) are used within Contextoids. The core propagation function \(\Pi_{EPP}\) (Def 5.2) and generative function \(\Phi_{gen}\) (Defs 5.3-5.5) operate on these abstract structures; they only interact with spatiotemporal concepts if explicitly encoded within particular Causaloid functions \(f_\chi\) or the rules of \(\Phi_{gen}\). This formally detaches EPP causality from a presupposed spacetime, fulfilling a key requirement for modeling non-Euclidean systems or systems where spacetime is emergent.

        
\textbf{Emergence of Temporal Order}
        
        While EPP does not assume a universal linear temporal order, an operational or effective temporal order can emerge within the system through several mechanisms:
        \begin{itemize}
            \item \textbf{Contextual Sequencing:} Sequences of \(\text{Time}(t)\) Contextoids, or relations (\(\text{kind}_e\)) like "temporally\_precedes" defined in \(E_C\), can establish a local or domain-specific temporal order within \(\mathcal{C}_{sys}\).
            \item \textbf{Propagation Dynamics:} The iterative application of \(\Pi_{EPP}\) inherently defines evaluation steps. While these steps are not necessarily universal time, they form a sequence of state transitions \(S_G \to S_G' \to S_G'' \dots\) that constitutes an internal process time.
            \item \textbf{Generative Function Dynamics:} Changes orchestrated by \(\Phi_{gen}\) also occur in sequence, creating a history of structural evolution.
        \end{itemize}
        Causaloid functions \(f_\chi\) can then be designed to query and utilize this emergent temporal information from the context or process history when relevant.


\textbf{Congruence with Classical Causality}
        
        Classical causality (A causes B if A precedes B and influences B) can be shown as a special case within the EPP formalism under specific conditions:
        \begin{enumerate}
            \item The Contextual Fabric \(\mathcal{C}_{sys}\) is structured to represent a classical (possibly dynamic, as in GR) spacetime with a well-defined linear temporal order.
            \item Causaloids \(\chi_A\) and \(\chi_B\) represent events A and B.
            \item A causal hyperedge \(e_g\) connects \(\chi_A\) to \(\chi_B\), with \(\text{logic}_e\) such that \(\chi_B\) is activated if \(\chi_A\) is active AND the context (queried via \(f_{\chi_B}\) or through \(\text{logic}_e\)) confirms that the event corresponding to \(\chi_A\) occurred at an earlier time than the event corresponding to \(\chi_B\) within the defined spacetime context.
            \item The Causaloid \(\chi_A\) can be designated "the cause" and \(\chi_B\) "the effect" based on this emergent, contextually-defined temporal precedence.
        \end{enumerate}
        Under these (and potentially other simplifying) assumptions, EPP's effect propagation aligns with classical cause-and-effect chains.

%% ======================================================================
%% The Causal State Machine (CSM)
%% ======================================================================

\subsection[The Causal State Machine (CSM) ]{The Causal State Machine (CSM)}
\label{sec:formalization_csm}

The Effect Propagation Process (EPP) framework provides mechanisms for modeling how effects propagate and causal structures evolve. To translate the insights derived from such processes into concrete operations or interventions, the \textbf{Causal State Machine (CSM)} is introduced. The CSM serves as an orchestrator, linking specific causal conditions (represented as Causal States) to predefined deterministic Actions. This section formalizes the CSM, reflecting its implementation within systems like DeepCausality\footnote{\url{deepcausality.com}}.

Let \(\mathbb{I}_{CSM}\) be a set of unique identifiers for Causal States within a CSM.

\noindent\textbf{Definition 6.1 (Causal State):} A \textbf{Causal State} \(q\) represents a specific condition whose truthfulness is determined by the evaluation of an associated Causaloid. It is defined as a tuple:
\[ q = (id_q, \text{data}_q, \chi_q, \text{version}_q) \]
where:
\begin{itemize}
    \item \( id_q \in \mathbb{I}_{CSM} \) is a unique identifier for this Causal State.
    \item \( \text{data}_q \in \mathcal{O}_{\text{type}} \) is the specific input observation or data value (compatible with the Causaloid's expected observation type \(\mathcal{O}_{\text{type}}\)) used for evaluating this state. This data may be intrinsic to the state or provided externally during evaluation.
    \item \( \chi_q \) is a reference to a Causaloid (as defined in Section \ref{ssec:causaloid_formal}), \( \chi_q = (id_{\chi_q}, f_{\chi_q}, \mathcal{C}_{refs_q}, \dots) \). The function \(f_{\chi_q}\) embodies the predicate for this state.
    \item \( \text{version}_q \in \mathbb{N} \) is an optional version number for the state definition.
\end{itemize}
The activation of Causal State \(q\) is determined by \( \text{eval}(q) = f_{\chi_q}(\text{data}_q, \text{ContextAccessor}(\mathcal{C}_{refs_q})) \).

\noindent\textbf{Definition 6.2 (Causal Action):} A \textbf{Causal Action} \(a\) represents a deterministic operation. It is defined as a tuple:
\[ a = (\text{exec}_a, \text{descr}_a, \text{version}_a) \]
where:
\begin{itemize}
    \item \( \text{exec}_a: \text{void} \to \text{Result}(\text{void, ActionError}) \) is the executable function representing the action. Its invocation may modify a \(\text{WorldState}\) implicitly.
    \item \( \text{descr}_a \) is a descriptive label for the action.
    \item \( \text{version}_a \in \mathbb{N} \) is an optional version number for the action definition.
\end{itemize}
The successful execution of \( \text{exec}_a \) yields \( \text{Ok}(\text{void}) \); failure yields an \( \text{Err}(\text{ActionError}) \).

\vspace{\baselineskip} % Add some space before the next definition

\noindent\textbf{Definition 6.3 (Causal State Machine):} A Causal State Machine \( M \) is defined by its collection of Causal State-Action pairs. Let \(\mathcal{Q}\) be the set of all possible Causal States and \(\mathcal{A}\) be the set of all possible Causal Actions.
\[ M = (\mathcal{SA}_M) \]
where:
\begin{itemize}
    \item \( \mathcal{SA}_M \subseteq \{(q, a) \mid q \in \mathcal{Q}, a \in \mathcal{A}\} \) is a finite set of ordered pairs, where each pair \( (q_j, a_j) \) associates a unique Causal State \(q_j\) (identified by \(id_{q_j}\)) with a Causal Action \(a_j\). In implementation, this is often represented as a map from \(id_q \to (q, a)\).
\end{itemize}
The set of CausaloidGraphs \(\mathcal{G}_M\) supervised by \(M\) is implicitly defined as the set of all CausaloidGraphs containing any Causaloid \(\chi_q\) referenced by any \(q\) in \(\mathcal{SA}_M\). Similarly, the set of relevant Context Hypergraphs \(\mathcal{C}_M\) is the union of all \(\mathcal{C}_{refs_q}\) for all \(q\) in \(\mathcal{SA}_M\).

    \subsubsection[CSM Operation: State Evaluation and Action Triggering]{CSM Operation: State Evaluation and Action Triggering}
    \label{ssec:csm_operation_formal_merged}

    The operation of the CSM involves evaluating its Causal States and triggering associated Causal Actions.

    \paragraph{Single State Evaluation and Action (\(\text{eval\_single\_state}\)):}
    Given a CSM \(M = (\mathcal{SA}_M)\), an identifier \(id_q^* \in \mathbb{I}_{CSM}\), and potentially new input data \(d_{in} \in \mathcal{O}_{\text{type}}\) for that state:
    \begin{enumerate}
        \item Retrieve the state-action pair \((q^*, a^*)\) from \(\mathcal{SA}_M\) such that \(id_{q^*} = id_q^*\). If no such pair exists, an error is indicated.
        \item The Causal State \(q^* = (id_{q^*}, \text{data}_{q^*}, \chi_{q^*}, \dots)\) is evaluated. The evaluation function is \(f_{\chi_{q^*}}\). The input data used is either \(d_{in}\) (if provided externally for this evaluation) or the state's intrinsic \(\text{data}_{q^*}\).
        Let \( \text{trigger} = f_{\chi_{q^*}}(\text{data}_{\text{eval}}, \text{ContextAccessor}(\mathcal{C}_{refs_{q^*}})) \).
        \item If \(\text{trigger} = \text{true}\), then the action \(a^*\) is executed by invoking its function \(\text{exec}_{a^*}(\,)\).
    \end{enumerate}


    \paragraph{All States Evaluation and Action (\(\text{eval\_all\_states}\)):}
    Given a CSM \(M = (\mathcal{SA}_M)\):
    \begin{enumerate}
        \item For each state-action pair \((q_j, a_j) \in \mathcal{SA}_M\):
            \begin{enumerate}
                \item The Causal State \(q_j = (id_{q_j}, \text{data}_{q_j}, \chi_{q_j}, \dots)\) is evaluated using its intrinsic data \(\text{data}_{q_j}\).
                Let \( \text{trigger}_j = f_{\chi_{q_j}}(\text{data}_{q_j}, \text{ContextAccessor}(\mathcal{C}_{refs_{q_j}})) \).
                \item If \(\text{trigger}_j = \text{true}\), then the action \(a_j\) is executed by invoking its function \(\text{exec}_{a_j}(\,)\).
            \end{enumerate}
    \end{enumerate}
    The order and concurrency of action execution in step 1(b) depend on the CSM's specific execution semantics (e.g., sequential, parallel, prioritized), which are an implementation detail beyond this core formalism. 

    \subsubsection[CSM Dynamics: Managing State-Action Pairs]{CSM Dynamics: Managing State-Action Pairs}
    \label{ssec:csm_dynamics_formal_merged}

    The set of state-action pairs \(\mathcal{SA}_M\) within a CSM \(M\) can be dynamic, allowing the CSM to adapt its stimulus-response behavior. Formal operations on \(\mathcal{SA}_M\) include:
    \begin{itemize}
        \item \textbf{AddStateAction}(\(M, q_{new}, a_{new}\)): \( \mathcal{SA}_M' = \mathcal{SA}_M \cup \{(q_{new}, a_{new})\} \), provided \(id_{q_{new}}\) is not already a key in \(\mathcal{SA}_M\).
        \item \textbf{RemoveStateAction}(\(M, id_q\)): \( \mathcal{SA}_M' = \mathcal{SA}_M \setminus \{(q, a) \mid id_q \text{ is the identifier of } q \} \).
        \item \textbf{UpdateStateAction}(\(M, id_q, q_{updated}, a_{updated}\)): Replaces the pair \((q,a)\) associated with \(id_q\) with \((q_{updated}, a_{updated})\).
    \end{itemize}


%% ======================================================================
%% Example
%% ======================================================================

\subsection{Example: Smoking, Tar, and Cancer}
\label{sec:formalization_example_smoking_tar_cancer}

To illustrate the core concepts of the Effect Propagation Process (EPP) formalism in a more concrete manner, we consider the well-known causal chain: Smoking \(\rightarrow\) Tar in Lungs \(\rightarrow\) Lung Cancer. This example demonstrates how Causaloids, Contexts (implicitly for this simplification), and a CausaloidGraph can represent this system, and how effect propagation leads to an inference. This example is simplified and inspired by the structure typically modeled as a Directed Acyclic Graph (DAG) in classical causal inference \cite{pearl2000causality}, here adapted to EPP principles.

    \subsubsection{Defining the Causal Problem}
    \label{ssec:example_problem_definition}
    We aim to model a system to infer the likelihood of lung cancer based on smoking habits, mediated by tar accumulation. The core causal hypotheses are:
    \begin{enumerate}
        \item Smoking (represented by nicotine levels) leads to increased tar in the lungs.
        \item Increased tar in the lungs leads to a higher risk of lung cancer.
    \end{enumerate}
    We will represent these as individual Causaloids within a CausaloidGraph. For simplicity, any contextual dependencies (e.g., thresholds for "high" nicotine or tar) are assumed to be encapsulated within the Causaloid functions themselves or drawn from an implicitly defined context for this illustrative purpose. Observations will be represented as numerical values.

    \subsubsection{Formalizing the Components}
    \label{ssec:example_formal_components}

\textbf{The Causaloid (chi)}
        
        We define two primary Causaloids:

        \begin{itemize}
            \item \textbf{\(\chi_{S \to T}\) (Smoking \(\rightarrow\) Tar):}
                \begin{itemize}
                    \item \(id_{\chi_{S \to T}} = 1\) (a unique identifier)
                    \item \(\text{type}_{\chi_{S \to T}} = \text{Singleton}\)
                    \item \(f_{\chi_{S \to T}}: \mathcal{O}_{\text{nicotine}} \to \{\text{true, false}\}\), where \(\mathcal{O}_{\text{nicotine}}\) is a numerical input representing nicotine level. The function \(f_{\chi_{S \to T}}\) evaluates to true if the nicotine level exceeds a predefined threshold (e.g., \(0.55\)), indicating a significant likelihood of tar presence due to smoking.
                    \item \(\mathcal{C}_{refs} = \emptyset\) (assuming context-free logic for this example, or that thresholds are part of \(f_{\chi_{S \to T}}\)).
                    \item \(\text{desc}_{\chi_{S \to T}}\) = "Causal relation between smoking and tar in the lung."
                    \item \(\mathcal{A}_{linked}, I_{linked}\) = Optional, representing linkage to foundational assumptions or prior inferences regarding this relationship.
                \end{itemize}
            \vspace{0.5em} 
            \item \textbf{\(\chi_{T \to C}\) (Tar \(\rightarrow\) Cancer):}
                \begin{itemize}
                    \item \(id_{\chi_{T \to C}} = 2\)
                    \item \(\text{type}_{\chi_{T \to C}} = \text{Singleton}\)
                    \item \(f_{\chi_{T \to C}}: \mathcal{O}_{\text{tar}} \to \{\text{true, false}\}\), where \(\mathcal{O}_{\text{tar}}\) is a numerical input representing tar level. The function \(f_{\chi_{T \to C}}\) evaluates to true if the tar level exceeds a predefined threshold (e.g., \(0.55\)).
                    \item \(\mathcal{C}_{refs} = \emptyset\).
                    \item \(\text{desc}_{\chi_{T \to C}}\) = "Causal relation between tar in the lung and lung cancer."
                    \item \(\mathcal{A}_{linked}, I_{linked}\) = Optional.
                \end{itemize}
        \end{itemize}
        The shared underlying logic for \(f_\chi\) in this example (input observation \(\ge\) threshold) is abstracted into each Causaloid's specific function.

\textbf{The CausaloidGraph (G)}
        
We construct a CausaloidGraph \(G_{STC} = (V_G, E_G, ID_G, \text{Name}_G)\) to represent the causal chain.
        \begin{itemize}
            \item \(V_G = \{v_{g1}, v_{g2}\}\), where:
                \begin{itemize}
                    \item \(v_{g1} = (id_{g1}, \text{payload}_{g1} = \chi_{S \to T})\)
                    \item \(v_{g2} = (id_{g2}, \text{payload}_{g2} = \chi_{T \to C})\)
                \end{itemize}
            \item \(E_G = \{e_{g1}\}\), representing the link from smoking/tar to tar/cancer. For this simple chain:
                \begin{itemize}
                    \item \(e_{g1} = (V_{\text{source}} = \{v_{g1}\}, V_{\text{target}} = \{v_{g2}\}, \text{logic}_{e1})\)
                    \item \(\text{logic}_{e1}\): Specifies that the evaluation of \(v_{g2}\) is conditioned by, or follows, the evaluation of \(v_{g1}\). In a simple direct propagation, if \(v_{g1}\) becomes active, this contributes to the conditions for evaluating \(v_{g2}\). (In a computational system with distinct inputs for each step, this logic might simply define the sequence or dependency).
                \end{itemize}
            \item \(ID_G\), \(\text{Name}_G\): Appropriate identifiers and names (e.g., "Smoking-Tar-Cancer Model").
        \end{itemize}
        This formal \(G_{STC}\) represents the structure. In a practical implementation, a collection of Causaloids (like a vector) reasoned over sequentially can instantiate such a linear graph.

    \subsubsection{Formalizing Effect Propagation (\(\Pi_{EPP}\))}
    \label{ssec:example_epp_propagation}

    Consider input observations \(O_{trig} = \{ (id_{\chi_{S \to T}}, o_{\text{nicotine}}), (id_{\chi_{T \to C}}, o_{\text{tar}}) \}\), where \(o_{\text{nicotine}}\) is the observed nicotine level and \(o_{\text{tar}}\) is the observed tar level.
    Let the initial state be \(S_G(v_{g1})=\text{inactive}\), \(S_G(v_{g2})=\text{inactive}\).
    
    The propagation \(\Pi_{EPP}((G_{STC}, S_G, \emptyset, O_{trig})) \to S_G'\) conceptually proceeds as follows, reflecting a sequential evaluation for this chain:
    \begin{enumerate}
        \item \textbf{Evaluate \(\chi_{S \to T}\) (node \(v_{g1}\)):}
            The function \(f_{\chi_{S \to T}}\) is evaluated with its corresponding input \(o_{\text{nicotine}}\) from \(O_{trig}\). 
            If \(f_{\chi_{S \to T}}(o_{\text{nicotine}}) = \text{true}\) (e.g., nicotine \(\ge 0.55\)), then \(S_G'(v_{g1}) = \text{active}\). Else, \(S_G'(v_{g1}) = \text{inactive}\).
        \item \textbf{Evaluate \(\chi_{T \to C}\) (node \(v_{g2}\)):}
            The function \(f_{\chi_{T \to C}}\) is evaluated with its corresponding input \(o_{\text{tar}}\) from \(O_{trig}\). The evaluation of \(v_{g2}\) in this chained model might also be implicitly conditioned on \(v_{g1}\) being active if the overall inference requires the full chain to hold.
            If \(f_{\chi_{T \to C}}(o_{\text{tar}}) = \text{true}\) (e.g., tar \(\ge 0.55\)), then \(S_G'(v_{g2}) = \text{active}\). Else, \(S_G'(v_{g2}) = \text{inactive}\).
        \item \textbf{Final Inference/System State:} The overall state of the system might be defined by a conjunction: Cancer risk is inferred if \(S_G'(v_{g1}) = \text{active} \land S_G'(v_{g2}) = \text{active}\). This reflects whether the complete causal pathway from smoking, via tar, to cancer is deemed active for the given observations.
    \end{enumerate}
    This illustrative propagation shows how individual Causaloid evaluations, based on specific inputs, contribute to the overall state of the causal model. The \(\text{logic}_e\) of the hyperedge \(e_{g1}\) here implies a dependency or sequential consideration in this chain.

    \subsubsection{Formalizing Observations}
    \label{ssec:example_observations_connection}
    This simplified example illustrates:
    \begin{itemize}
        \item \textbf{Causaloids (\(\chi\))} as operational units (Section \ref{ssec:causaloid_formal}), encapsulating specific causal functions (\(f_\chi\)).
        \item A simple \textbf{CausaloidGraph (\(G\))} structure (Section \ref{ssec:causaloidgraph_formal}) representing the causal chain.
        \item The \textbf{Effect Propagation Process (\(\Pi_{EPP}\))} (Section \ref{ssec:epp_process_formal}) as the evaluation of Causaloids in a sequence determined by the graph structure and input data.
        \item The \textbf{State of the CausaloidGraph (\(S_G\))} (Definition \ref{sssec:state_of_causaloidgraph_formal}) evolving based on these evaluations.
    \end{itemize}
    While this example omits explicit dynamic Contexts (\(\mathcal{C}_{sys}\)) or a dynamic Generative Function (\(\Phi_{gen}\)) for brevity, it demonstrates how the core EPP structural entities map to a well-understood causal scenario.
    
%% ======================================================================
%% Discussion
%% ======================================================================    
    
\subsection[Discussion]{Discussion} 
\label{sec:formalization_example_discussion}

The preceding sections have laid out a formal, set-theoretic definition of the Effect Propagation Process (EPP). This formalism is a direct response to the need for new conceptual and computational tools to understand causality in systems where classical assumptions of a fixed spatiotemporal background and linear temporal order are demonstrably insufficient. The motivation stems from both the frontiers of fundamental physics, where spacetime itself is considered emergent, and the practical challenges of modeling complex adaptive systems across various scientific and engineering domains.

The primary significance of the EPP formalism lies in its principled detachment from spacetime. By making Context (\(\mathcal{C}\)) an explicit, definable, and potentially non-Euclidean, dynamic fabric, and by defining Causaloids (\(\chi\)) as operational units of effect transfer independent of a priori temporal ordering, EPP achieves a level of generality that classical causal formalisms cannot. This enables the modeling of causality in non-physical or abstract domains, systems with complex multi-scale temporal dynamics and feedback loops, and, crucially, systems where causal relationships themselves can emerge, transform, or dissolve in response to evolving contexts (dynamic regime shifts). The recursive isomorphism of CausaloidGraphs (\(G\)) further enhances expressive power, allowing for modular construction of complex causal models, while the clear separation of causal logic, contextual data, and propagation mechanisms facilitates transparency.

This foundational formalism, while potent, has current limitations. It primarily defines a deterministic framework, though probabilistic behavior can be encapsulated within Causaloid functions or edge logic. A more deeply integrated probabilistic EPP, or a full formal treatment of quantum indefinite causal order, remains an area for future development. Furthermore, EPP does not, in itself, provide algorithms for automated causal discovery of its structures from raw data; it provides the language to represent and reason with such structures once hypothesized. Compared to established frameworks like Pearl's SCMs, EPP offers greater flexibility for cyclic and emergent systems but currently lacks an equivalent to the extensive \textit{do}-calculus, though the Causal State Machine (CSM) provides a mechanism for linking EPP inferences to actions. EPP generalizes classical notions where necessary but does not seek to replace them where they are already sufficient.

\newpage    

%% Validity 
%%\section{Validity}
\label{sec:validity}

The validity of the Effect Propagation Process must be assessed according to the nature of its contribution. The EPP is
a philosophy-informed, formal computational framework designed to provide a new, more expressive language for modeling
dynamic, contextual causal systems. Therefore, its validity rests on the criteria appropriate for such a framework,
analogous to how one would assess a mathematical logic or a new programming language:

\begin{itemize}
    \item \textbf{Internal Validity} (Soundness \& Consistency): The framework's internal validity is determined by its logical and
  mathematical soundness and coherence. As detailed in the accompanying specification, the EPP is built on a foundation
  of first-principles reasoning to ensure its components are consistent and its operations are sound.
  \item \textbf{External Validity} (Expressive Power \& Scoped Generalization): The framework's external validity is demonstrated by its
  robustness against alternative interpretations of its premises and how well it defines the boundaries of its
  generalization.
\end{itemize}



\subsection{Internal validity}
\label{sec:validity_internal}

The \textbf{soundness} of EPP derives from the first principled logical progression of the presented argument. The stated problem of inadequacy of classical causality in light of the challenges imposed by Quantum Gravity (emergent spacetime, indefinite causal order) is well-recognized and documented\cite{MriniHardyIndefinite}.. From this recognized problem, the argument for EPP progresses as stated below:

\begin{enumerate}
    \item Establishes classical causality and its historical critiques.
    \item Introduces the challenges from modern physics (GR, QG).
    \item Shows why these challenges render classical causality insufficient.
    \item Proposes EPP as a coherent response to these challenges.
    \item Contrasts EPP with classical causality.
    \item Discusses its ontological, epistemological, and teleological implications.
    \item Acknowledges and addresses threats to its validity.
\end{enumerate}

The soundness is further strengthened by its inspiration from established scientific theories (QFT, GR) and foundational work in quantum gravity. While quantum gravity as a scientific theory remains a work in progress, the conceptual challenges that arise from it are valid regardless of how any particular theory may explain the underlying quantum mechanisms.

The \textbf{consistency} of the EPP framework arises from a handful of carefully stated conclusions.

\subsubsection{Spacetime Agnosticism \& Causaloids}

\textbf{Premise:} On a quantum level, spacetime may not exist.

\textbf{Conclusion:} Therefore, remove spacetime from EPP.

\textbf{Premise:} EPP does not have a defined spacetime.

\textbf{Conclusion:} Cause and effect cannot be separated anymore because there is no a priori temporal order.

\textbf{Premise:} Cause and effect cannot be discerned because of missing temporal order.

\textbf{Conclusion:} Fold cause and effect into one entity, the causaloid, that is independent of temporal (and spatial) order.

Note, the last conclusion holds because of the temporal order required for classical causality. The only logical alternative conclusion from the premise of missing temporal order would be to abandon causality altogether. However, this conflicts with the reality in which causal relationships indeed exist; therefore, the alternative has been deemed unsound.

\subsubsection{Effect Propagation as the Essence of Causality}

\textbf{Premise:} The causaloid, as a building block, is independent of temporal (and spatial) order.

\textbf{Conclusion:} Define causality by what it does (propagate effects) instead of what it was thought to be (a temporal order dependent atomic relationship).

The careful reader may raise a concern over the choice of words (i.e., propagate effects), since “transferring information” or “transmitting influence” might be equally valid choices. True, that is indeed a fair point. However, the term information has specific meaning in information theory and computer science, and likewise, the term influence has specific meaning in social science; therefore the author settled cautiously on “effect” mainly to prevent conflating different meanings.

\subsubsection{Compatibility with Classical Causality}

The full logical argument of how classical causality is derived from EPP is in the section “Causality as Effect Propagation Process”.
While classical causality can be derived from EPP, the reverse is not true because EPP cannot be derived from classical causality because classical causality requires a background spacetime. That deduction proves that EPP is more abstract in the sense of more general than classical causality. Therefore, it follows that EPP naturally applies to areas where classical causality cannot be used any longer. The internal validity of EPP roots in its internal soundness and consistency that stems from its first-principles reasoning. Therefore, ambiguity, contradictory claims, and unjustified leaps in logic are avoided to the extent it is possible. Minor mistakes might be possible and the author is open to suggestions  to improve EPP further.

\subsection{External validity}
\label{sec:validity_external}

Establishing the boundaries of generalization as a proxy for external validity requires a delicate balance of realistically acknowledging what EPP can address versus avoiding overstating  any particular capability. Related to external validity is always the possibility of an alternative interpretation of the underlying premises.


\subsubsection{Falsifiability}

A crucial distinction must be made regarding the role of falsifiability. The EPP, as a formal foundation, is not itself an empirical scientific theory and therefore is not directly falsifiable. One cannot falsify the established philosophical foundation of the EPP—its metaphysics, ontology, and epistemology—without challenging its underlying axiomatic assumptions. Instead, the validity of the EPP rests on its internal consistency and its expressive power.

Falsifiability, however, is a critical property of the \textbf{specific, testable causal models that are constructed \textit{within} the EPP.} A model of an avionics system, for instance, represents a set of falsifiable hypotheses encoded as a `CausaloidGraph` and its associated `Context`. This model makes concrete predictions about system behavior that can be tested against simulation or real-world data. If the predictions fail, the \textit{model} is falsified, not the framework.

The EPP provides the formal language and the computational primitives necessary to empower engineers and domain experts to express and test causal hypotheses in complex, contextual realities that were previously beyond the reach of formal modeling.

\subsubsection{Alternative interpretations}

There are several potential alternative interpretations of the premises underlying the Effect Propagation Process framework.

\textbf{Russell was more right than acknowledged}

One can take the position that, if Russell deemed causality as a relict of a bygone era, then why not openly ask to abandon causality altogether and focus solely on descriptive and correlation-based data science?

DARPA disagrees\cite{DARPA_ANSR}:

\begin{quote}
    “In the real world, observations are often correlated and a product of an underlying causal mechanism, which can be modeled and understood.”
\end{quote}

The problem is not a simple choice between correlation and causation. The deeper issue, as contemporary philosophers
like Luciano Floridi have argued, is that the predominant mental model underlying complex system design remains rooted
in an outdated Aristotelian and Newtonian "Ur-philosophy" of fixed space and time. This leads to tools that are
inadequate for the intricate complexity of reality. To address this core critique, the EPP lifts causality into a
contextual generalization required to model dynamic systems that no longer adhere to a fixed background spacetime. This
includes systems with non-Euclidean geometries, multiple interacting contexts, and emergent causal structures, as found
in domains like avionics. The question of falsifiability, therefore, applies not to the EPP framework itself, but to the
specific, testable models that are constructed within it. A model of an avionics system built using the EPP is indeed
falsifiable against simulation data; the framework itself simply provides the language for expressing that model.

The author argues, a similar shift towards a richer contextualization of advanced dynamic models needs to happen in the
correlation-based methodologies of deep learning as well to build tools more suitable for today's reality. A
transfer of core EPP concepts, i.e., the contextual hypergraph, uniform Euclidean and non-Euclidean geometries, into the
foundations of deep learning is welcomed by the author.

\textbf{Pluralism of causal concepts}

Instead of a unified framework like EPP, one can argue in favor of the existing pluralistic reality where multiple 
disjoint concepts of (computational) causality exist for different levels of causal analysis.

As stated before, EPP does not seek to replace classical causality and all tools that are built atop the classical
definition of causality. Instead, EPP seeks to advance the core concept of causality to meet increasingly challenging
demands. Due to the novelty of this foundational work, a single coherent framework is preferred until it becomes clear
which parts may branch out and become more specialized domains over time.


\subsubsection{Boundaries of Generalization}

The primary boundary of the Effect Propagation Process is one of scope: it is a foundational framework, not a final, comprehensive theory of all dynamic causal phenomena. 
Its purpose is to provide the formal language and computational primitives necessary to begin a systematic exploration of dynamic causality, not to claim to have concluded it. 
The EPP foundation defines two clear boundaries:


\begin{itemize}
    \item \textbf{A Lower Boundary of Complexity:} The EPP is designed for a class of problems characterized by
    non-linear temporality, non-Euclidean structures, multiple contexts, and the potential for emergence. For systems
    that align well with the classical assumptions of a fixed spacetime and static causal rules, simpler, established
    methodologies such as Pearl's SCMs are often more appropriate and should be preferred. The EPP's power is best
    reserved for the complex realities that lie beyond the classical scope.

    \item \textbf{An Upper Boundary of Knowledge:} The EPP does not claim to have "solved" dynamic causality. On the
    contrary, by providing the first formal tools to model phenomena like causal/contextual co-emergence, it reveals
    the vastness of our current ignorance. The framework, in its early stages, makes it possible to ask new kinds of
    questions and to begin building and testing the first generation of truly dynamic causal theories for which the 
    specific characteristics need to be discovered
\end{itemize}

While the EPP draws conceptual inspiration from physics to address these new frontiers, it is a foundational and formal computational
framework. It does not propose that macroscopic systems are quantum mechanical, but rather draws inspiration from these powerful
concepts to engineer a more robust and flexible toolkit for modeling the complex, adaptive systems we face today.

\newpage

\subsubsection{Method Selection Criteria: Classical Causality vs. EPP}

In cases where methods of classical causality and conventional machine learning do not solve the problem at hand, methodologies rooted in EPP might be preferred. The following decision matrix supports the assessment of when to use which methods. This matrix provides guidance on selecting an appropriate causal modeling approach based on the temporal and spatial complexity of the system under investigation.

% Please add the following required packages to your document preamble:
\begin{table}[hb]
\begin{tabular}{llll}
Feature Assessed &
  System Characteristic &
  Classical Methods &
  EPP Methods \\ \hline
\multicolumn{1}{|l|}{Temporal Complexity} &
  \multicolumn{1}{l|}{} &
  \multicolumn{1}{l|}{} &
  \multicolumn{1}{l|}{} \\ \hline
\multicolumn{1}{|l|}{} &
  \multicolumn{1}{l|}{Single time scale + linear progression} &
  \multicolumn{1}{l|}{Sufficient} &
  \multicolumn{1}{l|}{} \\ \hline
\multicolumn{1}{|l|}{} &
  \multicolumn{1}{l|}{Multiple time scales OR non-linear temporal relationships} &
  \multicolumn{1}{l|}{May struggle} &
  \multicolumn{1}{l|}{Consider EPP} \\ \hline
\multicolumn{1}{|l|}{} &
  \multicolumn{1}{l|}{Multiple time scales AND non-linear temporal relationships} &
  \multicolumn{1}{l|}{Insufficient} &
  \multicolumn{1}{l|}{EPP Required} \\ \hline
\multicolumn{1}{|l|}{Spatial Structure} &
  \multicolumn{1}{l|}{} &
  \multicolumn{1}{l|}{} &
  \multicolumn{1}{l|}{} \\ \hline
\multicolumn{1}{|l|}{} &
  \multicolumn{1}{l|}{Euclidean space with fixed coordinates} &
  \multicolumn{1}{l|}{Sufficient} &
  \multicolumn{1}{l|}{} \\ \hline
\multicolumn{1}{|l|}{} &
  \multicolumn{1}{l|}{Non-Euclidean OR dynamic spatial relationships} &
  \multicolumn{1}{l|}{Limited/Difficult} &
  \multicolumn{1}{l|}{EPP Advantageous} \\ \hline
\multicolumn{1}{|l|}{} &
  \multicolumn{1}{l|}{Non-Euclidean AND dynamic spatial relationships} &
  \multicolumn{1}{l|}{Very Limited} &
  \multicolumn{1}{l|}{EPP Required} \\ \hline
\multicolumn{1}{|l|}{Combined Complexity} &
  \multicolumn{1}{l|}{} &
  \multicolumn{1}{l|}{} &
  \multicolumn{1}{l|}{} \\ \hline
\multicolumn{1}{|l|}{Scenario 1} &
  \multicolumn{1}{l|}{Linear Time \& Euclidean Space} &
  \multicolumn{1}{l|}{Preferred} &
  \multicolumn{1}{l|}{} \\ \hline
\multicolumn{1}{|l|}{Scenario 2} &
  \multicolumn{1}{l|}{Non-Linear Time OR Complex Spatial} &
  \multicolumn{1}{l|}{Limited/Difficult} &
  \multicolumn{1}{l|}{Consider EPP} \\ \hline
\multicolumn{1}{|l|}{Scenario 3} &
  \multicolumn{1}{l|}{Non-Linear Time AND Complex Spatial} &
  \multicolumn{1}{l|}{Insufficient} &
  \multicolumn{1}{l|}{EPP Required} \\ \hline
\end{tabular}
\caption{Causal Method Selection Matrix}
\label{tab:method_matrix}
\end{table}

\textbf{Emergent Causality:} If the problem involves emergent causal structures (where the causal graph itself is not fixed and changes dynamically based on context, i.e., dynamic regime shifts), EPP-based methodologies become the only viable option.


\clearpage

%% Implementation 
%% ======================================================================
%% Implementation of the  Effect Propagation Process 
%% ======================================================================

\section{The Implementation of the Effect Propagation Process}
\label{sec:implementation}

\subsection{Overview}
\label{sec:implementation_overview}

DeepCausality is an open-source framework, hosted at the Linux Foundation and accessible at \url{https://deepcausality.com}, designed to enable the construction, execution, and rigorous management of explicit, context-aware, and explainable causal models. It implements the Effect Propagation Process and with hat provides a pathway to reason about cause and effect within intricate, multi-dimensional, and dynamically evolving environments. 
The framework's unique hypergeometric nature refers to its core reliance on hypergraph structures for representing both the rich tapestry of context and the complex web of causal relationships,
 offering a new level of expressiveness and analytical depth.

DeepCausality’s core contributions are design to provide a robust foundation for causally-grounded intelligence:
\begin{itemize}
    \item \textbf{A Novel Hypergraph-based Context Engine:} At its heart, DeepCausality features a sophisticated engine for managing context. This moves beyond simple conditioning variables to enable the creation of intricate context hypergraphs populated by \textit{Contextoids} – specialized nodes representing rich, multi-dimensional information encompassing Data, Time, Space, and SpaceTime. This inherently supports dynamically adjustable contexts, the simultaneous integration of information from multiple distinct context hypergraphs (potentially with differing Euclidean or non-Euclidean geometries), and grounds causal reasoning in a highly nuanced and comprehensive understanding of the operational environment.
    \item \textbf{Structurally Composable Causal Modeling:} DeepCausality introduces \textit{Causaloids} – encapsulated, testable causal functions – as the fundamental building blocks of causal models. These are organized within \textit{CausaloidGraphs}, which are themselves hypergraphs explicitly representing intricate causal relationships. Crucially, this architecture employs recursive isomorphic causal data structures: nodes within a CausaloidGraph can themselves be entire sub-graphs or collections of other causes. This enables the intuitive, modular construction of deeply complex, layered causal systems where macro-level phenomena can be decomposed into interacting micro-level mechanisms, ensuring transparent composability.
    \item \textbf{The Causal State Machine (CSM) for Actionable Intelligence:} Bridging the gap between causal understanding and effective intervention, the CSM is architected to manage interactions between causal models and their contexts. Based on the collective causal inference derived—the identification of specific active causes or system states—the CSM deterministically initiates predefined actions, facilitating the creation of complex, dynamic control and supervision systems that respond with causally-reasoned precision.
    \item \textbf{Implementation in Rust for Performance and Reliability:} Recognizing the demanding requirements of a sophisticated causal reasoning engine operating on potentially vast and dynamic data, DeepCausality is implemented in Rust. This choice leverages Rust’s high-performance characteristics, memory safety guarantees, and expressive type system to build an efficient, robust, and reliable foundational causal engine.
    
\end{itemize}

This section details the complete architecture of DeepCausality, its conceptual foundations, and its Rust implementation.

\newpage
\subsection{UltraGraph}
\label{sec:implementation_ultragraph}

The importance of hypergraphs in the EPP led to the decision to implement a hypergraph in a dedicated Rust crate called UltraGraph. Initially, UltraGraph wrapped the MatrixGraph implementation of the 
Petgraph\footnote{\url{https://docs.rs/petgraph/latest/petgraph}} crate. This was a deliberate decision
to speed up the bootstrapping of the DeepCausality project while preserving the option to 
replace the implementation when the need arose. And indeed, as the requirements  of the
DeepCausality project kept advancing with the introduction of emergence, a new hypergraph implementation became necessary. The UltraGraph v0.8 release adds a ground-up rewrite of  UltraGraph inspired by the NWHypergraph (NWHy)\cite{liu2022nwhy} architecture that was further optimized for Rust's memory model. 
  
The key elements of the new UltraGraph implementation are:
  
\begin{itemize}
	\item The introduction of a dual-state graph life-cycle.
	\item The introduction of a SoA CsrAdjacency type.
	\item The separation of forward and backward CsrAdjacency.
\end{itemize}

\subsubsection{Dual-state graph life-cycle}


UltraGraph v0.8 introduces a dual-state architecture. This recognizes that graph-based systems that implement the EPP have two distinct phases: a dynamic "Evolve" phase, where the structure is built and modified, and a stable "Analyze" phase, where high-speed queries are essential. 

\textbf{The Dynamic Graph State:} 

This is the default state for every new graph. It is an adjacency-list-based structure
optimized for flexibility. Adding nodes and edges is a cheap O(1) operation, perfect for systems where the graph structure emerges dynamically over time.

\textbf{The Static Graph State:}

When the graph is ready, a one-time  call to \textit{.freeze()} transforms the graph into a hyper-optimized, immutable Compressed Sparse Row (CSR) format designed for peak performance. All algorithms check if a graph is in a frozen state; therefore, it is impossible by design to run any graph algorithm on a dynamic graph. 

\textbf{The graph life-cycle:}

A Graph begins in a flexible DynamicGraph state, optimized for fast, O(1) mutations as the structure evolves. When the state has been finalized, and causal reasoning can begin, a single \textit{.freeze()} call transforms the graph into a hyper-optimized, immutable CsmGraph based on a cache-friendly Struct of Arrays (SoA) memory layout. This step is the key to eliminating cache misses and unlocking near-linear scaling. 
If the graph needs to evolve further, simply call \textit{.unfreeze()}. There is a memory trade-off in the state transition because, in each state, the memory usage is roughly $(e+v)$ whereas during a state transition, the memory usage temporarily peaks at $(e+v)^{2}$ and that warrants careful consideration when a graph grows large, i.e., beyond 1 billion nodes. 


\subsubsection{SoA CsrAdjacency Type}

In conventional CSR-based graph representations (like NWHypergraph), adjacency information is typically packed together row-wise in a "single structure" per edge or neighbor, which is technically an Array of Structs (AoS) layout, or in simple terms, “rows of neighbors.” UltraGraph, however, takes a different approach by introducing a CsrAdjacency<W> type that implements a Struct of Arrays (SoA) pattern:

\begin{lstlisting}[language=Rust, label={list:CsrAdjacency}, caption={UltraGraph: CsrAdjacency}]
#[derive(Default)]
pub(crate) struct CsrAdjacency<W> {
    pub(crate) offsets: Vec<usize>,
    pub(crate) targets: Vec<usize>,
    pub(crate) weights: Vec<W>,
}
\end{lstlisting}

\newpage

The CsrAdjacency layout divides each component of the adjacency data (offsets, targets, and weights) into separate, contiguous memory regions:

\begin{itemize}
\item offsets: Starting positions of each node’s adjacency list.
\item targets: The target node indices for each edge.
\item weights: Edge weights (optional).
\end{itemize}

The Struct of Arrays design of the CsrAdjacency has two advantages:


\textbf{Better cache utilization:} When performing traversal or shortest path algorithms, the CPU can stream just the fields it needs (often offsets and targets) without pulling in unnecessary weight data and thus avoiding cache pollution and improving CPU data prefecht.

\textbf{SIMD-friendliness:} SoA layouts enable vectorized processing (e.g., with AVX) far more easily 
than Array of Structs (AoS). Also, a SoA layout is easier for the compiler to optimize and thus unlock
meaningful performance gains without the need for complex optimization. 


\subsubsection{Separation of Forward and Backward CsrAdjacency}

Moreover, UltraGraph uses two separate CsrAdjacency instances: one for successor or outbound edges and another for backward or inbound edges. This dual-CSR setup is more explicit and efficient than mixing directions within a single row layout because it reduces CPU cache pollution and thereby directly supports fast and efficient algorithm implementations. UltraGraph deliberately traded a bit more
memory for better algorithm performance, as shown in the benchmarks.


\begin{lstlisting}[language=Rust, label={list:SeparatedCsrAdjacency}, caption={UltraGraph: Forward and Backward CsrAdjacency}]
#[derive(Clone)]
pub struct CsmGraph<N, W>
where
    N: Clone,
    W: Clone + Default,
{
    // Node payloads, indexed directly by `usize`.
    nodes: Vec<N>,
    // CsrAdjacency structure for forward traversal (successors).
    forward_edges: CsrAdjacency<W>,
    // CSR structure for backward traversal (predecessors).
    backward_edges: CsrAdjacency<W>,
    // Index of the designated root node.
    root_index: Option<usize>,
}
\end{lstlisting}


The backward node list is particularly useful in causality-based inference algorithms, where backtracking is often required, and is thus particularly well suited for DeepCausality. Memory usage remains low due to the combined effects of the Struct of Arrays layout and the clean separation between forward and backward adjacency. This design leads to a predictable, flat memory layout with minimal overhead:

\begin{itemize}
	\item No per-node allocation overhead.
	\item No padding, no vtables, no boxed pointers.
	\item No HashMaps, linked lists, or other complex types.
	\item No indexing needed due to simple offset.
	\item When a node has no inbound or outbound nodes or weights, there is zero allocation, thus saving memory.
\end{itemize}


Memory fragmentation is largely prevented because of the freeze/unfreeze operation in the graph evolution life-cycle. Calling \textit{.freeze()} compacts the structure, which removes any prior allocation gaps from the dynamic phase and thus results in a clean, continuous memory structure.

\subsubsection{Discussion}

This new implementation of UltraGraph based on its innovative two-stage graph life-cycle and its CPU-cache-friendly design enables causal graph reasoning at scale. While it is unlikely in practice to see causal graphs exceeding a million nodes, a context may grow large. UltraGraph confidently handles graphs up to a 100 million nodes on consumer-grade hardware and would only need a specialized high-memory server when the graph size is expected to exceed a billion nodes. At this scale, the underlying CSR implementation will most likely benefit from contemporary ReRAM-based\cite{zheng2023phgraph} hardware accelerators specifically designed to accelerate hypergraphs with sparse representation. 
Furthermore, the two-stage graph life-cycle directly supports the implementation of emergence in DeepCausality because it cleanly separates the mutation from the analysis phase and thus enables the four-stage process outlined in the EPP Ontology.  

\newpage
\subsection{DeepCausality}
\label{sec:implementation_deep_causality}

\newpage
\subsection{Benchmarks}
\label{sec:implementation_benchmarks}

\subsubsection{UltraGraph Benchmarks}

All benchmarks were completed on a 2023 Macbook Pro with a M3 Max CPU.

\textbf{Dynamic Graph Benchmarks:}

The dynamic graph structure, when the graph is in an unfrozen state, is optimized for efficient mutation. The table below summarizes the performance characteristics of the key operations.

% In your preamble, make sure you have: \usepackage{booktabs}
\begin{table}[h!]
\centering
\caption{Dynamic Graph Mutation Performance}
\label{tab:dynamic-graph-perf}
\begin{tabular}{lrlrr}
\toprule
\textbf{Benchmark Name} & \textbf{Graph Size} & \textbf{Operation} & \textbf{Estimated Time (Median)} & \textbf{Outliers Detected} \\
\midrule
\texttt{small\_add\_node}  & 10    & \texttt{add\_node} & 29.099 ns & 14\% (14 / 100) \\
\texttt{medium\_add\_node} & 100   & \texttt{add\_node} & 45.864 ns & 12\% (12 / 100) \\
\texttt{large\_add\_node}  & 1,000 & \texttt{add\_node} & 39.293 ns & 11\% (11 / 100) \\
\texttt{small\_get\_node}  & 10    & \texttt{get\_node} & 3.9417 ns & 8\% (8 / 100)  \\
\texttt{medium\_get\_node} & 100   & \texttt{get\_node} & 3.9849 ns & 2\% (2 / 100)  \\
\texttt{large\_get\_node}  & 1,000 & \texttt{get\_node} & 3.9916 ns & 7\% (7 / 100)  \\
\bottomrule
\end{tabular}
\end{table}

Benchmark source code is available in the UltrGraph Github repository\footnote{\url{https://github.com/deepcausality-rs/deep_causality/tree/main/ultragraph/benches}}.


\textbf{Static Graph Benchmarks:}

The new architecture causes the largest and most significant performance gains for algorithms running over large graphs (100k or more nodes) because of its close alignment with contemporary hardware.
By combining an instantaneous O(1) lookup with a perfectly linear scan over a node's neighbors, Ultragraph creates the ideal scenario for the CPU's prefetcher to easily anticipates a straight-line sprint through memory. The result becomes more notable the more data the prefetcher can load ahead of time, thus the disproportional performance gains on larger graphs.

% In your preamble, make sure you have: \usepackage{booktabs}
\begin{table}[h!]
\centering
\caption{Memory Usage and Scaling Performance on a Linear Graph}
\label{tab:memory-scaling}
\begin{tabular}{lrrrr}
\toprule
\textbf{Number of Nodes} & \textbf{Memory Usage} & \textbf{\texttt{eval\_subgraph}} & \textbf{\texttt{eval\_path}} & \textbf{\texttt{eval\_cause}} \\
\midrule
\textbf{100,000}         & 55 MB                 & 0.68 ms                         & 0.57 ms                      & \textbf{5.4 ns}               \\
\textbf{1,000,000}       & 350 MB                & 11.12 ms                        & 6.95 ms                      & \textbf{5.5 ns}               \\
\textbf{10,000,000}      & 3 GB                  & 114 ms                          & 85.80 ms                     & \textbf{5.6 ns}               \\
\textbf{100,000,000}     & 32 GB                 & 1.23 s                          & 0.98 s                       & \textbf{5.5 ns}               \\
\bottomrule
\end{tabular}
\end{table}

Benchmark source code is available in the DeepCausality Github repository\footnote{\url{https://github.com/deepcausality-rs/deep_causality/tree/main/deep_causality/benches}}.

\textbf{Observations}:

\textbf{Constant Time to get a single node:} The benchmark evaluate\_single\_cause returns always takes about 5.5. ns regardless of
 whether the node lookup happens before or during the benchmark loop and regardless of whether blackbox is used or not. 
 The time does not change with the size of the graph because the implementation of the underlying get\_node is just two O(1) array lookup 
 to find the index and than a straight redirect to a virtual memory address, which in this case, is close to the physical limit of the hardware memory architecture. 


\textbf{Near-Linear Scalability:} Both the memory usage and the execution time for the subgraph and shortest\_path tasks appear to scale in a roughly linear fashion with the number of nodes. 
A 10x increase in nodes results in a roughly 10x increase in time and memory. 

\textbf{All benchmarks are single-threaded.} The performance shown in all benchmarks is from a single core. Initial experiments showed that for graphs up to 1 million nodes, the overhead of even highly-optimized parallel libraries like rayon resulted in a  net performance loss of 30\% or more compared to the single-threaded version. This is a testament to the extreme  efficiency of the CSR layout when paired with modern CPU caches and prefetchers.  The results suggest that meaningful gains from concurrency will only appear on very large graphs (likely 10M-50M nodes  and above). However, this requires a concurrency model carefully designed to avoid the cache-invalidation issues common in work-stealing schedulers (used by rayon and Tokio). 

% \subsubsection{DeepCausality Benchmarks}


\newpage
%% ======================================================================
%% Validation and verification 
%% ======================================================================

\subsection{Validation and Verification}
\label{sec:implementation_validation}

\subsubsection{Overview}

Regulated industries such as avionics, robotics, or defense have to comply with strict regulatory requirements for software certification. Rust as a language provides a significant amount of safety relevant features that are difficult to achieve with C/C++ while still providing compatibility with existing code C/C++. Furthermore, the existence of a certified Rust compiler enables the validation and verification process of Rust systems in regulated industries. Build on this foundation, the DeepCausality project, while not certified, provides a plausible pathway towards certification. 

\subsubsection{Toolchain}

Rust a a language was chose because of its memory safety and the availability of certified compilers from vendors such
as Ferrocene\footnote{\url{https://ferrocene.dev/en}} (ISO 26262 ASIL D, IEC 61508 SIL 3, IEC 62304 / Class C) HighTec EDV-Systeme\footnote{\url{https://hightec-rt.com/products/rust-development-platform}} (ISO 26262 ASIL D), and AdaCore\footnote{\url{https://www.adacore.com/gnatpro-rust}}. 

\subsubsection{Hermetic Builds and Tests}

Beyond the language-level assurances, DeepCausality implements a robust build and test infrastructure specifically designed for high-integrity environments. Regulatory standards like ISO 26262 (Automotive Safety), IEC 61508 (Functional Safety), DO-178C (Avionics Software), and IEC 62304 (Medical Device Software) demand that the software build process fulfils:

\begin{itemize}
	\item Reproducibility: Given the same inputs, the system must always produce the same outputs. 
	\item Traceability: Every output (including test results) must be traceable back to its exact inputs.
	\item Independence: Tests must be independent of each other. The failure of one component should not mask or cause the failure of another.
	\item Controlled Environment: The testing environment must be fully specified and controlled.
\end{itemize}


The entire DeepCausality mono-repository is built and tested using the Bazel build system. Bazel's core principle of running tests in fully isolated, hermetic sandboxes ensures that each test executes independently, free from external state or interference from other tests. Bazel enforces hermeticity by:

\begin{itemize}
	\item Reproducible Inputs: Bazel requires all inputs (source code, compilers, libraries, tools) to be explicitly declared. This makes builds and tests fully reproducible.
	\item Isolation: Each test runs in its own isolated environment (a "sandbox"). This guarantees isolation and prevents side effects.
	\item Traceability: Each artifact in Bazel can be fully traced via the build graph. Bazel supports search and through the build graph which means one can quickly find all affected dependencies in case an issue has been found. 
\end{itemize}

\subsubsection{Design Principles}

The DeepCausality project has been designed with regulatory requirements in mind to streamline the validation and verification process that precedes certification. Specifically, DeepCausality is built upon three core principles that establish a baseline for trustworthiness:

\begin{itemize}
	\item Zero External Dependencies
	\item Zero Unsafe Code
	\item Zero Macros in Libraries
\end{itemize}

\newpage

\subsubsection{Zero External Dependencies}

The DeepCausality codebase, including its custom tensor and random number generation primitives, compiles without external third-party dependencies by default. The decision to implement a custom CausalTensor as foundation for the various algorithms used in the DeepCausality project as well as the decision to implement a custom random number generation that underpins the Uncertain crate fundamentally enabled the zero dependency achievement. Among other benefits, this ensures that the Effect Ethos's deontic inference process operates with predictable and auditable determinism, which is critical for proofing end to end alignment. 
The zero external dependency principle reduces supply chain risk by eliminating unknown code from external sources and effective mitigates a key security concern in high-integrity systems. It ensures that the entire intellectual supply chain is auditable and simplifies cross-platform compilation for diverse embedded hardware targets.


The only exception is an optional feature flag, os-random, in the deep\_causality\_rand crate. When enabled, this flag introduces a dependency on a thin Rust wrapper for libc to access the operating system’s secure random number generator. This is clearly documented and is strictly opt-in for developers who require a cryptographically secure random numbers provided by the host system. Alternatively, the deep\_causality\_rand crate also contains a flexible trait that exposes the random number generator (RNG) functionality. For regulated industries, the trait-based architecture for RNG allows them to seamlessly swap in their own certified hardware RNG binding by simply implementing one trait, which means DeepCausality becomes compatible with  existing certified hardware. 

For the development process, there are a handful of external dependencies i.e. Criterion for running benchmarks or CSV for reading data files, but these are very limited, all contained outside the DeepCausality codebase, and only applicable to the development process with the understanding that none of these will be part of a production build.  

\subsubsection{Zero Unsafe Code}

The entire DeepCausality codebase including all internal dependencies, examples, and tests, are implemented entirely in safe Rust. This preserves Rust's guarantee of memory safety and therefore eliminating an entire class of catastrophic failures (e.g., buffer overflows, data races) at compile time. It significantly reduces the burden of memory safety audits, which are typically one of the most complex and costly aspects of high-integrity software.

\subsubsection{Zero Macros in Libraries}

The entire DeepCausality codebase is free of macros to ensures maximum code clarity and audibility. The entire codebase can be read, understood, and analyzed line-by-line, which is essential for formal verification processes and human review in regulated environments. The only area where macros had to be used is in bulk testing for generic traits as, for example, in the deep\_causality\_num trait. The reason is quite obvious, core numerical traits such as "mul" that defines the multiplication operation for every numerical type would require identical tests for every possible numerical type in Rust. In those cases, a macro is used to derive the test mainly to keep the test files largely free of duplicates. It is important to understand that this practice is contained only to tests and only applies to very specific foundational crates i.e. deep\_causality\_rand and deep\_causality\_num. 


These key principles are complemented by a rigorous testing regimen that maintains a 3-month rolling average code coverage of 97\% across its 80,000+ lines of Rust code. While code coverage does not constitute formal verification, it is a critical quantitative indicator of testing thoroughness. For certification bodies, this high coverage, coupled with the inherent safety of Rust and the absence of external dependencies, substantially lowers the time, effort and cost associated with verification and validation activities. Combined, these three key principles enable a number of properties particular relevant to regulated industries:

\begin{itemize}
	\item Supply Chain Security: By being entirely self-contained, DeepCausality guarantees the integrity of its code. 
	\item Cross-Compilation \& Embedded Systems: The std-only approach makes DeepCausality inherently portable across a vast array of embedded target platforms supported by Rust. 
	\item Performance \& Development Velocity: Because the DeepCausality codebase is self-contained, the entire mono-repository (est. 80K Lines of Code), completes a full clean build within 1o seconds translates directly to developer productivity. 
\end{itemize}

The DeepCausality project provides a clear and plausible pathway towards regulatory certification because of the existence of a certified Rust compiler, the already established hermetic build system, and the achieved zero dependencies, unsafe, and macros best practice across the entire project codebase. Systems build with DeepCausality that seek certification still need to comply with all regulatory requirements, but because of the careful preparations already made, the process of validation and verification becomes manageable for an experienced team.

\newpage


%% Future Work
\section{Future Work}
\label{sec:future_work}


The preceding chapters have established the Effect Propagation Process as a conceptual and formal framework for modeling dynamic causality.
The framework's philosophical underpinnings have informed its implementation in DeepCausality. The framework's core principle,
Higher-Order Emergence, provides a formal language for describing systems capable of recursively evolving their own causal and contextual structures.

The EPP, in its most advanced modalities, operates in a reality where the classical pillars of verification and trust are no longer guaranteed,
as established in the Epistemology. However,the very power of this principle creates a set of three profound crisis,
as foreshadowed in the metaphysics:

\begin{enumerate}
    \item \textbf{The Crisis of Justification:} In a system where new causal rules and contexts are constantly emerging, the fixed principles needed for classical justification disappear.
    \item \textbf{The Crisis of Truth:} In a system that co-evolves with its factual Context, the stable, external reality required for a correspondence theory of truth dissolves.
    \item \textbf{The Crisis of Explainability:} It might not be possible any longer to explain the outcome because of the previous crisis of truth and the crisis of justification.
\end{enumerate}


These crises are fundamental properties of higher-order emergence encoded in the EPP and thus demands a new class of ontological primitive for a normative framework that shifts the anchor from epistemology (what is true) to teleology (what is its purpose).
The Effect Propagation Process therefore proposes two new, first-class ontological primitives:

\begin{itemize}
    \item \textbf{The Teloid:} A computable unit of purpose. Functioning as a prospective guard of intent, a Teloid would be a
    verifiable function that intercepts a proposed action from a Causal State Machine and evaluates it against a defined
    goal or policy before execution. This introduces a real, deliberative step of teleological verification against stated
    intent deeply integrated into the system's core reasoning engine.
    \item \textbf{The Effect Ethos:}  A framework for validating outcomes. Functioning as a retrospective validator, the Effect Ethos
    would assess the holistic, emergent state of the system after a reasoning cycle to ensure fundamental principles such
    as safety, fairness, or regulatory compliance have been upheld. The Effect Ethos would leverage the EPP's isomorphic
    design to construct a verifiable 'machine ethos' from simpler Teloid primitives, creating a composable and mechanistic
    ethical framework from first principles. Instead of external post-hoc analysis, the proposed Effect Ethos would become
    an integral part of the EPP and its implementation DeepCausality.
\end{itemize}

When combined, the Teloid and Effect Ethos, form a plausible architecture within which ethics becomes a computable and
verifiable. The distinction between a prospective "Teloid" (guarding actions) and a retrospective "Effect Ethos"
(validating outcomes) exists for a specific reason. A proposes action A proposed action, i.e., "shut down air-flow,"
can be vetted upfront against a set of codified rules to prevent catastrophic failures before they can happen.
However, a reasoning outcome, especially when the reasoning is conducted throughout a complex causal hypergraph
connected to multiple static and dynamic contexts, can only be evaluated after completion.
Many real-world ethical dilemmas involve balancing a locally "correct" action (which a Teloid might permit) against a
holistically undesirable emergent outcome that the Effect Ethos may prevent. For instance, a series of
individually-approved financial trades could, in aggregate, run against global risk management.
The Effect Ethos provides the necessary tools for this kind of holistic and balanced systemic validation.

The concepts of the Teloid and Effect Ethos are directly recognizable as "computable policy" and "auditable safety
layers" that broadly translate into two new categories:

\begin{itemize}
    \item \textbf{Compliance-as-Code:} The idea of modular Teloids for regulations (e.g., a "Reg-T Teloid") that could be audited
    directly would lower regulatory risk (fines) and operational cost (standardization).
    \item \textbf{Verifiable Safety for Autonomous Systems:} This provides a concrete architecture for satisfying safety standards (
    like ISO 26262 for automotive), which is currently a major challenge for any autonomous systems.
\end{itemize}


One practical application of Compliance-as-Code would be the formal verification of adherence to regulatory requirements
directly embedded into the model itself. It is not unthinkable that regulators might want to see audits of the codifying
teloids as a means to ascertain and monitor regulatory compliance. Another practical application is the development of
modular reference Teloids that codify specific regulations for certain domains with mandatory industry rules,
for example in finance, to lower the cost of compliance. For autonomous systems, embedding specific safety rules
becomes not only streamlined, but easier to audit, verify, and simulate. Lastly, while neither the Teloid nor the Effect
Ethos can directly answer the question of whether a specific inference or proposed action is the right thing
with respect to its context, at least these are feasible primitives to build a solution to answer those questions.

Challenges will arise mostly from formalization and verification of the proposed Teloid and Effect Ethos. Specifically,
at least the following questions need to be addressed in future development:

\begin{itemize}
    \item How do we formally verify the Teloid itself
    \item How do we prove that a composite Effect Ethos is complete and covers all necessary edge cases?
    \item How do we prove, even if a composite Effect Ethos is correct, that it will be deterministically applied?
\end{itemize}

The Teloid and Effect Ethos are presented as future work since developing because these immense challenges clearly
fall outside the scope of the presented EPP, but still warrant further consideration.
While the formalization is
subject for extensive future work, the implementation can re-use existing concepts and primitives already built in
DeepCausality and thus substantiate the feasibility of the proposal. For the actual implementation, the EPP and its
implementation DeepCausality, provides the staging ground because:

\begin{itemize}
    \item Real-world safety problems are not confined to simple geometries. Avionics and robotics safety needs native support
    for non-Euclidean geometries.
    \item Ethics never occurs in a vacuum. Therefore, an Effect Ethos requires multi-contextual support.
    \item Holistic ethical outcomes are emergent properties, thus dynamic and emergent causality are necessary to capture these.
\end{itemize}

The capability for higher-order emergence carries the risk of uncontrolled or undesirable system evolution. The "Crisis of Truth" is not a theoretical abstraction but a practical safety concern. The proposed architecture of the Teloid and Effect Ethos is the primary mechanism for managing the risks that result from dynamic emergence. The Teloid can be engineered to constrain the generative process by rejecting proposed structural modifications that violate predefined safety, ethical, or operational policies. However, no set of prospective rules can be proven complete. The retrospective Effect Ethos provides a second layer of defense, assessing holistic outcomes where individually correct actions might lead to an undesirable emergent state.

It is crucial, however, to recognize the pragmatic reality of applying EPP: real-world systems will be hybrid models. The majority of their components will be static or governed by predictable dynamics. Only a small but critical subset of the system will be designed to be truly emergent.
Traditional brute-force testing is computationally infeasible due to combinatorial explosion.
Likewise, formal verification, while powerful for deterministic systems, may not be applicable to a system whose state space can evolve dynamically relative to a dynamic context.
The most viable and rigorous path forward is adversarial stress-testing of the teloids and effect ethos.
It is possible to systematically search for emergent loopholes and stress-test the Effect Ethos
by using Deep Reinforcement Learning to intelligently and adversarially explore the state space of the learned world model.

Adversarial stress-testing does not offer absolute safety guarantees. The potential for unforeseen behavior in a sufficiently complex system remains, as risk is intrinsic to the nature of dynamic emergence. It represent, however, a principled and practical engineering discipline for managing that unavoidable risk.
The alternative is to either forgo the benefits of adaptive dynamic systems or to deploy them without a comparably rigorous validation strategy.
The proposed Teloid and Effect Ethos, validated through adversarial stress-testing, serve as the tools for navigating causal emergence responsibly.

Managing the intrinsic risk of emergent causality is not a challenge for a single methodology; the problem represents an ongoing challenge for the fields of AI safety, formal verification, and causality. The EPP, with its transparent and auditable architecture, is therefore offered as a high-fidelity testbed for exploring these foundational issues.
The author acknowledges that the exploration of causal emergence requires deep inquiry, probing questions, and different perspectives from a multitude of diverse stakeholders.
The transparent and open-governance of the DeepCausality project, hosted at the LF AI \& Data Foundation, provides a vendor-neutral venue for facilitating such an essential discussion.

\newpage

%% Epilogue
\section{Epilogue}
\label{sec:epilogue}

 The Effect Propagation Process, as detailed in the preceding sections, provides a consistent, first-principles foundation for dynamic causality and a new philosophy grounded language to study dynamic, emergent, and contextual causality.
 
The metaphysics of Effect Propagation Process establishes the essence of the EPP and establishes its core dynamics that lead to an orthogonal design that consistently applies to all levels of the EPP. 

The ontology of the Effect Propagation Process foreshadows the complex structures the EPP is designed to model by scaling the modality of the EPP. For a static EPP, a positivist epistemology remains sufficient. For a dynamic EPP, the epistemology evolves towards an interpretivism perspective, and for an emergent EPP, a pragmatism perspective on the epistemology becomes necessary.

The epistemology of the EPP explores the meaning truth and how it scales with the modality of the EPP. For a static EPP, the meaning of truth aligns with the classical correspondence theory. However, in a dynamic EPP, the meaning of truth shifts towards a coherent adaptability approach. In an emergent EPP, the meaning of truth evolves towards pragmatic efficacy where the validity of relativistic, emergent causal relationships is established by their functional utility.


The implementation of the EPP in the DeepCausality\footnote{\url{https://deepcausality.com}} demonstrates the practicality of hypergeometric computational causality for fast context-aware causal reasoning across Euclidean and non-Euclidean spaces. Furthermore, DeepCausality, as a reference implementation, provides an excellent foundation to study dynamic causality further. And indeed, the new foundation of dynamic causality already suggests  several domains of inquiry: 

\begin{itemize}
	\item \textbf{The Formalization of Causal Emergence:} While the EPP introduced causal emergence, its foundation is far from settled therefore the study of causal emergence promises further discoveries.
	\item \textbf{The Dynamics of Context:} The EPP established external context and its modalities, but especially the intersection between a dynamically changing environment and a dynamic context offers promising area for study. This is particular valuable for control systems in autonomous unmanned vehicles that have to adapt to new terrain. 
	\item \textbf{A Calculus of Purpose:} The Teloid and Effect Ethos are introduced as ontological primitives. The next logical step is the formalization of a calculus to establish the foundation for verifiable teleological reasoning that enables the design and validation of systems whose adherence to specified intent or regulations is mathematically provable. This line of study might be particular valuable for regulated industries such as robotics, avionics, and defense. 
	\item \textbf{The Teleology of Emergence:} Directly related to the calculus of purpose follows the study to teleological boundaries of causal emergence which is particular relevant to the development of safe emergent algorithms. 
	\item \textbf{Certification of Regulatory Requirements:} The Teloid provides a direct, transparent, traceable, and auditable link between a system's behavior and a codified safety-critical rule. Therefore, further study is warranted to explore the applicability of the EPP and Teloid foundation to the world of rigorous certification (e.g., DO-178C) to solidify operational trust.  
\end{itemize}


The presented EPP serves as an open invitation. It is an invitation to the architects and engineers of complex systems who require formal proof of safety and explainability. It is an invitation to the underwriters of risk, who must quantify, model, and price complex risks that is contextual and dynamic. And it is an invitation to the regulators and policymakers tasked with ensuring the safe and compliant integration of complex systems into the fabric of society. Trust is the single most valuable asset any society has and the EPP exists to facilitate trust building and certification of complex dynamic causal systems. 

\newpage

%% Appendix A
\appendix
\section{Appendix: Preliminary Performance Benchmarks}
\label{app:benchmarks}

The following tables summarize preliminary performance benchmarks for key DeepCausality operations, executed on an Apple MacBook Pro with an M3 Max processor. These benchmarks were run using the standard Rust `criterion` library (`cargo bench`). The benchmark code is available on Github\footnote{https://github.com/deepcausality-rs/deep\_causality/tree/main/deep\_causality/benches}. It is important to note that these are synthetic benchmarks, primarily utilizing linear or simple multi-layer graph structures, and do not involve complex context interactions or GPU acceleration. They provide an initial indication of the framework's CPU-based performance characteristics at different scales.

The scales are defined as follows:
\begin{itemize}
    \item Small: 10 Causaloids (for collections/maps/graphs)
    \item Medium: 1,000 Causaloids
    \item Large: 10,000 Causaloids
\end{itemize}

Table~\ref{tab:bench_collections} presents results for reasoning over collections (Vector/Array) and Maps (HashMap/BTreeMap) containing Causaloids. Table~\ref{tab:bench_linear_graphs} details performance for various reasoning tasks on linear `CausaloidGraph` structures. Table~\ref{tab:bench_multilayer_graphs} shows results for reasoning on small multi-layer graphs (benchmarks for medium/large multi-layer graphs were not included in the provided results). All times reported are the typical execution times (median/mean) derived from multiple samples.

% Table for Collection and Map Reasoning
\begin{table}[htbp]
    \centering
    \caption{Benchmark Results: Reasoning over Causaloid Collections and Maps.}
    \label{tab:bench_collections}
    \begin{tabular}{llr}
        \toprule
        Operation Description                 & Scale (Size) & Typical Time \\
        \midrule
        Collection (`Vec`/`Array`) Reasoning  & Small (10)   & \SI{101.01}{ns} \\
                                              & Medium (1k)  & \SI{3.9123}{\micro s} \\
                                              & Large (10k)  & \SI{43.979}{\micro s} \\
        \midrule
        Map (`HashMap`/`BTreeMap`) Reasoning & Small (10)   & \SI{50.365}{ns} \\
                                              & Medium (1k)  & \SI{4.6652}{\micro s} \\
                                              & Large (10k)  & \SI{50.797}{\micro s} \\
        \bottomrule
    \end{tabular}
\end{table}

% Table for Linear Graph Reasoning
\begin{table}[htbp]
    \centering
    \caption{Benchmark Results: Reasoning over Linear Causaloid Graphs.}
    \label{tab:bench_linear_graphs}
    \begin{tabular}{llr}
        \toprule
        Reasoning Task on Graph           & Scale (Nodes) & Typical Time \\
        \midrule
        Reason over All Causes            & Small (10)    & \SI{2.7601}{\micro s} \\
                                          & Medium (1k)   & \SI{509.94}{\micro s} \\
                                          & Large (10k)   & \SI{70.221}{ms} \\
        \midrule
        Reason over Subgraph from Cause   & Small (10)    & \SI{1.5070}{\micro s} \\
                                          & Medium (1k)   & \SI{245.25}{\micro s} \\
                                          & Large (10k)   & \SI{34.933}{ms} \\
        \midrule
        Reason Shortest Path Between Causes & Small (10)    & \SI{1.6900}{\micro s} \\
                                          & Medium (1k)   & \SI{286.08}{\micro s} \\
                                          & Large (10k)   & \SI{35.424}{ms} \\
        \midrule
        Reason over Single Cause          & Small (10)    & \SI{10.349}{ns} \\
                                          & Medium (1k)   & \SI{9.9537}{ns} \\
                                          & Large (10k)   & \SI{10.010}{ns} \\
        \bottomrule
    \end{tabular}
\end{table}

\newpage

% Table for Multi-Layer Graph Reasoning (Small Scale Only)
\begin{table}[htbp]
    \centering
    \caption{Benchmark Results: Reasoning over Small Multi-Layer Causaloid Graphs.}
    \label{tab:bench_multilayer_graphs}
    \begin{tabular}{llr}
        \toprule
        Reasoning Task on Graph           & Scale (Nodes) & Typical Time \\
        \midrule
        Reason over All Causes            & Small (10)    & \SI{1.2483}{\micro s} \\
        Reason over Subgraph from Cause   & Small (10)    & \SI{489.42}{ns} \\
        Reason Shortest Path Between Causes & Small (10)    & \SI{427.45}{ns} \\
        Reason over Single Cause          & Small (10)    & \SI{10.252}{ns} \\
        \bottomrule
    \end{tabular}
\end{table}

These preliminary results demonstrate efficient CPU-bound performance for reasoning tasks on the tested structures and scales, particularly highlighting the near-constant O(1) time for single cause lookups, irrespective of graph size. The scaling for full graph traversals appears roughly linear or slightly super-linear for the linear graphs tested up to 10,000 nodes. However, as noted in the Discussion, further benchmarking on more complex, densely connected hypergraphs and under various context interaction scenarios is necessary to fully characterize scalability limits. The raw benchmark output also indicated the presence of outliers in some measurements, suggesting potential sources of variability (e.g., cache effects, OS scheduling) that warrant deeper investigation in future performance analyses.


%% Glossary
\section*{Glossary}
\addcontentsline{toc}{section}{Glossary} % Add it to the Table of Contents

This glossary provides definitions for the key formal terms and symbols used throughout this paper to describe the Effect Propagation Process (EPP) and its components.

\begin{description}[style=nextline] % Using [style=nextline] if you have enumitem package and want this style

    \item[Effect Propagation Process (EPP)] 
    The overarching philosophical framework and the formalized dynamic process (\(\Pi_{EPP}\)) of effect transfer within a CausaloidGraph, conditioned by the Contextual Fabric.

    \item[Contextual Fabric (\(\mathcal{C}\))] 
    The structured environment, composed of Context Hypergraphs, within which effects propagate and causal relationships are conditioned. It is formalized as a Context Collection \(\mathcal{C}_{sys}\).

    \item[Contextoid (\(v\))] 
    The atomic unit of contextual information, defined as a tuple \(v = (id_v, \text{payload}_v, \text{adj}_v)\). It encapsulates data, time, space, or spacetime values. (See Definition 3.1)

    \item[Contextoid Payload (\(\text{payload}_v\))] 
    The tagged union representing the actual data, temporal, spatial, or spatiotemporal value held by a Contextoid.

    \item[Context Hypergraph (\(C\))] 
    A structured collection of Contextoids (\(V_C\)) and the N-ary relationships (Hyperedges \(E_C\)) between them, defined as \(C = (V_C, E_C, ID_C, \text{Name}_C)\). (See Definition 3.2)

    \item[Context Collection (\(\mathcal{C}_{sys}\))] 
    A finite set of distinct Context Hypergraphs, \(\mathcal{C}_{sys} = \{C_1, C_2, \dots, C_k\}\), representing the total available contextual information. (See Definition 3.4)

    \item[Context Accessor (\(\text{ContextAccessor}(\mathcal{C}_{refs})\))] 
    A functional interface providing read-access for Causaloids to a specified subset of Context Hypergraphs (\(\mathcal{C}_{refs} \subseteq \mathcal{C}_{sys}\)). (See Definition 3.5)

    \item[Causaloid (\(\chi\))] 
    The fundamental, operational unit of causal interaction, defined as \( \chi = (id_\chi, \text{type}_\chi, f_\chi, \mathcal{C}_{refs}, \text{desc}_\chi, \mathcal{A}_{linked}, I_{linked}) \). It encapsulates a testable mechanism for effect transfer. (See Definition 4.1)

    \item[Causaloid Type (\(\text{type}_\chi\))] 
    Specifies the structural nature of a Causaloid (Singleton, Collection, or Graph), determining how its causal function \(f_\chi\) is realized.

    \item[Causal Function (\(f_\chi\))] 
    The operational logic associated with a Causaloid \(\chi\), which maps input observations (\(\mathcal{O}_{\text{type}}\)) and accessed context to an activation status (\(\{\text{true, false}\}\)). (See Section 4.2.2)

    \item[CausaloidGraph (\(G\))] 
    A hypergraph, defined as \( G = (V_G, E_G, ID_G, \text{Name}_G) \), whose nodes (\(V_G\)) typically encapsulate Causaloids and whose hyperedges (\(E_G\)) define pathways and logic for effect propagation. Supports recursive isomorphism. (See Definition 4.2)

    \item[Causal Node (\(v_g\))] 
    A node within a CausaloidGraph \(G\), defined as \(v_g = (id_g, \text{payload}_g)\), where \(\text{payload}_g\) can be a Causaloid, a collection of Causaloids, or another CausaloidGraph.

    \item[Causal Hyperedge (\(e_g\))] 
    A hyperedge within a CausaloidGraph \(G\), defined as \(e_g = (V_{\text{source}}, V_{\text{target}}, \text{logic}_e)\), representing a directed functional relationship.

    \item[State of CausaloidGraph (\(S_G\))] 
    A function \(S_G: V_G \to \{\text{active}, \text{inactive}\}\) mapping each causal node in a CausaloidGraph to its current activation status. (See Definition 4.3)

    \item[Effect (\(\varepsilon\))] 
    Primarily the activation state (active/inactive) of a Causaloid within EPP, and the transfer of this status or derived information. (See Section 5.1.1)

    \item[Input Observations/Triggers (\(O_{input}, O_{trig}\))] 
    External data or events (\(O_{input}\) being the set of all possible, \(O_{trig}\) being the current set) that can initiate or influence the Effect Propagation Process. (See Definition 5.1)

    \item[EPP Transition Function (\(\Pi_{EPP}\))] 
    The core dynamic function \( \Pi_{EPP} : (G, S_G, \mathcal{C}_{sys}, O_{trig}) \to S_G' \), describing how the state of a CausaloidGraph evolves. (See Definition 5.2)

    \item[Operational Generative Function (\(\Phi_{gen}\))] 
    A formal function or set of functions representing meta-level rules for the dynamic evolution of the Contextual Fabric (\(\Phi_{gen\_C}\)), CausaloidGraphs (\(\Phi_{gen\_G}\)), or both (\(\Phi_{gen\_Total}\)). (See Section 5.2)

    \item[Causal State Machine (CSM, \(M\))] 
    An operational component, defined by its set of state-action pairs \(\mathcal{SA}_M\), that links recognized Causal States (\(q\)) to deterministic Causal Actions (\(a\)). (See Definition 6.3)

    \item[Causal State (\(q\))] 
    A specific condition within a CSM, \( q = (id_q, \text{data}_q, \chi_q, \text{version}_q) \), whose activation is determined by its associated Causaloid \(\chi_q\). (See Definition 6.1)

    \item[Causal Action (\(a\))] 
    A deterministic operation within a CSM, \( a = (\text{exec}_a, \text{descr}_a, \text{version}_a) \), triggered by an active Causal State. (See Definition 6.2)

\end{description}

%% Bibliography
\newpage
\bibliographystyle{unsrt}  
\addcontentsline{toc}{section}{References} % Add it to the Table of Contents
\bibliography{../epp_bib/references}  

\end{document}
