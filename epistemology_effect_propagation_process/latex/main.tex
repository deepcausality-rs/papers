\documentclass{article}


\usepackage{PRIMEarxiv}
\usepackage[table,xcdraw]{xcolor}
\usepackage[utf8]{inputenc} % allow utf-8 input
\usepackage[T1]{fontenc}    % use 8-bit T1 fonts
\usepackage{hyperref}       % hyperlinks
\usepackage{url}            % simple URL typesetting
\usepackage{booktabs}       % professional-quality tables
\usepackage{amsfonts}       % blackboard math symbols
\usepackage{nicefrac}       % compact symbols for 1/2, etc.
\usepackage{microtype}      % microtypography
\usepackage{lipsum}
\usepackage{fancyhdr}       % header
\usepackage{graphicx}       % graphics
\graphicspath{{media/}}     % organize your images and other figures under media/ folder

%Header
\pagestyle{fancy}
\thispagestyle{empty}
\rhead{ \textit{ }} 

% Update your Headers here
\fancyhead[LO]{The Epistemology of the Effect Propagation Process}
% \fancyhead[RE]{Firstauthor and Secondauthor} % Firstauthor et al. if more than 2 - must use \documentclass[twoside]{article}

  
%% Title
\title{The Epistemology of the Effect Propagation Process}

%% Author
\author{
  Marvin Hansen \\
  \texttt{marvin.hansen@gmail.com} \\
   Date: \today
}


\begin{document}
\maketitle

\begin{abstract}
The Effect Propagation Process (EPP) provides a spacetime-agnostic framework for computational causality, distinct from classical causality in its explicit handling of effect transmission. The EPP can operate in static, dynamic, or emergent modalities with profound epistemological implications.


The epistemology of the EPP begins with positivist foundations in static configurations where truth aligns with a predefined model. In a dynamic EPP, epistemology requires an interpretive stance where truth emphasizes coherent adaptability amidst non-linear temporal evolution. Lastly, its epistemology evolves further into a pragmatic understanding in co-emergent systems where the truth of relativistic, emergent causal relationships is validated by their functional efficacy.


The EPP framework serves as a philosophical foundation for advanced causal reasoning and enables relativistic, emergent causal relationships across complex non-linear time representations. Its epistemology enables novel explorations of understanding evolving causal knowledge in emergent systems and highlights the necessity for further research into their empirical validation.
\end{abstract}

\section{Definitions}
\label{sec:Definitions}

\subsection{Epistemology}
\label{subsec:Def_Epistemology}

Epistemology refers to the study of understanding knowledge and sets out to explain the sources of knowledge, how we derive knowledge, and the distinction between justified belief and opinion.

\subsection{Knowledge}
\label{subsec:Def_Knowledge}

A consensus is that knowledge is defined as justified true belief. Thus, the three key components of knowledge include truth, belief, and justification. By truth, it is meant that something is in correspondence to facts or reality. Belief refers to the state of mind that a person has about reality and truth.
When considering a proposition, belief results in three different states of mind: We may believe and accept it as true, disbelief and rejection as false, or withhold belief for further judgment. The last key component of knowledge is justification.
A true belief must entail a high degree of justification to count as knowledge. Sources of knowledge include, but are not limited, to perception, memory, reason, and testimony. Knowledge must come from a reliable source to be considered true. Epistemological reasoning broadly falls into two categories: a priori (non-empirical i.e. deduction) and a posteriori (empirical) justification of knowledge.


\subsection{Positivism, Interpretivism, and Pragmatism.}
\label{subsec:Def_Positivism}

Epistemological approaches to acquiring knowledge in research fall into three categories: positivism, interpretivism, and pragmatism. Positivism concerns itself with observable facts based on the scientific method and thus seeks to achieve generalizability and objectivity. Interpretivism maintains that our knowledge depends greatly on our interpretation of observations of human actions, experiences, and environments thus making interpretive research more subjective. Pragmatism focuses on practical effects or solutions to address problems that are suitable for existing situations or conditions. The epistemology of pragmatism is that knowledge is a self-correcting process based on experience thus, it must be evaluated and revised in view of subsequent experience.

\subsection{Classical Causality}
\label{subsec:Classical_Causality}

The classical definition of causality, taken from Judea Pearl's foundational work\cite{pearl2009causality}:


\begin{quote}
\begin{center}
        IF (cause) A then (effect) B

    AND 
    
    IF NOT (cause) A, then NOT (effect) B
\end{center}
\end{quote}


The classical definition does not entail a transmitter of the causal effect and thus assumes an engulfing spacetime. It is a subtle, but important assumption that, among others, requires time linearity to preserve the introduced time asymmetry. Time asymmetry follows from the directional definition (via the negation, if-not-then-not) that the causal relationship must flow from the cause to the effect.


\subsection{Effect Propagation Process}
\label{subsec:Effect_Propagation_Process}

The effect propagation process generalizes causality by removing the spacetime constraint by replacing the implicitly assumed spacetime with the stated propagation which serves as an effect transmitter. The generalized definition reads:

\begin{quote}
\begin{center}
        If X happens, then its effect propagates to Effect E.
    
    AND
    
    If X does not happen, then its effect does not propagate to Effect E.
\end{center}
\end{quote}


In this definition, X does not have a designated label and instead is described in terms of its emitting effect. Therefore, X can be seen as a preceding effect, which then propagates its effect further, and thus causality becomes an effect propagation process. The propagation does not entail any notion of time linearity because the emitting effect flows from one effect to another, but crucially, in the absence of the previous cause/effect distinction, the implied assumption of linear temporal order has been removed.
Furthermore, the definition does not imply any spatial constraints and thus makes the definition fundamentally spacetime agnostic. The logical constraints (if-then and if-not-then-not) remain present to preserve the essence of causality and maintain a clear distinction from correlation.

Because X and E are seen as effects, there is no meaningful distinction between cause and effect any longer. This directly leads to the EPP’s adoption of the Causaloid, a uniform entity proposed by Hardy\cite{HardyDynamicCausalStructure}, that merges the ‘cause' and 'effect' into one entity. Instead of dealing with two nearly identical concepts discernible from each other by temporal order, the causaloid is one concept that defines causality in terms of its effect transfer without presupposing a fixed spacetime background\cite{HardyDynamicCausalStructure}.

In EPP, effect propagation denotes the operation of a specific, definable mechanism (via the causaloid) that links states through the underlying fabric regardless of how that fabric might be defined. The first part, the causaloid, operationalizes effect transfer within a system without relying on any metaphysical causal power. The effect transfer encapsulated in the causaloid is intrinsic to its mechanistic definition. The identification of relevant causaloids requires explicit modeling and hypothesis testing or (future) discovery processes. 

EPP provides the formal structure for the representation of Causeloids once discovered. The second part, the non-defined fabric, through which effects propagate, serves the purpose of externalizing the fabric as an explicitly defined context. Instead of assuming an implicit background spacetime, the EPP externalizes the fabric through which effects propagate as a specific context that is defined by a different generative function. Therefore, EPP fundamentally embraces definable, testable, and verifiable functions (Causaloids) that define the functional relationships describing how a system changes if these functions are active.

Furthermore, the Effect Propagation Process might be static in the sense of being defined ahead of time, dynamic in the sense of being defined at runtime from a generative function, or emerging in the sense that the EPP results from previous generative stages of itself.

\newpage

\subsection{Externalized Context}
\label{subsec:Externalized_Context}

The Effect Propagation Process generalizes the preceding internal implicit spacetime assumption by replacing it with an external and explicitly defined context.
A context can be static or dynamic, depending on the situation. The context structure is defined beforehand for a static context, whereas for a dynamic context, the structure is generated dynamically at runtime from a generative function.

Unlike the EPP, a context is constrained from being recursive. Self-referential recursion in the absence of a stop criterion may lead to infinite time or even spacetime loops, which conflict with explainability and touches upon exotic scientific ideas. Therefore, for EPP's current formulation, context is constrained from being recursive to prevent problematic corner cases and to ensure a straightforward implementation of the presented concepts.

Regardless of the specific structure, the context is explicitly defined, either by a (static) structure or by a generating function that generates the context structure dynamically.

Data in any context changes over time, hence requiring constant updating. A change of context structure may happen and may affect the effect propagation process. Lastly, the need to handle multiple contexts emerges from the simple need to address multiple sources of contextual data.

With the separation of the Effect Propagation Process and its context, several interrelations emerge that directly impact the understanding of derived knowledge. Furthermore, since the context serves as the factual source of the EPP, and the EPP defines generalized causal inference, the resulting epistemology depends on the structure of the context and the EPP.


\section{The Epistemology of the Effect Propagation Process}
\label{sec:Epistemology}

The presented epistemology entails the study of understanding the sources of knowledge, how we derive knowledge, and how we determine the truthfulness of the derived knowledge within a contextualized Effect Propagation Process. The epistemology changes depending on whether the context is static or dynamic, and, equally profound, whether the EPP is static, dynamic, or emergent.

\textbf{Ontology of Knowledge sources}

In the EPP, the context is designed as the source of factual knowledge. For context, facts may remain invariant (e.g. the value of Pi) or receive continuous updates. The designation whether a context is static or dynamic refers to its structure, not to the factual data in it. Furthermore, a context might be shared between two or more defined EPP and an EPP may use one or more context(s) thus simplifying modeling complex domains.
The EPP encodes each causal relationship in a designated Causaloid. The Causaloid encodes the causal rule, whereas the context encodes supporting data required to apply the rules. The Causaloid may use external data or data from the context to apply its rule.

For example, a context may encode a continuous signal feed from a LIDAR sensor and the Causaloid encodes a rule to infer whether an obstacle has been detected. In this case, the context provides all data. In another scenario, a context may encode several known defect patterns, a Causaloid tests incoming image data for the defect data from the context, but uses incoming real-time image feeds from a manufacturing monitoring system to determine if any of the produced items contain known defects. In this case, the Causaloid relies on context and external data. Therefore, the Effect Propagation Process emits a flexible knowledge ontology by relying on one or more contexts and potentially multiple external data sources.

\textbf{Knowledge Derivation}

The EPP derivates knowledge by applying data to the Causaloid that models that causal relationship to determine whether the causal relation holds true within the applicable context. Consequently, multi-stage reasoning maps directly to the topology of the EPP itself because each effect from a Causaloid propagates further through the EPP topology, which is the structure of the EPP manifested as all connected Causaloids.
From this perspective, a “line of reasoning” literally becomes a pathway through the EPP topology.
Through the topological approach of knowledge derivation, the Effect Propagation Process provides a flexible way to model complex, contextual, multi-causal domains.

\newpage

\textbf{Justification of derived knowledge}

Discerning the truthfulness of knowledge is one key element of epistemology. The EPP with its explicit context, explicit causal function (Causaloid), and explicit support for external data provides all pre-requirements to support the full explainability of each inference. Furthermore, in the case of multi-stage reasoning, the sequence of applied Causaloids establishes the order or explainability.
Fundamentally, the EPP leads to explainable causal inference because of complete data, context, and inference function when assuming a static EPP. For a dynamic or emergent EPP, explainability might not be guaranteed for all potential state transitions. An implementation of the EPP has to specify the exact details to support explainable inference and where to establish sensible constraints on explainability. The gravitas of the EPP epistemology is rooted in its flexible, contextualized ontology, a powerful knowledge derivation mechanism, and its intrinsic support of explainable causal inference. The epistemology varies depending on whether the EPP process is static, dynamic, or emergent.

\subsection{Static EPP Epistemology}
\label{subsec:Static_EPP}

For a static Effect Propagation Process, the knowledge is explicitly modeled during the design stage and confined in the context. The quantitative nature of explicitly modeled context and EPP leads to the positivism of the resulting epistemology.

\subsubsection{Static context}

A static context emits an invariant structure after it's defined, therefore a static EPP combined with a static context allows for the strongest deterministic guarantees albeit at the expense of flexibility. Static contexts remain an invaluable tool to model contextual data that remain structurally invariant, which is a common situation when integrating external data sources. The content, structure, richness, and accuracy of that static context profoundly determine the epistemology of what can be known through the EPP.

\subsubsection{Dynamic context}

In a dynamic context, the context structure itself evolves e.g. new elements (i.e. quarter of a year) are added as the data feed progresses. By definition, a dynamic context relies on a generating function to gauge the dynamic changes of the context. The impact on the epistemology of a static EPP remains minimal though. Fundamentally, dynamic contexts are used when structural elements occur at either regular intervals or otherwise determinable occurrences, and therefore, the EPP can model these elements regardless of whether they have been added to the context yet. For example, a Causaloid that determines whether the sum of the previous three monthly financial reports matches the quarterly financial reports for the current quarter might be a precondition if the “current” quarter in the context has been updated. Therefore, dynamic contexts simplify domain modeling while leaving the epistemology modeled in a static EPP intact.

\subsection{Dynamic EPP Epistemology}
\label{subsec:Dynamic_EPP}

For a dynamic Effect Propagation Process, the dynamics are captured in generative functions that evolve either the EPP, the context, or both. Conceptually, these generative functions could range from deterministic, rule-based algorithms that construct or modify Causaloids and Context structures based on predefined logic or specific triggers, to more adaptive mechanisms. For instance, a generative function for a dynamic Context might be a higher-order function that, given the current state and new inputs, returns an updated Context graph, a practice well established in functional programming to build dynamic systems.

The ontology of knowledge may evolve as a result of the evolving EPP and the impact of the epistemology remains deepening not only the EPP evolution itself but the interaction with its context as it can happen that both, the EPP and its context evolve dynamically.

\subsubsection{Static context}

For a dynamic EPP, a static context may serve as the foundational layer that captures core data that remain structurally invariant. As with a static EPP, the static context determines fundamentally what determines the epistemology of what can be known through the dynamic EPP.

\newpage

\subsubsection{Dynamic context}

For a dynamic context, though, the impact on the epistemology captured in a dynamic EPP can be profound. For example, with the advent of model context protocol (MCP), which lets LLMs call into tools to retrieve or modify data, a causaloid in a dynamic EPP may trigger an MCP invocation, which then updates the context by expanding its structure, and then triggers a generative function that creates a new Causaloid based on the retrieved contextual data, which then analyzes either a newly created part of the context, or a new external data feed created by the MCP. As a consequence, the epistemology in this case depends on a dynamic EPP-Context co-evolution.


\subsubsection{Dynamic Co-Evolution}


When both, the EPP and the engulfing context evolve dynamically in what can be seen as a co-evolution, then no fixed epistemology can be established anymore because the inference based on generated Causaloid over newly added sub-structures of the context may or may not occur depending on the occurrence of the underlying trigger event(s). One could estimate a potential epistemology by using a Rubin causal model\cite{rubin2005causal} (RCM) by comparing potential reasoning outcomes under different scenarios in which a Causaloid was generated versus when not. More profoundly, it might not be possible any longer to use automated explainability to discern the appropriateness and relevance of the generated Causaloids and contextual shifts in response to external changes. This introduces an element of interpretivism to the resulting epistemology: the derived knowledge requires the observer to apply a conceptual framework for understanding the system's complex and dynamic evolving behavior.

\subsection{Emergent EPP Epistemology}
\label{subsec:Emergent_EPP}

Unlike a dynamic EPP, an emergent EPP does not evolve anymore based on pre-determined triggers that initiate pre-defined generative functions. Instead, an EPP is considered emergent when the underlying generation process leads to novel causal configurations, reasoning pathways, or new generative principles for Causaloids context interactions that were not explicitly encoded beforehand.
The generation process may incorporate principles from evolutionary computation, novelty search, or machine learning embedded in the EPP itself. While the full exploration of AI-driven generative functions for EPP remains future work, the foundational idea of using programmable functions to dynamically define and evolve both the EPP and its context is a natural extension of EPP's core design philosophy. Regardless of the mechanism, emergent EPP does not interact with its Context using pre-defined procedures but instead relies on procedures generated by the EPP itself. The key indicator of emergence is that its novel behavior was not foreseeable by its initial designers.

\subsubsection{Static context}

Like a static or dynamic EPP, when the static context has been defined upfront, it determines fundamentally what determines the epistemology of what can be known through the dynamic EPP within the contextual boundaries. Unlike a static or dynamic EPP, an emergent EPP may or may not generate a new static context and that indeed alters the Epistemology emerging from an emerging EPP. 

\subsubsection{Dynamic context}

Likewise, when an emerging EPP creates or modifies a dynamic context, the emerging Epistemology cannot be determined any longer because of the resulting co-emergence of the EPP and its context.

\subsubsection{Dynamic Co-Emergence}

An EPP co-emerges with its context when the underlying generation process leads to novel causal configurations that were not explicitly encoded beforehand. This can happen when the EPP contains methods of machine learning that evolve the EPP itself in response to a dynamically changing context. As a result of the dynamic, non-deterministic self-modification of the EPP itself, the spectrum of subsequent factual representation in the context and the emerging causal structures cannot be predicted any longer.
Therefore, determining the truthfulness of the emerging causality imposes a non-trivial challenge that adds an elevated level of pragmatism to the epistemology. The pragmatism becomes necessary because it is not guaranteed that the underlying dynamic context always leads to a truthful representation of the world it seeks to model, but the generated causal relationships may not always be correct either. Both can happen due to generative errors during the EPP. Generative errors may result from complex interactions that contain steps that, in isolation, are correct, but when combined in a certain order may lead to an incorrect outcome. This is a typical characteristic of increasingly complex dynamic systems that need to be considered by taking an operational stance on truth.

\section{Operationalized meaning of truth in EPP}
\label{sec:Operationalized_Truth}

The meaning of truth evolves depending on the modality of the EPP because the underlying reference for a true statement varies depending on the chosen modality. Per the definition of knowledge, a true belief must entail a high degree of justification and come from a reliable source to count as knowledge. It is the underlying justification process that depends on the modality of the EPP that causes the shift in the meaning of truth to vary.

For a static Effect Propagation Process, the meaning of truth aligns with the classical correspondence theory. That means, that if the context encodes accurate facts and the causal relationships are true, all derived forms of knowledge must be true.
Justification rests upon the verifiable mapping between the EPP's explicit model encoded in its context and the part of reality it purports to represent. The static EPP implicitly operates under the assumption that its model is a faithful mirror of objective facts. Here, the truth of an inference is determined by the adherence to the contextual facts and encoded causal relationships. As a result, a static EPP leads to deterministic verifiability within the confined boundaries of its context.

As the EPP transitions into a dynamic modality, the meaning of truth begins to shift towards a coherent adaptability to dynamic interaction with a changing context. A dynamically modified causal relationship is deemed true if it maintains consistency with the facts in its evolving context. In a dynamic modality,  the justification of knowledge becomes contextually and temporally aware. Therefore, truth is assessed by the EPP's capacity to maintain a relevant and internally consistent causal understanding amidst navigating a temporal dynamic context.  This leans towards a coherence theory of truth, where coherence itself must be evaluated relative to the EPP’s intricate temporal structures.

For a contextual co-emergent EPP, the meaning of truth shifts further toward pragmatic efficacy. This shift becomes necessary because of the emergence of relativistic causal relationships from the EPP that co-evolve with its context. Here, establishing an objective a priori truth becomes elusive since the fabric that would traditionally serve as a stable reference for truth is itself emerging dynamically alongside the causal inference made from it. Instead, the truth of an emergent causal inference is established by its utility in enabling the EPP to navigate its environment within its temporally complex context.
This pragmatic efficacy means that truth, defined by its functional value, becomes inherently system-relative and context-dependent.

Indeed, a functional value could serve as a fitness function guiding the emergent process itself thus raising fundamental questions about alignment. Consequently, pragmatic efficacy can lead to multiple, functionally 'true' yet distinct causal understandings, each valid within its own emergent trajectory and its relativistic interrelation with its context.

This dynamic interplay, where the EPP generates both its context and the Causaloids that encode the causal relationships that operate within that context presents a research opportunity.
It allows for the exploration of relativistic emergent causality and how coherent and pragmatically effective causal understandings can arise in systems that lack a fixed predefined spacetime. This might be of interest in theories of fundamental physics where spacetime itself may be an emergent phenomenon arising from more fundamental processes.


\section{Discussion}
\label{sec:Discussion}

The Effect Propagation Process with its static, dynamic, and emergent modalities leads to a remarkably rich and multifaceted epistemological landscape that ranges from positivism to pragmatic emergence.

When a static EPP operates in a static context, the epistemology aligns with positivism. Knowledge is derived from explicitly modeled, observable facts and follows explicit rules, leading to verifiable and deterministic causal chains. The objective is a generalizable inference based on a predefined world model and is best suited for confined use cases such as control systems that operate autonomously within defined boundaries.

However, with the introduction of dynamics via a dynamic context or a dynamic EPP, the system actively constructs and revises its causal structure in response to, and in concert with, a changing environment. As a result, system boundaries evolve and shift dynamically. While the explainability of individual steps might be maintained, the relevance of newly generated Causaloids and contextual interpretations necessitate an observer's engagement.
The observer's engagement adds an element of interpretivism because the meaning and validity of the system's knowledge are co-constructed through the observer's understanding of its complex, evolving behavior.

The 'observer' in these interpretive scenarios is not uniform by definition. Instead, the observer could be the EPP's initial designer who focuses on mechanistic integrity, an end-user who evaluates the domain-specific utility, a regulator who is scrutinizing the EPP for compliance and safety, or even another AI system that integrates EPP outputs into its operational logic. Each of these distinct observers brings a unique interpretive lens and, an advanced EPP may need to support multiple, potentially differing, observational perspectives and their corresponding reporting requirements. This leads to a critical observation: the 'justified knowledge' derived from interpretive engagement with an EPP is not absolute. Instead, the 'truth' or 'validity' of a complex causal inference can become dependent on the observer's framework, with different interpretations holding legitimate, albeit perspectival, justification, thus demanding a clear understanding of whose interpretation is being applied and for what purpose.

In an EPP that dynamically co-emerges with its context, the system may generate novel configurations and reasoning unforeseen by its designers. The justification of knowledge in such a system leans on sophisticated interpretive methodologies to discern meaning and validity from dynamic complex emergent structures. Even if the system’s genesis is not explainable by its initial design any longer, the emergent causal 'understanding' is fundamentally validated by its success in dealing with its co-evolving environment. The system doesn't just adapt; it becomes its own evolving theory of its world.

It is important to recognize that these epistemological stances, positivism, interpretivism, and pragmatism, are not mutually exclusive stages in an EPP. In practice, sophisticated EPP implementations will likely lead to hybrid systems with a static foundation, several different dynamic parts, and potentially emergent parts. At each level of the system, the appropriate epistemological view helps to analyze the mode at hand. For a static subsystem, the positivist view suffices, for a dynamic subsystem requires interpretivist engagement from the designer(s), and for emergent parts, pragmatic validation becomes paramount.

These three epistemological stages have profound implications for the role of the human in the process. With static EPPs, the human is the architect, directly encoding causal knowledge. With dynamic EPPs, the role shifts towards a meta-designer, setting up generative principles and then observing and interpreting the evolving system as it unfolds. Dynamic Bayesian Networks\cite{dagum1992dynamic} may help to develop tools to model transition probabilities between emergent states. The interpretivism perspective becomes the new center of deriving meaning from the dynamic EPP. Lastly, when dealing with emergent EPPs, the role shifts again towards a pragmatic approach to the emerging self-correcting knowledge. to understand, validate, and safely harness that emergent knowledge.

The epistemology of co-emergent EPP and its context raises multiple questions:

\begin{itemize}
    \item When the EPP evolves with its context, how it ensure the resulting causal inference remains valid?
    \item How can the emergent process be safeguarded?
    \item How can the emergent process remain aligned with established principles of AI safety?
\end{itemize}

At this early stage, none of the questions can be answered and more research in the area of ensuring safety in dynamic causal emergence over dynamic effect propagation is warranted.

Because of the inherent indefinite nature of EPP context co-emergence, one option to ensure safety is to codify the co-emergence as much as possible in an external control system that establishes principled boundaries to safeguard the process. Principled boundaries can be thought of as similar in spirit to Asimov's Laws of Robotics which the system must adhere to under all conditions. However, this presupposes a mechanism capable of verifying that it is possible to verify that each emergent step complies with a set of principled boundaries, and that might require additional research.

\newpage

\section{Conclusion}
\label{sec:Conclusion}


The epistemology of the Effect Propagation Process reflects the complex systems it is designed to model by scaling with the modality of the EPP. For a static EPP, a positivist epistemology remains sufficient. For a dynamic EPP, the epistemology evolves towards an interpretivism perspective, and for an emergent EPP, a pragmatism perspective on the epistemology becomes necessary.

Likewise, for the justification of knowledge in an EPP, the underlying notion of truth scales with the modality of the EPP. For a static EPP, the meaning of truth aligns with the classical correspondence theory. However, in a dynamic EPP, the meaning of truth shifts towards a coherent adaptability approach. In an emergent EPP, the meaning of truth evolves towards pragmatic efficacy where the validity of relativistic, emergent causal relationships is established by their functional utility.

The shifting epistemology implies that the role of human involvement equally scales with the EPP modality. For a static EPP, the human remains the designer, whereas for a dynamic EPP, the role extends towards an observer to understand system dynamics by observation and analysis. For an emergent EPP, however, the role of an observer might be delegated to an artificial intelligence while the human role evolves further towards assessing the quality and appropriateness of the emerging EPP.

The rich epistemology of the EPP enables novel explorations of understanding evolving causal knowledge in emergent systems. Further research is warranted to ensure the safety and alignment of emergent systems.
 
%Bibliography
\bibliographystyle{unsrt}  
\bibliography{references}  


\end{document}