\documentclass{article}


\usepackage{PRIMEarxiv}
\usepackage[table,xcdraw]{xcolor}
\usepackage[utf8]{inputenc} % allow utf-8 input
\usepackage[T1]{fontenc}    % use 8-bit T1 fonts
\usepackage{hyperref}       % hyperlinks
\usepackage{url}            % simple URL typesetting
\usepackage{booktabs}       % professional-quality tables
\usepackage{amsfonts}       % blackboard math symbols
\usepackage{nicefrac}       % compact symbols for 1/2, etc.
\usepackage{microtype}      % microtypography
\usepackage{lipsum}
\usepackage{fancyhdr}       % header
\usepackage{graphicx}       % graphics
\graphicspath{{media/}}     % organize your images and other figures under media/ folder

%Header
\pagestyle{fancy}
\thispagestyle{empty}
\rhead{ \textit{ }} 

% Update your Headers here
\fancyhead[LO]{Effect Propagation Process: A Philosophical Framework}
  
%% Title
\title{Effect Propagation Process: A Generalized Framework for Causal Inference in Dynamic Systems}

%% Author
\author{
  Marvin Hansen \\
  \texttt{marvin.hansen@gmail.com} \\
   Date: \today
}


\begin{document}
\maketitle

\begin{abstract}
The study of causality dates back to Aristotle and was later formalized in its commonly known form by Seneca. Seneca’s assumption of a background space and time was later challenged by relationalists like Leibniz and critiqued scientifically by Russell. However, contemporary research in quantum gravity suggests that spacetime itself might be emergent from the quantum level 
and causal structures can be indefinite, thus fundamentally challenging the assumption of a background spacetime. 
While these concepts may sound specific to quantum physics, the underlying problem of a fixed spacetime assumption's conflict with representing causality in non-linear structures also occurs in advanced system engineering. 
Three specific problems arise in advanced system engineering that previous methods of causality cannot address anymore: 
non-Euclidean data representation, non-linear time representation, and emerging causality.\newline
In response, the author introduces the "Effect Propagation Process" (EPP) framework to address these new challenges. 
The EPP departs from the classical notion of time-linear causality, borrows new concepts from physics, and proposes 
a view of  causality as a spacetime-agnostic continuous effect propagation process. The Effect Propagation Process offers 
a unified philosophical framework of causality that remains compatible with classical causality and conceptually congruent 
with physics theories of quantum gravity.\newline
The EPP was developed to serve as the foundational theory for DeepCausality, a Rust-based computational library designed 
to handle causal inference in complex and dynamic systems. The engineering challenges faced during its development, 
such as representing non-linear time structures, handling dynamic feedback loops, and modeling emergent causal shifts, 
necessitated a new generalization of causality. In absence of precedence of spacetime agnostic causality, 
the EPP was formulated as a direct response to these demands, providing both the philosophical grounding and the 
architectural blueprint for a new category of context-aware causal reasoning. The DeepCausality project solves 
the inevitable structural complexity challenge with recursive isomorphic causal data structures. However, because the aspect of isomorphic causal datastructures is implementation specific to Rust, it is not covered in the presented EPP framework.\newline
Philosophically, the epistemology of the EPP begins with positivist foundations in static configurations where truth aligns with 
a predefined model. In a dynamic EPP, epistemology requires an interpretive stance where truth emphasizes coherent adaptability 
amidst non-linear temporal evolution. Lastly, its epistemology evolves further into a pragmatic understanding in co-emergent 
systems where the truth of relativistic, emergent causal relationships is validated by their functional efficacy. 
The new epistemology lays the groundwork for analyzing the dynamic emergence of causality that results from the application of the EPP.
Practical application of the Effect Propagation Process enables the handling of uncertainties, causal reasoning 
over non-Euclidean data, and non-linear complex feedback loops in dynamic systems.\newline
The Effect Propagation Process does not seek to replace the classical notion of causality; instead it seeks to advance and generalize the concept of causality to address the new challenges of causal inference over non-Euclidean data, non-linear temporal structures, and handling of emergent causality.
\end{abstract}

% keywords 
\keywords{Causality \and Post-Quantum Causality \and Philosophy of Causality \and Causal Structures,
\and Effect Propagation Process, \and Epistemology \and Ontology }


\newpage

\section{Motivation}
\label{sec:motivation}


What warrants the development of a new philosophy of causality?

The pre-existing philosophy of causality served mankind for millennia, and one might be tempted to conclude that this is all there is to know about cause and effect. However, the origin of the Effect Propagation Process (EPP) did not start in philosophy, but in three distinct problems. The first problem is related to non-Euclidean data representation, the second problem is rooted in causal inference over non-linear time representations, and the third in handling dynamic causal structures.

\subsection{Non-Euclidean representation}

The first problem applies equally to computational causality and deep learning; therefore, it is best illustrated with familiar  large language models. Language embeddings remain foundational to contemporary large language models (LLMs), but these require a reduction into a vector space because many prevalent LLM architectures operate efficiently in vector spaces, thus making the reduction from non-Euclidean realms (language) into a Euclidean representation (Vector space) mandatory. Instead of advancing LLM architectures to natively handle non-Euclidean representations, the industry has focused on leveraging Vector databases for storing and retrieving embeddings derived from LLMs. Graph neural networks operate on non-Euclidean spaces, but as these are not yet commonly adopted as core components in mainstream LLM architectures, the problem prevails.

\subsection{Non-linear Time}

However, when generalizing space beyond Euclidean, then the second problem emerges: how to represent time? More profoundly, can we separate time and space from data? Out of this insight, the idea emerged to separate space, time, and data into a separate context usable by multiple models. As space and time were elevated from an implicit background into an explicit first-class context representation, then moving beyond correlation towards causality became the next obvious step. At this moment, a profound problem emerged: When space is non-Euclidean, and time might not be a simple linear progression, then how do we establish a causal relationship?

As it turned out, establishing a clear causal relationship became problematic within the classical definition of causality, which fundamentally relies on a linear time-asymmetric ordering (cause preceding effect) within its assumed background spacetime.

One might challenge the presumption of non-linear time progression, but in complex systems with dynamic feedback loops, it’s perfectly possible to see context structures that entail non-linear time regions. Non-linear time regions can occur when the background time is represented as a temporal hypergraph that holds multiple time resolutions simultaneously. The simultaneous presence of time units at different scales breaks the simple time-linear assumption (all time has the same unit, and moves therefore at the same rate) that computational causality tools commonly make.

Furthermore, in a temporal hypergraph, the unit of time is scale dependent which means in order to compare temporal values one must consider the scale to make a valid comparison between equally scaled values (hour X compared to hour Y). Less obvious, a temporal hypergraph, by design, holds all past and present temporal values simultaneously within its structure. This co-existence of multiple temporal points simplifies non-trivial temporal arithmetic over hetero-scaled time units, yet it imposes a vexing problem: How do you know if a time value in a node of the graph is current or past?

The problem is non-trivial because, as time progresses, the engulfing context engine continually generates the non-Euclidean temporal hypergraph representation with the implication that, at one lookup, the value of a temporal node is current, but at the next one, it might be past; however, the exact temporal distance at which a “present” value becomes “past” depends on the node's time scale. A node holding a temporal value “hour” will be valid for 60 minutes; that means a lookup every ten minutes will yield the current hour 6 times, but handling the seventh lookup leads to a fundamental design decision that illustrates the implied complexity.

A “dynamic-position” design means, when a new hour starts, a new node will be added; therefore, the seventh lookup returns a past value. By implication, the index of the new node needs to be looked up to retrieve the value of the new current hour. Therefore, a “dynamic-position” design requires a dynamic index to locate current values.

\newpage

A “static-position” design means, when a new hour starts, the previous current value will be overwritten with the understanding that the seventh lookup returns the new current value. By implication, the index of the current value always remains static. Therefore, a “static-position” design requires a fixed index, i.e., a lookup table to locate current values. Use cases exist for both scenarios and in practice, temporal hypergraphs use a mixture of static and dynamic positioning to handle different types of relative values, e.g., current day, last year, next hour, and similar..

Exacerbating the problem further, feedback loops across different time scales using different relative values may dynamically modify the temporal hypergraph itself to capture non-regular observations or to add estimations at future time values that have not yet occurred.   
At this point, it becomes abundantly clear that the assumption of a simple, unidirectional linear temporal order required for establishing cause-and-effect becomes untenable.

At the same time, causal relations remain valid in those non-linear regions. Additionally, the designation of a cause purely based on strict temporal order feels arbitrary in a temporal hypergraph in which past, present, and estimated future temporal values across multiple time scales all exist simultaneously. Regardless of static or dynamic location of relative values, the definition of causality had to evolve to match the reality of modeling causal structures across complex multimodal hypergraphs.


\subsection{No a-priori causal structures}

When modeling dynamic feedback loops across different time scales, the third problem emerges eventually: Not all causal structures are known upfront. There are cases where the causal structure emerges from the prevalent context predominantly in response to a change in externalities. For example, in the financial industry, a shifting volume imbalance indicates an emerging regime change. The exact cause for the shifting volume imbalance can be attributed to externalities such as breaking news. However, in absence of internal references, the subsequent  causal structure emerges as part of the unfolding regime change. There is no possible way to know the new structure upfront therefore, it cannot reliably be modeled ahead of time. Likewise, in correlation-based methods, a similar phenomenon unfolds because, in absence of internal references, the deep learning model cannot predict correctly anymore because data during an emerging regime change fall outside its training data distribution. The mechanism is different, but the outcome is the same: that previously reliable models crater in novel situations.

The previously identified limitation of temporal order directly applies to dynamic regime changes because, if the new causal structure has not yet emerged, how can we know its temporal structure beforehand? In short, we cannot know reliably anything about emerging causality up to the moment it emerges.

These three problems, non-Euclidean representation, non-linear temporal structures, and emerging causality deeply interrelate with each other, thus defy overly simplistic solutions. For example, advanced graph neural networks work on non-Euclidean data representation, but fail on non-linear temporal structures. One might be tempted to build non-linear-time graph neural networks, but this does not address the problem of non-Euclidean data representation and emerging causality. There is research to combine methods from computational causality with deep learning, but these approaches are focused on non-Euclidean representation without integrating the challenges of non-linear temporality and emergent causal logic.


\newpage

\subsection{Why Existing Methods Struggle?}

The established methods of causal inference remain important milestones and they work well within their defined domain; they were not architected for the challenges imposed by handling non-Euclidean representation, non-linear temporal structures, and emerging causality. Therefore, the subsequent analysis does not criticize any of the existing methods, as they remain valid. Instead, the analysis states how the underlying assumption(s) of each method conflict with the previously established problems.

\subsubsection{Granger Causality}

Granger Causality\cite{granger1969causal} is used for time-series data where past values of one time-series are used to predict values of another time-series. In a strict sense, Granger Causality tests if one variable (say X) predicts another variable (Y) through a series of t-tests and F-tests on lagged values of variable X.

\textbf{Assumptions:}

Granger causality assumes a stable causal structure and a linear, uniform time representation with consistent intervals.

\textbf{Implications:}

\begin{itemize}
    \item \textbf{Non-Linear Time:} Granger causality cannot handle non-linear time representation
    \item \textbf{Non-Euclidean Representation:} Granger causality operates on time-series values within a Euclidean representation. It cannot be applied to a non-Euclidean representation.
    \item \textbf{Emergent Causality:} Because of the assumption of a stable causal structure, Granger causality cannot operate on emergent causal structures.
\end{itemize}


\subsubsection{Pearl's Causal DAGs and Structural Causal Models (SCMs)}

Judea Pearl pioneered the formalization of causality that underpins the majority of contemporary methods of computational causality. The framework of Structural Causal Models\cite{pearl2000causality} (SCMs) rests upon the assumptions of Directed Acyclic Graphs (DAGs).

\textbf{Assumptions:}

Pearl's causal framework is exceptionally powerful for reasoning about interventions given a known or discovered causal model.
The framework assumes: 

\begin{itemize}
    \item Acyclicity (DAG): Causal relationships are acyclical in a directed acyclic graph.
    \item Static Causal Structure: The causal graph, once defined, is assumed to be static.
    \item Fixed background spacetime: Variables in the DAG are assumed to be embedded within a fixed background spacetime.
\end{itemize}


\textbf{Implications:}

\begin{itemize}
    \item \textbf{Non-Linear Time:} The acyclical assumption explicitly prohibits feedback loops and the fixed background spacetime assumption prevents any form of non-linear time.
    \item \textbf{Non-Euclidean Representation:} The DAG structure could potentially allow for non-Euclidean representation, but existing tooling assumes Euclidean structures and thus does not allow non-Euclidean representation
    \item \textbf{Emergent Causality:} The assumption of static causal structure prevents any handling of causal emergence.
\end{itemize}

\newpage

\subsubsection{Rubin causal model (RCM)}

The Rubin causal model\cite{rubin2005causal} (RCM) is a statistical framework for estimating causal effects by comparing potential outcomes under different treatment assignments.

\textbf{Assumptions:}

RCM assumes stable units, no interference between units (SUTVA), and that treatment assignment is ignorable conditional on observed covariates. RCM also assumes a stable causal structure and a fixed background spacetime.

\textbf{Implications:}

\begin{itemize}
    \item \textbf{Non-Linear Time:} RCM is not designed to handle non-linear or multi-scale time representation.
    \item \textbf{Non-Euclidean Representation:} RCM is designed for variables representing treatments, outcomes, and covariates, without any mechanism for complex relational types. Therefore, RCM cannot infer across non-Euclidean data representations.
    \item \textbf{Emergent Causality:} RCM estimates effects within a stable causal structure and therefore cannot handle emergent causality.
\end{itemize}

\subsubsection{Dynamic Bayesian Networks}

Dynamic Bayesian Networks\cite{dagum1992dynamic} (DBN) extend Bayesian Networks to model temporal processes by defining transition probabilities between states.

\textbf{Assumptions:}

Dynamic Bayesian Networks, similar to Granger causality, require a linear and uniform time representation. DBNs model changes in the state of variables over time according to a fixed probabilistic causal structure assumed to be static.


\textbf{Implications:}

\begin{itemize}
    \item \textbf{Non-Linear Time:} DBN cannot handle non-linear time.
    \item \textbf{Non-Euclidean Representation:} DBNs reason over probabilities, not non-Euclidean space itself.
    \item \textbf{Emergent Causality:}  DBNs assume a static causal structure and thus cannot handle emergent causality
\end{itemize}

As stated earlier, all these methods remain valid in their respective domains. Granger causality remains vital in time series forecasting. Pearl’s causal DAG remains relevant, among others, in empirical healthcare. Dynamic Bayesian Networks find applications in healthcare, data mining, and robotics. This analysis, however, shows why each method cannot be applied to the previously established problems. It is important to recognize that, as Russell indicated, it is the classical interpretation of causality itself that warrants fundamental rethinking.

\subsection{Why a new philosophy of causality?}

Upon studying Hardy’s work on causality for quantum gravity\cite{HardyDynamicCausalStructure}, the author noticed the clear emphasis on operational causality; a sensible choice given the entire field of quantum gravity can be best described as work in progress.
However, the lack of a philosophical foundation became apparent when attempting to combine Hardy’s unified causality with non-linear time, and uniform context over Euclidean and non-Euclidean structures: There was no “big idea” that would help to bring everything together. However, a new idea was much needed to solve the vexing problem of handling causal inference across complex hypergraph structures let alone tackling emergent causality. The presented Effect Propagation Process closes this foundational gap and brings together previously disjoint ideas into one coherent philosophical framework of causality that elegantly solves the non-Euclidean representation problem by defining causality as a fundamentally spacetime-agnostic process.\newline
The EPP framework is conceptual and exploratory by design and offers a philosophically coherent generalization of causality that addresses the limitations of classical models under emerging scientific paradigms. By no means does the EPP claims empirical completeness, but instead borrows concept from theoretical developments in quantum gravity to tackle the challenges imposed by those domains where conventional methods of computational causality no longer work. Like all foundational frameworks, its value derives from opening a new way of thinking about causality in domains where the classical spacetime assumption no longer holds.\newline
It is important to understand that, without something similar to EPP, conventional causality cannot work across complex hypergraph structures and solve new challenges of causal inference over non-Euclidean data, non-linear temporal structures, and handling emergent causality in dynamic systems. The foundational work inspired by Quantum Gravity is a direct testament to the problem complexity that EPP solves.
 
\newpage

\section{History of Causality}
\label{sec:history}

Plato is believed to be the first who explored the cause (aitia) in the book of Timaeus (c 360 BC). In it, Plato stipulates that multiple indispensable factors, the model, the maker (Demiurge), the material, and the space (receptacle), explain how the physical world with all the things in it are made\cite{PlatoTimaeusSEP}. Aristotle (c 350 BC) formalized the notion of causality in his Metaphysics\cite{AristotleMetaphysicsSEP} with the  "Four Causes”\cite{AristotleCausalitySEP}. These are:

\begin{enumerate}
    \item The material cause or that which is given in reply to the question “What is it made out of?”
    \item The formal cause or that which is given in reply to the question "What is it?”. What is singled out in the answer is the essence of the what-it-is-to-be something.
    \item The efficient cause or that which is given in reply to the question: “Where does change (or motion) come from?”. What is singled out in the answer is the whence of change (or motion).
    \item The final cause, the end purpose, is given in reply to the question: “What is its good?”. What is singled out in the answer is that for the sake of which something is done or takes place.
\end{enumerate}

Seneca (c 56 AD) argues in letter 65\cite{SenecaLetters} that cause and effect operate within a stage (space) and follow an order (time). Remove the stage or the order, and the conventional understanding of 'making something' or 'causing something' breaks down. His argument highlights time and space as indispensable prerequisites for classical causality. His focus on space and time as necessary conditions served as a precursor for physical concepts treating spacetime as a background for causal processes.

The idea of space and time as a background for causality, however, did not remain unchallenged. Gottfried Wilhelm Leibniz (1646-1716) rejected the concept of absolute space and absolute time as independent, fundamental constructs. Instead, Leibniz proposed\cite{LeibnizPhysicsSEP} a relational view in which space is the simultaneous relation of coexisting things and time is the relational order of successive things.
Through rigorous first principles analysis, Leibniz argued that the concept of absolute space and time was logically untenable.
His relational perspective offered a significant alternative to the preeminent Newtonian worldview of his time.


Albert Einstein (1879 - 1955) departed from Newtonian physics through his theory of General Relativity\cite{EinsteinPapers1915}  (GR) in which he established that space and time are one manifold, spacetime, that is bent by gravitational influence from large masses. General Relativity preserves the prerequisite of a spatiotemporal context for causality, echoing Seneca's key insight. However, the notion of dynamic spacetime requires a dynamic view of causality to fit into the dynamic spacetime manifold.

Bertrand Russell (1872 - 1970) observed that successful physics has its roots in sophisticated, law-based descriptions of how a system evolves dynamically. In modern physics, the focus is on the state of a system (e.g., position, velocity, field strength across space) and how that entire state evolves continuously and dynamically. Therefore, for Russell, the idea of classical  causality, a strict happen-before relation, does not match the reality of modern physics anymore. Consequently, Russell wrote in his 1912 essay "On the Notion of Cause"\cite{RussellOnCause}:


\begin{quote}
    \begin{center}
            \textit{The law of causality, [...], is a relic of a bygone age"} - Bertrand Russell
    \end{center}
\end{quote}


Many modern physics laws are time symmetric, and that means if state S1 at time t1 is related to state S2 at time t2 by a law, it's equally true that state S2 at time t2 is related to state S1 at time t1. This relationship isn't a simple, linear, one-way street from a necessary "cause" to a dependent "effect." Knowing the state at any time allows you to calculate the state at any other time, past or future. Therefore, which state is the "cause" and which one is the "effect" becomes arbitrary.

Russell was not opposed to causality itself; instead his primary argument was that the traditional philosophical interpretation of causality as a fundamental, temporally asymmetric, and directed link is not what he observed in physics. His critique resonates with challenges encountered in contemporary quantum research.

\newpage

\section{The impact of Quantum Gravity on Causality}
\label{sec:impact_quant_grav}

Quantum Field Theory\cite{peskin2018introduction} (QFT)  stands out as one of the most rigorously tested theories with unparalleled predictive accuracy in the history of science. QFT predictions in Quantum Electrodynamics have been experimentally verified up to an astonishing accuracy of one part in a billion or better. This experimental success proves that the underlying concepts such as the primacy of fields, the emergence of particles, and inherent indeterminacy used across different areas of QFT indeed result in demonstrably more accurate models.

The physics standard model, built on top of QFT, despite being one of the most successful theories of all time, accurately describes three of the four known fundamental forces: electromagnetism, the strong nuclear force, and the weak nuclear force. Notably absent remains gravity due to complex discrepancies between Einstein’s (pre-quantum) theory of relativity and Quantum Field Theory. It is important to point out that General Relativity in itself is successful in the sense of high accuracy and strong predictive power, albeit in a non-quantum realm.

The unification of general relativity with quantum field theory would complete the standard physics model, but doing so faces a non-trivial impasse, as Lucian Hardy\cite{HardyDynamicCausalStructure} formulated: “Quantum theory is a probabilistic theory with fixed causal structure. General relativity is a deterministic theory but where the causal structure is dynamic.” Furthermore, the emergence of Quantum Gravity directly challenges the traditional separation of causality through the introduction of indefinite causal order with time symmetry\cite{MriniHardyIndefinite}.

Indefinite causal order means the order of causal events is not fixed. It might be that A happens before B, or that B happens before A, or a quantum superposition where both orders (A before B, and B before A) coexist simultaneously. Time symmetry or time reversal symmetry refers to the theoretical symmetry of physical laws under inverted time. That means, the laws of physics remain the same when time is reversed, a property known as T-symmetry. Many laws of physics are formulated in terms of time symmetry, thus allowing to calculate previous system states in time using the same set of equations. For example, laws like Newtonian mechanics, Maxwell's equations of electromagnetism, or Einstein's equations for General Relativity are T-symmetric.

The fundamental dynamical laws of quantum physics (excluding the weak nuclear force) are accepted as T-symmetric. However, the transition of a quantum state from a probabilistic T-symmetric state into a definite and time-asymmetric state, while observed, isn't well understood and subject to ongoing research in the field of quantum measurement.

Quantum superposition of states, on the other hand, is extremely well-observed, well-documented, and as a result, it is accepted as a fundamental building block of quantum physics. Quantum superposition inspires the exploration of concepts like indefinite causal structures in theories aiming to unify quantum mechanics and gravity. Therefore, the classical conceptualization of cause and effect embedded into a background spacetime might not yet exist at the fundamental level of quantum gravity. In Quantum Gravity, space and time are not external conditions, but potential emergent properties of the internal quantum structure itself. The problem is no longer whether spacetime is static or dynamic, but that spacetime itself emerges from the quantum level and thus positions itself as a higher-order effect of a generative process.

\begin{quote}
    Instead of asking "Where and When does this cause operate?", quantum gravity asks "What is the underlying process from which the notions of 'where' and 'when', and thus causal order, emerge?”
\end{quote}

Russell saw physics moving towards laws governing states, a view echoed in quantum gravity's search for fundamental rules governing the structure from which spacetime and causal order emerge. At this stage, the understanding of causality evolved from a structure that required the existence of space towards a dynamic generative process from which causality emerges. This emergent causality does not rely on a pre-existing spacetime, but is grounded in a more fundamental level of reality—a set of underlying rules (i.e., conceptualized as a 'generating function') that determine the fundamental manifestation of spatiotemporal properties.

If the 'output' in terms of emergent properties doesn’t include states that resemble classical spacetime, then the conditions Seneca deemed necessary for causality would not appear. From the quantum perspective, 'space and time' that Seneca identified as necessary conditions are not fundamental inputs to causality any longer, but are outputs from the underlying quantum generative process.

The conceptualization of this fundamental level as a "generating function" captures the idea of a quantum process from which the necessary condition of classical causality's spatiotemporal structure arises. It's a shift from asking "What causes X given spacetime?" to "What process generates spacetime (and thus enables X to be caused)?".


\section{Causality as Effect Propagation Process}
\label{sec:epp}

The notion of a generative process that underlies the fabric of spacetime leads to the implication that causality has to evolve beyond the strict “before-after” relation towards a spacetime-agnostic view. The classical definition of causality, taken from Judea Pearl's foundational work\cite{pearl2000causality}: 

\begin{quote}
    IF (cause) A then (effect) B.
    
    AND 
    
    IF NOT (cause) A, then NOT (effect) B.
\end{quote}

When removing time from causality, it is indeed no longer possible to discern cause from effect because, in the absence of time, there is no “happen-before” relation any longer, and therefore, the designation of cause or effect indeed becomes arbitrary, just as Russell hinted at earlier on. When removing space from causality, the location of a cause or effect in space is not possible anymore because space itself is no longer available. The absence of spacetime raises the question: 
\begin{quote}
\begin{center}
    \textit{What is the essence of causality?}
\end{center}
\end{quote}

Logically, the answer comes in three parts:

\begin{enumerate}
    \item Causality is a process.
    \item Causality deals with effects.
    \item Causality describes how effects propagate.
\end{enumerate}

The first one is self-explanatory because causality occurs in dynamic systems that change and therefore, causality must be a process.

The second one is less obvious, because one might think that causality is all about the “cause” that brings the effects into existence. However, let’s think the other way around: We know that X is the cause of effect E, because E happens when X happens and because E does not happen when X does not happen either. Therefore, we can describe a cause in terms of its effects. Therefore, it is true that causality deals with effects.

The third one, effect propagation, stretches the imagination and is less obvious. When we rewrite the previous definition of classical causality in terms of effect propagation, we see that there is no loss of information:

\begin{quote}
    If X happens, then its effect propagates to Effect E.

    AND
    
    If X does not happen, then its effect does not propagate to Effect E.
\end{quote}

In this definition, X does not have a designated label and instead is described in terms of its emitting effect. Therefore, X can be seen as a preceding effect, which then propagates its effect further. Therefore, causality becomes an effect propagation process. 

In this definition, X does not have a designated label and instead is described in terms of its emitting effect. Therefore, X can be seen as a preceding effect, which then propagates its effect further. Therefore, causality becomes an effect propagation process.

The effect propagation process definition is more general and treats the classical happen-before definition of causality as a specialized derived form. When you designate the preceding effect to be a “cause”, then you can rewrite the general definition back into the classical definition thus the general and the specialized definition of causality remain congruent. This detail is important because the generalized effect propagation process definition would not be sound if it were unable to express a specialized variation. Framing causality as an effect propagation process leads to several implications:


\textbf{Focus on effect propagation:}

In EPP,'effect propagation' denotes the operation of a specific, definable mechanism (via the causaloid) that links states through the underlying fabric regardless of how that fabric might be defined.

The first part, the causaloid, operationalizes effect transfer within a system without relying on any metaphysical causal power. The effect transfer encapsulated in the causaloid is  intrinsic to its mechanistic definition. The identification of relevant causaloids requires explicit modeling and hypothesis testing or (future) discovery processes. EPP provides the formatl structure for representation of Causeloids once discovered.

The second part, the non-defined fabric, through which effects propagate, serves the purpose of externalizing the fabric as a specific, definable context. Instead of assuming an implicit background spacetime, the EPP externalizes the fabric through which effects propagate as a specific context that is defined by a different generative function.
Therefore, EPP fundamentally moves away from invoking a metaphysical causal power towards definable, testable, and verifiable functions (Causaloids) that define the functional relationships describing how a system changes if these functions   are active.


\textbf{Operational generative function:}

In EPP, the aforementioned generative function is strictly operational for modeling dynamic systems. EPP, as a philosophical framework, does not entail ontological aspirations to specify the exact ultimate generative function of the universe and instead leaves these questions to fundamental theories of physics. Instead, the generative function serves as a conceptual necessity to get around the static limitations implied in the classical definition of causality.

In practical applications of EPP, the underlying structure refers to the explicit model of the systems fundamental components which then serves as the operational fabric through which effects propagates. Whether this explicit model is static, dynamic, or emergent is secondary for the application of EPP.

\textbf{Detachment from fixed spacetime:}

In quantum gravity, where spacetime geometry might be in a superposition or non-existent at the fundamental level, "propagation" isn't a movement along a geodesic in a manifold. It's the propagation of effects through a network of states or elements defined by the fundamental structure. This is a prerequisite for handling indefinite causal order.


\textbf{Handling indefinite causal order:}

In situations where the causal structure itself is indefinite (a superposition on the quantum level, a not-yet-emerged state in GR), "effect propagation" can be understood as the influence propagating through a superposition of possible pathways through the fundamental structure. The "effect" isn't tied to a single, definite causal link but is a result of the propagation through all possible pathways.

\textbf{Alignment with emergence:}

When the fundamental structure gives rise to classical spacetime, this fundamental "effect propagation" would then manifest as propagation through spacetime (e.g., waves, particles, forces traveling from one spatiotemporal point to another). Classical cause-and-effect becomes a macroscopic view of this underlying process. This results from the logical congruence between the general effect propagation process definition and the classical definition of causality.

\textbf{Causaloid as building block}

In the Effect Propagation Process framework, due to the detachment from a fixed spacetime, this fundamental temporal order is absent. Consequently, the entire classical concept of causality, where a cause must happen before its effect, can no longer be fundamentally established. The distinction between a definitive 'Cause' and a definitive 'Effect' becomes untenable as Russell foresaw. When the separation between cause and effect becomes untenable, then the obvious question arises: why even preserve an untenable separation?

Therefore, the Effect Propagation Process framework adopts the causaloid, a uniform entity proposed by Hardy\cite{HardyDynamicCausalStructure}, that merges the ‘cause' and 'effect' into one entity. Instead of dealing with two nearly identical concepts discernible from each other by temporal order, the causaloid is one concept that defines causality in terms of its effect transfer without presupposing a fixed spacetime background. It is important to note that the EPP framework adopts the conceptual role of the Causaloid as a spacetime-agnostic unit of causal interaction, inspired by Hardy’s work on Quantum Gravity, but it does not uses Hardy's formal definition. Instead, the EPP formulates the Causaloid as an abstract structure, thereby decoupling it from any particular physical theory while preserving its core philosophical utility and making it practical implementable in software.
\newline
In the post-quantum context, the term "propagation" does not imply movement through a pre-existing space or time in the classical sense. Instead, it refers to the fundamental process by which an effect is transferred within the underlying structure of reality itself. This fundamental process is what gives rise to the appearance of propagation through spacetime in the classical view. Furthermore, while classical causality relies on a definite temporal order, the Effect Propagation Process treats temporal order as an emergent property, arising from the fundamental process itself.

While the Effect Propagation Process involves the transfer of effects within the fundamental structure, it is crucial to distinguish this from mere accidental correlation. The process reflects the fundamental way the underlying structure of reality establishes dependencies between its components and how it is gives rise to the non-accidental relationships we recognize as observed causal relations.This fundamental determination, rather than simple co-occurrence, is what the "Effect Propagation Process" captures at the deepest level.\newline
Consequently, causality is understood as an effect propagation process that emerges from the fundamental structure or set of rules (akin to a generating function) from which spatiotemporal relationships emerge. This philosophical concept of the Effect Propagation Process finds support in physical theories that propose fundamental structures underlying spacetime, such as Causal Set Theory, and generalizes the idea of effect transfer present in Quantum Field Theory.

Lastly, the Effect Propagation Process offers a philosophical interpretation for mathematical tools that describe non-classical causal behavior, such as Process Matrices. However, the Effect Propagation Process remains agnostic to the operational details of any particular physics theory and, instead, it offers a coherent way of thinking about causality that aligns with the absence of finite spacetime in the quantum realm.

\section{The Epistemology of the Effect Propagation Process:}
\label{sec:epp_epistemology}

Epistemological approaches to acquiring knowledge in research fall into three categories: positivism, interpretivism, and pragmatism. Positivism concerns itself with observable facts based on the scientific method and thus seeks to achieve generalizability and objectivity. Interpretivism maintains that our knowledge depends greatly on our interpretation of observations of human actions, experiences, and environments thus making interpretive research more subjective. Pragmatism focuses on practical effects or solutions to address problems that are suitable for existing situations or conditions. The epistemology of pragmatism is that knowledge is a self-correcting process based on experience thus, it must be evaluated and revised in view of subsequent experience.

The  presented EPP epistemology changes depending on whether the context is static or dynamic, and, equally profound, whether the EPP is static, dynamic, or emergent. A detailed exploration of the epistemological structure of the EPP including formal definitions, knowledge ontologies, multi-modal truth justification, and the role of observer perspectives, is provided in a dedicated companion paper\cite{Hansen2025EPP_Epistemology}. 

\textbf{Ontology of Knowledge sources}

In the EPP, the context is designed as the source of factual knowledge. For context, facts may remain invariant (e.g. the value of Pi) or receive continuous updates. The designation whether a context is static or dynamic refers to its structure, not to the factual data in it. Furthermore, a context might be shared between two or more defined EPP and an EPP may use one or more context(s) thus simplifying modeling complex domains.
The EPP encodes each causal relationship in a designated Causaloid. The Causaloid encodes the causal rule, whereas the context encodes supporting data required to apply the rules. The Causaloid may use external data or data from the context to apply its rule.
For example, a context may encode a continuous signal feed from a LIDAR sensor and the Causaloid encodes a rule to infer whether an obstacle has been detected. In this case, the context provides all data. In another scenario, a context may encode several known defect patterns, a Causaloid tests incoming image data for the defect data from the context, but uses incoming real-time image feeds from a manufacturing monitoring system to determine if any of the produced items contain known defects. In this case, the Causaloid relies on context and external data. Therefore, the Effect Propagation Process emits a flexible knowledge ontology by relying on one or more contexts and potentially multiple external data sources.

\textbf{Knowledge Derivation}

The EPP derivates knowledge by applying data to the Causaloid that models that causal relationship to determine whether the causal relation holds true within the applicable context. Consequently, multi-stage reasoning maps directly to the topology of the EPP itself because each effect from a Causaloid propagates further through the EPP topology, which is the structure of the EPP manifested as all connected Causaloids.
From this perspective, a “line of reasoning” literally becomes a pathway through the EPP topology.
Through the topological approach of knowledge derivation, the Effect Propagation Process provides a flexible way to model complex, contextual, multi-causal domains.

\textbf{Justification of Derived Knowledge}

Discerning the truthfulness of knowledge is one key element of epistemology. The EPP with its explicit context, explicit causal function (Causaloid), and explicit support for external data provides all pre-requirements to support the full explainability of each inference. Furthermore, in the case of multi-stage reasoning, the sequence of applied Causaloids establishes the order or explainability.
Fundamentally, the EPP leads to explainable causal inference because of complete data, context, and inference function when assuming a static EPP. For a dynamic or emergent EPP, explainability might not be guaranteed for all potential state transitions. An implementation of the EPP has to specify the exact details to support explainable inference and where to establish sensible constraints on explainability.

The gravitas of the EPP epistemology is rooted in its flexible, contextualized ontology, a powerful knowledge derivation mechanism, and its intrinsic support of explainable causal inference. The epistemology varies depending on whether the EPP process is static, dynamic, or emergent.

\newpage

\subsection{Static EPP Epistemology}

For a static Effect Propagation Process, the knowledge is explicitly modeled during the design stage and confined in the context. The quantitative nature of explicitly modeled context and EPP leads to the positivism of the resulting epistemology.

\textbf{Static context}

A static context emits an invariant structure after it's defined, therefore a static EPP combined with a static context allows for the strongest deterministic guarantees albeit at the expense of flexibility. Static contexts remain an invaluable tool to model contextual data that remain structurally invariant, which is a common situation when integrating external data sources. The content, structure, richness, and accuracy of that static context profoundly determine the epistemology of what can be known through the EPP.

\textbf{Dynamic context}

In a dynamic context, the context structure itself evolves e.g. new elements (i.e. quarter of a year) are added as the data feed progresses. By definition, a dynamic context relies on a generating function to gauge the dynamic changes of the context. The impact on the epistemology of a static EPP remains minimal though.
Fundamentally, dynamic contexts are used when structural elements occur at either regular intervals or otherwise determinable occurrences, and therefore, the EPP can model these elements regardless of whether they have been added to the context yet.
For example, a Causaloid that determines whether the sum of the previous three monthly financial reports matches the quarterly financial reports for the current quarter might be a precondition if the “current” quarter in the context has been updated. Therefore, dynamic contexts simplify domain modeling while leaving the epistemology modeled in a static EPP intact.

\subsection{Dynamic EPP Epistemology}

For a dynamic Effect Propagation Process, the dynamics are captured in generative functions that evolve either the EPP, the context, or both. Conceptually, these generative functions could range from deterministic, rule-based algorithms that construct or modify Causaloids and Context structures based on predefined logic or specific triggers, to more adaptive mechanisms. For instance, a generative function for a dynamic Context might be a higher-order function that, given the current state and new inputs, returns an updated Context graph, a practice well established in functional programming to build dynamic systems.

The ontology of knowledge may evolve as a result of the evolving EPP and the impact of the epistemology remains deepening not only the EPP evolution itself but the interaction with its context as it can happen that both, the EPP and its context evolve dynamically.

\textbf{Static context}

For a dynamic EPP, a static context may serve as the foundational layer that captures core data that remain structurally invariant. As with a static EPP, the static context determines fundamentally what determines the epistemology of what can be known through the dynamic EPP.

\textbf{Dynamic context}

For a dynamic context, though, the impact on the epistemology captured in a dynamic EPP can be profound. For example, with the advent of model context protocol (MCP), which lets LLMs call into tools to retrieve or modify data, a causaloid in a dynamic EPP may trigger an MCP invocation, which then updates the context by expanding its structure, and then triggers a generative function that creates a new Causaloid based on the retrieved contextual data, which then analyzes either a newly created part of the context, or a new external data feed created by the MCP. As a consequence, the epistemology in this case depends on a dynamic EPP-Context co-evolution.

\textbf{Dynamic Co-Evolution}

When both, the EPP and the engulfing context evolve dynamically in what can be seen as a co-evolution, then no fixed epistemology can be established anymore because the inference based on generated Causaloid over newly added sub-structures of the context may or may not occur depending on the occurrence of the underlying trigger event(s). One could estimate a potential epistemology by using a Rubin causal model\cite{rubin2005causal} (RCM) by comparing potential reasoning outcomes under different scenarios in which a Causaloid was generated versus when not.

More profoundly, it might not be possible any longer to use automated explainability to discern the appropriateness and relevance of the generated Causaloids and contextual shifts in response to external changes. This introduces an element of interpretivism to the resulting epistemology: the derived knowledge requires the observer to apply a conceptual framework for understanding the system's complex and dynamic evolving behavior.

\subsection{Emergent EPP Epistemology}

Unlike a dynamic EPP, an emergent EPP does not evolve anymore based on pre-determined triggers that initiate pre-defined generative functions. Instead, an EPP is considered emergent when the underlying generation process leads to novel causal configurations, reasoning pathways, or new generative principles for Causaloids context interactions that were not explicitly encoded beforehand.
The generation process may incorporate principles from evolutionary computation, novelty search, or machine learning embedded in the EPP itself. While the full exploration of AI-driven generative functions for EPP remains future work, the foundational idea of using programmable functions to dynamically define and evolve both the EPP and its context is a natural extension of EPP's core design philosophy.

Regardless of the mechanism, emergent EPP does not interact with its Context using pre-defined procedures but instead relies on procedures generated by the EPP itself. The key indicator of emergence is that its novel behavior was not foreseeable by its initial designers.

\textbf{Static context}

Like a static or dynamic EPP, when the static context has been defined upfront, it determines fundamentally what determines the epistemology of what can be known through the dynamic EPP within the contextual boundaries.

Unlike a static or dynamic EPP, an emergent EPP may or may not generate a new static context and that indeed alters the Epistemology emerging from an emerging EPP.

\textbf{Dynamic context}

Likewise, when an emerging EPP creates or modifies a dynamic context, the emerging Epistemology cannot be determined any longer because of the resulting co-emergence of the EPP and its context.

\textbf{Dynamic Co-Emergence}

An EPP co-emerges with its context when the underlying generation process leads to novel causal configurations that were not explicitly encoded beforehand. This can happen when the EPP contains methods of machine learning that evolve the EPP itself in response to a dynamically changing context. As a result of the dynamic, non-deterministic self-modification of the EPP itself, the spectrum of subsequent factual representation in the context and the emerging causal structures cannot be predicted any longer.

Therefore, determining the truthfulness of the emerging causality imposes a non-trivial challenge that adds an elevated level of pragmatism to the epistemology. The pragmatism becomes necessary because it is not guaranteed that the underlying dynamic context always leads to a truthful representation of the world it seeks to model, but the generated causal relationships may not always be correct either. Both can happen due to generative errors during the EPP. Generative errors may result from complex interactions that contain steps that, in isolation, are correct, but when combined in a certain order may lead to an incorrect outcome. This is a typical characteristic of increasingly complex dynamic systems that need to be considered by taking an operational stance on truth.

\subsection{Operationalized meaning of truth in EPP}

The meaning of truth evolves depending on the modality of the EPP because the underlying reference for a true statement varies depending on the chosen modality. Per the definition of knowledge, a true belief must entail a high degree of justification and come from a reliable source to count as knowledge. It is the underlying justification process that depends on the modality of the EPP that causes the shift in the meaning of truth to vary.

For a static Effect Propagation Process, the meaning of truth aligns with the classical correspondence theory. That means, that if the context encodes accurate facts and the causal relationships are true, all derived forms of knowledge must be true.
Justification rests upon the verifiable mapping between the EPP's explicit model encoded in its context and the part of reality it purports to represent. The static EPP implicitly operates under the assumption that its model is a faithful mirror of objective facts. Here, the truth of an inference is determined by the adherence to the contextual facts and encoded causal relationships. As a result, a static EPP leads to deterministic verifiability within the confined boundaries of its context.
As the EPP transitions into a dynamic modality, the meaning of truth begins to shift towards a coherent adaptability to dynamic interaction with a changing context. A dynamically modified causal relationship is deemed true if it maintains consistency with the facts in its evolving context. In a dynamic modality,  the justification of knowledge becomes contextually and temporally aware. Therefore, truth is assessed by the EPP's capacity to maintain a relevant and internally consistent causal understanding amidst navigating a temporal dynamic context.  This leans towards a coherence theory of truth, where coherence itself must be evaluated relative to the EPP’s intricate temporal structures.

\newpage

For a contextual co-emergent EPP, the meaning of truth shifts further toward pragmatic efficacy. This shift becomes necessary because of the emergence of relativistic causal relationships from the EPP that co-evolve with its context. Here, establishing an objective a priori truth becomes elusive since the fabric that would traditionally serve as a stable reference for truth is itself emerging dynamically alongside the causal inference made from it. Instead, the truth of an emergent causal inference is established by its utility in enabling the EPP to navigate its environment within its temporally complex context.
This pragmatic efficacy means that truth, defined by its functional value, becomes inherently system-relative and context-dependent.

Indeed, a functional value could serve as a fitness function guiding the emergent process itself thus raising fundamental questions about alignment. Consequently, pragmatic efficacy can lead to multiple, functionally 'true' yet distinct causal understandings, each valid within its own emergent trajectory and its relativistic interrelation with its context.

This dynamic interplay, where the EPP generates both its context and the Causaloids that encode the causal relationships that operate within that context presents a research opportunity. It allows for the exploration of relativistic emergent causality and how coherent and pragmatically effective causal understandings can arise in systems that lack a fixed predefined spacetime. This might be of interest in theories of fundamental physics where spacetime itself may be an emergent phenomenon arising from more fundamental processes.

\section{Causal Emergence}
\label{sec:causal_emergence}

The problem of modeling no a-priori causal structures motivates a different view of causality that sets the stage for tackling  causal emergence. The Effect Propagation Process framework's detachment from fixed spacetime and its focus on a generative function establish the foundation" for causal emergence. Static causal discovery, for example Pearl’s DAGs framework, assumes a fixed causal structure and thus cannot handle causal emergence. Granger causality assumes that time-dependent variables change, but the causal structure remains fixed and therefore cannot handle causal emergence either. This argument holds true for any dynamic system with fixed causality because of the inability of the underlying methodology to handle spacetime-agnostic causal structure. The Effect Propagation Process instead proposes that the causal relationships themselves emerge, change, and may even disappear. The implications of this approach lead to a fundamental reassessment of how to operationalize causality:

\textbf{Causal discovery}

Instead of trying to find a fixed causal structure, EPP models how causal relationships emerge from underlying (dynamic) processes. The existing work on causal discovery remains valid; the EPP, however, takes the idea takes the idea one step further by incorporating a dynamic generative process. Further research will verify the utility of this perspective, but at least it expands the notion of discovering a static structure to describing a dynamic process.

\textbf{Causal transferability}

Instead of trying to capture the exact conditions under which a causal relationship holds true, the EPP specifies all presumptions as a generative function, which makes it fundamentally testable and thus transferability can be decided.

\textbf{Causal dynamics}

The inspiration from causality in quantum gravity was carefully chosen because of its unique ability to reconcile dynamic and static structures and its handling of deterministic and probabilistic modality. The underlying idea in quantum gravity is that the spacetime fabric of reality itself emerges from an underlying process. While we do not have the scientific methods and technology to verify this idea on the quantum level, we can carefully, within boundaries, transfer the idea to the EPP notion that causality itself emerges from an underlying generative process and, therefore, model the dynamics of causal emergence. The properties of EPP become apparent when looking from the lens of modeling the dynamics of causal emergence.

\begin{itemize}
    \item \textbf{Temporal order becomes irrelevant:} Because causal relations can emerge from an underlying process
    \item \textbf{Spacetime-agnostic becomes necessary:} Because the generative process is concerned with establishing relations of effect propagation, the exact fabric through which those effects propagate is conceptualized as an external context; therefore, EPP itself has to be spacetime-agnostic.
    \item \textbf{Hardy's Causaloids are necessary:} Because the EPP itself is spacetime-agnostic, a different representation of causal relationship that is also spacetime-agnostic becomes necessary and the causaloid proposed by Lucian Hardy has been deemed the best fit.
    \item \textbf{Centrality of the generative function:} Because the EPP can represent causal relations as either static, dynamic, or emergent, the generative function takes on a central role to express those causal relations. Furthermore, a generative function may generate the engulfing context as a specific fabric for the effect propagation process.
\end{itemize}


The Effect Propagation Process framework constitutes a  foundational shift to viewing causality itself as emergent and thus redefines what causality means in dynamic systems. Because of its flexibility, EPP can express static causal relationships similar to Pearl’s Causal DAG, it can handle dynamic causal systems similar to Dynamic Bayesian Networks, but then goes further and enables dynamic causal emergence. Dynamic causal emergence has real-world applications:

\begin{itemize}
    \item \textbf{Financial Markets:}  Causal relationships between assets change based on market conditions. EPP can be used to model how these relationships emerge and dissolve.
    \item \textbf{Biological Systems:} Gene regulatory networks where modulating relationships emerge based on cellular state.
    \item \textbf{Social Systems:} How influence relationships emerge and change in social networks.
\end{itemize}

\section{Contrast}
\label{sec:contrast}

The classical concept of causality existed for millennia and served mankind well until the dawn of the quantum era. The Effect Propagation Process does not seek to replace the classical notion of causality, but instead seeks to advance and generalize the concept of causality to match the now more advanced understanding of fundamental science. To this end, this section contrasts the Effect Propagation Process (EPP) with the classical definition of causality to pinpoint the exact differences:

\subsection{Detachment from Fixed Spacetime}

\textbf{Classical causality:} 
Classical causality assumes an implicit Euclidean context with a fixed spacetime. This is a direct consequence of Seneca’s definition. The advent of general relativity changed the notion of a fixed spacetime background towards a dynamic spacetime background with the sole implication that causality became dynamic relative to its engulfing spacetime.

\textbf{EPP:} 
The EPP removes the constraint of spacetime altogether and does not assume any particular fabric in which effects propagate. While classic causality cannot operate in arbitrary, non-spatiotemporal, non-Euclidean structures, EPP  can.


\subsection{Agnostic to Temporal Order}

\textbf{Classical causality:} 
Classical causality requires a strict linear temporal order, i.e., a cause must precede its effect.

\textbf{EPP:} 
 As EPP removes spacetime altogether, it does not impose any particular temporal order. By extension, there is also no imposed spatial order with the understanding that the EPP would require the underlying fabric to define temporal or spatial order if it were to use these orders during effect propagation.
Instead of seeing temporal order as a fixed background, temporal order is seen as an emergent property. This removes the constraint of fitting all causal interactions into a rigid before-after sequence, opening the door to modeling complex systems with feedback loops.

\subsection{Causaloid as Uniform Building Block}

\textbf{Classical causality:} 
Classical causality discern a cause from an effect merely by the imbued temporal order which stipulates that a cause must happen before its effect.

\textbf{EPP:} 
In the absence of a temporal order in EPP, there is no meaningful way to discern a cause from an effect, and consequently, the EPP does not make this distinction any longer. Instead, the EPP relies on Hardy’s causaloid which folds cause and effect into a single entity of testable effect transfer.
The absence of a temporal order does not imply the absence of time; it means the absence of a strict order. That means, EPP allows the modeling of complex systems with bi-directional causal influence, a feat that is hard to accomplish with the classical definition of causality.

\newpage

\subsection{Focus on the Generative Function}

\textbf{Classical causality:} 
Classical causality tries to simplify reality into tractable causal chains operating on observable states.


\textbf{EPP:} 
EPP does not presume the existence of a linear causal chain. Instead, it assumes the existence of a generating function from which the observable fabric (i.e., spacetime) and causal relationships emerge. The effects then propagate through the observable fabric.

\subsection{Embracing Indefinite Causal Order}

\textbf{Classical causality:} 
Classical causality assumes the existing of a definitive causal structure that, once discovered, can be modeled. This leads directly to the notoriously hard causal discovery problem as it is not yet clear how to find a definitive causal structure assuming it exists.

\textbf{EPP:} 
EPP does not assume the pre-existence of a definitive causal structure. Instead, it embraces indefinite causal order in which it is not yet clear if A happened before B or B happened before A. Conceptually, this mechanism not only allows for the existence of superposed causal pathways, but enables the handling of emergent causal structures. While the practicality of superposed causal pathways remains an open topic, there is a clear use case for emergent causal structures in dynamic systems. Moving away from a presumed definitive causal structure towards emergent causal structures in dynamic systems elevates causal modeling to deal with the intricate uncertainty of dynamic systems.

\section{Validity}
\label{sec:validity}

In empirical science, internal validity focuses on the extent to which a study accurately establishes a causal relationship between variables, whereas external validity considers the extent to which the findings can be generalized.
In the absence of empirical evidence, it is appropriate to consider the soundness and consistency as internal validity and consider the boundaries of generalization as a proxy for external validity.

\subsection{Internal validity}
\label{sec:validity_internal}

The \textbf{soundness} of EPP derives from the first principled logical progression of the presented argument. The stated problem of inadequacy of classical causality in light of the challenges imposed by Quantum Gravity (emergent spacetime, indefinite causal order) is well-recognized and documented\cite{MriniHardyIndefinite}.. From this recognized problem, the argument for EPP progresses as stated below:

\begin{enumerate}
    \item Establishes classical causality and its historical critiques.
    \item Introduces the challenges from modern physics (GR, QG).
    \item Shows why these challenges render classical causality insufficient.
    \item Proposes EPP as a coherent response to these challenges.
    \item Contrasts EPP with classical causality.
    \item Discusses its ontological, epistemological, and teleological implications.
    \item Acknowledges and addresses threats to its validity.
\end{enumerate}

The soundness is further strengthened by its inspiration from established scientific theories (QFT, GR) and foundational work in quantum gravity. While quantum gravity as a scientific theory remains a work in progress, the conceptual challenges that arise from it are valid regardless of how any particular theory may explain the underlying quantum mechanisms.

\newpage

The \textbf{consistency} of the EPP framework arises from a handful of carefully stated conclusions.

\subsubsection{Spacetime Agnosticism \& Causaloids}

\textbf{Premise:} On a quantum level, spacetime may not exist.

\textbf{Conclusion:} Therefore, remove spacetime from EPP.

\textbf{Premise:} EPP does not have a defined spacetime.

\textbf{Conclusion:} Cause and effect cannot be separated anymore because there is no a priori temporal order.

\textbf{Premise:} Cause and effect cannot be discerned because of missing temporal order.

\textbf{Conclusion:} Fold cause and effect into one entity, the causaloid, that is independent of temporal (and spatial) order.

Note, the last conclusion holds because of the temporal order required for classical causality. The only logical alternative conclusion from the premise of missing temporal order would be to abandon causality altogether. However, this conflicts with the reality in which causal relationships indeed exist; therefore, the alternative has been deemed unsound.

\subsubsection{Effect Propagation as the Essence of Causality}

\textbf{Premise:} The causaloid, as a building block, is independent of temporal (and spatial) order.

\textbf{Conclusion:} Define causality by what it does (propagate effects) instead of what it was thought to be (a temporal order dependent atomic relationship).

The careful reader may raise a concern over the choice of words (i.e., propagate effects), since “transferring information” or “transmitting influence” might be equally valid choices. True, that is indeed a fair point. However, the term information has specific meaning in information theory and computer science, and likewise, the term influence has specific meaning in social science; therefore the author settled cautiously on “effect” mainly to prevent conflating different meanings.

\subsubsection{Compatibility with Classical Causality}

The full logical argument of how classical causality is derived from EPP is in the section “Causality as Effect Propagation Process”.
While classical causality can be derived from EPP, the reverse is not true because EPP cannot be derived from classical causality because classical causality requires a background spacetime. That deduction proves that EPP is more abstract in the sense of more general than classical causality. Therefore, it follows that EPP naturally applies to areas where classical causality cannot be used any longer. The internal validity of EPP roots in its internal soundness and consistency that stems from its first-principles reasoning. Therefore, ambiguity, contradictory claims, and unjustified leaps in logic are avoided to the extent it is possible. Minor mistakes might be possible and the author is open to suggestions  to improve EPP further.

\newpage

\subsection{External validity}
\label{sec:validity_external}

Establishing the boundaries of generalization as a proxy for external validity requires a delicate balance of realistically acknowledging what EPP can address versus avoiding overstating  any particular capability. Related to external validity is always the possibility of an alternative interpretation of the underlying premises.

\subsubsection{Alternative interpretations}

There are several potential alternative interpretations of the premises underlying the Effect Propagation Process framework.

\textbf{Russell was more right than acknowledged}

One can take the position that, if Russell deemed causality as a relict of a bygone era, then why not openly ask to abandon causality altogether and focus solely on descriptive and correlation-based data science?

DARPA disagrees\cite{DARPA_ANSR}:

\begin{quote}
    “In the real world, observations are often correlated and a product of an underlying causal mechanism, which can be modeled and understood.”
\end{quote}

The problem is not correlation versus causation as an either-or distinction. Instead, EPP solves the problem of lifting causality into a spacetime-agonistic generalization that is required to model increasingly complex dynamic systems that do not longer adhere to a fixed background spacetime.

\textbf{Pluralism of causal concepts}

Instead of a unified framework like EPP, one can argue in favor of the existing pluralistic reality where multiple disjoint concepts of (computational) causality exist for different levels of causal analysis.

As stated before, EPP does not seek to replace classical causality and all tools that are built atop the classical definition. Instead, EPP seeks to advance the core concept of causality to meet increasingly challenging demands. Due to the novelty of this foundational work, a single coherent framework is preferred until it becomes clear which parts may branch out into more specialized applications.


\textbf{Different Interpretations of Quantum Gravity}

It is perfectly possible that over time, a different interpretation of quantum gravity may arise because QG is far from settled and further advancement is expected. However, EPP aligns with the general trend of QG research, but isn't rooted in any particular theory of quantum gravity.

The only fundamental work EPP assumes to remain valid is Hardy’s causaloid and, for good reason, because it is the only known attempt to generalize causality to fit into the realm of quantum mechanics and the realm of general relativity. It is important to stress that the causaloid is a necessary stepping stone to establish a foundation for post-quantum causality in the absence of empirical validated scientific facts.

If Hardy’s causaloid would ever be invalidated by verified experimental results, EPP would indeed need a revision. The author remains open to suggestions to revise EPP in that case to re-establish a stronger foundation.

\newpage

\subsubsection{Boundaries of Generalization}

The main boundary to EPP comes down to preventing the application of any quantum principle to macroscopic systems.

EPP does not endorse nor attempt to apply any quantum principle to non-quantum systems. Instead, it takes inspiration from quantum gravity to advance causality. Furthermore, it is crucial to understand that EPP does not propose that macroscopic complex systems are quantum mechanical in the way they physically operate. Instead, EPP attempts to transfer concepts from quantum gravity to advance causality further to model advanced dynamic systems with complex feedback loops that are notoriously hard to model with conventional causality tools.

Another boundary relates to the lower limit of complexity that can be solved with EPP. For example, abandoning the cause-effect distinction via the causaloid is too radical for many applications. EPP remains compatible with classical causality and therefore can potentially be used where classical causality would apply. However, when a simpler classical causality methodology can solve the problem at hand, it should be preferred over any more complex methodology.

To prevent conceptual overreach, the author does not claim to have invented post-quantum causality even though its inherent independence of spacetime would suggest this view. Furthermore, the author does not claim that the Effect Propagation Process framework would solve open questions in quantum physics or quantum gravity. However, because of the early stage of quantum gravity research, it is far from settled how causality fits into the suggested conceptualization of emergent spacetime. Instead, the Effect Propagation Process, as a philosophical framework inspired by concepts of quantum gravity, is defined as a spacetime-agnostic continuous effect propagation process.

\subsubsection{Method Selection Criteria: Classical Causality vs. EPP}

In cases where methods of classical causality and conventional machine learning do not solve the problem at hand, methodologies rooted in EPP might be preferred. The following decision matrix supports the assessment of when to use which methods. This matrix provides guidance on selecting an appropriate causal modeling approach based on the temporal and spatial complexity of the system under investigation.

% Please add the following required packages to your document preamble:
\begin{table}[hb]
\begin{tabular}{llll}
Feature Assessed &
  System Characteristic &
  Classical Methods &
  EPP Methods \\ \hline
\multicolumn{1}{|l|}{Temporal Complexity} &
  \multicolumn{1}{l|}{} &
  \multicolumn{1}{l|}{} &
  \multicolumn{1}{l|}{} \\ \hline
\multicolumn{1}{|l|}{} &
  \multicolumn{1}{l|}{Single time scale + linear progression} &
  \multicolumn{1}{l|}{Sufficient} &
  \multicolumn{1}{l|}{} \\ \hline
\multicolumn{1}{|l|}{} &
  \multicolumn{1}{l|}{Multiple time scales OR non-linear temporal relationships} &
  \multicolumn{1}{l|}{May struggle} &
  \multicolumn{1}{l|}{Consider EPP} \\ \hline
\multicolumn{1}{|l|}{} &
  \multicolumn{1}{l|}{Multiple time scales AND non-linear temporal relationships} &
  \multicolumn{1}{l|}{Insufficient} &
  \multicolumn{1}{l|}{EPP Required} \\ \hline
\multicolumn{1}{|l|}{Spatial Structure} &
  \multicolumn{1}{l|}{} &
  \multicolumn{1}{l|}{} &
  \multicolumn{1}{l|}{} \\ \hline
\multicolumn{1}{|l|}{} &
  \multicolumn{1}{l|}{Euclidean space with fixed coordinates} &
  \multicolumn{1}{l|}{Sufficient} &
  \multicolumn{1}{l|}{} \\ \hline
\multicolumn{1}{|l|}{} &
  \multicolumn{1}{l|}{Non-Euclidean OR dynamic spatial relationships} &
  \multicolumn{1}{l|}{Limited/Difficult} &
  \multicolumn{1}{l|}{EPP Advantageous} \\ \hline
\multicolumn{1}{|l|}{} &
  \multicolumn{1}{l|}{Non-Euclidean AND dynamic spatial relationships} &
  \multicolumn{1}{l|}{Very Limited} &
  \multicolumn{1}{l|}{EPP Required} \\ \hline
\multicolumn{1}{|l|}{Combined Complexity} &
  \multicolumn{1}{l|}{} &
  \multicolumn{1}{l|}{} &
  \multicolumn{1}{l|}{} \\ \hline
\multicolumn{1}{|l|}{Scenario 1} &
  \multicolumn{1}{l|}{Linear Time \& Euclidean Space} &
  \multicolumn{1}{l|}{Preferred} &
  \multicolumn{1}{l|}{} \\ \hline
\multicolumn{1}{|l|}{Scenario 2} &
  \multicolumn{1}{l|}{Non-Linear Time OR Complex Spatial} &
  \multicolumn{1}{l|}{Limited/Difficult} &
  \multicolumn{1}{l|}{Consider EPP} \\ \hline
\multicolumn{1}{|l|}{Scenario 3} &
  \multicolumn{1}{l|}{Non-Linear Time AND Complex Spatial} &
  \multicolumn{1}{l|}{Insufficient} &
  \multicolumn{1}{l|}{EPP Required} \\ \hline
\end{tabular}
\caption{Causal Method Selection Matrix}
\label{tab:method_matrix}
\end{table}

\textbf{Emergent Causality:} If the problem involves emergent causal structures (where the causal graph itself is not fixed and changes dynamically based on context, i.e., dynamic regime shifts), EPP-based methodologies become the only viable option.

\newpage

\section{Conclusion}
\label{sec:conclusion}

The Effect Propagation Process framework rethinks causality as a continuous transfer of effects originating from a potentially non-spatiotemporal underlying structure. 

The EPP is implemented in the DeepCausality\footnote{\url{https://deepcausality.com}} Rust library, which is a hypergeometric computational causality library that enables fast and deterministic context-aware causal reasoning across Euclidean and non-Euclidean spaces. This operational realization substantiates the EPP’s core principles and demonstrates he EPP’s applicability to real-world complex systems.  DeepCausality finds its application in modeling dynamic control systems used in financial markets, advanced analytics, and complex control systems.

The framework navigates the challenges of spacetime agnosticism, aligns with the concept of emergence, accommodates indefinite causal order, and remains compatible with classical causality. Furthermore, the notion of an Effect Propagation Process reshapes the ontology of causality by suggesting that the most fundamental reality is not spacetime, but the generative process that materializes the spacetime context through which effects propagate.

The epistemology of the Effect Propagation Process reflects the complex systems it is designed to model by scaling with the modality of the EPP. For a static EPP, a positivist epistemology remains sufficient. For a dynamic EPP, the epistemology evolves towards an interpretivism perspective, and for an emergent EPP, a pragmatism perspective on the epistemology becomes necessary.

Likewise, for the justification of knowledge in an EPP, the underlying notion of truth scales with the modality of the EPP. For a static EPP, the meaning of truth aligns with the classical correspondence theory. However, in a dynamic EPP, the meaning of truth shifts towards a coherent adaptability approach. In an emergent EPP, the meaning of truth evolves towards pragmatic efficacy where the validity of relativistic, emergent causal relationships is established by their functional utility.

The framework provides a robust conceptual grounding for exploring the nature of causal effect propagation in a universe that may fundamentally defy classical intuition while leaving traditional teleology as a likely emergent property. It redefines the scope and methods of epistemology, shifting the focus of causal knowledge from observing event sequences in spacetime to inferring the rules and dynamics of this deeper process.

The Effect Propagation Process offers a unified philosophical language of causality that is powerful enough to handle new challenges, remains compatible with classical causality, and conceptually aligns with contemporary theories of quantum gravity. Future work may explores the realm of dynamic emergent causality further. 


\newpage

%Bibliography
\bibliographystyle{unsrt}  
\bibliography{references}  


\end{document}
